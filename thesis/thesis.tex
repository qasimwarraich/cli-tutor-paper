\documentclass{seal_thesis}

\thesisType{Master's Thesis}
\date{\today}
\title{CLI-Tutor}
\subtitle{Can interactive learning make the command line more approachable?}
\author{Qasim Warraich}
\home{Lahore} % Geburtsort
\country{Pakistan}
\legi{18-787-796}
\prof{Prof. Dr. Harald C. Gall}
\assistent{Dr. Carol V. Alexandru-Funakoshi}
\email{qasim.warraich@uzh.ch}
\url{gitlab.com/qasimwarraich/cli-tutor}
\begindate{2022-03-14}
\enddate{2022-09-14}

%% OUTLINE
% - Abstract
% - Overview
%   - problem description
%   - introduce tool
%   - outline thesis
% - Introduction
% - Tool (CLI-Tutor)
%   - overview
%       - cirriculum
%       - lesson design
%       - usability considerations
%       - safety considerations
%   - web version
% - Design and Implementation
%   - overview
%   - original tinkering and considerations
%   - features and considerations
%   - architecture of cli-tool
%   - architecture of web-tool
% - User Study
%    - methodology
%       - assignment
%       - survey structure
%    - participants
%    - findings
% - Related Work and Reflections
%   - related work
%   - improvements
%   - future work
% - Conclusion
% - Appendix
%   - Detailed dev considerations

\begin{document}
\maketitle

\frontmatter

% acknowledgements

\begin{acknowledgements}
I would like to acknowledge my thesis advisor Dr Carol V. Alexandru-Funakoshi
for his continual support, encouragement and effortless management of time
zones throughout this thesis work. Additionally, I would like to thank my dear
friends Dominique and Jasmin for their design advice, testing support and
encouragement. Lastly, I would like to express my gratitude to all the
participants of the user study associated with this thesis work. Thank you for
your time, valuable feedback and kind words.

\end{acknowledgements}


% abstract(s)
\begin{abstract}

Despite the arguably dated appearance, difficult learning curve and practical
non-existence in the modern personal computing space, Command Line Interfaces
(CLIs) have more than stood the test of time in the software development world.
There are a multitude of extremely popular tools and applications that
primarily focus on the command line as an interaction medium. Some examples
include version control software like git\footnote{git source code version
control tool: \href{https://git-scm.com/}{https://git-scm.com/} }, compilers
and interpreters for programming languages, package managers and various core
utilities that are popular in areas such as software development, scripting and
system administration. Command line interfaces are also utilised in areas
outside of software development. For example the infamous Bloomberg Terminal in
the financial sector and for general computing purposes such as email (mutt,
neomutt) and text editing (Vim, Neovim, Wordstar, etc...). 

As mentioned before, the use of the command line as an interaction paradigm has
effectively disappeared from a mainstream personal computer usage perspective.
This reality contributes greatly to the intimidation factor and learning difficulty for
those interested in getting into software engineering or system administration.
This unfamiliarity, paired with the inevitability of usage of CLIs in the
development space highlights a need to make the command line more accessible to
new users for whom text-based interaction with their computer is an alien
concept. In recent years interactive learning utilising tools such as sandboxed
environments have been gaining in popularity and have the potential to be a
suitable medium for learning command line basics through actual usage, examples
and practice.

In this work, we have created just such an interactive tutoring tool tailored
for the command line. "CLI-Tutor" is a forgiving CLI application that aims to
teach topics such as shell basics and Unix-like core utility usage through the
use of guided lessons with interactive examples and feedback at the command
line.

\end{abstract}

% TODO: PROPERLY TRANSLATE THIS 
\begin{zusammenfassung}
\sloppy

Trotz des wohl veralteten Aussehens, schwierige Lernkurve und praktisch
Nichtexistenz im allgemeinen Personal-Computing-Bereich,
Befehlszeilenschnittstellen (CLIs) haben sich in der Welt der
Softwareentwicklung mehr als bewährt. Es gibt eine Vielzahl äußerst beliebter
Tools und Anwendungen, die konzentrieren sich in erster Linie auf die CLI als
Interaktionsmedium. Einige Beispiele enthalten Versionskontrollsoftware wie
`git', Compiler und Interpreter für Programmiersprachen, Paketmanager und
verschiedene Kerndienstprogramme beliebt in Bereichen wie Softwareentwicklung,
Scripting und System Verwaltung.

Wie bereits erwähnt, hat die Verwendung der Befehlszeile als
Interaktionsparadigma praktisch aus der Perspektive der Mainstream-PC-Nutzung
verschwunden. Dies trägt stark zum Einschüchterungsfaktor und zur
Lernschwierigkeit bei die daran interessiert sind, in die Softwareentwicklung
oder Systemadministration einzusteigen. Diese Ungewohntheit, gepaart mit der
Zwangsläufigkeit der Nutzung von CLIs im Der Entwicklungsbereich unterstreicht
die Notwendigkeit, die Befehlszeile zugänglicher zu machen neue Benutzer, für
die die textbasierte Interaktion mit ihrem Computer ein Fremdwort ist Konzept.
In den letzten Jahren interaktives Lernen unter Verwendung von Tools wie
Sandboxed Umgebungen erfreuen sich zunehmender Beliebtheit und haben das
Potenzial, a geeignetes Medium zum Erlernen der CLI durch praktische Anwendung,
Beispiele und üben.

\end{zusammenfassung}


\tableofcontents
\listoffigures
\listoftables
\lstlistoflistings

\mainmatter

% chapters
\chapter{Overview}
%
\section{Problem Description}
This tool aims to determine whether an interactive learning method may ease the
introduction into command line interfaces for novice users, particularly
mitigating the 'scare factor' experienced by first-time users. We do so by
creating a forgiving CLI with the goal of teaching topics such as shell
scripting basics and Unix-like core utility usage through the use of
interactive examples. We draw inspiration from the
`vimtutor'\cite{pierce_ware_smith_moolenaar_2019} Figure \ref{vimtutor} utility
shipping alongside the popular terminal-based text editor Vim. 

The proposed tool shall allow for opt-in analytics that are sent back to a data
collection service for the purpose of learning which mistakes are most commonly
made, and to improve the tool accordingly. To validate the tool and answer our
research questions, a user study will be conducted, most likely with bachelor's
students at The University of Zurich. A secondary goal is to embed the learning
tool into a prototypical web application in order to make it more accessible
and portable.
%
\fig[1\textwidth]{img/vimtutor}{Screen shot of vimtutor}{vimtutor}

\section{Interactive Learning Tool}
\subsection{Introduction to CLI-Tutor}
\subsubsection{Subsubsection}

\paragraph{Paragraph.} Always with a point.


% - Introduction
% - RQs
\chapter{Introduction}
\label{chap:intro}

Command line interfaces are still widely employed and created for a variety of
reasons. Command line interfaces allow for rapid development and the
implementation of a very large amount of features that would require entail
much more detailed and complicated implementation decisions if they were to be
integrated into a graphical interface. Command line programs also benefit a lot
from their inherent simplicity. The simple notion of text input and output
being the main forms of interactions allow for an extremely flexible interface
for chaining tools together. This composition of programs is referred to as
piping ( SOME UNIX PHILOSOPHY MEME STUFF HERE ). This compositional simplicity
paired with the ability to specify arguments and modify program behaviour on
instantiation opens the room for powerful batch processing capabilities,
automation and the ability to deal with large amounts of input and output
efficiently.

The learning curve to such systems is however a strong barrier towards
employments of these benefits. This learning curve is made steeper by the
paradigm shift in the personal computing space towards graphical user
interfaces.



\section{Interactive Learning Tools}
To validate the tool and answer our
research questions, a user study will be conducted, most likely with bachelor's
students at the University of Zurich. A secondary goal is to embed the learning
tool into a prototypical web application in order to make it more accessible
and portable.
This tool aims to determine whether an interactive learning method may ease the
introduction into command line interfaces for novice users, particularly
mitigating the 'scare factor' experienced by first-time users. 

\section{Requirements}

% NOTE: Paragraphs titles should always have a point (.) after the title.
\paragraph{Paragraph.} Always with a point.


\chapter{The CLI-Tutor Tool}
% - Tool (CLI-Tutor)
%   - overview
%       - curriculum
%       - lesson design
%            - deliberate emphasis on  easy vocabulary
%            - running examples
%            - refresher from previous lesson
%       - usability considerations
%       - safety considerations
%       - vocab and validation
%   - web version
%
\label{chap:clitutor}

\begin{lstlisting}[float=htbp, keepspaces, frame=single, language={}, label=lst:clihelp, caption=Output of the help flag of \textit{CLI-Tutor} running in a docker container.]

cli-student@3bc86f9090f9:~/tutor$ cli-tutor -h

        _ _       __        __
  _____/ (_)     / /___  __/ /_____  _____
 / ___/ / /_____/ __/ / / / __/ __ \/ ___/
/ /__/ / /_____/ /_/ /_/ / /_/ /_/ / /
\___/_/_/      \__/\__,_/\__/\____/_/

A simple command line tutor application that aims to introduce users to the
    basics of command line interaction.
    Web version is available at https://clitutor.chistole.ch

Usage:
  cli-tutor [flags]
  cli-tutor [command]

Available Commands:
  completion  Generate the autocompletion script for the specified shell
  help        Help about any command
  info        Prints information about the tool and log collection
  repo        Prints a url to the git repository for this tool
  version     Print the version number of cli-tutor

Flags:
  -h, --help            help for cli-tutor
  -n, --no-upload-log   Do not send a copy of the log to the developer
  -x, --no-welcome      Do not show welcome message when entering tutor

Use "cli-tutor [command] --help" for more information about a command.
\end{lstlisting}


\textit{CLI-Tutor} is a command line application written in the Go
programming language. It is an interactive tutorial focused on introducing
novices to the Linux command line environment. The application is intended to
be a forgiving but faithful representation of a shell running
\textit{Bash}\footnote{\textit{Bourne Again SHell}:
\url{https://www.gnu.org/software/bash/}}. In this chapter, we will discuss the non-technical design considerations and choices made during the development of \textit{CLI-Tutor}.

\section{Overview}
\subsection{Curriculum}

\textit{CLI-Tutor} is intended to be used with zero prerequisite knowledge of
the command line. To achieve this low barrier to entry lessons had to be
designed from the perspective of a complete novice. The curriculum consists of
lessons that introduce the very basics of textual interaction and shell usage.
As of writing \textit{CLI-Tutor} has 5 full lessons implemented as a proof of
concept and serve as the curriculum in the user study performed in this work.
The tool is designed to be easily extensible with new lessons being easy to
contribute. How this is achieved will be discussed later in
\autoref{chap:design}.

\subsubsection{List of implemented lessons} 

\paragraph{Basics of Textual Interaction.} This lesson covers the very
foundational concepts of textual interaction. The user is introduced to the
term \textit{CLI} and explained what a \textit{shell} is using an ASCII
graphic. The user is then introduced to the concept of issuing a command. To
sow interest the user is then asked to execute some commands to illustrate that
the terminal is really interacting with their operating system. The Users are
asked to use the \textit{curl} from command to pull down a weather report from
the \url{wttr.in} as an example of what is possible with the command line.
Furthermore, the user is introduced to some important in-built commands
of \textit{CLI-Tutor} and taught how to clear the screen. Another feature
introduced to the user is \textit{zen-mode}, a feature that prevents the screen
from being cluttered with output of previous commands to prevent the user from
getting overwhelmed. This feature is turned on by default in lesson 1 but is
then turned off unless explicitly requested for the remainder of the curriculum.

\paragraph{Getting more familiar with the shell.} This lesson jumps deeper into
the concept of issuing commands and the "grammar" behind a command. To make
this more intuitive the metaphor of a sentence is used, adapted from the
documentation of the \textit{cobra} command line tool
library\cite{franciacobra}. The lesson introduces users to important concepts
such as sub commands, and flags. The user is then asked to perform a series of
task involving the \textit{wc} word counting core utility. These actions are
performed on a sample file that \textit{CLI-Tutor} creates and serves as a
running example throughout the tutor.

\paragraph{Basics of the file system and practicing commands.} Lesson 3 is all
about the file system. The file system and file system operations. The user is
first introduced to the prompt and given an explanation of all the different
sections of the prompt line. The prompt in \textit{CLI-Tutor} is modeled after
the stock prompt of many popular Linux distributions, consisting of a username,
hostname and current working directory path. The user is introduced to the idea
of a hierarchical tree-like file system and is taught how to navigate around.
The \textit{ls} command is also discussed and used to illustrate the concept of
hidden files. \textit{CLI-Tutor} also includes a hidden file that it places in
the directory the tutor is launched from to help illustrate the concept of
hidden files. All the example files included in lessons are deleted upon
exiting the tutor. The lesson closes with a long example for the user to try
out that includes all the concepts discussed in the lesson.

\paragraph{Shell shortcuts and tricks.} This lesson is aimed to be a bit more
fun and introduces the user to some tricks and shortcuts available in the shell
and the \textit{readline}\cite{ramey_fox_readline} library used for input in a
\textit{bash} shell environment. The user is introduced to the concept of a
shell history file and also of the reverse history search feature with
\textit{Ctrl+r}. The cancel \textit{Ctrl+c} command is also introduced as well
as the \textit{!!} operator. As a last step, tab completion is also introduced.

\paragraph{Helping your self.} This is the last lesson of the current version
of \textit{CLI-Tutor} and covers how a user can go about seeking help at the
command line and helping themselves. The concept of pagers and some important
keybindings of the \textit{less} pager program is first explained before
introducing the \textit{man} command to the user. Users are also taught about help flags and encouraged to checkout the help command shown at the very beginning of this \hyperref[chap:clitutor]{Chapter} (see: \autoref{lst:clihelp}).

\subsection{Lesson Design}
\subsection{Usability Considerations}
\subsection{Safety Considerations}
\section{Web Version}


\begin{figure}[htbp]
    \centering
    \includegraphics[width=0.8\textwidth]{img/climenu}
    \caption{Screen shot of \textit{CLI-Tutor} menu screen.}
    \label{fig:clitutormenu}
\end{figure}
\begin{figure}[htbp]
	\centering
	\includegraphics[width=1\textwidth]{img/cliwebfull}
	\caption{Screen shot of \textit{CLI-Tutor} menu screen in the web version.}
	\label{fig:webversion}
\end{figure}

\begin{figure}[htbp]
	\centering
	\includegraphics[width=1\textwidth]{img/cliexpansionfull}
	\caption{Screen shot of a \textit{CLI-Tutor} lesson showing values interpolated into the lesson.}
	\label{fig:webversion}
\end{figure}



\chapter{Design And Implementation}
\section{Section}
%
\subsubsection{Subsubsection}

\paragraph{Paragraph.} Always with a point.



\chapter{User Study}
% - User Study
%    - methodology
%       - assignment
%       - survey structure
%    - participants
% DATAPOINTS: Interest in interactive learnign or percentage of people who find
% it more effective than reading
\label{chap:userstudy}

In order to test and validate the effectiveness of our solution, a user study
was conducted. The goal of this user study was two part. Firstly, we were
interested in assessing the usability and response to \textit{CLI-Tutor}.
Secondly, we wanted to ascertain if interactive learning would be a more
effective medium to teach command line interaction than the traditional means
such as online documentation or books. In this chapter, we will describe our
user study in detail.


\section{Methodology}

The user study for this thesis work was conducted remotely and asynchronously.
We designed an online survey using the \textit{LimeSurvey}\cite{schmitzlime}
tool made available to us by the University of Zurich.

The user study focused on the comparison between interactive learning
approaches such as that of \textit{CLI-Tutor} and traditional ones, which are
mostly reading based. In the modern software development space, online
documentation is the status quo and the medium we choose to compare our
solution against. As discussed in \autoref{chap:design}, \textit{CLI-Tutor}
uses \textit{Markdown} to specify lessons. Many static documentation generation
websites use Markdown files to generate documentation from. This is also true
for our chosen generator. This enables us to make objectively compare the
interaction medium rather than the lesson content, since the exact same lessons
can be used in both tools. Furthermore, due to the popularity of
\textit{MkDocs} it is a very realistic representation of documentation that
individuals may encounter in the wild.


\subsection{Interactive versus Non-interactive}

Our user study intended to perform an A/B testing comparing learning mediums.
To support this we randomly assigned our participants to one of two groups,
interactive and non-interactive. 


\subsection{Structure}

In this section, we provide a structural overview of our online survey. The
entire survey is available in \autoref{chap:appendixa}.

Our online survey was divided into the following sections:

\begin{itemize}

    \item User Familiarity: In this section users answered questions relating
        to their experience, interest and preferences to provide us with some
        information regarding each individual participant.

    \item Assignment: All participants where divided into one of two groups,
        interactive and non-interactive. The assignment value is unknown to the
        participant at the time of starting the survey. Our survey tool then
        conditionally rendered a URL for the participants to follow to the next
        section of the survey.

    \item Tutorial: At this stage, once the participants have been assigned to
        one of the two groups, they will either be sent to our web application
        running \textit{CLI-Tutor} or to our documentation website. If assigned
        to the non-interactive group...

    \item Evaluation: The evaluation stage is where participants were asked a
        series of basic questions relating to the lessons they took in the
        previous stage.

    \item Feedback: In this section participants were able to provide feedback
        regarding their experience. All but one of the questions in this
        section were identical for both user groups. The non-interactive group
        were asked one additional question regarding interactive learning: \textit{Do 
        you think an interactive command line tutorial application would
    improve the learning process?}

    \item Feedback Opposite (optional): This section was optional and included
        only one quick feedback question. At the end of the survey,
        participants were then given a chance to try out the opposite tool to
        which they were assigned for the bulk of the user study. No evaluation
        or tutorial was mandated here and participants were given one free text
        response to report on their feelings using the alternative tool.
\end{itemize}

\subsection{Assignment}

Assignment was achieved through the use of pseudorandom number generation, a
feature of \textit{LimeSurvey}. 19 were assigned to the interactive group and 15 to the non-interactive group.

\section{Participants}

In total, 34 participants took part in our user study. Recruitment was
primarily done through University channels though some participants were also
sourced through work email and word of mouth.


\section{Section} In addition to these two views, there is also a prompt to
enter an identifier when the program is first launched. This was only added to
the program to help identify log files for the user study conducted as a part
of this thesis work.

\paragraph{Paragraph.} Always with a point.


\begin{figure}[H]
	\centering
	\scalebox{0.75}{%% Creator: Matplotlib, PGF backend
%%
%% To include the figure in your LaTeX document, write
%%   \input{<filename>.pgf}
%%
%% Make sure the required packages are loaded in your preamble
%%   \usepackage{pgf}
%%
%% Also ensure that all the required font packages are loaded; for instance,
%% the lmodern package is sometimes necessary when using math font.
%%   \usepackage{lmodern}
%%
%% Figures using additional raster images can only be included by \input if
%% they are in the same directory as the main LaTeX file. For loading figures
%% from other directories you can use the `import` package
%%   \usepackage{import}
%%
%% and then include the figures with
%%   \import{<path to file>}{<filename>.pgf}
%%
%% Matplotlib used the following preamble
%%   \usepackage{fontspec}
%%   \setmainfont{DejaVuSerif.ttf}[Path=\detokenize{/home/spam/miniconda3/envs/mpl/lib/python3.10/site-packages/matplotlib/mpl-data/fonts/ttf/}]
%%   \setsansfont{DejaVuSans.ttf}[Path=\detokenize{/home/spam/miniconda3/envs/mpl/lib/python3.10/site-packages/matplotlib/mpl-data/fonts/ttf/}]
%%   \setmonofont{DejaVuSansMono.ttf}[Path=\detokenize{/home/spam/miniconda3/envs/mpl/lib/python3.10/site-packages/matplotlib/mpl-data/fonts/ttf/}]
%%
\begingroup%
\makeatletter%
\begin{pgfpicture}%
\pgfpathrectangle{\pgfpointorigin}{\pgfqpoint{5.906660in}{5.000000in}}%
\pgfusepath{use as bounding box, clip}%
\begin{pgfscope}%
\pgfsetbuttcap%
\pgfsetmiterjoin%
\definecolor{currentfill}{rgb}{1.000000,1.000000,1.000000}%
\pgfsetfillcolor{currentfill}%
\pgfsetlinewidth{0.000000pt}%
\definecolor{currentstroke}{rgb}{1.000000,1.000000,1.000000}%
\pgfsetstrokecolor{currentstroke}%
\pgfsetdash{}{0pt}%
\pgfpathmoveto{\pgfqpoint{0.000000in}{0.000000in}}%
\pgfpathlineto{\pgfqpoint{5.906660in}{0.000000in}}%
\pgfpathlineto{\pgfqpoint{5.906660in}{5.000000in}}%
\pgfpathlineto{\pgfqpoint{0.000000in}{5.000000in}}%
\pgfpathlineto{\pgfqpoint{0.000000in}{0.000000in}}%
\pgfpathclose%
\pgfusepath{fill}%
\end{pgfscope}%
\begin{pgfscope}%
\definecolor{textcolor}{rgb}{0.150000,0.150000,0.150000}%
\pgfsetstrokecolor{textcolor}%
\pgfsetfillcolor{textcolor}%
\pgftext[x=3.027163in,y=0.411111in,,top]{\color{textcolor}\sffamily\fontsize{12.000000}{14.400000}\selectfont Computer Science UniversityUniverity Experience}%
\end{pgfscope}%
\begin{pgfscope}%
\pgfsetbuttcap%
\pgfsetmiterjoin%
\definecolor{currentfill}{rgb}{0.552941,0.827451,0.780392}%
\pgfsetfillcolor{currentfill}%
\pgfsetlinewidth{1.003750pt}%
\definecolor{currentstroke}{rgb}{1.000000,1.000000,1.000000}%
\pgfsetstrokecolor{currentstroke}%
\pgfsetdash{}{0pt}%
\pgfpathmoveto{\pgfqpoint{4.567163in}{2.475000in}}%
\pgfpathcurveto{\pgfqpoint{4.567163in}{2.713219in}}{\pgfqpoint{4.511889in}{2.948220in}}{\pgfqpoint{4.405702in}{3.161463in}}%
\pgfpathcurveto{\pgfqpoint{4.299515in}{3.374705in}}{\pgfqpoint{4.145283in}{3.560429in}}{\pgfqpoint{3.955175in}{3.703981in}}%
\pgfpathcurveto{\pgfqpoint{3.765067in}{3.847533in}}{\pgfqpoint{3.544218in}{3.945035in}}{\pgfqpoint{3.310053in}{3.988794in}}%
\pgfpathcurveto{\pgfqpoint{3.075889in}{4.032554in}}{\pgfqpoint{2.834733in}{4.021389in}}{\pgfqpoint{2.605613in}{3.956180in}}%
\pgfpathlineto{\pgfqpoint{3.027163in}{2.475000in}}%
\pgfpathlineto{\pgfqpoint{4.567163in}{2.475000in}}%
\pgfpathlineto{\pgfqpoint{4.567163in}{2.475000in}}%
\pgfpathclose%
\pgfusepath{stroke,fill}%
\end{pgfscope}%
\begin{pgfscope}%
\pgfsetbuttcap%
\pgfsetmiterjoin%
\definecolor{currentfill}{rgb}{1.000000,1.000000,0.701961}%
\pgfsetfillcolor{currentfill}%
\pgfsetlinewidth{1.003750pt}%
\definecolor{currentstroke}{rgb}{1.000000,1.000000,1.000000}%
\pgfsetstrokecolor{currentstroke}%
\pgfsetdash{}{0pt}%
\pgfpathmoveto{\pgfqpoint{2.605613in}{3.956180in}}%
\pgfpathcurveto{\pgfqpoint{2.376493in}{3.890972in}}{\pgfqpoint{2.165597in}{3.773481in}}{\pgfqpoint{1.989567in}{3.612978in}}%
\pgfpathcurveto{\pgfqpoint{1.813536in}{3.452475in}}{\pgfqpoint{1.677124in}{3.253294in}}{\pgfqpoint{1.591094in}{3.031153in}}%
\pgfpathcurveto{\pgfqpoint{1.505064in}{2.809011in}}{\pgfqpoint{1.471740in}{2.569908in}}{\pgfqpoint{1.493751in}{2.332708in}}%
\pgfpathcurveto{\pgfqpoint{1.515762in}{2.095508in}}{\pgfqpoint{1.592513in}{1.866620in}}{\pgfqpoint{1.717949in}{1.664101in}}%
\pgfpathlineto{\pgfqpoint{3.027163in}{2.475000in}}%
\pgfpathlineto{\pgfqpoint{2.605613in}{3.956180in}}%
\pgfpathlineto{\pgfqpoint{2.605613in}{3.956180in}}%
\pgfpathclose%
\pgfusepath{stroke,fill}%
\end{pgfscope}%
\begin{pgfscope}%
\pgfsetbuttcap%
\pgfsetmiterjoin%
\definecolor{currentfill}{rgb}{0.745098,0.729412,0.854902}%
\pgfsetfillcolor{currentfill}%
\pgfsetlinewidth{1.003750pt}%
\definecolor{currentstroke}{rgb}{1.000000,1.000000,1.000000}%
\pgfsetstrokecolor{currentstroke}%
\pgfsetdash{}{0pt}%
\pgfpathmoveto{\pgfqpoint{1.717949in}{1.664101in}}%
\pgfpathcurveto{\pgfqpoint{1.843385in}{1.461582in}}{\pgfqpoint{2.014117in}{1.290903in}}{\pgfqpoint{2.216676in}{1.165531in}}%
\pgfpathcurveto{\pgfqpoint{2.419234in}{1.040158in}}{\pgfqpoint{2.648147in}{0.963479in}}{\pgfqpoint{2.885354in}{0.941543in}}%
\pgfpathcurveto{\pgfqpoint{3.122560in}{0.919607in}}{\pgfqpoint{3.361653in}{0.953006in}}{\pgfqpoint{3.583768in}{1.039106in}}%
\pgfpathcurveto{\pgfqpoint{3.805882in}{1.125206in}}{\pgfqpoint{4.005020in}{1.261680in}}{\pgfqpoint{4.165467in}{1.437762in}}%
\pgfpathlineto{\pgfqpoint{3.027163in}{2.475000in}}%
\pgfpathlineto{\pgfqpoint{1.717949in}{1.664101in}}%
\pgfpathlineto{\pgfqpoint{1.717949in}{1.664101in}}%
\pgfpathclose%
\pgfusepath{stroke,fill}%
\end{pgfscope}%
\begin{pgfscope}%
\pgfsetbuttcap%
\pgfsetmiterjoin%
\definecolor{currentfill}{rgb}{0.984314,0.501961,0.447059}%
\pgfsetfillcolor{currentfill}%
\pgfsetlinewidth{1.003750pt}%
\definecolor{currentstroke}{rgb}{1.000000,1.000000,1.000000}%
\pgfsetstrokecolor{currentstroke}%
\pgfsetdash{}{0pt}%
\pgfpathmoveto{\pgfqpoint{4.165467in}{1.437762in}}%
\pgfpathcurveto{\pgfqpoint{4.165467in}{1.437762in}}{\pgfqpoint{4.165467in}{1.437762in}}{\pgfqpoint{4.165467in}{1.437762in}}%
\pgfpathlineto{\pgfqpoint{3.027163in}{2.475000in}}%
\pgfpathlineto{\pgfqpoint{4.165467in}{1.437762in}}%
\pgfpathlineto{\pgfqpoint{4.165467in}{1.437762in}}%
\pgfpathclose%
\pgfusepath{stroke,fill}%
\end{pgfscope}%
\begin{pgfscope}%
\pgfsetbuttcap%
\pgfsetmiterjoin%
\definecolor{currentfill}{rgb}{0.501961,0.694118,0.827451}%
\pgfsetfillcolor{currentfill}%
\pgfsetlinewidth{1.003750pt}%
\definecolor{currentstroke}{rgb}{1.000000,1.000000,1.000000}%
\pgfsetstrokecolor{currentstroke}%
\pgfsetdash{}{0pt}%
\pgfpathmoveto{\pgfqpoint{4.165467in}{1.437762in}}%
\pgfpathcurveto{\pgfqpoint{4.293579in}{1.578356in}}{\pgfqpoint{4.394541in}{1.741476in}}{\pgfqpoint{4.463232in}{1.918848in}}%
\pgfpathcurveto{\pgfqpoint{4.531924in}{2.096219in}}{\pgfqpoint{4.567163in}{2.284792in}}{\pgfqpoint{4.567163in}{2.475001in}}%
\pgfpathlineto{\pgfqpoint{3.027163in}{2.475000in}}%
\pgfpathlineto{\pgfqpoint{4.165467in}{1.437762in}}%
\pgfpathlineto{\pgfqpoint{4.165467in}{1.437762in}}%
\pgfpathclose%
\pgfusepath{stroke,fill}%
\end{pgfscope}%
\begin{pgfscope}%
\pgfsetbuttcap%
\pgfsetmiterjoin%
\definecolor{currentfill}{rgb}{0.992157,0.705882,0.384314}%
\pgfsetfillcolor{currentfill}%
\pgfsetlinewidth{1.003750pt}%
\definecolor{currentstroke}{rgb}{1.000000,1.000000,1.000000}%
\pgfsetstrokecolor{currentstroke}%
\pgfsetdash{}{0pt}%
\pgfpathmoveto{\pgfqpoint{4.567163in}{2.475001in}}%
\pgfpathcurveto{\pgfqpoint{4.567163in}{2.475001in}}{\pgfqpoint{4.567163in}{2.475001in}}{\pgfqpoint{4.567163in}{2.475001in}}%
\pgfpathlineto{\pgfqpoint{3.027163in}{2.475000in}}%
\pgfpathlineto{\pgfqpoint{4.567163in}{2.475001in}}%
\pgfpathlineto{\pgfqpoint{4.567163in}{2.475001in}}%
\pgfpathclose%
\pgfusepath{stroke,fill}%
\end{pgfscope}%
\begin{pgfscope}%
\definecolor{textcolor}{rgb}{0.150000,0.150000,0.150000}%
\pgfsetstrokecolor{textcolor}%
\pgfsetfillcolor{textcolor}%
\pgftext[x=3.583970in,y=3.212389in,,]{\color{textcolor}\sffamily\fontsize{12.000000}{14.400000}\selectfont 29.4\%}%
\end{pgfscope}%
\begin{pgfscope}%
\definecolor{textcolor}{rgb}{0.150000,0.150000,0.150000}%
\pgfsetstrokecolor{textcolor}%
\pgfsetfillcolor{textcolor}%
\pgftext[x=2.165522in,y=2.808692in,,]{\color{textcolor}\sffamily\fontsize{12.000000}{14.400000}\selectfont 29.4\%}%
\end{pgfscope}%
\begin{pgfscope}%
\definecolor{textcolor}{rgb}{0.150000,0.150000,0.150000}%
\pgfsetstrokecolor{textcolor}%
\pgfsetfillcolor{textcolor}%
\pgftext[x=2.942078in,y=1.554926in,,]{\color{textcolor}\sffamily\fontsize{12.000000}{14.400000}\selectfont 29.4\%}%
\end{pgfscope}%
\begin{pgfscope}%
\definecolor{textcolor}{rgb}{0.150000,0.150000,0.150000}%
\pgfsetstrokecolor{textcolor}%
\pgfsetfillcolor{textcolor}%
\pgftext[x=3.710146in,y=1.852657in,,]{\color{textcolor}\sffamily\fontsize{12.000000}{14.400000}\selectfont 0.0\%}%
\end{pgfscope}%
\begin{pgfscope}%
\definecolor{textcolor}{rgb}{0.150000,0.150000,0.150000}%
\pgfsetstrokecolor{textcolor}%
\pgfsetfillcolor{textcolor}%
\pgftext[x=3.888805in,y=2.141309in,,]{\color{textcolor}\sffamily\fontsize{12.000000}{14.400000}\selectfont 11.8\%}%
\end{pgfscope}%
\begin{pgfscope}%
\definecolor{textcolor}{rgb}{0.150000,0.150000,0.150000}%
\pgfsetstrokecolor{textcolor}%
\pgfsetfillcolor{textcolor}%
\pgftext[x=3.951163in,y=2.475000in,,]{\color{textcolor}\sffamily\fontsize{12.000000}{14.400000}\selectfont 0.0\%}%
\end{pgfscope}%
\begin{pgfscope}%
\pgfsetbuttcap%
\pgfsetmiterjoin%
\definecolor{currentfill}{rgb}{1.000000,1.000000,1.000000}%
\pgfsetfillcolor{currentfill}%
\pgfsetfillopacity{0.800000}%
\pgfsetlinewidth{1.003750pt}%
\definecolor{currentstroke}{rgb}{0.800000,0.800000,0.800000}%
\pgfsetstrokecolor{currentstroke}%
\pgfsetstrokeopacity{0.800000}%
\pgfsetdash{}{0pt}%
\pgfpathmoveto{\pgfqpoint{3.271524in}{0.458500in}}%
\pgfpathlineto{\pgfqpoint{5.827163in}{0.458500in}}%
\pgfpathquadraticcurveto{\pgfqpoint{5.852163in}{0.458500in}}{\pgfqpoint{5.852163in}{0.483500in}}%
\pgfpathlineto{\pgfqpoint{5.852163in}{1.571829in}}%
\pgfpathquadraticcurveto{\pgfqpoint{5.852163in}{1.596829in}}{\pgfqpoint{5.827163in}{1.596829in}}%
\pgfpathlineto{\pgfqpoint{3.271524in}{1.596829in}}%
\pgfpathquadraticcurveto{\pgfqpoint{3.246524in}{1.596829in}}{\pgfqpoint{3.246524in}{1.571829in}}%
\pgfpathlineto{\pgfqpoint{3.246524in}{0.483500in}}%
\pgfpathquadraticcurveto{\pgfqpoint{3.246524in}{0.458500in}}{\pgfqpoint{3.271524in}{0.458500in}}%
\pgfpathlineto{\pgfqpoint{3.271524in}{0.458500in}}%
\pgfpathclose%
\pgfusepath{stroke,fill}%
\end{pgfscope}%
\begin{pgfscope}%
\pgfsetbuttcap%
\pgfsetmiterjoin%
\definecolor{currentfill}{rgb}{0.552941,0.827451,0.780392}%
\pgfsetfillcolor{currentfill}%
\pgfsetlinewidth{1.003750pt}%
\definecolor{currentstroke}{rgb}{1.000000,1.000000,1.000000}%
\pgfsetstrokecolor{currentstroke}%
\pgfsetdash{}{0pt}%
\pgfpathmoveto{\pgfqpoint{3.296524in}{1.451858in}}%
\pgfpathlineto{\pgfqpoint{3.546524in}{1.451858in}}%
\pgfpathlineto{\pgfqpoint{3.546524in}{1.539358in}}%
\pgfpathlineto{\pgfqpoint{3.296524in}{1.539358in}}%
\pgfpathlineto{\pgfqpoint{3.296524in}{1.451858in}}%
\pgfpathclose%
\pgfusepath{stroke,fill}%
\end{pgfscope}%
\begin{pgfscope}%
\definecolor{textcolor}{rgb}{0.150000,0.150000,0.150000}%
\pgfsetstrokecolor{textcolor}%
\pgfsetfillcolor{textcolor}%
\pgftext[x=3.646524in,y=1.451858in,left,base]{\color{textcolor}\sffamily\fontsize{9.000000}{10.800000}\selectfont No CS Degree, 29.4 \%}%
\end{pgfscope}%
\begin{pgfscope}%
\pgfsetbuttcap%
\pgfsetmiterjoin%
\definecolor{currentfill}{rgb}{1.000000,1.000000,0.701961}%
\pgfsetfillcolor{currentfill}%
\pgfsetlinewidth{1.003750pt}%
\definecolor{currentstroke}{rgb}{1.000000,1.000000,1.000000}%
\pgfsetstrokecolor{currentstroke}%
\pgfsetdash{}{0pt}%
\pgfpathmoveto{\pgfqpoint{3.296524in}{1.268387in}}%
\pgfpathlineto{\pgfqpoint{3.546524in}{1.268387in}}%
\pgfpathlineto{\pgfqpoint{3.546524in}{1.355887in}}%
\pgfpathlineto{\pgfqpoint{3.296524in}{1.355887in}}%
\pgfpathlineto{\pgfqpoint{3.296524in}{1.268387in}}%
\pgfpathclose%
\pgfusepath{stroke,fill}%
\end{pgfscope}%
\begin{pgfscope}%
\definecolor{textcolor}{rgb}{0.150000,0.150000,0.150000}%
\pgfsetstrokecolor{textcolor}%
\pgfsetfillcolor{textcolor}%
\pgftext[x=3.646524in,y=1.268387in,left,base]{\color{textcolor}\sffamily\fontsize{9.000000}{10.800000}\selectfont Bachelors's Degree, 29.4 \%}%
\end{pgfscope}%
\begin{pgfscope}%
\pgfsetbuttcap%
\pgfsetmiterjoin%
\definecolor{currentfill}{rgb}{0.745098,0.729412,0.854902}%
\pgfsetfillcolor{currentfill}%
\pgfsetlinewidth{1.003750pt}%
\definecolor{currentstroke}{rgb}{1.000000,1.000000,1.000000}%
\pgfsetstrokecolor{currentstroke}%
\pgfsetdash{}{0pt}%
\pgfpathmoveto{\pgfqpoint{3.296524in}{1.084915in}}%
\pgfpathlineto{\pgfqpoint{3.546524in}{1.084915in}}%
\pgfpathlineto{\pgfqpoint{3.546524in}{1.172415in}}%
\pgfpathlineto{\pgfqpoint{3.296524in}{1.172415in}}%
\pgfpathlineto{\pgfqpoint{3.296524in}{1.084915in}}%
\pgfpathclose%
\pgfusepath{stroke,fill}%
\end{pgfscope}%
\begin{pgfscope}%
\definecolor{textcolor}{rgb}{0.150000,0.150000,0.150000}%
\pgfsetstrokecolor{textcolor}%
\pgfsetfillcolor{textcolor}%
\pgftext[x=3.646524in,y=1.084915in,left,base]{\color{textcolor}\sffamily\fontsize{9.000000}{10.800000}\selectfont Master's Degree, 29.4 \%}%
\end{pgfscope}%
\begin{pgfscope}%
\pgfsetbuttcap%
\pgfsetmiterjoin%
\definecolor{currentfill}{rgb}{0.984314,0.501961,0.447059}%
\pgfsetfillcolor{currentfill}%
\pgfsetlinewidth{1.003750pt}%
\definecolor{currentstroke}{rgb}{1.000000,1.000000,1.000000}%
\pgfsetstrokecolor{currentstroke}%
\pgfsetdash{}{0pt}%
\pgfpathmoveto{\pgfqpoint{3.296524in}{0.901444in}}%
\pgfpathlineto{\pgfqpoint{3.546524in}{0.901444in}}%
\pgfpathlineto{\pgfqpoint{3.546524in}{0.988944in}}%
\pgfpathlineto{\pgfqpoint{3.296524in}{0.988944in}}%
\pgfpathlineto{\pgfqpoint{3.296524in}{0.901444in}}%
\pgfpathclose%
\pgfusepath{stroke,fill}%
\end{pgfscope}%
\begin{pgfscope}%
\definecolor{textcolor}{rgb}{0.150000,0.150000,0.150000}%
\pgfsetstrokecolor{textcolor}%
\pgfsetfillcolor{textcolor}%
\pgftext[x=3.646524in,y=0.901444in,left,base]{\color{textcolor}\sffamily\fontsize{9.000000}{10.800000}\selectfont Doctorate Degree, 0.0 \%}%
\end{pgfscope}%
\begin{pgfscope}%
\pgfsetbuttcap%
\pgfsetmiterjoin%
\definecolor{currentfill}{rgb}{0.501961,0.694118,0.827451}%
\pgfsetfillcolor{currentfill}%
\pgfsetlinewidth{1.003750pt}%
\definecolor{currentstroke}{rgb}{1.000000,1.000000,1.000000}%
\pgfsetstrokecolor{currentstroke}%
\pgfsetdash{}{0pt}%
\pgfpathmoveto{\pgfqpoint{3.296524in}{0.717972in}}%
\pgfpathlineto{\pgfqpoint{3.546524in}{0.717972in}}%
\pgfpathlineto{\pgfqpoint{3.546524in}{0.805472in}}%
\pgfpathlineto{\pgfqpoint{3.296524in}{0.805472in}}%
\pgfpathlineto{\pgfqpoint{3.296524in}{0.717972in}}%
\pgfpathclose%
\pgfusepath{stroke,fill}%
\end{pgfscope}%
\begin{pgfscope}%
\definecolor{textcolor}{rgb}{0.150000,0.150000,0.150000}%
\pgfsetstrokecolor{textcolor}%
\pgfsetfillcolor{textcolor}%
\pgftext[x=3.646524in,y=0.717972in,left,base]{\color{textcolor}\sffamily\fontsize{9.000000}{10.800000}\selectfont Other Engineering Degree, 11.8 \%}%
\end{pgfscope}%
\begin{pgfscope}%
\pgfsetbuttcap%
\pgfsetmiterjoin%
\definecolor{currentfill}{rgb}{0.992157,0.705882,0.384314}%
\pgfsetfillcolor{currentfill}%
\pgfsetlinewidth{1.003750pt}%
\definecolor{currentstroke}{rgb}{1.000000,1.000000,1.000000}%
\pgfsetstrokecolor{currentstroke}%
\pgfsetdash{}{0pt}%
\pgfpathmoveto{\pgfqpoint{3.296524in}{0.534501in}}%
\pgfpathlineto{\pgfqpoint{3.546524in}{0.534501in}}%
\pgfpathlineto{\pgfqpoint{3.546524in}{0.622001in}}%
\pgfpathlineto{\pgfqpoint{3.296524in}{0.622001in}}%
\pgfpathlineto{\pgfqpoint{3.296524in}{0.534501in}}%
\pgfpathclose%
\pgfusepath{stroke,fill}%
\end{pgfscope}%
\begin{pgfscope}%
\definecolor{textcolor}{rgb}{0.150000,0.150000,0.150000}%
\pgfsetstrokecolor{textcolor}%
\pgfsetfillcolor{textcolor}%
\pgftext[x=3.646524in,y=0.534501in,left,base]{\color{textcolor}\sffamily\fontsize{9.000000}{10.800000}\selectfont Yes but did not finish, 0.0 \%}%
\end{pgfscope}%
\end{pgfpicture}%
\makeatother%
\endgroup%
}
	\caption{The distribution of programming experience amongst study participants.}
	\label{fig:programmingexp}
\end{figure}

\begin{figure}[H]
	\centering
	\scalebox{0.8}{%% Creator: Matplotlib, PGF backend
%%
%% To include the figure in your LaTeX document, write
%%   \input{<filename>.pgf}
%%
%% Make sure the required packages are loaded in your preamble
%%   \usepackage{pgf}
%%
%% Also ensure that all the required font packages are loaded; for instance,
%% the lmodern package is sometimes necessary when using math font.
%%   \usepackage{lmodern}
%%
%% Figures using additional raster images can only be included by \input if
%% they are in the same directory as the main LaTeX file. For loading figures
%% from other directories you can use the `import` package
%%   \usepackage{import}
%%
%% and then include the figures with
%%   \import{<path to file>}{<filename>.pgf}
%%
%% Matplotlib used the following preamble
%%   \usepackage{fontspec}
%%   \setmainfont{DejaVuSerif.ttf}[Path=\detokenize{/home/spam/miniconda3/envs/mpl/lib/python3.10/site-packages/matplotlib/mpl-data/fonts/ttf/}]
%%   \setsansfont{DejaVuSans.ttf}[Path=\detokenize{/home/spam/miniconda3/envs/mpl/lib/python3.10/site-packages/matplotlib/mpl-data/fonts/ttf/}]
%%   \setmonofont{DejaVuSansMono.ttf}[Path=\detokenize{/home/spam/miniconda3/envs/mpl/lib/python3.10/site-packages/matplotlib/mpl-data/fonts/ttf/}]
%%
\begingroup%
\makeatletter%
\begin{pgfpicture}%
\pgfpathrectangle{\pgfpointorigin}{\pgfqpoint{5.906660in}{5.000000in}}%
\pgfusepath{use as bounding box, clip}%
\begin{pgfscope}%
\pgfsetbuttcap%
\pgfsetmiterjoin%
\definecolor{currentfill}{rgb}{1.000000,1.000000,1.000000}%
\pgfsetfillcolor{currentfill}%
\pgfsetlinewidth{0.000000pt}%
\definecolor{currentstroke}{rgb}{1.000000,1.000000,1.000000}%
\pgfsetstrokecolor{currentstroke}%
\pgfsetdash{}{0pt}%
\pgfpathmoveto{\pgfqpoint{0.000000in}{0.000000in}}%
\pgfpathlineto{\pgfqpoint{5.906660in}{0.000000in}}%
\pgfpathlineto{\pgfqpoint{5.906660in}{5.000000in}}%
\pgfpathlineto{\pgfqpoint{0.000000in}{5.000000in}}%
\pgfpathlineto{\pgfqpoint{0.000000in}{0.000000in}}%
\pgfpathclose%
\pgfusepath{fill}%
\end{pgfscope}%
\begin{pgfscope}%
\definecolor{textcolor}{rgb}{0.150000,0.150000,0.150000}%
\pgfsetstrokecolor{textcolor}%
\pgfsetfillcolor{textcolor}%
\pgftext[x=3.027163in,y=0.411111in,,top]{\color{textcolor}\sffamily\fontsize{12.000000}{14.400000}\selectfont Computer Science UniversityUniverity Experience}%
\end{pgfscope}%
\begin{pgfscope}%
\pgfsetbuttcap%
\pgfsetmiterjoin%
\definecolor{currentfill}{rgb}{0.552941,0.827451,0.780392}%
\pgfsetfillcolor{currentfill}%
\pgfsetlinewidth{1.003750pt}%
\definecolor{currentstroke}{rgb}{1.000000,1.000000,1.000000}%
\pgfsetstrokecolor{currentstroke}%
\pgfsetdash{}{0pt}%
\pgfpathmoveto{\pgfqpoint{4.567163in}{2.475000in}}%
\pgfpathcurveto{\pgfqpoint{4.567163in}{2.713219in}}{\pgfqpoint{4.511889in}{2.948220in}}{\pgfqpoint{4.405702in}{3.161463in}}%
\pgfpathcurveto{\pgfqpoint{4.299515in}{3.374705in}}{\pgfqpoint{4.145283in}{3.560429in}}{\pgfqpoint{3.955175in}{3.703981in}}%
\pgfpathcurveto{\pgfqpoint{3.765067in}{3.847533in}}{\pgfqpoint{3.544218in}{3.945035in}}{\pgfqpoint{3.310053in}{3.988794in}}%
\pgfpathcurveto{\pgfqpoint{3.075889in}{4.032554in}}{\pgfqpoint{2.834733in}{4.021389in}}{\pgfqpoint{2.605613in}{3.956180in}}%
\pgfpathlineto{\pgfqpoint{3.027163in}{2.475000in}}%
\pgfpathlineto{\pgfqpoint{4.567163in}{2.475000in}}%
\pgfpathlineto{\pgfqpoint{4.567163in}{2.475000in}}%
\pgfpathclose%
\pgfusepath{stroke,fill}%
\end{pgfscope}%
\begin{pgfscope}%
\pgfsetbuttcap%
\pgfsetmiterjoin%
\definecolor{currentfill}{rgb}{1.000000,1.000000,0.701961}%
\pgfsetfillcolor{currentfill}%
\pgfsetlinewidth{1.003750pt}%
\definecolor{currentstroke}{rgb}{1.000000,1.000000,1.000000}%
\pgfsetstrokecolor{currentstroke}%
\pgfsetdash{}{0pt}%
\pgfpathmoveto{\pgfqpoint{2.605613in}{3.956180in}}%
\pgfpathcurveto{\pgfqpoint{2.376493in}{3.890972in}}{\pgfqpoint{2.165597in}{3.773481in}}{\pgfqpoint{1.989567in}{3.612978in}}%
\pgfpathcurveto{\pgfqpoint{1.813536in}{3.452475in}}{\pgfqpoint{1.677124in}{3.253294in}}{\pgfqpoint{1.591094in}{3.031153in}}%
\pgfpathcurveto{\pgfqpoint{1.505064in}{2.809011in}}{\pgfqpoint{1.471740in}{2.569908in}}{\pgfqpoint{1.493751in}{2.332708in}}%
\pgfpathcurveto{\pgfqpoint{1.515762in}{2.095508in}}{\pgfqpoint{1.592513in}{1.866620in}}{\pgfqpoint{1.717949in}{1.664101in}}%
\pgfpathlineto{\pgfqpoint{3.027163in}{2.475000in}}%
\pgfpathlineto{\pgfqpoint{2.605613in}{3.956180in}}%
\pgfpathlineto{\pgfqpoint{2.605613in}{3.956180in}}%
\pgfpathclose%
\pgfusepath{stroke,fill}%
\end{pgfscope}%
\begin{pgfscope}%
\pgfsetbuttcap%
\pgfsetmiterjoin%
\definecolor{currentfill}{rgb}{0.745098,0.729412,0.854902}%
\pgfsetfillcolor{currentfill}%
\pgfsetlinewidth{1.003750pt}%
\definecolor{currentstroke}{rgb}{1.000000,1.000000,1.000000}%
\pgfsetstrokecolor{currentstroke}%
\pgfsetdash{}{0pt}%
\pgfpathmoveto{\pgfqpoint{1.717949in}{1.664101in}}%
\pgfpathcurveto{\pgfqpoint{1.843385in}{1.461582in}}{\pgfqpoint{2.014117in}{1.290903in}}{\pgfqpoint{2.216676in}{1.165531in}}%
\pgfpathcurveto{\pgfqpoint{2.419234in}{1.040158in}}{\pgfqpoint{2.648147in}{0.963479in}}{\pgfqpoint{2.885354in}{0.941543in}}%
\pgfpathcurveto{\pgfqpoint{3.122560in}{0.919607in}}{\pgfqpoint{3.361653in}{0.953006in}}{\pgfqpoint{3.583768in}{1.039106in}}%
\pgfpathcurveto{\pgfqpoint{3.805882in}{1.125206in}}{\pgfqpoint{4.005020in}{1.261680in}}{\pgfqpoint{4.165467in}{1.437762in}}%
\pgfpathlineto{\pgfqpoint{3.027163in}{2.475000in}}%
\pgfpathlineto{\pgfqpoint{1.717949in}{1.664101in}}%
\pgfpathlineto{\pgfqpoint{1.717949in}{1.664101in}}%
\pgfpathclose%
\pgfusepath{stroke,fill}%
\end{pgfscope}%
\begin{pgfscope}%
\pgfsetbuttcap%
\pgfsetmiterjoin%
\definecolor{currentfill}{rgb}{0.984314,0.501961,0.447059}%
\pgfsetfillcolor{currentfill}%
\pgfsetlinewidth{1.003750pt}%
\definecolor{currentstroke}{rgb}{1.000000,1.000000,1.000000}%
\pgfsetstrokecolor{currentstroke}%
\pgfsetdash{}{0pt}%
\pgfpathmoveto{\pgfqpoint{4.165467in}{1.437762in}}%
\pgfpathcurveto{\pgfqpoint{4.165467in}{1.437762in}}{\pgfqpoint{4.165467in}{1.437762in}}{\pgfqpoint{4.165467in}{1.437762in}}%
\pgfpathlineto{\pgfqpoint{3.027163in}{2.475000in}}%
\pgfpathlineto{\pgfqpoint{4.165467in}{1.437762in}}%
\pgfpathlineto{\pgfqpoint{4.165467in}{1.437762in}}%
\pgfpathclose%
\pgfusepath{stroke,fill}%
\end{pgfscope}%
\begin{pgfscope}%
\pgfsetbuttcap%
\pgfsetmiterjoin%
\definecolor{currentfill}{rgb}{0.501961,0.694118,0.827451}%
\pgfsetfillcolor{currentfill}%
\pgfsetlinewidth{1.003750pt}%
\definecolor{currentstroke}{rgb}{1.000000,1.000000,1.000000}%
\pgfsetstrokecolor{currentstroke}%
\pgfsetdash{}{0pt}%
\pgfpathmoveto{\pgfqpoint{4.165467in}{1.437762in}}%
\pgfpathcurveto{\pgfqpoint{4.293579in}{1.578356in}}{\pgfqpoint{4.394541in}{1.741476in}}{\pgfqpoint{4.463232in}{1.918848in}}%
\pgfpathcurveto{\pgfqpoint{4.531924in}{2.096219in}}{\pgfqpoint{4.567163in}{2.284792in}}{\pgfqpoint{4.567163in}{2.475001in}}%
\pgfpathlineto{\pgfqpoint{3.027163in}{2.475000in}}%
\pgfpathlineto{\pgfqpoint{4.165467in}{1.437762in}}%
\pgfpathlineto{\pgfqpoint{4.165467in}{1.437762in}}%
\pgfpathclose%
\pgfusepath{stroke,fill}%
\end{pgfscope}%
\begin{pgfscope}%
\pgfsetbuttcap%
\pgfsetmiterjoin%
\definecolor{currentfill}{rgb}{0.992157,0.705882,0.384314}%
\pgfsetfillcolor{currentfill}%
\pgfsetlinewidth{1.003750pt}%
\definecolor{currentstroke}{rgb}{1.000000,1.000000,1.000000}%
\pgfsetstrokecolor{currentstroke}%
\pgfsetdash{}{0pt}%
\pgfpathmoveto{\pgfqpoint{4.567163in}{2.475001in}}%
\pgfpathcurveto{\pgfqpoint{4.567163in}{2.475001in}}{\pgfqpoint{4.567163in}{2.475001in}}{\pgfqpoint{4.567163in}{2.475001in}}%
\pgfpathlineto{\pgfqpoint{3.027163in}{2.475000in}}%
\pgfpathlineto{\pgfqpoint{4.567163in}{2.475001in}}%
\pgfpathlineto{\pgfqpoint{4.567163in}{2.475001in}}%
\pgfpathclose%
\pgfusepath{stroke,fill}%
\end{pgfscope}%
\begin{pgfscope}%
\definecolor{textcolor}{rgb}{0.150000,0.150000,0.150000}%
\pgfsetstrokecolor{textcolor}%
\pgfsetfillcolor{textcolor}%
\pgftext[x=3.583970in,y=3.212389in,,]{\color{textcolor}\sffamily\fontsize{12.000000}{14.400000}\selectfont 29.4\%}%
\end{pgfscope}%
\begin{pgfscope}%
\definecolor{textcolor}{rgb}{0.150000,0.150000,0.150000}%
\pgfsetstrokecolor{textcolor}%
\pgfsetfillcolor{textcolor}%
\pgftext[x=2.165522in,y=2.808692in,,]{\color{textcolor}\sffamily\fontsize{12.000000}{14.400000}\selectfont 29.4\%}%
\end{pgfscope}%
\begin{pgfscope}%
\definecolor{textcolor}{rgb}{0.150000,0.150000,0.150000}%
\pgfsetstrokecolor{textcolor}%
\pgfsetfillcolor{textcolor}%
\pgftext[x=2.942078in,y=1.554926in,,]{\color{textcolor}\sffamily\fontsize{12.000000}{14.400000}\selectfont 29.4\%}%
\end{pgfscope}%
\begin{pgfscope}%
\definecolor{textcolor}{rgb}{0.150000,0.150000,0.150000}%
\pgfsetstrokecolor{textcolor}%
\pgfsetfillcolor{textcolor}%
\pgftext[x=3.710146in,y=1.852657in,,]{\color{textcolor}\sffamily\fontsize{12.000000}{14.400000}\selectfont 0.0\%}%
\end{pgfscope}%
\begin{pgfscope}%
\definecolor{textcolor}{rgb}{0.150000,0.150000,0.150000}%
\pgfsetstrokecolor{textcolor}%
\pgfsetfillcolor{textcolor}%
\pgftext[x=3.888805in,y=2.141309in,,]{\color{textcolor}\sffamily\fontsize{12.000000}{14.400000}\selectfont 11.8\%}%
\end{pgfscope}%
\begin{pgfscope}%
\definecolor{textcolor}{rgb}{0.150000,0.150000,0.150000}%
\pgfsetstrokecolor{textcolor}%
\pgfsetfillcolor{textcolor}%
\pgftext[x=3.951163in,y=2.475000in,,]{\color{textcolor}\sffamily\fontsize{12.000000}{14.400000}\selectfont 0.0\%}%
\end{pgfscope}%
\begin{pgfscope}%
\pgfsetbuttcap%
\pgfsetmiterjoin%
\definecolor{currentfill}{rgb}{1.000000,1.000000,1.000000}%
\pgfsetfillcolor{currentfill}%
\pgfsetfillopacity{0.800000}%
\pgfsetlinewidth{1.003750pt}%
\definecolor{currentstroke}{rgb}{0.800000,0.800000,0.800000}%
\pgfsetstrokecolor{currentstroke}%
\pgfsetstrokeopacity{0.800000}%
\pgfsetdash{}{0pt}%
\pgfpathmoveto{\pgfqpoint{3.271524in}{0.458500in}}%
\pgfpathlineto{\pgfqpoint{5.827163in}{0.458500in}}%
\pgfpathquadraticcurveto{\pgfqpoint{5.852163in}{0.458500in}}{\pgfqpoint{5.852163in}{0.483500in}}%
\pgfpathlineto{\pgfqpoint{5.852163in}{1.571829in}}%
\pgfpathquadraticcurveto{\pgfqpoint{5.852163in}{1.596829in}}{\pgfqpoint{5.827163in}{1.596829in}}%
\pgfpathlineto{\pgfqpoint{3.271524in}{1.596829in}}%
\pgfpathquadraticcurveto{\pgfqpoint{3.246524in}{1.596829in}}{\pgfqpoint{3.246524in}{1.571829in}}%
\pgfpathlineto{\pgfqpoint{3.246524in}{0.483500in}}%
\pgfpathquadraticcurveto{\pgfqpoint{3.246524in}{0.458500in}}{\pgfqpoint{3.271524in}{0.458500in}}%
\pgfpathlineto{\pgfqpoint{3.271524in}{0.458500in}}%
\pgfpathclose%
\pgfusepath{stroke,fill}%
\end{pgfscope}%
\begin{pgfscope}%
\pgfsetbuttcap%
\pgfsetmiterjoin%
\definecolor{currentfill}{rgb}{0.552941,0.827451,0.780392}%
\pgfsetfillcolor{currentfill}%
\pgfsetlinewidth{1.003750pt}%
\definecolor{currentstroke}{rgb}{1.000000,1.000000,1.000000}%
\pgfsetstrokecolor{currentstroke}%
\pgfsetdash{}{0pt}%
\pgfpathmoveto{\pgfqpoint{3.296524in}{1.451858in}}%
\pgfpathlineto{\pgfqpoint{3.546524in}{1.451858in}}%
\pgfpathlineto{\pgfqpoint{3.546524in}{1.539358in}}%
\pgfpathlineto{\pgfqpoint{3.296524in}{1.539358in}}%
\pgfpathlineto{\pgfqpoint{3.296524in}{1.451858in}}%
\pgfpathclose%
\pgfusepath{stroke,fill}%
\end{pgfscope}%
\begin{pgfscope}%
\definecolor{textcolor}{rgb}{0.150000,0.150000,0.150000}%
\pgfsetstrokecolor{textcolor}%
\pgfsetfillcolor{textcolor}%
\pgftext[x=3.646524in,y=1.451858in,left,base]{\color{textcolor}\sffamily\fontsize{9.000000}{10.800000}\selectfont No CS Degree, 29.4 \%}%
\end{pgfscope}%
\begin{pgfscope}%
\pgfsetbuttcap%
\pgfsetmiterjoin%
\definecolor{currentfill}{rgb}{1.000000,1.000000,0.701961}%
\pgfsetfillcolor{currentfill}%
\pgfsetlinewidth{1.003750pt}%
\definecolor{currentstroke}{rgb}{1.000000,1.000000,1.000000}%
\pgfsetstrokecolor{currentstroke}%
\pgfsetdash{}{0pt}%
\pgfpathmoveto{\pgfqpoint{3.296524in}{1.268387in}}%
\pgfpathlineto{\pgfqpoint{3.546524in}{1.268387in}}%
\pgfpathlineto{\pgfqpoint{3.546524in}{1.355887in}}%
\pgfpathlineto{\pgfqpoint{3.296524in}{1.355887in}}%
\pgfpathlineto{\pgfqpoint{3.296524in}{1.268387in}}%
\pgfpathclose%
\pgfusepath{stroke,fill}%
\end{pgfscope}%
\begin{pgfscope}%
\definecolor{textcolor}{rgb}{0.150000,0.150000,0.150000}%
\pgfsetstrokecolor{textcolor}%
\pgfsetfillcolor{textcolor}%
\pgftext[x=3.646524in,y=1.268387in,left,base]{\color{textcolor}\sffamily\fontsize{9.000000}{10.800000}\selectfont Bachelors's Degree, 29.4 \%}%
\end{pgfscope}%
\begin{pgfscope}%
\pgfsetbuttcap%
\pgfsetmiterjoin%
\definecolor{currentfill}{rgb}{0.745098,0.729412,0.854902}%
\pgfsetfillcolor{currentfill}%
\pgfsetlinewidth{1.003750pt}%
\definecolor{currentstroke}{rgb}{1.000000,1.000000,1.000000}%
\pgfsetstrokecolor{currentstroke}%
\pgfsetdash{}{0pt}%
\pgfpathmoveto{\pgfqpoint{3.296524in}{1.084915in}}%
\pgfpathlineto{\pgfqpoint{3.546524in}{1.084915in}}%
\pgfpathlineto{\pgfqpoint{3.546524in}{1.172415in}}%
\pgfpathlineto{\pgfqpoint{3.296524in}{1.172415in}}%
\pgfpathlineto{\pgfqpoint{3.296524in}{1.084915in}}%
\pgfpathclose%
\pgfusepath{stroke,fill}%
\end{pgfscope}%
\begin{pgfscope}%
\definecolor{textcolor}{rgb}{0.150000,0.150000,0.150000}%
\pgfsetstrokecolor{textcolor}%
\pgfsetfillcolor{textcolor}%
\pgftext[x=3.646524in,y=1.084915in,left,base]{\color{textcolor}\sffamily\fontsize{9.000000}{10.800000}\selectfont Master's Degree, 29.4 \%}%
\end{pgfscope}%
\begin{pgfscope}%
\pgfsetbuttcap%
\pgfsetmiterjoin%
\definecolor{currentfill}{rgb}{0.984314,0.501961,0.447059}%
\pgfsetfillcolor{currentfill}%
\pgfsetlinewidth{1.003750pt}%
\definecolor{currentstroke}{rgb}{1.000000,1.000000,1.000000}%
\pgfsetstrokecolor{currentstroke}%
\pgfsetdash{}{0pt}%
\pgfpathmoveto{\pgfqpoint{3.296524in}{0.901444in}}%
\pgfpathlineto{\pgfqpoint{3.546524in}{0.901444in}}%
\pgfpathlineto{\pgfqpoint{3.546524in}{0.988944in}}%
\pgfpathlineto{\pgfqpoint{3.296524in}{0.988944in}}%
\pgfpathlineto{\pgfqpoint{3.296524in}{0.901444in}}%
\pgfpathclose%
\pgfusepath{stroke,fill}%
\end{pgfscope}%
\begin{pgfscope}%
\definecolor{textcolor}{rgb}{0.150000,0.150000,0.150000}%
\pgfsetstrokecolor{textcolor}%
\pgfsetfillcolor{textcolor}%
\pgftext[x=3.646524in,y=0.901444in,left,base]{\color{textcolor}\sffamily\fontsize{9.000000}{10.800000}\selectfont Doctorate Degree, 0.0 \%}%
\end{pgfscope}%
\begin{pgfscope}%
\pgfsetbuttcap%
\pgfsetmiterjoin%
\definecolor{currentfill}{rgb}{0.501961,0.694118,0.827451}%
\pgfsetfillcolor{currentfill}%
\pgfsetlinewidth{1.003750pt}%
\definecolor{currentstroke}{rgb}{1.000000,1.000000,1.000000}%
\pgfsetstrokecolor{currentstroke}%
\pgfsetdash{}{0pt}%
\pgfpathmoveto{\pgfqpoint{3.296524in}{0.717972in}}%
\pgfpathlineto{\pgfqpoint{3.546524in}{0.717972in}}%
\pgfpathlineto{\pgfqpoint{3.546524in}{0.805472in}}%
\pgfpathlineto{\pgfqpoint{3.296524in}{0.805472in}}%
\pgfpathlineto{\pgfqpoint{3.296524in}{0.717972in}}%
\pgfpathclose%
\pgfusepath{stroke,fill}%
\end{pgfscope}%
\begin{pgfscope}%
\definecolor{textcolor}{rgb}{0.150000,0.150000,0.150000}%
\pgfsetstrokecolor{textcolor}%
\pgfsetfillcolor{textcolor}%
\pgftext[x=3.646524in,y=0.717972in,left,base]{\color{textcolor}\sffamily\fontsize{9.000000}{10.800000}\selectfont Other Engineering Degree, 11.8 \%}%
\end{pgfscope}%
\begin{pgfscope}%
\pgfsetbuttcap%
\pgfsetmiterjoin%
\definecolor{currentfill}{rgb}{0.992157,0.705882,0.384314}%
\pgfsetfillcolor{currentfill}%
\pgfsetlinewidth{1.003750pt}%
\definecolor{currentstroke}{rgb}{1.000000,1.000000,1.000000}%
\pgfsetstrokecolor{currentstroke}%
\pgfsetdash{}{0pt}%
\pgfpathmoveto{\pgfqpoint{3.296524in}{0.534501in}}%
\pgfpathlineto{\pgfqpoint{3.546524in}{0.534501in}}%
\pgfpathlineto{\pgfqpoint{3.546524in}{0.622001in}}%
\pgfpathlineto{\pgfqpoint{3.296524in}{0.622001in}}%
\pgfpathlineto{\pgfqpoint{3.296524in}{0.534501in}}%
\pgfpathclose%
\pgfusepath{stroke,fill}%
\end{pgfscope}%
\begin{pgfscope}%
\definecolor{textcolor}{rgb}{0.150000,0.150000,0.150000}%
\pgfsetstrokecolor{textcolor}%
\pgfsetfillcolor{textcolor}%
\pgftext[x=3.646524in,y=0.534501in,left,base]{\color{textcolor}\sffamily\fontsize{9.000000}{10.800000}\selectfont Yes but did not finish, 0.0 \%}%
\end{pgfscope}%
\end{pgfpicture}%
\makeatother%
\endgroup%
}
	\caption{University level Computer Science experience amongst study participants.}
	\label{fig:uniexp}
\end{figure}

\begin{figure}[H]
	\centering
	\begin{minipage}{0.45\textwidth}
		\centering
		\scalebox{0.7}{%% Creator: Matplotlib, PGF backend
%%
%% To include the figure in your LaTeX document, write
%%   \input{<filename>.pgf}
%%
%% Make sure the required packages are loaded in your preamble
%%   \usepackage{pgf}
%%
%% Also ensure that all the required font packages are loaded; for instance,
%% the lmodern package is sometimes necessary when using math font.
%%   \usepackage{lmodern}
%%
%% Figures using additional raster images can only be included by \input if
%% they are in the same directory as the main LaTeX file. For loading figures
%% from other directories you can use the `import` package
%%   \usepackage{import}
%%
%% and then include the figures with
%%   \import{<path to file>}{<filename>.pgf}
%%
%% Matplotlib used the following preamble
%%   \usepackage{fontspec}
%%   \setmainfont{DejaVuSerif.ttf}[Path=\detokenize{/home/spam/miniconda3/envs/mpl/lib/python3.10/site-packages/matplotlib/mpl-data/fonts/ttf/}]
%%   \setsansfont{DejaVuSans.ttf}[Path=\detokenize{/home/spam/miniconda3/envs/mpl/lib/python3.10/site-packages/matplotlib/mpl-data/fonts/ttf/}]
%%   \setmonofont{DejaVuSansMono.ttf}[Path=\detokenize{/home/spam/miniconda3/envs/mpl/lib/python3.10/site-packages/matplotlib/mpl-data/fonts/ttf/}]
%%
\begingroup%
\makeatletter%
\begin{pgfpicture}%
\pgfpathrectangle{\pgfpointorigin}{\pgfqpoint{5.906660in}{5.000000in}}%
\pgfusepath{use as bounding box, clip}%
\begin{pgfscope}%
\pgfsetbuttcap%
\pgfsetmiterjoin%
\definecolor{currentfill}{rgb}{1.000000,1.000000,1.000000}%
\pgfsetfillcolor{currentfill}%
\pgfsetlinewidth{0.000000pt}%
\definecolor{currentstroke}{rgb}{1.000000,1.000000,1.000000}%
\pgfsetstrokecolor{currentstroke}%
\pgfsetdash{}{0pt}%
\pgfpathmoveto{\pgfqpoint{0.000000in}{0.000000in}}%
\pgfpathlineto{\pgfqpoint{5.906660in}{0.000000in}}%
\pgfpathlineto{\pgfqpoint{5.906660in}{5.000000in}}%
\pgfpathlineto{\pgfqpoint{0.000000in}{5.000000in}}%
\pgfpathlineto{\pgfqpoint{0.000000in}{0.000000in}}%
\pgfpathclose%
\pgfusepath{fill}%
\end{pgfscope}%
\begin{pgfscope}%
\definecolor{textcolor}{rgb}{0.150000,0.150000,0.150000}%
\pgfsetstrokecolor{textcolor}%
\pgfsetfillcolor{textcolor}%
\pgftext[x=3.027163in,y=0.411111in,,top]{\color{textcolor}\sffamily\fontsize{12.000000}{14.400000}\selectfont Computer Science UniversityUniverity Experience}%
\end{pgfscope}%
\begin{pgfscope}%
\pgfsetbuttcap%
\pgfsetmiterjoin%
\definecolor{currentfill}{rgb}{0.552941,0.827451,0.780392}%
\pgfsetfillcolor{currentfill}%
\pgfsetlinewidth{1.003750pt}%
\definecolor{currentstroke}{rgb}{1.000000,1.000000,1.000000}%
\pgfsetstrokecolor{currentstroke}%
\pgfsetdash{}{0pt}%
\pgfpathmoveto{\pgfqpoint{4.567163in}{2.475000in}}%
\pgfpathcurveto{\pgfqpoint{4.567163in}{2.713219in}}{\pgfqpoint{4.511889in}{2.948220in}}{\pgfqpoint{4.405702in}{3.161463in}}%
\pgfpathcurveto{\pgfqpoint{4.299515in}{3.374705in}}{\pgfqpoint{4.145283in}{3.560429in}}{\pgfqpoint{3.955175in}{3.703981in}}%
\pgfpathcurveto{\pgfqpoint{3.765067in}{3.847533in}}{\pgfqpoint{3.544218in}{3.945035in}}{\pgfqpoint{3.310053in}{3.988794in}}%
\pgfpathcurveto{\pgfqpoint{3.075889in}{4.032554in}}{\pgfqpoint{2.834733in}{4.021389in}}{\pgfqpoint{2.605613in}{3.956180in}}%
\pgfpathlineto{\pgfqpoint{3.027163in}{2.475000in}}%
\pgfpathlineto{\pgfqpoint{4.567163in}{2.475000in}}%
\pgfpathlineto{\pgfqpoint{4.567163in}{2.475000in}}%
\pgfpathclose%
\pgfusepath{stroke,fill}%
\end{pgfscope}%
\begin{pgfscope}%
\pgfsetbuttcap%
\pgfsetmiterjoin%
\definecolor{currentfill}{rgb}{1.000000,1.000000,0.701961}%
\pgfsetfillcolor{currentfill}%
\pgfsetlinewidth{1.003750pt}%
\definecolor{currentstroke}{rgb}{1.000000,1.000000,1.000000}%
\pgfsetstrokecolor{currentstroke}%
\pgfsetdash{}{0pt}%
\pgfpathmoveto{\pgfqpoint{2.605613in}{3.956180in}}%
\pgfpathcurveto{\pgfqpoint{2.376493in}{3.890972in}}{\pgfqpoint{2.165597in}{3.773481in}}{\pgfqpoint{1.989567in}{3.612978in}}%
\pgfpathcurveto{\pgfqpoint{1.813536in}{3.452475in}}{\pgfqpoint{1.677124in}{3.253294in}}{\pgfqpoint{1.591094in}{3.031153in}}%
\pgfpathcurveto{\pgfqpoint{1.505064in}{2.809011in}}{\pgfqpoint{1.471740in}{2.569908in}}{\pgfqpoint{1.493751in}{2.332708in}}%
\pgfpathcurveto{\pgfqpoint{1.515762in}{2.095508in}}{\pgfqpoint{1.592513in}{1.866620in}}{\pgfqpoint{1.717949in}{1.664101in}}%
\pgfpathlineto{\pgfqpoint{3.027163in}{2.475000in}}%
\pgfpathlineto{\pgfqpoint{2.605613in}{3.956180in}}%
\pgfpathlineto{\pgfqpoint{2.605613in}{3.956180in}}%
\pgfpathclose%
\pgfusepath{stroke,fill}%
\end{pgfscope}%
\begin{pgfscope}%
\pgfsetbuttcap%
\pgfsetmiterjoin%
\definecolor{currentfill}{rgb}{0.745098,0.729412,0.854902}%
\pgfsetfillcolor{currentfill}%
\pgfsetlinewidth{1.003750pt}%
\definecolor{currentstroke}{rgb}{1.000000,1.000000,1.000000}%
\pgfsetstrokecolor{currentstroke}%
\pgfsetdash{}{0pt}%
\pgfpathmoveto{\pgfqpoint{1.717949in}{1.664101in}}%
\pgfpathcurveto{\pgfqpoint{1.843385in}{1.461582in}}{\pgfqpoint{2.014117in}{1.290903in}}{\pgfqpoint{2.216676in}{1.165531in}}%
\pgfpathcurveto{\pgfqpoint{2.419234in}{1.040158in}}{\pgfqpoint{2.648147in}{0.963479in}}{\pgfqpoint{2.885354in}{0.941543in}}%
\pgfpathcurveto{\pgfqpoint{3.122560in}{0.919607in}}{\pgfqpoint{3.361653in}{0.953006in}}{\pgfqpoint{3.583768in}{1.039106in}}%
\pgfpathcurveto{\pgfqpoint{3.805882in}{1.125206in}}{\pgfqpoint{4.005020in}{1.261680in}}{\pgfqpoint{4.165467in}{1.437762in}}%
\pgfpathlineto{\pgfqpoint{3.027163in}{2.475000in}}%
\pgfpathlineto{\pgfqpoint{1.717949in}{1.664101in}}%
\pgfpathlineto{\pgfqpoint{1.717949in}{1.664101in}}%
\pgfpathclose%
\pgfusepath{stroke,fill}%
\end{pgfscope}%
\begin{pgfscope}%
\pgfsetbuttcap%
\pgfsetmiterjoin%
\definecolor{currentfill}{rgb}{0.984314,0.501961,0.447059}%
\pgfsetfillcolor{currentfill}%
\pgfsetlinewidth{1.003750pt}%
\definecolor{currentstroke}{rgb}{1.000000,1.000000,1.000000}%
\pgfsetstrokecolor{currentstroke}%
\pgfsetdash{}{0pt}%
\pgfpathmoveto{\pgfqpoint{4.165467in}{1.437762in}}%
\pgfpathcurveto{\pgfqpoint{4.165467in}{1.437762in}}{\pgfqpoint{4.165467in}{1.437762in}}{\pgfqpoint{4.165467in}{1.437762in}}%
\pgfpathlineto{\pgfqpoint{3.027163in}{2.475000in}}%
\pgfpathlineto{\pgfqpoint{4.165467in}{1.437762in}}%
\pgfpathlineto{\pgfqpoint{4.165467in}{1.437762in}}%
\pgfpathclose%
\pgfusepath{stroke,fill}%
\end{pgfscope}%
\begin{pgfscope}%
\pgfsetbuttcap%
\pgfsetmiterjoin%
\definecolor{currentfill}{rgb}{0.501961,0.694118,0.827451}%
\pgfsetfillcolor{currentfill}%
\pgfsetlinewidth{1.003750pt}%
\definecolor{currentstroke}{rgb}{1.000000,1.000000,1.000000}%
\pgfsetstrokecolor{currentstroke}%
\pgfsetdash{}{0pt}%
\pgfpathmoveto{\pgfqpoint{4.165467in}{1.437762in}}%
\pgfpathcurveto{\pgfqpoint{4.293579in}{1.578356in}}{\pgfqpoint{4.394541in}{1.741476in}}{\pgfqpoint{4.463232in}{1.918848in}}%
\pgfpathcurveto{\pgfqpoint{4.531924in}{2.096219in}}{\pgfqpoint{4.567163in}{2.284792in}}{\pgfqpoint{4.567163in}{2.475001in}}%
\pgfpathlineto{\pgfqpoint{3.027163in}{2.475000in}}%
\pgfpathlineto{\pgfqpoint{4.165467in}{1.437762in}}%
\pgfpathlineto{\pgfqpoint{4.165467in}{1.437762in}}%
\pgfpathclose%
\pgfusepath{stroke,fill}%
\end{pgfscope}%
\begin{pgfscope}%
\pgfsetbuttcap%
\pgfsetmiterjoin%
\definecolor{currentfill}{rgb}{0.992157,0.705882,0.384314}%
\pgfsetfillcolor{currentfill}%
\pgfsetlinewidth{1.003750pt}%
\definecolor{currentstroke}{rgb}{1.000000,1.000000,1.000000}%
\pgfsetstrokecolor{currentstroke}%
\pgfsetdash{}{0pt}%
\pgfpathmoveto{\pgfqpoint{4.567163in}{2.475001in}}%
\pgfpathcurveto{\pgfqpoint{4.567163in}{2.475001in}}{\pgfqpoint{4.567163in}{2.475001in}}{\pgfqpoint{4.567163in}{2.475001in}}%
\pgfpathlineto{\pgfqpoint{3.027163in}{2.475000in}}%
\pgfpathlineto{\pgfqpoint{4.567163in}{2.475001in}}%
\pgfpathlineto{\pgfqpoint{4.567163in}{2.475001in}}%
\pgfpathclose%
\pgfusepath{stroke,fill}%
\end{pgfscope}%
\begin{pgfscope}%
\definecolor{textcolor}{rgb}{0.150000,0.150000,0.150000}%
\pgfsetstrokecolor{textcolor}%
\pgfsetfillcolor{textcolor}%
\pgftext[x=3.583970in,y=3.212389in,,]{\color{textcolor}\sffamily\fontsize{12.000000}{14.400000}\selectfont 29.4\%}%
\end{pgfscope}%
\begin{pgfscope}%
\definecolor{textcolor}{rgb}{0.150000,0.150000,0.150000}%
\pgfsetstrokecolor{textcolor}%
\pgfsetfillcolor{textcolor}%
\pgftext[x=2.165522in,y=2.808692in,,]{\color{textcolor}\sffamily\fontsize{12.000000}{14.400000}\selectfont 29.4\%}%
\end{pgfscope}%
\begin{pgfscope}%
\definecolor{textcolor}{rgb}{0.150000,0.150000,0.150000}%
\pgfsetstrokecolor{textcolor}%
\pgfsetfillcolor{textcolor}%
\pgftext[x=2.942078in,y=1.554926in,,]{\color{textcolor}\sffamily\fontsize{12.000000}{14.400000}\selectfont 29.4\%}%
\end{pgfscope}%
\begin{pgfscope}%
\definecolor{textcolor}{rgb}{0.150000,0.150000,0.150000}%
\pgfsetstrokecolor{textcolor}%
\pgfsetfillcolor{textcolor}%
\pgftext[x=3.710146in,y=1.852657in,,]{\color{textcolor}\sffamily\fontsize{12.000000}{14.400000}\selectfont 0.0\%}%
\end{pgfscope}%
\begin{pgfscope}%
\definecolor{textcolor}{rgb}{0.150000,0.150000,0.150000}%
\pgfsetstrokecolor{textcolor}%
\pgfsetfillcolor{textcolor}%
\pgftext[x=3.888805in,y=2.141309in,,]{\color{textcolor}\sffamily\fontsize{12.000000}{14.400000}\selectfont 11.8\%}%
\end{pgfscope}%
\begin{pgfscope}%
\definecolor{textcolor}{rgb}{0.150000,0.150000,0.150000}%
\pgfsetstrokecolor{textcolor}%
\pgfsetfillcolor{textcolor}%
\pgftext[x=3.951163in,y=2.475000in,,]{\color{textcolor}\sffamily\fontsize{12.000000}{14.400000}\selectfont 0.0\%}%
\end{pgfscope}%
\begin{pgfscope}%
\pgfsetbuttcap%
\pgfsetmiterjoin%
\definecolor{currentfill}{rgb}{1.000000,1.000000,1.000000}%
\pgfsetfillcolor{currentfill}%
\pgfsetfillopacity{0.800000}%
\pgfsetlinewidth{1.003750pt}%
\definecolor{currentstroke}{rgb}{0.800000,0.800000,0.800000}%
\pgfsetstrokecolor{currentstroke}%
\pgfsetstrokeopacity{0.800000}%
\pgfsetdash{}{0pt}%
\pgfpathmoveto{\pgfqpoint{3.271524in}{0.458500in}}%
\pgfpathlineto{\pgfqpoint{5.827163in}{0.458500in}}%
\pgfpathquadraticcurveto{\pgfqpoint{5.852163in}{0.458500in}}{\pgfqpoint{5.852163in}{0.483500in}}%
\pgfpathlineto{\pgfqpoint{5.852163in}{1.571829in}}%
\pgfpathquadraticcurveto{\pgfqpoint{5.852163in}{1.596829in}}{\pgfqpoint{5.827163in}{1.596829in}}%
\pgfpathlineto{\pgfqpoint{3.271524in}{1.596829in}}%
\pgfpathquadraticcurveto{\pgfqpoint{3.246524in}{1.596829in}}{\pgfqpoint{3.246524in}{1.571829in}}%
\pgfpathlineto{\pgfqpoint{3.246524in}{0.483500in}}%
\pgfpathquadraticcurveto{\pgfqpoint{3.246524in}{0.458500in}}{\pgfqpoint{3.271524in}{0.458500in}}%
\pgfpathlineto{\pgfqpoint{3.271524in}{0.458500in}}%
\pgfpathclose%
\pgfusepath{stroke,fill}%
\end{pgfscope}%
\begin{pgfscope}%
\pgfsetbuttcap%
\pgfsetmiterjoin%
\definecolor{currentfill}{rgb}{0.552941,0.827451,0.780392}%
\pgfsetfillcolor{currentfill}%
\pgfsetlinewidth{1.003750pt}%
\definecolor{currentstroke}{rgb}{1.000000,1.000000,1.000000}%
\pgfsetstrokecolor{currentstroke}%
\pgfsetdash{}{0pt}%
\pgfpathmoveto{\pgfqpoint{3.296524in}{1.451858in}}%
\pgfpathlineto{\pgfqpoint{3.546524in}{1.451858in}}%
\pgfpathlineto{\pgfqpoint{3.546524in}{1.539358in}}%
\pgfpathlineto{\pgfqpoint{3.296524in}{1.539358in}}%
\pgfpathlineto{\pgfqpoint{3.296524in}{1.451858in}}%
\pgfpathclose%
\pgfusepath{stroke,fill}%
\end{pgfscope}%
\begin{pgfscope}%
\definecolor{textcolor}{rgb}{0.150000,0.150000,0.150000}%
\pgfsetstrokecolor{textcolor}%
\pgfsetfillcolor{textcolor}%
\pgftext[x=3.646524in,y=1.451858in,left,base]{\color{textcolor}\sffamily\fontsize{9.000000}{10.800000}\selectfont No CS Degree, 29.4 \%}%
\end{pgfscope}%
\begin{pgfscope}%
\pgfsetbuttcap%
\pgfsetmiterjoin%
\definecolor{currentfill}{rgb}{1.000000,1.000000,0.701961}%
\pgfsetfillcolor{currentfill}%
\pgfsetlinewidth{1.003750pt}%
\definecolor{currentstroke}{rgb}{1.000000,1.000000,1.000000}%
\pgfsetstrokecolor{currentstroke}%
\pgfsetdash{}{0pt}%
\pgfpathmoveto{\pgfqpoint{3.296524in}{1.268387in}}%
\pgfpathlineto{\pgfqpoint{3.546524in}{1.268387in}}%
\pgfpathlineto{\pgfqpoint{3.546524in}{1.355887in}}%
\pgfpathlineto{\pgfqpoint{3.296524in}{1.355887in}}%
\pgfpathlineto{\pgfqpoint{3.296524in}{1.268387in}}%
\pgfpathclose%
\pgfusepath{stroke,fill}%
\end{pgfscope}%
\begin{pgfscope}%
\definecolor{textcolor}{rgb}{0.150000,0.150000,0.150000}%
\pgfsetstrokecolor{textcolor}%
\pgfsetfillcolor{textcolor}%
\pgftext[x=3.646524in,y=1.268387in,left,base]{\color{textcolor}\sffamily\fontsize{9.000000}{10.800000}\selectfont Bachelors's Degree, 29.4 \%}%
\end{pgfscope}%
\begin{pgfscope}%
\pgfsetbuttcap%
\pgfsetmiterjoin%
\definecolor{currentfill}{rgb}{0.745098,0.729412,0.854902}%
\pgfsetfillcolor{currentfill}%
\pgfsetlinewidth{1.003750pt}%
\definecolor{currentstroke}{rgb}{1.000000,1.000000,1.000000}%
\pgfsetstrokecolor{currentstroke}%
\pgfsetdash{}{0pt}%
\pgfpathmoveto{\pgfqpoint{3.296524in}{1.084915in}}%
\pgfpathlineto{\pgfqpoint{3.546524in}{1.084915in}}%
\pgfpathlineto{\pgfqpoint{3.546524in}{1.172415in}}%
\pgfpathlineto{\pgfqpoint{3.296524in}{1.172415in}}%
\pgfpathlineto{\pgfqpoint{3.296524in}{1.084915in}}%
\pgfpathclose%
\pgfusepath{stroke,fill}%
\end{pgfscope}%
\begin{pgfscope}%
\definecolor{textcolor}{rgb}{0.150000,0.150000,0.150000}%
\pgfsetstrokecolor{textcolor}%
\pgfsetfillcolor{textcolor}%
\pgftext[x=3.646524in,y=1.084915in,left,base]{\color{textcolor}\sffamily\fontsize{9.000000}{10.800000}\selectfont Master's Degree, 29.4 \%}%
\end{pgfscope}%
\begin{pgfscope}%
\pgfsetbuttcap%
\pgfsetmiterjoin%
\definecolor{currentfill}{rgb}{0.984314,0.501961,0.447059}%
\pgfsetfillcolor{currentfill}%
\pgfsetlinewidth{1.003750pt}%
\definecolor{currentstroke}{rgb}{1.000000,1.000000,1.000000}%
\pgfsetstrokecolor{currentstroke}%
\pgfsetdash{}{0pt}%
\pgfpathmoveto{\pgfqpoint{3.296524in}{0.901444in}}%
\pgfpathlineto{\pgfqpoint{3.546524in}{0.901444in}}%
\pgfpathlineto{\pgfqpoint{3.546524in}{0.988944in}}%
\pgfpathlineto{\pgfqpoint{3.296524in}{0.988944in}}%
\pgfpathlineto{\pgfqpoint{3.296524in}{0.901444in}}%
\pgfpathclose%
\pgfusepath{stroke,fill}%
\end{pgfscope}%
\begin{pgfscope}%
\definecolor{textcolor}{rgb}{0.150000,0.150000,0.150000}%
\pgfsetstrokecolor{textcolor}%
\pgfsetfillcolor{textcolor}%
\pgftext[x=3.646524in,y=0.901444in,left,base]{\color{textcolor}\sffamily\fontsize{9.000000}{10.800000}\selectfont Doctorate Degree, 0.0 \%}%
\end{pgfscope}%
\begin{pgfscope}%
\pgfsetbuttcap%
\pgfsetmiterjoin%
\definecolor{currentfill}{rgb}{0.501961,0.694118,0.827451}%
\pgfsetfillcolor{currentfill}%
\pgfsetlinewidth{1.003750pt}%
\definecolor{currentstroke}{rgb}{1.000000,1.000000,1.000000}%
\pgfsetstrokecolor{currentstroke}%
\pgfsetdash{}{0pt}%
\pgfpathmoveto{\pgfqpoint{3.296524in}{0.717972in}}%
\pgfpathlineto{\pgfqpoint{3.546524in}{0.717972in}}%
\pgfpathlineto{\pgfqpoint{3.546524in}{0.805472in}}%
\pgfpathlineto{\pgfqpoint{3.296524in}{0.805472in}}%
\pgfpathlineto{\pgfqpoint{3.296524in}{0.717972in}}%
\pgfpathclose%
\pgfusepath{stroke,fill}%
\end{pgfscope}%
\begin{pgfscope}%
\definecolor{textcolor}{rgb}{0.150000,0.150000,0.150000}%
\pgfsetstrokecolor{textcolor}%
\pgfsetfillcolor{textcolor}%
\pgftext[x=3.646524in,y=0.717972in,left,base]{\color{textcolor}\sffamily\fontsize{9.000000}{10.800000}\selectfont Other Engineering Degree, 11.8 \%}%
\end{pgfscope}%
\begin{pgfscope}%
\pgfsetbuttcap%
\pgfsetmiterjoin%
\definecolor{currentfill}{rgb}{0.992157,0.705882,0.384314}%
\pgfsetfillcolor{currentfill}%
\pgfsetlinewidth{1.003750pt}%
\definecolor{currentstroke}{rgb}{1.000000,1.000000,1.000000}%
\pgfsetstrokecolor{currentstroke}%
\pgfsetdash{}{0pt}%
\pgfpathmoveto{\pgfqpoint{3.296524in}{0.534501in}}%
\pgfpathlineto{\pgfqpoint{3.546524in}{0.534501in}}%
\pgfpathlineto{\pgfqpoint{3.546524in}{0.622001in}}%
\pgfpathlineto{\pgfqpoint{3.296524in}{0.622001in}}%
\pgfpathlineto{\pgfqpoint{3.296524in}{0.534501in}}%
\pgfpathclose%
\pgfusepath{stroke,fill}%
\end{pgfscope}%
\begin{pgfscope}%
\definecolor{textcolor}{rgb}{0.150000,0.150000,0.150000}%
\pgfsetstrokecolor{textcolor}%
\pgfsetfillcolor{textcolor}%
\pgftext[x=3.646524in,y=0.534501in,left,base]{\color{textcolor}\sffamily\fontsize{9.000000}{10.800000}\selectfont Yes but did not finish, 0.0 \%}%
\end{pgfscope}%
\end{pgfpicture}%
\makeatother%
\endgroup%
}
		\caption{first figure}
	\end{minipage}\hspace{-1em}
	\begin{minipage}{0.45\textwidth}
		\centering
		\scalebox{0.7}{%% Creator: Matplotlib, PGF backend
%%
%% To include the figure in your LaTeX document, write
%%   \input{<filename>.pgf}
%%
%% Make sure the required packages are loaded in your preamble
%%   \usepackage{pgf}
%%
%% Also ensure that all the required font packages are loaded; for instance,
%% the lmodern package is sometimes necessary when using math font.
%%   \usepackage{lmodern}
%%
%% Figures using additional raster images can only be included by \input if
%% they are in the same directory as the main LaTeX file. For loading figures
%% from other directories you can use the `import` package
%%   \usepackage{import}
%%
%% and then include the figures with
%%   \import{<path to file>}{<filename>.pgf}
%%
%% Matplotlib used the following preamble
%%   \usepackage{fontspec}
%%   \setmainfont{DejaVuSerif.ttf}[Path=\detokenize{/home/spam/miniconda3/envs/mpl/lib/python3.10/site-packages/matplotlib/mpl-data/fonts/ttf/}]
%%   \setsansfont{DejaVuSans.ttf}[Path=\detokenize{/home/spam/miniconda3/envs/mpl/lib/python3.10/site-packages/matplotlib/mpl-data/fonts/ttf/}]
%%   \setmonofont{DejaVuSansMono.ttf}[Path=\detokenize{/home/spam/miniconda3/envs/mpl/lib/python3.10/site-packages/matplotlib/mpl-data/fonts/ttf/}]
%%
\begingroup%
\makeatletter%
\begin{pgfpicture}%
\pgfpathrectangle{\pgfpointorigin}{\pgfqpoint{5.906660in}{5.000000in}}%
\pgfusepath{use as bounding box, clip}%
\begin{pgfscope}%
\pgfsetbuttcap%
\pgfsetmiterjoin%
\definecolor{currentfill}{rgb}{1.000000,1.000000,1.000000}%
\pgfsetfillcolor{currentfill}%
\pgfsetlinewidth{0.000000pt}%
\definecolor{currentstroke}{rgb}{1.000000,1.000000,1.000000}%
\pgfsetstrokecolor{currentstroke}%
\pgfsetdash{}{0pt}%
\pgfpathmoveto{\pgfqpoint{0.000000in}{0.000000in}}%
\pgfpathlineto{\pgfqpoint{5.906660in}{0.000000in}}%
\pgfpathlineto{\pgfqpoint{5.906660in}{5.000000in}}%
\pgfpathlineto{\pgfqpoint{0.000000in}{5.000000in}}%
\pgfpathlineto{\pgfqpoint{0.000000in}{0.000000in}}%
\pgfpathclose%
\pgfusepath{fill}%
\end{pgfscope}%
\begin{pgfscope}%
\definecolor{textcolor}{rgb}{0.150000,0.150000,0.150000}%
\pgfsetstrokecolor{textcolor}%
\pgfsetfillcolor{textcolor}%
\pgftext[x=3.027163in,y=0.411111in,,top]{\color{textcolor}\sffamily\fontsize{12.000000}{14.400000}\selectfont Computer Science UniversityUniverity Experience}%
\end{pgfscope}%
\begin{pgfscope}%
\pgfsetbuttcap%
\pgfsetmiterjoin%
\definecolor{currentfill}{rgb}{0.552941,0.827451,0.780392}%
\pgfsetfillcolor{currentfill}%
\pgfsetlinewidth{1.003750pt}%
\definecolor{currentstroke}{rgb}{1.000000,1.000000,1.000000}%
\pgfsetstrokecolor{currentstroke}%
\pgfsetdash{}{0pt}%
\pgfpathmoveto{\pgfqpoint{4.567163in}{2.475000in}}%
\pgfpathcurveto{\pgfqpoint{4.567163in}{2.713219in}}{\pgfqpoint{4.511889in}{2.948220in}}{\pgfqpoint{4.405702in}{3.161463in}}%
\pgfpathcurveto{\pgfqpoint{4.299515in}{3.374705in}}{\pgfqpoint{4.145283in}{3.560429in}}{\pgfqpoint{3.955175in}{3.703981in}}%
\pgfpathcurveto{\pgfqpoint{3.765067in}{3.847533in}}{\pgfqpoint{3.544218in}{3.945035in}}{\pgfqpoint{3.310053in}{3.988794in}}%
\pgfpathcurveto{\pgfqpoint{3.075889in}{4.032554in}}{\pgfqpoint{2.834733in}{4.021389in}}{\pgfqpoint{2.605613in}{3.956180in}}%
\pgfpathlineto{\pgfqpoint{3.027163in}{2.475000in}}%
\pgfpathlineto{\pgfqpoint{4.567163in}{2.475000in}}%
\pgfpathlineto{\pgfqpoint{4.567163in}{2.475000in}}%
\pgfpathclose%
\pgfusepath{stroke,fill}%
\end{pgfscope}%
\begin{pgfscope}%
\pgfsetbuttcap%
\pgfsetmiterjoin%
\definecolor{currentfill}{rgb}{1.000000,1.000000,0.701961}%
\pgfsetfillcolor{currentfill}%
\pgfsetlinewidth{1.003750pt}%
\definecolor{currentstroke}{rgb}{1.000000,1.000000,1.000000}%
\pgfsetstrokecolor{currentstroke}%
\pgfsetdash{}{0pt}%
\pgfpathmoveto{\pgfqpoint{2.605613in}{3.956180in}}%
\pgfpathcurveto{\pgfqpoint{2.376493in}{3.890972in}}{\pgfqpoint{2.165597in}{3.773481in}}{\pgfqpoint{1.989567in}{3.612978in}}%
\pgfpathcurveto{\pgfqpoint{1.813536in}{3.452475in}}{\pgfqpoint{1.677124in}{3.253294in}}{\pgfqpoint{1.591094in}{3.031153in}}%
\pgfpathcurveto{\pgfqpoint{1.505064in}{2.809011in}}{\pgfqpoint{1.471740in}{2.569908in}}{\pgfqpoint{1.493751in}{2.332708in}}%
\pgfpathcurveto{\pgfqpoint{1.515762in}{2.095508in}}{\pgfqpoint{1.592513in}{1.866620in}}{\pgfqpoint{1.717949in}{1.664101in}}%
\pgfpathlineto{\pgfqpoint{3.027163in}{2.475000in}}%
\pgfpathlineto{\pgfqpoint{2.605613in}{3.956180in}}%
\pgfpathlineto{\pgfqpoint{2.605613in}{3.956180in}}%
\pgfpathclose%
\pgfusepath{stroke,fill}%
\end{pgfscope}%
\begin{pgfscope}%
\pgfsetbuttcap%
\pgfsetmiterjoin%
\definecolor{currentfill}{rgb}{0.745098,0.729412,0.854902}%
\pgfsetfillcolor{currentfill}%
\pgfsetlinewidth{1.003750pt}%
\definecolor{currentstroke}{rgb}{1.000000,1.000000,1.000000}%
\pgfsetstrokecolor{currentstroke}%
\pgfsetdash{}{0pt}%
\pgfpathmoveto{\pgfqpoint{1.717949in}{1.664101in}}%
\pgfpathcurveto{\pgfqpoint{1.843385in}{1.461582in}}{\pgfqpoint{2.014117in}{1.290903in}}{\pgfqpoint{2.216676in}{1.165531in}}%
\pgfpathcurveto{\pgfqpoint{2.419234in}{1.040158in}}{\pgfqpoint{2.648147in}{0.963479in}}{\pgfqpoint{2.885354in}{0.941543in}}%
\pgfpathcurveto{\pgfqpoint{3.122560in}{0.919607in}}{\pgfqpoint{3.361653in}{0.953006in}}{\pgfqpoint{3.583768in}{1.039106in}}%
\pgfpathcurveto{\pgfqpoint{3.805882in}{1.125206in}}{\pgfqpoint{4.005020in}{1.261680in}}{\pgfqpoint{4.165467in}{1.437762in}}%
\pgfpathlineto{\pgfqpoint{3.027163in}{2.475000in}}%
\pgfpathlineto{\pgfqpoint{1.717949in}{1.664101in}}%
\pgfpathlineto{\pgfqpoint{1.717949in}{1.664101in}}%
\pgfpathclose%
\pgfusepath{stroke,fill}%
\end{pgfscope}%
\begin{pgfscope}%
\pgfsetbuttcap%
\pgfsetmiterjoin%
\definecolor{currentfill}{rgb}{0.984314,0.501961,0.447059}%
\pgfsetfillcolor{currentfill}%
\pgfsetlinewidth{1.003750pt}%
\definecolor{currentstroke}{rgb}{1.000000,1.000000,1.000000}%
\pgfsetstrokecolor{currentstroke}%
\pgfsetdash{}{0pt}%
\pgfpathmoveto{\pgfqpoint{4.165467in}{1.437762in}}%
\pgfpathcurveto{\pgfqpoint{4.165467in}{1.437762in}}{\pgfqpoint{4.165467in}{1.437762in}}{\pgfqpoint{4.165467in}{1.437762in}}%
\pgfpathlineto{\pgfqpoint{3.027163in}{2.475000in}}%
\pgfpathlineto{\pgfqpoint{4.165467in}{1.437762in}}%
\pgfpathlineto{\pgfqpoint{4.165467in}{1.437762in}}%
\pgfpathclose%
\pgfusepath{stroke,fill}%
\end{pgfscope}%
\begin{pgfscope}%
\pgfsetbuttcap%
\pgfsetmiterjoin%
\definecolor{currentfill}{rgb}{0.501961,0.694118,0.827451}%
\pgfsetfillcolor{currentfill}%
\pgfsetlinewidth{1.003750pt}%
\definecolor{currentstroke}{rgb}{1.000000,1.000000,1.000000}%
\pgfsetstrokecolor{currentstroke}%
\pgfsetdash{}{0pt}%
\pgfpathmoveto{\pgfqpoint{4.165467in}{1.437762in}}%
\pgfpathcurveto{\pgfqpoint{4.293579in}{1.578356in}}{\pgfqpoint{4.394541in}{1.741476in}}{\pgfqpoint{4.463232in}{1.918848in}}%
\pgfpathcurveto{\pgfqpoint{4.531924in}{2.096219in}}{\pgfqpoint{4.567163in}{2.284792in}}{\pgfqpoint{4.567163in}{2.475001in}}%
\pgfpathlineto{\pgfqpoint{3.027163in}{2.475000in}}%
\pgfpathlineto{\pgfqpoint{4.165467in}{1.437762in}}%
\pgfpathlineto{\pgfqpoint{4.165467in}{1.437762in}}%
\pgfpathclose%
\pgfusepath{stroke,fill}%
\end{pgfscope}%
\begin{pgfscope}%
\pgfsetbuttcap%
\pgfsetmiterjoin%
\definecolor{currentfill}{rgb}{0.992157,0.705882,0.384314}%
\pgfsetfillcolor{currentfill}%
\pgfsetlinewidth{1.003750pt}%
\definecolor{currentstroke}{rgb}{1.000000,1.000000,1.000000}%
\pgfsetstrokecolor{currentstroke}%
\pgfsetdash{}{0pt}%
\pgfpathmoveto{\pgfqpoint{4.567163in}{2.475001in}}%
\pgfpathcurveto{\pgfqpoint{4.567163in}{2.475001in}}{\pgfqpoint{4.567163in}{2.475001in}}{\pgfqpoint{4.567163in}{2.475001in}}%
\pgfpathlineto{\pgfqpoint{3.027163in}{2.475000in}}%
\pgfpathlineto{\pgfqpoint{4.567163in}{2.475001in}}%
\pgfpathlineto{\pgfqpoint{4.567163in}{2.475001in}}%
\pgfpathclose%
\pgfusepath{stroke,fill}%
\end{pgfscope}%
\begin{pgfscope}%
\definecolor{textcolor}{rgb}{0.150000,0.150000,0.150000}%
\pgfsetstrokecolor{textcolor}%
\pgfsetfillcolor{textcolor}%
\pgftext[x=3.583970in,y=3.212389in,,]{\color{textcolor}\sffamily\fontsize{12.000000}{14.400000}\selectfont 29.4\%}%
\end{pgfscope}%
\begin{pgfscope}%
\definecolor{textcolor}{rgb}{0.150000,0.150000,0.150000}%
\pgfsetstrokecolor{textcolor}%
\pgfsetfillcolor{textcolor}%
\pgftext[x=2.165522in,y=2.808692in,,]{\color{textcolor}\sffamily\fontsize{12.000000}{14.400000}\selectfont 29.4\%}%
\end{pgfscope}%
\begin{pgfscope}%
\definecolor{textcolor}{rgb}{0.150000,0.150000,0.150000}%
\pgfsetstrokecolor{textcolor}%
\pgfsetfillcolor{textcolor}%
\pgftext[x=2.942078in,y=1.554926in,,]{\color{textcolor}\sffamily\fontsize{12.000000}{14.400000}\selectfont 29.4\%}%
\end{pgfscope}%
\begin{pgfscope}%
\definecolor{textcolor}{rgb}{0.150000,0.150000,0.150000}%
\pgfsetstrokecolor{textcolor}%
\pgfsetfillcolor{textcolor}%
\pgftext[x=3.710146in,y=1.852657in,,]{\color{textcolor}\sffamily\fontsize{12.000000}{14.400000}\selectfont 0.0\%}%
\end{pgfscope}%
\begin{pgfscope}%
\definecolor{textcolor}{rgb}{0.150000,0.150000,0.150000}%
\pgfsetstrokecolor{textcolor}%
\pgfsetfillcolor{textcolor}%
\pgftext[x=3.888805in,y=2.141309in,,]{\color{textcolor}\sffamily\fontsize{12.000000}{14.400000}\selectfont 11.8\%}%
\end{pgfscope}%
\begin{pgfscope}%
\definecolor{textcolor}{rgb}{0.150000,0.150000,0.150000}%
\pgfsetstrokecolor{textcolor}%
\pgfsetfillcolor{textcolor}%
\pgftext[x=3.951163in,y=2.475000in,,]{\color{textcolor}\sffamily\fontsize{12.000000}{14.400000}\selectfont 0.0\%}%
\end{pgfscope}%
\begin{pgfscope}%
\pgfsetbuttcap%
\pgfsetmiterjoin%
\definecolor{currentfill}{rgb}{1.000000,1.000000,1.000000}%
\pgfsetfillcolor{currentfill}%
\pgfsetfillopacity{0.800000}%
\pgfsetlinewidth{1.003750pt}%
\definecolor{currentstroke}{rgb}{0.800000,0.800000,0.800000}%
\pgfsetstrokecolor{currentstroke}%
\pgfsetstrokeopacity{0.800000}%
\pgfsetdash{}{0pt}%
\pgfpathmoveto{\pgfqpoint{3.271524in}{0.458500in}}%
\pgfpathlineto{\pgfqpoint{5.827163in}{0.458500in}}%
\pgfpathquadraticcurveto{\pgfqpoint{5.852163in}{0.458500in}}{\pgfqpoint{5.852163in}{0.483500in}}%
\pgfpathlineto{\pgfqpoint{5.852163in}{1.571829in}}%
\pgfpathquadraticcurveto{\pgfqpoint{5.852163in}{1.596829in}}{\pgfqpoint{5.827163in}{1.596829in}}%
\pgfpathlineto{\pgfqpoint{3.271524in}{1.596829in}}%
\pgfpathquadraticcurveto{\pgfqpoint{3.246524in}{1.596829in}}{\pgfqpoint{3.246524in}{1.571829in}}%
\pgfpathlineto{\pgfqpoint{3.246524in}{0.483500in}}%
\pgfpathquadraticcurveto{\pgfqpoint{3.246524in}{0.458500in}}{\pgfqpoint{3.271524in}{0.458500in}}%
\pgfpathlineto{\pgfqpoint{3.271524in}{0.458500in}}%
\pgfpathclose%
\pgfusepath{stroke,fill}%
\end{pgfscope}%
\begin{pgfscope}%
\pgfsetbuttcap%
\pgfsetmiterjoin%
\definecolor{currentfill}{rgb}{0.552941,0.827451,0.780392}%
\pgfsetfillcolor{currentfill}%
\pgfsetlinewidth{1.003750pt}%
\definecolor{currentstroke}{rgb}{1.000000,1.000000,1.000000}%
\pgfsetstrokecolor{currentstroke}%
\pgfsetdash{}{0pt}%
\pgfpathmoveto{\pgfqpoint{3.296524in}{1.451858in}}%
\pgfpathlineto{\pgfqpoint{3.546524in}{1.451858in}}%
\pgfpathlineto{\pgfqpoint{3.546524in}{1.539358in}}%
\pgfpathlineto{\pgfqpoint{3.296524in}{1.539358in}}%
\pgfpathlineto{\pgfqpoint{3.296524in}{1.451858in}}%
\pgfpathclose%
\pgfusepath{stroke,fill}%
\end{pgfscope}%
\begin{pgfscope}%
\definecolor{textcolor}{rgb}{0.150000,0.150000,0.150000}%
\pgfsetstrokecolor{textcolor}%
\pgfsetfillcolor{textcolor}%
\pgftext[x=3.646524in,y=1.451858in,left,base]{\color{textcolor}\sffamily\fontsize{9.000000}{10.800000}\selectfont No CS Degree, 29.4 \%}%
\end{pgfscope}%
\begin{pgfscope}%
\pgfsetbuttcap%
\pgfsetmiterjoin%
\definecolor{currentfill}{rgb}{1.000000,1.000000,0.701961}%
\pgfsetfillcolor{currentfill}%
\pgfsetlinewidth{1.003750pt}%
\definecolor{currentstroke}{rgb}{1.000000,1.000000,1.000000}%
\pgfsetstrokecolor{currentstroke}%
\pgfsetdash{}{0pt}%
\pgfpathmoveto{\pgfqpoint{3.296524in}{1.268387in}}%
\pgfpathlineto{\pgfqpoint{3.546524in}{1.268387in}}%
\pgfpathlineto{\pgfqpoint{3.546524in}{1.355887in}}%
\pgfpathlineto{\pgfqpoint{3.296524in}{1.355887in}}%
\pgfpathlineto{\pgfqpoint{3.296524in}{1.268387in}}%
\pgfpathclose%
\pgfusepath{stroke,fill}%
\end{pgfscope}%
\begin{pgfscope}%
\definecolor{textcolor}{rgb}{0.150000,0.150000,0.150000}%
\pgfsetstrokecolor{textcolor}%
\pgfsetfillcolor{textcolor}%
\pgftext[x=3.646524in,y=1.268387in,left,base]{\color{textcolor}\sffamily\fontsize{9.000000}{10.800000}\selectfont Bachelors's Degree, 29.4 \%}%
\end{pgfscope}%
\begin{pgfscope}%
\pgfsetbuttcap%
\pgfsetmiterjoin%
\definecolor{currentfill}{rgb}{0.745098,0.729412,0.854902}%
\pgfsetfillcolor{currentfill}%
\pgfsetlinewidth{1.003750pt}%
\definecolor{currentstroke}{rgb}{1.000000,1.000000,1.000000}%
\pgfsetstrokecolor{currentstroke}%
\pgfsetdash{}{0pt}%
\pgfpathmoveto{\pgfqpoint{3.296524in}{1.084915in}}%
\pgfpathlineto{\pgfqpoint{3.546524in}{1.084915in}}%
\pgfpathlineto{\pgfqpoint{3.546524in}{1.172415in}}%
\pgfpathlineto{\pgfqpoint{3.296524in}{1.172415in}}%
\pgfpathlineto{\pgfqpoint{3.296524in}{1.084915in}}%
\pgfpathclose%
\pgfusepath{stroke,fill}%
\end{pgfscope}%
\begin{pgfscope}%
\definecolor{textcolor}{rgb}{0.150000,0.150000,0.150000}%
\pgfsetstrokecolor{textcolor}%
\pgfsetfillcolor{textcolor}%
\pgftext[x=3.646524in,y=1.084915in,left,base]{\color{textcolor}\sffamily\fontsize{9.000000}{10.800000}\selectfont Master's Degree, 29.4 \%}%
\end{pgfscope}%
\begin{pgfscope}%
\pgfsetbuttcap%
\pgfsetmiterjoin%
\definecolor{currentfill}{rgb}{0.984314,0.501961,0.447059}%
\pgfsetfillcolor{currentfill}%
\pgfsetlinewidth{1.003750pt}%
\definecolor{currentstroke}{rgb}{1.000000,1.000000,1.000000}%
\pgfsetstrokecolor{currentstroke}%
\pgfsetdash{}{0pt}%
\pgfpathmoveto{\pgfqpoint{3.296524in}{0.901444in}}%
\pgfpathlineto{\pgfqpoint{3.546524in}{0.901444in}}%
\pgfpathlineto{\pgfqpoint{3.546524in}{0.988944in}}%
\pgfpathlineto{\pgfqpoint{3.296524in}{0.988944in}}%
\pgfpathlineto{\pgfqpoint{3.296524in}{0.901444in}}%
\pgfpathclose%
\pgfusepath{stroke,fill}%
\end{pgfscope}%
\begin{pgfscope}%
\definecolor{textcolor}{rgb}{0.150000,0.150000,0.150000}%
\pgfsetstrokecolor{textcolor}%
\pgfsetfillcolor{textcolor}%
\pgftext[x=3.646524in,y=0.901444in,left,base]{\color{textcolor}\sffamily\fontsize{9.000000}{10.800000}\selectfont Doctorate Degree, 0.0 \%}%
\end{pgfscope}%
\begin{pgfscope}%
\pgfsetbuttcap%
\pgfsetmiterjoin%
\definecolor{currentfill}{rgb}{0.501961,0.694118,0.827451}%
\pgfsetfillcolor{currentfill}%
\pgfsetlinewidth{1.003750pt}%
\definecolor{currentstroke}{rgb}{1.000000,1.000000,1.000000}%
\pgfsetstrokecolor{currentstroke}%
\pgfsetdash{}{0pt}%
\pgfpathmoveto{\pgfqpoint{3.296524in}{0.717972in}}%
\pgfpathlineto{\pgfqpoint{3.546524in}{0.717972in}}%
\pgfpathlineto{\pgfqpoint{3.546524in}{0.805472in}}%
\pgfpathlineto{\pgfqpoint{3.296524in}{0.805472in}}%
\pgfpathlineto{\pgfqpoint{3.296524in}{0.717972in}}%
\pgfpathclose%
\pgfusepath{stroke,fill}%
\end{pgfscope}%
\begin{pgfscope}%
\definecolor{textcolor}{rgb}{0.150000,0.150000,0.150000}%
\pgfsetstrokecolor{textcolor}%
\pgfsetfillcolor{textcolor}%
\pgftext[x=3.646524in,y=0.717972in,left,base]{\color{textcolor}\sffamily\fontsize{9.000000}{10.800000}\selectfont Other Engineering Degree, 11.8 \%}%
\end{pgfscope}%
\begin{pgfscope}%
\pgfsetbuttcap%
\pgfsetmiterjoin%
\definecolor{currentfill}{rgb}{0.992157,0.705882,0.384314}%
\pgfsetfillcolor{currentfill}%
\pgfsetlinewidth{1.003750pt}%
\definecolor{currentstroke}{rgb}{1.000000,1.000000,1.000000}%
\pgfsetstrokecolor{currentstroke}%
\pgfsetdash{}{0pt}%
\pgfpathmoveto{\pgfqpoint{3.296524in}{0.534501in}}%
\pgfpathlineto{\pgfqpoint{3.546524in}{0.534501in}}%
\pgfpathlineto{\pgfqpoint{3.546524in}{0.622001in}}%
\pgfpathlineto{\pgfqpoint{3.296524in}{0.622001in}}%
\pgfpathlineto{\pgfqpoint{3.296524in}{0.534501in}}%
\pgfpathclose%
\pgfusepath{stroke,fill}%
\end{pgfscope}%
\begin{pgfscope}%
\definecolor{textcolor}{rgb}{0.150000,0.150000,0.150000}%
\pgfsetstrokecolor{textcolor}%
\pgfsetfillcolor{textcolor}%
\pgftext[x=3.646524in,y=0.534501in,left,base]{\color{textcolor}\sffamily\fontsize{9.000000}{10.800000}\selectfont Yes but did not finish, 0.0 \%}%
\end{pgfscope}%
\end{pgfpicture}%
\makeatother%
\endgroup%
}
		\caption{second figure}
	\end{minipage}
\end{figure}



\begin{figure}[H]
	\scalebox{0.72}{%% Creator: Matplotlib, PGF backend
%%
%% To include the figure in your LaTeX document, write
%%   \input{<filename>.pgf}
%%
%% Make sure the required packages are loaded in your preamble
%%   \usepackage{pgf}
%%
%% Also ensure that all the required font packages are loaded; for instance,
%% the lmodern package is sometimes necessary when using math font.
%%   \usepackage{lmodern}
%%
%% Figures using additional raster images can only be included by \input if
%% they are in the same directory as the main LaTeX file. For loading figures
%% from other directories you can use the `import` package
%%   \usepackage{import}
%%
%% and then include the figures with
%%   \import{<path to file>}{<filename>.pgf}
%%
%% Matplotlib used the following preamble
%%   \usepackage{fontspec}
%%   \setmainfont{DejaVuSerif.ttf}[Path=\detokenize{/home/spam/miniconda3/envs/mpl/lib/python3.10/site-packages/matplotlib/mpl-data/fonts/ttf/}]
%%   \setsansfont{DejaVuSans.ttf}[Path=\detokenize{/home/spam/miniconda3/envs/mpl/lib/python3.10/site-packages/matplotlib/mpl-data/fonts/ttf/}]
%%   \setmonofont{DejaVuSansMono.ttf}[Path=\detokenize{/home/spam/miniconda3/envs/mpl/lib/python3.10/site-packages/matplotlib/mpl-data/fonts/ttf/}]
%%
\begingroup%
\makeatletter%
\begin{pgfpicture}%
\pgfpathrectangle{\pgfpointorigin}{\pgfqpoint{5.906660in}{5.000000in}}%
\pgfusepath{use as bounding box, clip}%
\begin{pgfscope}%
\pgfsetbuttcap%
\pgfsetmiterjoin%
\definecolor{currentfill}{rgb}{1.000000,1.000000,1.000000}%
\pgfsetfillcolor{currentfill}%
\pgfsetlinewidth{0.000000pt}%
\definecolor{currentstroke}{rgb}{1.000000,1.000000,1.000000}%
\pgfsetstrokecolor{currentstroke}%
\pgfsetdash{}{0pt}%
\pgfpathmoveto{\pgfqpoint{0.000000in}{0.000000in}}%
\pgfpathlineto{\pgfqpoint{5.906660in}{0.000000in}}%
\pgfpathlineto{\pgfqpoint{5.906660in}{5.000000in}}%
\pgfpathlineto{\pgfqpoint{0.000000in}{5.000000in}}%
\pgfpathlineto{\pgfqpoint{0.000000in}{0.000000in}}%
\pgfpathclose%
\pgfusepath{fill}%
\end{pgfscope}%
\begin{pgfscope}%
\definecolor{textcolor}{rgb}{0.150000,0.150000,0.150000}%
\pgfsetstrokecolor{textcolor}%
\pgfsetfillcolor{textcolor}%
\pgftext[x=3.027163in,y=0.411111in,,top]{\color{textcolor}\sffamily\fontsize{12.000000}{14.400000}\selectfont Computer Science UniversityUniverity Experience}%
\end{pgfscope}%
\begin{pgfscope}%
\pgfsetbuttcap%
\pgfsetmiterjoin%
\definecolor{currentfill}{rgb}{0.552941,0.827451,0.780392}%
\pgfsetfillcolor{currentfill}%
\pgfsetlinewidth{1.003750pt}%
\definecolor{currentstroke}{rgb}{1.000000,1.000000,1.000000}%
\pgfsetstrokecolor{currentstroke}%
\pgfsetdash{}{0pt}%
\pgfpathmoveto{\pgfqpoint{4.567163in}{2.475000in}}%
\pgfpathcurveto{\pgfqpoint{4.567163in}{2.713219in}}{\pgfqpoint{4.511889in}{2.948220in}}{\pgfqpoint{4.405702in}{3.161463in}}%
\pgfpathcurveto{\pgfqpoint{4.299515in}{3.374705in}}{\pgfqpoint{4.145283in}{3.560429in}}{\pgfqpoint{3.955175in}{3.703981in}}%
\pgfpathcurveto{\pgfqpoint{3.765067in}{3.847533in}}{\pgfqpoint{3.544218in}{3.945035in}}{\pgfqpoint{3.310053in}{3.988794in}}%
\pgfpathcurveto{\pgfqpoint{3.075889in}{4.032554in}}{\pgfqpoint{2.834733in}{4.021389in}}{\pgfqpoint{2.605613in}{3.956180in}}%
\pgfpathlineto{\pgfqpoint{3.027163in}{2.475000in}}%
\pgfpathlineto{\pgfqpoint{4.567163in}{2.475000in}}%
\pgfpathlineto{\pgfqpoint{4.567163in}{2.475000in}}%
\pgfpathclose%
\pgfusepath{stroke,fill}%
\end{pgfscope}%
\begin{pgfscope}%
\pgfsetbuttcap%
\pgfsetmiterjoin%
\definecolor{currentfill}{rgb}{1.000000,1.000000,0.701961}%
\pgfsetfillcolor{currentfill}%
\pgfsetlinewidth{1.003750pt}%
\definecolor{currentstroke}{rgb}{1.000000,1.000000,1.000000}%
\pgfsetstrokecolor{currentstroke}%
\pgfsetdash{}{0pt}%
\pgfpathmoveto{\pgfqpoint{2.605613in}{3.956180in}}%
\pgfpathcurveto{\pgfqpoint{2.376493in}{3.890972in}}{\pgfqpoint{2.165597in}{3.773481in}}{\pgfqpoint{1.989567in}{3.612978in}}%
\pgfpathcurveto{\pgfqpoint{1.813536in}{3.452475in}}{\pgfqpoint{1.677124in}{3.253294in}}{\pgfqpoint{1.591094in}{3.031153in}}%
\pgfpathcurveto{\pgfqpoint{1.505064in}{2.809011in}}{\pgfqpoint{1.471740in}{2.569908in}}{\pgfqpoint{1.493751in}{2.332708in}}%
\pgfpathcurveto{\pgfqpoint{1.515762in}{2.095508in}}{\pgfqpoint{1.592513in}{1.866620in}}{\pgfqpoint{1.717949in}{1.664101in}}%
\pgfpathlineto{\pgfqpoint{3.027163in}{2.475000in}}%
\pgfpathlineto{\pgfqpoint{2.605613in}{3.956180in}}%
\pgfpathlineto{\pgfqpoint{2.605613in}{3.956180in}}%
\pgfpathclose%
\pgfusepath{stroke,fill}%
\end{pgfscope}%
\begin{pgfscope}%
\pgfsetbuttcap%
\pgfsetmiterjoin%
\definecolor{currentfill}{rgb}{0.745098,0.729412,0.854902}%
\pgfsetfillcolor{currentfill}%
\pgfsetlinewidth{1.003750pt}%
\definecolor{currentstroke}{rgb}{1.000000,1.000000,1.000000}%
\pgfsetstrokecolor{currentstroke}%
\pgfsetdash{}{0pt}%
\pgfpathmoveto{\pgfqpoint{1.717949in}{1.664101in}}%
\pgfpathcurveto{\pgfqpoint{1.843385in}{1.461582in}}{\pgfqpoint{2.014117in}{1.290903in}}{\pgfqpoint{2.216676in}{1.165531in}}%
\pgfpathcurveto{\pgfqpoint{2.419234in}{1.040158in}}{\pgfqpoint{2.648147in}{0.963479in}}{\pgfqpoint{2.885354in}{0.941543in}}%
\pgfpathcurveto{\pgfqpoint{3.122560in}{0.919607in}}{\pgfqpoint{3.361653in}{0.953006in}}{\pgfqpoint{3.583768in}{1.039106in}}%
\pgfpathcurveto{\pgfqpoint{3.805882in}{1.125206in}}{\pgfqpoint{4.005020in}{1.261680in}}{\pgfqpoint{4.165467in}{1.437762in}}%
\pgfpathlineto{\pgfqpoint{3.027163in}{2.475000in}}%
\pgfpathlineto{\pgfqpoint{1.717949in}{1.664101in}}%
\pgfpathlineto{\pgfqpoint{1.717949in}{1.664101in}}%
\pgfpathclose%
\pgfusepath{stroke,fill}%
\end{pgfscope}%
\begin{pgfscope}%
\pgfsetbuttcap%
\pgfsetmiterjoin%
\definecolor{currentfill}{rgb}{0.984314,0.501961,0.447059}%
\pgfsetfillcolor{currentfill}%
\pgfsetlinewidth{1.003750pt}%
\definecolor{currentstroke}{rgb}{1.000000,1.000000,1.000000}%
\pgfsetstrokecolor{currentstroke}%
\pgfsetdash{}{0pt}%
\pgfpathmoveto{\pgfqpoint{4.165467in}{1.437762in}}%
\pgfpathcurveto{\pgfqpoint{4.165467in}{1.437762in}}{\pgfqpoint{4.165467in}{1.437762in}}{\pgfqpoint{4.165467in}{1.437762in}}%
\pgfpathlineto{\pgfqpoint{3.027163in}{2.475000in}}%
\pgfpathlineto{\pgfqpoint{4.165467in}{1.437762in}}%
\pgfpathlineto{\pgfqpoint{4.165467in}{1.437762in}}%
\pgfpathclose%
\pgfusepath{stroke,fill}%
\end{pgfscope}%
\begin{pgfscope}%
\pgfsetbuttcap%
\pgfsetmiterjoin%
\definecolor{currentfill}{rgb}{0.501961,0.694118,0.827451}%
\pgfsetfillcolor{currentfill}%
\pgfsetlinewidth{1.003750pt}%
\definecolor{currentstroke}{rgb}{1.000000,1.000000,1.000000}%
\pgfsetstrokecolor{currentstroke}%
\pgfsetdash{}{0pt}%
\pgfpathmoveto{\pgfqpoint{4.165467in}{1.437762in}}%
\pgfpathcurveto{\pgfqpoint{4.293579in}{1.578356in}}{\pgfqpoint{4.394541in}{1.741476in}}{\pgfqpoint{4.463232in}{1.918848in}}%
\pgfpathcurveto{\pgfqpoint{4.531924in}{2.096219in}}{\pgfqpoint{4.567163in}{2.284792in}}{\pgfqpoint{4.567163in}{2.475001in}}%
\pgfpathlineto{\pgfqpoint{3.027163in}{2.475000in}}%
\pgfpathlineto{\pgfqpoint{4.165467in}{1.437762in}}%
\pgfpathlineto{\pgfqpoint{4.165467in}{1.437762in}}%
\pgfpathclose%
\pgfusepath{stroke,fill}%
\end{pgfscope}%
\begin{pgfscope}%
\pgfsetbuttcap%
\pgfsetmiterjoin%
\definecolor{currentfill}{rgb}{0.992157,0.705882,0.384314}%
\pgfsetfillcolor{currentfill}%
\pgfsetlinewidth{1.003750pt}%
\definecolor{currentstroke}{rgb}{1.000000,1.000000,1.000000}%
\pgfsetstrokecolor{currentstroke}%
\pgfsetdash{}{0pt}%
\pgfpathmoveto{\pgfqpoint{4.567163in}{2.475001in}}%
\pgfpathcurveto{\pgfqpoint{4.567163in}{2.475001in}}{\pgfqpoint{4.567163in}{2.475001in}}{\pgfqpoint{4.567163in}{2.475001in}}%
\pgfpathlineto{\pgfqpoint{3.027163in}{2.475000in}}%
\pgfpathlineto{\pgfqpoint{4.567163in}{2.475001in}}%
\pgfpathlineto{\pgfqpoint{4.567163in}{2.475001in}}%
\pgfpathclose%
\pgfusepath{stroke,fill}%
\end{pgfscope}%
\begin{pgfscope}%
\definecolor{textcolor}{rgb}{0.150000,0.150000,0.150000}%
\pgfsetstrokecolor{textcolor}%
\pgfsetfillcolor{textcolor}%
\pgftext[x=3.583970in,y=3.212389in,,]{\color{textcolor}\sffamily\fontsize{12.000000}{14.400000}\selectfont 29.4\%}%
\end{pgfscope}%
\begin{pgfscope}%
\definecolor{textcolor}{rgb}{0.150000,0.150000,0.150000}%
\pgfsetstrokecolor{textcolor}%
\pgfsetfillcolor{textcolor}%
\pgftext[x=2.165522in,y=2.808692in,,]{\color{textcolor}\sffamily\fontsize{12.000000}{14.400000}\selectfont 29.4\%}%
\end{pgfscope}%
\begin{pgfscope}%
\definecolor{textcolor}{rgb}{0.150000,0.150000,0.150000}%
\pgfsetstrokecolor{textcolor}%
\pgfsetfillcolor{textcolor}%
\pgftext[x=2.942078in,y=1.554926in,,]{\color{textcolor}\sffamily\fontsize{12.000000}{14.400000}\selectfont 29.4\%}%
\end{pgfscope}%
\begin{pgfscope}%
\definecolor{textcolor}{rgb}{0.150000,0.150000,0.150000}%
\pgfsetstrokecolor{textcolor}%
\pgfsetfillcolor{textcolor}%
\pgftext[x=3.710146in,y=1.852657in,,]{\color{textcolor}\sffamily\fontsize{12.000000}{14.400000}\selectfont 0.0\%}%
\end{pgfscope}%
\begin{pgfscope}%
\definecolor{textcolor}{rgb}{0.150000,0.150000,0.150000}%
\pgfsetstrokecolor{textcolor}%
\pgfsetfillcolor{textcolor}%
\pgftext[x=3.888805in,y=2.141309in,,]{\color{textcolor}\sffamily\fontsize{12.000000}{14.400000}\selectfont 11.8\%}%
\end{pgfscope}%
\begin{pgfscope}%
\definecolor{textcolor}{rgb}{0.150000,0.150000,0.150000}%
\pgfsetstrokecolor{textcolor}%
\pgfsetfillcolor{textcolor}%
\pgftext[x=3.951163in,y=2.475000in,,]{\color{textcolor}\sffamily\fontsize{12.000000}{14.400000}\selectfont 0.0\%}%
\end{pgfscope}%
\begin{pgfscope}%
\pgfsetbuttcap%
\pgfsetmiterjoin%
\definecolor{currentfill}{rgb}{1.000000,1.000000,1.000000}%
\pgfsetfillcolor{currentfill}%
\pgfsetfillopacity{0.800000}%
\pgfsetlinewidth{1.003750pt}%
\definecolor{currentstroke}{rgb}{0.800000,0.800000,0.800000}%
\pgfsetstrokecolor{currentstroke}%
\pgfsetstrokeopacity{0.800000}%
\pgfsetdash{}{0pt}%
\pgfpathmoveto{\pgfqpoint{3.271524in}{0.458500in}}%
\pgfpathlineto{\pgfqpoint{5.827163in}{0.458500in}}%
\pgfpathquadraticcurveto{\pgfqpoint{5.852163in}{0.458500in}}{\pgfqpoint{5.852163in}{0.483500in}}%
\pgfpathlineto{\pgfqpoint{5.852163in}{1.571829in}}%
\pgfpathquadraticcurveto{\pgfqpoint{5.852163in}{1.596829in}}{\pgfqpoint{5.827163in}{1.596829in}}%
\pgfpathlineto{\pgfqpoint{3.271524in}{1.596829in}}%
\pgfpathquadraticcurveto{\pgfqpoint{3.246524in}{1.596829in}}{\pgfqpoint{3.246524in}{1.571829in}}%
\pgfpathlineto{\pgfqpoint{3.246524in}{0.483500in}}%
\pgfpathquadraticcurveto{\pgfqpoint{3.246524in}{0.458500in}}{\pgfqpoint{3.271524in}{0.458500in}}%
\pgfpathlineto{\pgfqpoint{3.271524in}{0.458500in}}%
\pgfpathclose%
\pgfusepath{stroke,fill}%
\end{pgfscope}%
\begin{pgfscope}%
\pgfsetbuttcap%
\pgfsetmiterjoin%
\definecolor{currentfill}{rgb}{0.552941,0.827451,0.780392}%
\pgfsetfillcolor{currentfill}%
\pgfsetlinewidth{1.003750pt}%
\definecolor{currentstroke}{rgb}{1.000000,1.000000,1.000000}%
\pgfsetstrokecolor{currentstroke}%
\pgfsetdash{}{0pt}%
\pgfpathmoveto{\pgfqpoint{3.296524in}{1.451858in}}%
\pgfpathlineto{\pgfqpoint{3.546524in}{1.451858in}}%
\pgfpathlineto{\pgfqpoint{3.546524in}{1.539358in}}%
\pgfpathlineto{\pgfqpoint{3.296524in}{1.539358in}}%
\pgfpathlineto{\pgfqpoint{3.296524in}{1.451858in}}%
\pgfpathclose%
\pgfusepath{stroke,fill}%
\end{pgfscope}%
\begin{pgfscope}%
\definecolor{textcolor}{rgb}{0.150000,0.150000,0.150000}%
\pgfsetstrokecolor{textcolor}%
\pgfsetfillcolor{textcolor}%
\pgftext[x=3.646524in,y=1.451858in,left,base]{\color{textcolor}\sffamily\fontsize{9.000000}{10.800000}\selectfont No CS Degree, 29.4 \%}%
\end{pgfscope}%
\begin{pgfscope}%
\pgfsetbuttcap%
\pgfsetmiterjoin%
\definecolor{currentfill}{rgb}{1.000000,1.000000,0.701961}%
\pgfsetfillcolor{currentfill}%
\pgfsetlinewidth{1.003750pt}%
\definecolor{currentstroke}{rgb}{1.000000,1.000000,1.000000}%
\pgfsetstrokecolor{currentstroke}%
\pgfsetdash{}{0pt}%
\pgfpathmoveto{\pgfqpoint{3.296524in}{1.268387in}}%
\pgfpathlineto{\pgfqpoint{3.546524in}{1.268387in}}%
\pgfpathlineto{\pgfqpoint{3.546524in}{1.355887in}}%
\pgfpathlineto{\pgfqpoint{3.296524in}{1.355887in}}%
\pgfpathlineto{\pgfqpoint{3.296524in}{1.268387in}}%
\pgfpathclose%
\pgfusepath{stroke,fill}%
\end{pgfscope}%
\begin{pgfscope}%
\definecolor{textcolor}{rgb}{0.150000,0.150000,0.150000}%
\pgfsetstrokecolor{textcolor}%
\pgfsetfillcolor{textcolor}%
\pgftext[x=3.646524in,y=1.268387in,left,base]{\color{textcolor}\sffamily\fontsize{9.000000}{10.800000}\selectfont Bachelors's Degree, 29.4 \%}%
\end{pgfscope}%
\begin{pgfscope}%
\pgfsetbuttcap%
\pgfsetmiterjoin%
\definecolor{currentfill}{rgb}{0.745098,0.729412,0.854902}%
\pgfsetfillcolor{currentfill}%
\pgfsetlinewidth{1.003750pt}%
\definecolor{currentstroke}{rgb}{1.000000,1.000000,1.000000}%
\pgfsetstrokecolor{currentstroke}%
\pgfsetdash{}{0pt}%
\pgfpathmoveto{\pgfqpoint{3.296524in}{1.084915in}}%
\pgfpathlineto{\pgfqpoint{3.546524in}{1.084915in}}%
\pgfpathlineto{\pgfqpoint{3.546524in}{1.172415in}}%
\pgfpathlineto{\pgfqpoint{3.296524in}{1.172415in}}%
\pgfpathlineto{\pgfqpoint{3.296524in}{1.084915in}}%
\pgfpathclose%
\pgfusepath{stroke,fill}%
\end{pgfscope}%
\begin{pgfscope}%
\definecolor{textcolor}{rgb}{0.150000,0.150000,0.150000}%
\pgfsetstrokecolor{textcolor}%
\pgfsetfillcolor{textcolor}%
\pgftext[x=3.646524in,y=1.084915in,left,base]{\color{textcolor}\sffamily\fontsize{9.000000}{10.800000}\selectfont Master's Degree, 29.4 \%}%
\end{pgfscope}%
\begin{pgfscope}%
\pgfsetbuttcap%
\pgfsetmiterjoin%
\definecolor{currentfill}{rgb}{0.984314,0.501961,0.447059}%
\pgfsetfillcolor{currentfill}%
\pgfsetlinewidth{1.003750pt}%
\definecolor{currentstroke}{rgb}{1.000000,1.000000,1.000000}%
\pgfsetstrokecolor{currentstroke}%
\pgfsetdash{}{0pt}%
\pgfpathmoveto{\pgfqpoint{3.296524in}{0.901444in}}%
\pgfpathlineto{\pgfqpoint{3.546524in}{0.901444in}}%
\pgfpathlineto{\pgfqpoint{3.546524in}{0.988944in}}%
\pgfpathlineto{\pgfqpoint{3.296524in}{0.988944in}}%
\pgfpathlineto{\pgfqpoint{3.296524in}{0.901444in}}%
\pgfpathclose%
\pgfusepath{stroke,fill}%
\end{pgfscope}%
\begin{pgfscope}%
\definecolor{textcolor}{rgb}{0.150000,0.150000,0.150000}%
\pgfsetstrokecolor{textcolor}%
\pgfsetfillcolor{textcolor}%
\pgftext[x=3.646524in,y=0.901444in,left,base]{\color{textcolor}\sffamily\fontsize{9.000000}{10.800000}\selectfont Doctorate Degree, 0.0 \%}%
\end{pgfscope}%
\begin{pgfscope}%
\pgfsetbuttcap%
\pgfsetmiterjoin%
\definecolor{currentfill}{rgb}{0.501961,0.694118,0.827451}%
\pgfsetfillcolor{currentfill}%
\pgfsetlinewidth{1.003750pt}%
\definecolor{currentstroke}{rgb}{1.000000,1.000000,1.000000}%
\pgfsetstrokecolor{currentstroke}%
\pgfsetdash{}{0pt}%
\pgfpathmoveto{\pgfqpoint{3.296524in}{0.717972in}}%
\pgfpathlineto{\pgfqpoint{3.546524in}{0.717972in}}%
\pgfpathlineto{\pgfqpoint{3.546524in}{0.805472in}}%
\pgfpathlineto{\pgfqpoint{3.296524in}{0.805472in}}%
\pgfpathlineto{\pgfqpoint{3.296524in}{0.717972in}}%
\pgfpathclose%
\pgfusepath{stroke,fill}%
\end{pgfscope}%
\begin{pgfscope}%
\definecolor{textcolor}{rgb}{0.150000,0.150000,0.150000}%
\pgfsetstrokecolor{textcolor}%
\pgfsetfillcolor{textcolor}%
\pgftext[x=3.646524in,y=0.717972in,left,base]{\color{textcolor}\sffamily\fontsize{9.000000}{10.800000}\selectfont Other Engineering Degree, 11.8 \%}%
\end{pgfscope}%
\begin{pgfscope}%
\pgfsetbuttcap%
\pgfsetmiterjoin%
\definecolor{currentfill}{rgb}{0.992157,0.705882,0.384314}%
\pgfsetfillcolor{currentfill}%
\pgfsetlinewidth{1.003750pt}%
\definecolor{currentstroke}{rgb}{1.000000,1.000000,1.000000}%
\pgfsetstrokecolor{currentstroke}%
\pgfsetdash{}{0pt}%
\pgfpathmoveto{\pgfqpoint{3.296524in}{0.534501in}}%
\pgfpathlineto{\pgfqpoint{3.546524in}{0.534501in}}%
\pgfpathlineto{\pgfqpoint{3.546524in}{0.622001in}}%
\pgfpathlineto{\pgfqpoint{3.296524in}{0.622001in}}%
\pgfpathlineto{\pgfqpoint{3.296524in}{0.534501in}}%
\pgfpathclose%
\pgfusepath{stroke,fill}%
\end{pgfscope}%
\begin{pgfscope}%
\definecolor{textcolor}{rgb}{0.150000,0.150000,0.150000}%
\pgfsetstrokecolor{textcolor}%
\pgfsetfillcolor{textcolor}%
\pgftext[x=3.646524in,y=0.534501in,left,base]{\color{textcolor}\sffamily\fontsize{9.000000}{10.800000}\selectfont Yes but did not finish, 0.0 \%}%
\end{pgfscope}%
\end{pgfpicture}%
\makeatother%
\endgroup%
}
	\caption{spam}
	\label{fig:uniexp}
\end{figure}

\begin{figure}[H]
	\scalebox{0.72}{%% Creator: Matplotlib, PGF backend
%%
%% To include the figure in your LaTeX document, write
%%   \input{<filename>.pgf}
%%
%% Make sure the required packages are loaded in your preamble
%%   \usepackage{pgf}
%%
%% Also ensure that all the required font packages are loaded; for instance,
%% the lmodern package is sometimes necessary when using math font.
%%   \usepackage{lmodern}
%%
%% Figures using additional raster images can only be included by \input if
%% they are in the same directory as the main LaTeX file. For loading figures
%% from other directories you can use the `import` package
%%   \usepackage{import}
%%
%% and then include the figures with
%%   \import{<path to file>}{<filename>.pgf}
%%
%% Matplotlib used the following preamble
%%   \usepackage{fontspec}
%%   \setmainfont{DejaVuSerif.ttf}[Path=\detokenize{/home/spam/miniconda3/envs/mpl/lib/python3.10/site-packages/matplotlib/mpl-data/fonts/ttf/}]
%%   \setsansfont{DejaVuSans.ttf}[Path=\detokenize{/home/spam/miniconda3/envs/mpl/lib/python3.10/site-packages/matplotlib/mpl-data/fonts/ttf/}]
%%   \setmonofont{DejaVuSansMono.ttf}[Path=\detokenize{/home/spam/miniconda3/envs/mpl/lib/python3.10/site-packages/matplotlib/mpl-data/fonts/ttf/}]
%%
\begingroup%
\makeatletter%
\begin{pgfpicture}%
\pgfpathrectangle{\pgfpointorigin}{\pgfqpoint{5.906660in}{5.000000in}}%
\pgfusepath{use as bounding box, clip}%
\begin{pgfscope}%
\pgfsetbuttcap%
\pgfsetmiterjoin%
\definecolor{currentfill}{rgb}{1.000000,1.000000,1.000000}%
\pgfsetfillcolor{currentfill}%
\pgfsetlinewidth{0.000000pt}%
\definecolor{currentstroke}{rgb}{1.000000,1.000000,1.000000}%
\pgfsetstrokecolor{currentstroke}%
\pgfsetdash{}{0pt}%
\pgfpathmoveto{\pgfqpoint{0.000000in}{0.000000in}}%
\pgfpathlineto{\pgfqpoint{5.906660in}{0.000000in}}%
\pgfpathlineto{\pgfqpoint{5.906660in}{5.000000in}}%
\pgfpathlineto{\pgfqpoint{0.000000in}{5.000000in}}%
\pgfpathlineto{\pgfqpoint{0.000000in}{0.000000in}}%
\pgfpathclose%
\pgfusepath{fill}%
\end{pgfscope}%
\begin{pgfscope}%
\definecolor{textcolor}{rgb}{0.150000,0.150000,0.150000}%
\pgfsetstrokecolor{textcolor}%
\pgfsetfillcolor{textcolor}%
\pgftext[x=3.027163in,y=0.411111in,,top]{\color{textcolor}\sffamily\fontsize{12.000000}{14.400000}\selectfont Computer Science UniversityUniverity Experience}%
\end{pgfscope}%
\begin{pgfscope}%
\pgfsetbuttcap%
\pgfsetmiterjoin%
\definecolor{currentfill}{rgb}{0.552941,0.827451,0.780392}%
\pgfsetfillcolor{currentfill}%
\pgfsetlinewidth{1.003750pt}%
\definecolor{currentstroke}{rgb}{1.000000,1.000000,1.000000}%
\pgfsetstrokecolor{currentstroke}%
\pgfsetdash{}{0pt}%
\pgfpathmoveto{\pgfqpoint{4.567163in}{2.475000in}}%
\pgfpathcurveto{\pgfqpoint{4.567163in}{2.713219in}}{\pgfqpoint{4.511889in}{2.948220in}}{\pgfqpoint{4.405702in}{3.161463in}}%
\pgfpathcurveto{\pgfqpoint{4.299515in}{3.374705in}}{\pgfqpoint{4.145283in}{3.560429in}}{\pgfqpoint{3.955175in}{3.703981in}}%
\pgfpathcurveto{\pgfqpoint{3.765067in}{3.847533in}}{\pgfqpoint{3.544218in}{3.945035in}}{\pgfqpoint{3.310053in}{3.988794in}}%
\pgfpathcurveto{\pgfqpoint{3.075889in}{4.032554in}}{\pgfqpoint{2.834733in}{4.021389in}}{\pgfqpoint{2.605613in}{3.956180in}}%
\pgfpathlineto{\pgfqpoint{3.027163in}{2.475000in}}%
\pgfpathlineto{\pgfqpoint{4.567163in}{2.475000in}}%
\pgfpathlineto{\pgfqpoint{4.567163in}{2.475000in}}%
\pgfpathclose%
\pgfusepath{stroke,fill}%
\end{pgfscope}%
\begin{pgfscope}%
\pgfsetbuttcap%
\pgfsetmiterjoin%
\definecolor{currentfill}{rgb}{1.000000,1.000000,0.701961}%
\pgfsetfillcolor{currentfill}%
\pgfsetlinewidth{1.003750pt}%
\definecolor{currentstroke}{rgb}{1.000000,1.000000,1.000000}%
\pgfsetstrokecolor{currentstroke}%
\pgfsetdash{}{0pt}%
\pgfpathmoveto{\pgfqpoint{2.605613in}{3.956180in}}%
\pgfpathcurveto{\pgfqpoint{2.376493in}{3.890972in}}{\pgfqpoint{2.165597in}{3.773481in}}{\pgfqpoint{1.989567in}{3.612978in}}%
\pgfpathcurveto{\pgfqpoint{1.813536in}{3.452475in}}{\pgfqpoint{1.677124in}{3.253294in}}{\pgfqpoint{1.591094in}{3.031153in}}%
\pgfpathcurveto{\pgfqpoint{1.505064in}{2.809011in}}{\pgfqpoint{1.471740in}{2.569908in}}{\pgfqpoint{1.493751in}{2.332708in}}%
\pgfpathcurveto{\pgfqpoint{1.515762in}{2.095508in}}{\pgfqpoint{1.592513in}{1.866620in}}{\pgfqpoint{1.717949in}{1.664101in}}%
\pgfpathlineto{\pgfqpoint{3.027163in}{2.475000in}}%
\pgfpathlineto{\pgfqpoint{2.605613in}{3.956180in}}%
\pgfpathlineto{\pgfqpoint{2.605613in}{3.956180in}}%
\pgfpathclose%
\pgfusepath{stroke,fill}%
\end{pgfscope}%
\begin{pgfscope}%
\pgfsetbuttcap%
\pgfsetmiterjoin%
\definecolor{currentfill}{rgb}{0.745098,0.729412,0.854902}%
\pgfsetfillcolor{currentfill}%
\pgfsetlinewidth{1.003750pt}%
\definecolor{currentstroke}{rgb}{1.000000,1.000000,1.000000}%
\pgfsetstrokecolor{currentstroke}%
\pgfsetdash{}{0pt}%
\pgfpathmoveto{\pgfqpoint{1.717949in}{1.664101in}}%
\pgfpathcurveto{\pgfqpoint{1.843385in}{1.461582in}}{\pgfqpoint{2.014117in}{1.290903in}}{\pgfqpoint{2.216676in}{1.165531in}}%
\pgfpathcurveto{\pgfqpoint{2.419234in}{1.040158in}}{\pgfqpoint{2.648147in}{0.963479in}}{\pgfqpoint{2.885354in}{0.941543in}}%
\pgfpathcurveto{\pgfqpoint{3.122560in}{0.919607in}}{\pgfqpoint{3.361653in}{0.953006in}}{\pgfqpoint{3.583768in}{1.039106in}}%
\pgfpathcurveto{\pgfqpoint{3.805882in}{1.125206in}}{\pgfqpoint{4.005020in}{1.261680in}}{\pgfqpoint{4.165467in}{1.437762in}}%
\pgfpathlineto{\pgfqpoint{3.027163in}{2.475000in}}%
\pgfpathlineto{\pgfqpoint{1.717949in}{1.664101in}}%
\pgfpathlineto{\pgfqpoint{1.717949in}{1.664101in}}%
\pgfpathclose%
\pgfusepath{stroke,fill}%
\end{pgfscope}%
\begin{pgfscope}%
\pgfsetbuttcap%
\pgfsetmiterjoin%
\definecolor{currentfill}{rgb}{0.984314,0.501961,0.447059}%
\pgfsetfillcolor{currentfill}%
\pgfsetlinewidth{1.003750pt}%
\definecolor{currentstroke}{rgb}{1.000000,1.000000,1.000000}%
\pgfsetstrokecolor{currentstroke}%
\pgfsetdash{}{0pt}%
\pgfpathmoveto{\pgfqpoint{4.165467in}{1.437762in}}%
\pgfpathcurveto{\pgfqpoint{4.165467in}{1.437762in}}{\pgfqpoint{4.165467in}{1.437762in}}{\pgfqpoint{4.165467in}{1.437762in}}%
\pgfpathlineto{\pgfqpoint{3.027163in}{2.475000in}}%
\pgfpathlineto{\pgfqpoint{4.165467in}{1.437762in}}%
\pgfpathlineto{\pgfqpoint{4.165467in}{1.437762in}}%
\pgfpathclose%
\pgfusepath{stroke,fill}%
\end{pgfscope}%
\begin{pgfscope}%
\pgfsetbuttcap%
\pgfsetmiterjoin%
\definecolor{currentfill}{rgb}{0.501961,0.694118,0.827451}%
\pgfsetfillcolor{currentfill}%
\pgfsetlinewidth{1.003750pt}%
\definecolor{currentstroke}{rgb}{1.000000,1.000000,1.000000}%
\pgfsetstrokecolor{currentstroke}%
\pgfsetdash{}{0pt}%
\pgfpathmoveto{\pgfqpoint{4.165467in}{1.437762in}}%
\pgfpathcurveto{\pgfqpoint{4.293579in}{1.578356in}}{\pgfqpoint{4.394541in}{1.741476in}}{\pgfqpoint{4.463232in}{1.918848in}}%
\pgfpathcurveto{\pgfqpoint{4.531924in}{2.096219in}}{\pgfqpoint{4.567163in}{2.284792in}}{\pgfqpoint{4.567163in}{2.475001in}}%
\pgfpathlineto{\pgfqpoint{3.027163in}{2.475000in}}%
\pgfpathlineto{\pgfqpoint{4.165467in}{1.437762in}}%
\pgfpathlineto{\pgfqpoint{4.165467in}{1.437762in}}%
\pgfpathclose%
\pgfusepath{stroke,fill}%
\end{pgfscope}%
\begin{pgfscope}%
\pgfsetbuttcap%
\pgfsetmiterjoin%
\definecolor{currentfill}{rgb}{0.992157,0.705882,0.384314}%
\pgfsetfillcolor{currentfill}%
\pgfsetlinewidth{1.003750pt}%
\definecolor{currentstroke}{rgb}{1.000000,1.000000,1.000000}%
\pgfsetstrokecolor{currentstroke}%
\pgfsetdash{}{0pt}%
\pgfpathmoveto{\pgfqpoint{4.567163in}{2.475001in}}%
\pgfpathcurveto{\pgfqpoint{4.567163in}{2.475001in}}{\pgfqpoint{4.567163in}{2.475001in}}{\pgfqpoint{4.567163in}{2.475001in}}%
\pgfpathlineto{\pgfqpoint{3.027163in}{2.475000in}}%
\pgfpathlineto{\pgfqpoint{4.567163in}{2.475001in}}%
\pgfpathlineto{\pgfqpoint{4.567163in}{2.475001in}}%
\pgfpathclose%
\pgfusepath{stroke,fill}%
\end{pgfscope}%
\begin{pgfscope}%
\definecolor{textcolor}{rgb}{0.150000,0.150000,0.150000}%
\pgfsetstrokecolor{textcolor}%
\pgfsetfillcolor{textcolor}%
\pgftext[x=3.583970in,y=3.212389in,,]{\color{textcolor}\sffamily\fontsize{12.000000}{14.400000}\selectfont 29.4\%}%
\end{pgfscope}%
\begin{pgfscope}%
\definecolor{textcolor}{rgb}{0.150000,0.150000,0.150000}%
\pgfsetstrokecolor{textcolor}%
\pgfsetfillcolor{textcolor}%
\pgftext[x=2.165522in,y=2.808692in,,]{\color{textcolor}\sffamily\fontsize{12.000000}{14.400000}\selectfont 29.4\%}%
\end{pgfscope}%
\begin{pgfscope}%
\definecolor{textcolor}{rgb}{0.150000,0.150000,0.150000}%
\pgfsetstrokecolor{textcolor}%
\pgfsetfillcolor{textcolor}%
\pgftext[x=2.942078in,y=1.554926in,,]{\color{textcolor}\sffamily\fontsize{12.000000}{14.400000}\selectfont 29.4\%}%
\end{pgfscope}%
\begin{pgfscope}%
\definecolor{textcolor}{rgb}{0.150000,0.150000,0.150000}%
\pgfsetstrokecolor{textcolor}%
\pgfsetfillcolor{textcolor}%
\pgftext[x=3.710146in,y=1.852657in,,]{\color{textcolor}\sffamily\fontsize{12.000000}{14.400000}\selectfont 0.0\%}%
\end{pgfscope}%
\begin{pgfscope}%
\definecolor{textcolor}{rgb}{0.150000,0.150000,0.150000}%
\pgfsetstrokecolor{textcolor}%
\pgfsetfillcolor{textcolor}%
\pgftext[x=3.888805in,y=2.141309in,,]{\color{textcolor}\sffamily\fontsize{12.000000}{14.400000}\selectfont 11.8\%}%
\end{pgfscope}%
\begin{pgfscope}%
\definecolor{textcolor}{rgb}{0.150000,0.150000,0.150000}%
\pgfsetstrokecolor{textcolor}%
\pgfsetfillcolor{textcolor}%
\pgftext[x=3.951163in,y=2.475000in,,]{\color{textcolor}\sffamily\fontsize{12.000000}{14.400000}\selectfont 0.0\%}%
\end{pgfscope}%
\begin{pgfscope}%
\pgfsetbuttcap%
\pgfsetmiterjoin%
\definecolor{currentfill}{rgb}{1.000000,1.000000,1.000000}%
\pgfsetfillcolor{currentfill}%
\pgfsetfillopacity{0.800000}%
\pgfsetlinewidth{1.003750pt}%
\definecolor{currentstroke}{rgb}{0.800000,0.800000,0.800000}%
\pgfsetstrokecolor{currentstroke}%
\pgfsetstrokeopacity{0.800000}%
\pgfsetdash{}{0pt}%
\pgfpathmoveto{\pgfqpoint{3.271524in}{0.458500in}}%
\pgfpathlineto{\pgfqpoint{5.827163in}{0.458500in}}%
\pgfpathquadraticcurveto{\pgfqpoint{5.852163in}{0.458500in}}{\pgfqpoint{5.852163in}{0.483500in}}%
\pgfpathlineto{\pgfqpoint{5.852163in}{1.571829in}}%
\pgfpathquadraticcurveto{\pgfqpoint{5.852163in}{1.596829in}}{\pgfqpoint{5.827163in}{1.596829in}}%
\pgfpathlineto{\pgfqpoint{3.271524in}{1.596829in}}%
\pgfpathquadraticcurveto{\pgfqpoint{3.246524in}{1.596829in}}{\pgfqpoint{3.246524in}{1.571829in}}%
\pgfpathlineto{\pgfqpoint{3.246524in}{0.483500in}}%
\pgfpathquadraticcurveto{\pgfqpoint{3.246524in}{0.458500in}}{\pgfqpoint{3.271524in}{0.458500in}}%
\pgfpathlineto{\pgfqpoint{3.271524in}{0.458500in}}%
\pgfpathclose%
\pgfusepath{stroke,fill}%
\end{pgfscope}%
\begin{pgfscope}%
\pgfsetbuttcap%
\pgfsetmiterjoin%
\definecolor{currentfill}{rgb}{0.552941,0.827451,0.780392}%
\pgfsetfillcolor{currentfill}%
\pgfsetlinewidth{1.003750pt}%
\definecolor{currentstroke}{rgb}{1.000000,1.000000,1.000000}%
\pgfsetstrokecolor{currentstroke}%
\pgfsetdash{}{0pt}%
\pgfpathmoveto{\pgfqpoint{3.296524in}{1.451858in}}%
\pgfpathlineto{\pgfqpoint{3.546524in}{1.451858in}}%
\pgfpathlineto{\pgfqpoint{3.546524in}{1.539358in}}%
\pgfpathlineto{\pgfqpoint{3.296524in}{1.539358in}}%
\pgfpathlineto{\pgfqpoint{3.296524in}{1.451858in}}%
\pgfpathclose%
\pgfusepath{stroke,fill}%
\end{pgfscope}%
\begin{pgfscope}%
\definecolor{textcolor}{rgb}{0.150000,0.150000,0.150000}%
\pgfsetstrokecolor{textcolor}%
\pgfsetfillcolor{textcolor}%
\pgftext[x=3.646524in,y=1.451858in,left,base]{\color{textcolor}\sffamily\fontsize{9.000000}{10.800000}\selectfont No CS Degree, 29.4 \%}%
\end{pgfscope}%
\begin{pgfscope}%
\pgfsetbuttcap%
\pgfsetmiterjoin%
\definecolor{currentfill}{rgb}{1.000000,1.000000,0.701961}%
\pgfsetfillcolor{currentfill}%
\pgfsetlinewidth{1.003750pt}%
\definecolor{currentstroke}{rgb}{1.000000,1.000000,1.000000}%
\pgfsetstrokecolor{currentstroke}%
\pgfsetdash{}{0pt}%
\pgfpathmoveto{\pgfqpoint{3.296524in}{1.268387in}}%
\pgfpathlineto{\pgfqpoint{3.546524in}{1.268387in}}%
\pgfpathlineto{\pgfqpoint{3.546524in}{1.355887in}}%
\pgfpathlineto{\pgfqpoint{3.296524in}{1.355887in}}%
\pgfpathlineto{\pgfqpoint{3.296524in}{1.268387in}}%
\pgfpathclose%
\pgfusepath{stroke,fill}%
\end{pgfscope}%
\begin{pgfscope}%
\definecolor{textcolor}{rgb}{0.150000,0.150000,0.150000}%
\pgfsetstrokecolor{textcolor}%
\pgfsetfillcolor{textcolor}%
\pgftext[x=3.646524in,y=1.268387in,left,base]{\color{textcolor}\sffamily\fontsize{9.000000}{10.800000}\selectfont Bachelors's Degree, 29.4 \%}%
\end{pgfscope}%
\begin{pgfscope}%
\pgfsetbuttcap%
\pgfsetmiterjoin%
\definecolor{currentfill}{rgb}{0.745098,0.729412,0.854902}%
\pgfsetfillcolor{currentfill}%
\pgfsetlinewidth{1.003750pt}%
\definecolor{currentstroke}{rgb}{1.000000,1.000000,1.000000}%
\pgfsetstrokecolor{currentstroke}%
\pgfsetdash{}{0pt}%
\pgfpathmoveto{\pgfqpoint{3.296524in}{1.084915in}}%
\pgfpathlineto{\pgfqpoint{3.546524in}{1.084915in}}%
\pgfpathlineto{\pgfqpoint{3.546524in}{1.172415in}}%
\pgfpathlineto{\pgfqpoint{3.296524in}{1.172415in}}%
\pgfpathlineto{\pgfqpoint{3.296524in}{1.084915in}}%
\pgfpathclose%
\pgfusepath{stroke,fill}%
\end{pgfscope}%
\begin{pgfscope}%
\definecolor{textcolor}{rgb}{0.150000,0.150000,0.150000}%
\pgfsetstrokecolor{textcolor}%
\pgfsetfillcolor{textcolor}%
\pgftext[x=3.646524in,y=1.084915in,left,base]{\color{textcolor}\sffamily\fontsize{9.000000}{10.800000}\selectfont Master's Degree, 29.4 \%}%
\end{pgfscope}%
\begin{pgfscope}%
\pgfsetbuttcap%
\pgfsetmiterjoin%
\definecolor{currentfill}{rgb}{0.984314,0.501961,0.447059}%
\pgfsetfillcolor{currentfill}%
\pgfsetlinewidth{1.003750pt}%
\definecolor{currentstroke}{rgb}{1.000000,1.000000,1.000000}%
\pgfsetstrokecolor{currentstroke}%
\pgfsetdash{}{0pt}%
\pgfpathmoveto{\pgfqpoint{3.296524in}{0.901444in}}%
\pgfpathlineto{\pgfqpoint{3.546524in}{0.901444in}}%
\pgfpathlineto{\pgfqpoint{3.546524in}{0.988944in}}%
\pgfpathlineto{\pgfqpoint{3.296524in}{0.988944in}}%
\pgfpathlineto{\pgfqpoint{3.296524in}{0.901444in}}%
\pgfpathclose%
\pgfusepath{stroke,fill}%
\end{pgfscope}%
\begin{pgfscope}%
\definecolor{textcolor}{rgb}{0.150000,0.150000,0.150000}%
\pgfsetstrokecolor{textcolor}%
\pgfsetfillcolor{textcolor}%
\pgftext[x=3.646524in,y=0.901444in,left,base]{\color{textcolor}\sffamily\fontsize{9.000000}{10.800000}\selectfont Doctorate Degree, 0.0 \%}%
\end{pgfscope}%
\begin{pgfscope}%
\pgfsetbuttcap%
\pgfsetmiterjoin%
\definecolor{currentfill}{rgb}{0.501961,0.694118,0.827451}%
\pgfsetfillcolor{currentfill}%
\pgfsetlinewidth{1.003750pt}%
\definecolor{currentstroke}{rgb}{1.000000,1.000000,1.000000}%
\pgfsetstrokecolor{currentstroke}%
\pgfsetdash{}{0pt}%
\pgfpathmoveto{\pgfqpoint{3.296524in}{0.717972in}}%
\pgfpathlineto{\pgfqpoint{3.546524in}{0.717972in}}%
\pgfpathlineto{\pgfqpoint{3.546524in}{0.805472in}}%
\pgfpathlineto{\pgfqpoint{3.296524in}{0.805472in}}%
\pgfpathlineto{\pgfqpoint{3.296524in}{0.717972in}}%
\pgfpathclose%
\pgfusepath{stroke,fill}%
\end{pgfscope}%
\begin{pgfscope}%
\definecolor{textcolor}{rgb}{0.150000,0.150000,0.150000}%
\pgfsetstrokecolor{textcolor}%
\pgfsetfillcolor{textcolor}%
\pgftext[x=3.646524in,y=0.717972in,left,base]{\color{textcolor}\sffamily\fontsize{9.000000}{10.800000}\selectfont Other Engineering Degree, 11.8 \%}%
\end{pgfscope}%
\begin{pgfscope}%
\pgfsetbuttcap%
\pgfsetmiterjoin%
\definecolor{currentfill}{rgb}{0.992157,0.705882,0.384314}%
\pgfsetfillcolor{currentfill}%
\pgfsetlinewidth{1.003750pt}%
\definecolor{currentstroke}{rgb}{1.000000,1.000000,1.000000}%
\pgfsetstrokecolor{currentstroke}%
\pgfsetdash{}{0pt}%
\pgfpathmoveto{\pgfqpoint{3.296524in}{0.534501in}}%
\pgfpathlineto{\pgfqpoint{3.546524in}{0.534501in}}%
\pgfpathlineto{\pgfqpoint{3.546524in}{0.622001in}}%
\pgfpathlineto{\pgfqpoint{3.296524in}{0.622001in}}%
\pgfpathlineto{\pgfqpoint{3.296524in}{0.534501in}}%
\pgfpathclose%
\pgfusepath{stroke,fill}%
\end{pgfscope}%
\begin{pgfscope}%
\definecolor{textcolor}{rgb}{0.150000,0.150000,0.150000}%
\pgfsetstrokecolor{textcolor}%
\pgfsetfillcolor{textcolor}%
\pgftext[x=3.646524in,y=0.534501in,left,base]{\color{textcolor}\sffamily\fontsize{9.000000}{10.800000}\selectfont Yes but did not finish, 0.0 \%}%
\end{pgfscope}%
\end{pgfpicture}%
\makeatother%
\endgroup%
}
	\caption{more and more}
	\label{fig:moreandmore}
\end{figure}

\begin{figure}[H]
	\centering
	\scalebox{0.72}{%% Creator: Matplotlib, PGF backend
%%
%% To include the figure in your LaTeX document, write
%%   \input{<filename>.pgf}
%%
%% Make sure the required packages are loaded in your preamble
%%   \usepackage{pgf}
%%
%% Also ensure that all the required font packages are loaded; for instance,
%% the lmodern package is sometimes necessary when using math font.
%%   \usepackage{lmodern}
%%
%% Figures using additional raster images can only be included by \input if
%% they are in the same directory as the main LaTeX file. For loading figures
%% from other directories you can use the `import` package
%%   \usepackage{import}
%%
%% and then include the figures with
%%   \import{<path to file>}{<filename>.pgf}
%%
%% Matplotlib used the following preamble
%%   \usepackage{fontspec}
%%   \setmainfont{DejaVuSerif.ttf}[Path=\detokenize{/home/spam/miniconda3/envs/mpl/lib/python3.10/site-packages/matplotlib/mpl-data/fonts/ttf/}]
%%   \setsansfont{DejaVuSans.ttf}[Path=\detokenize{/home/spam/miniconda3/envs/mpl/lib/python3.10/site-packages/matplotlib/mpl-data/fonts/ttf/}]
%%   \setmonofont{DejaVuSansMono.ttf}[Path=\detokenize{/home/spam/miniconda3/envs/mpl/lib/python3.10/site-packages/matplotlib/mpl-data/fonts/ttf/}]
%%
\begingroup%
\makeatletter%
\begin{pgfpicture}%
\pgfpathrectangle{\pgfpointorigin}{\pgfqpoint{5.906660in}{5.000000in}}%
\pgfusepath{use as bounding box, clip}%
\begin{pgfscope}%
\pgfsetbuttcap%
\pgfsetmiterjoin%
\definecolor{currentfill}{rgb}{1.000000,1.000000,1.000000}%
\pgfsetfillcolor{currentfill}%
\pgfsetlinewidth{0.000000pt}%
\definecolor{currentstroke}{rgb}{1.000000,1.000000,1.000000}%
\pgfsetstrokecolor{currentstroke}%
\pgfsetdash{}{0pt}%
\pgfpathmoveto{\pgfqpoint{0.000000in}{0.000000in}}%
\pgfpathlineto{\pgfqpoint{5.906660in}{0.000000in}}%
\pgfpathlineto{\pgfqpoint{5.906660in}{5.000000in}}%
\pgfpathlineto{\pgfqpoint{0.000000in}{5.000000in}}%
\pgfpathlineto{\pgfqpoint{0.000000in}{0.000000in}}%
\pgfpathclose%
\pgfusepath{fill}%
\end{pgfscope}%
\begin{pgfscope}%
\definecolor{textcolor}{rgb}{0.150000,0.150000,0.150000}%
\pgfsetstrokecolor{textcolor}%
\pgfsetfillcolor{textcolor}%
\pgftext[x=3.027163in,y=0.411111in,,top]{\color{textcolor}\sffamily\fontsize{12.000000}{14.400000}\selectfont Computer Science UniversityUniverity Experience}%
\end{pgfscope}%
\begin{pgfscope}%
\pgfsetbuttcap%
\pgfsetmiterjoin%
\definecolor{currentfill}{rgb}{0.552941,0.827451,0.780392}%
\pgfsetfillcolor{currentfill}%
\pgfsetlinewidth{1.003750pt}%
\definecolor{currentstroke}{rgb}{1.000000,1.000000,1.000000}%
\pgfsetstrokecolor{currentstroke}%
\pgfsetdash{}{0pt}%
\pgfpathmoveto{\pgfqpoint{4.567163in}{2.475000in}}%
\pgfpathcurveto{\pgfqpoint{4.567163in}{2.713219in}}{\pgfqpoint{4.511889in}{2.948220in}}{\pgfqpoint{4.405702in}{3.161463in}}%
\pgfpathcurveto{\pgfqpoint{4.299515in}{3.374705in}}{\pgfqpoint{4.145283in}{3.560429in}}{\pgfqpoint{3.955175in}{3.703981in}}%
\pgfpathcurveto{\pgfqpoint{3.765067in}{3.847533in}}{\pgfqpoint{3.544218in}{3.945035in}}{\pgfqpoint{3.310053in}{3.988794in}}%
\pgfpathcurveto{\pgfqpoint{3.075889in}{4.032554in}}{\pgfqpoint{2.834733in}{4.021389in}}{\pgfqpoint{2.605613in}{3.956180in}}%
\pgfpathlineto{\pgfqpoint{3.027163in}{2.475000in}}%
\pgfpathlineto{\pgfqpoint{4.567163in}{2.475000in}}%
\pgfpathlineto{\pgfqpoint{4.567163in}{2.475000in}}%
\pgfpathclose%
\pgfusepath{stroke,fill}%
\end{pgfscope}%
\begin{pgfscope}%
\pgfsetbuttcap%
\pgfsetmiterjoin%
\definecolor{currentfill}{rgb}{1.000000,1.000000,0.701961}%
\pgfsetfillcolor{currentfill}%
\pgfsetlinewidth{1.003750pt}%
\definecolor{currentstroke}{rgb}{1.000000,1.000000,1.000000}%
\pgfsetstrokecolor{currentstroke}%
\pgfsetdash{}{0pt}%
\pgfpathmoveto{\pgfqpoint{2.605613in}{3.956180in}}%
\pgfpathcurveto{\pgfqpoint{2.376493in}{3.890972in}}{\pgfqpoint{2.165597in}{3.773481in}}{\pgfqpoint{1.989567in}{3.612978in}}%
\pgfpathcurveto{\pgfqpoint{1.813536in}{3.452475in}}{\pgfqpoint{1.677124in}{3.253294in}}{\pgfqpoint{1.591094in}{3.031153in}}%
\pgfpathcurveto{\pgfqpoint{1.505064in}{2.809011in}}{\pgfqpoint{1.471740in}{2.569908in}}{\pgfqpoint{1.493751in}{2.332708in}}%
\pgfpathcurveto{\pgfqpoint{1.515762in}{2.095508in}}{\pgfqpoint{1.592513in}{1.866620in}}{\pgfqpoint{1.717949in}{1.664101in}}%
\pgfpathlineto{\pgfqpoint{3.027163in}{2.475000in}}%
\pgfpathlineto{\pgfqpoint{2.605613in}{3.956180in}}%
\pgfpathlineto{\pgfqpoint{2.605613in}{3.956180in}}%
\pgfpathclose%
\pgfusepath{stroke,fill}%
\end{pgfscope}%
\begin{pgfscope}%
\pgfsetbuttcap%
\pgfsetmiterjoin%
\definecolor{currentfill}{rgb}{0.745098,0.729412,0.854902}%
\pgfsetfillcolor{currentfill}%
\pgfsetlinewidth{1.003750pt}%
\definecolor{currentstroke}{rgb}{1.000000,1.000000,1.000000}%
\pgfsetstrokecolor{currentstroke}%
\pgfsetdash{}{0pt}%
\pgfpathmoveto{\pgfqpoint{1.717949in}{1.664101in}}%
\pgfpathcurveto{\pgfqpoint{1.843385in}{1.461582in}}{\pgfqpoint{2.014117in}{1.290903in}}{\pgfqpoint{2.216676in}{1.165531in}}%
\pgfpathcurveto{\pgfqpoint{2.419234in}{1.040158in}}{\pgfqpoint{2.648147in}{0.963479in}}{\pgfqpoint{2.885354in}{0.941543in}}%
\pgfpathcurveto{\pgfqpoint{3.122560in}{0.919607in}}{\pgfqpoint{3.361653in}{0.953006in}}{\pgfqpoint{3.583768in}{1.039106in}}%
\pgfpathcurveto{\pgfqpoint{3.805882in}{1.125206in}}{\pgfqpoint{4.005020in}{1.261680in}}{\pgfqpoint{4.165467in}{1.437762in}}%
\pgfpathlineto{\pgfqpoint{3.027163in}{2.475000in}}%
\pgfpathlineto{\pgfqpoint{1.717949in}{1.664101in}}%
\pgfpathlineto{\pgfqpoint{1.717949in}{1.664101in}}%
\pgfpathclose%
\pgfusepath{stroke,fill}%
\end{pgfscope}%
\begin{pgfscope}%
\pgfsetbuttcap%
\pgfsetmiterjoin%
\definecolor{currentfill}{rgb}{0.984314,0.501961,0.447059}%
\pgfsetfillcolor{currentfill}%
\pgfsetlinewidth{1.003750pt}%
\definecolor{currentstroke}{rgb}{1.000000,1.000000,1.000000}%
\pgfsetstrokecolor{currentstroke}%
\pgfsetdash{}{0pt}%
\pgfpathmoveto{\pgfqpoint{4.165467in}{1.437762in}}%
\pgfpathcurveto{\pgfqpoint{4.165467in}{1.437762in}}{\pgfqpoint{4.165467in}{1.437762in}}{\pgfqpoint{4.165467in}{1.437762in}}%
\pgfpathlineto{\pgfqpoint{3.027163in}{2.475000in}}%
\pgfpathlineto{\pgfqpoint{4.165467in}{1.437762in}}%
\pgfpathlineto{\pgfqpoint{4.165467in}{1.437762in}}%
\pgfpathclose%
\pgfusepath{stroke,fill}%
\end{pgfscope}%
\begin{pgfscope}%
\pgfsetbuttcap%
\pgfsetmiterjoin%
\definecolor{currentfill}{rgb}{0.501961,0.694118,0.827451}%
\pgfsetfillcolor{currentfill}%
\pgfsetlinewidth{1.003750pt}%
\definecolor{currentstroke}{rgb}{1.000000,1.000000,1.000000}%
\pgfsetstrokecolor{currentstroke}%
\pgfsetdash{}{0pt}%
\pgfpathmoveto{\pgfqpoint{4.165467in}{1.437762in}}%
\pgfpathcurveto{\pgfqpoint{4.293579in}{1.578356in}}{\pgfqpoint{4.394541in}{1.741476in}}{\pgfqpoint{4.463232in}{1.918848in}}%
\pgfpathcurveto{\pgfqpoint{4.531924in}{2.096219in}}{\pgfqpoint{4.567163in}{2.284792in}}{\pgfqpoint{4.567163in}{2.475001in}}%
\pgfpathlineto{\pgfqpoint{3.027163in}{2.475000in}}%
\pgfpathlineto{\pgfqpoint{4.165467in}{1.437762in}}%
\pgfpathlineto{\pgfqpoint{4.165467in}{1.437762in}}%
\pgfpathclose%
\pgfusepath{stroke,fill}%
\end{pgfscope}%
\begin{pgfscope}%
\pgfsetbuttcap%
\pgfsetmiterjoin%
\definecolor{currentfill}{rgb}{0.992157,0.705882,0.384314}%
\pgfsetfillcolor{currentfill}%
\pgfsetlinewidth{1.003750pt}%
\definecolor{currentstroke}{rgb}{1.000000,1.000000,1.000000}%
\pgfsetstrokecolor{currentstroke}%
\pgfsetdash{}{0pt}%
\pgfpathmoveto{\pgfqpoint{4.567163in}{2.475001in}}%
\pgfpathcurveto{\pgfqpoint{4.567163in}{2.475001in}}{\pgfqpoint{4.567163in}{2.475001in}}{\pgfqpoint{4.567163in}{2.475001in}}%
\pgfpathlineto{\pgfqpoint{3.027163in}{2.475000in}}%
\pgfpathlineto{\pgfqpoint{4.567163in}{2.475001in}}%
\pgfpathlineto{\pgfqpoint{4.567163in}{2.475001in}}%
\pgfpathclose%
\pgfusepath{stroke,fill}%
\end{pgfscope}%
\begin{pgfscope}%
\definecolor{textcolor}{rgb}{0.150000,0.150000,0.150000}%
\pgfsetstrokecolor{textcolor}%
\pgfsetfillcolor{textcolor}%
\pgftext[x=3.583970in,y=3.212389in,,]{\color{textcolor}\sffamily\fontsize{12.000000}{14.400000}\selectfont 29.4\%}%
\end{pgfscope}%
\begin{pgfscope}%
\definecolor{textcolor}{rgb}{0.150000,0.150000,0.150000}%
\pgfsetstrokecolor{textcolor}%
\pgfsetfillcolor{textcolor}%
\pgftext[x=2.165522in,y=2.808692in,,]{\color{textcolor}\sffamily\fontsize{12.000000}{14.400000}\selectfont 29.4\%}%
\end{pgfscope}%
\begin{pgfscope}%
\definecolor{textcolor}{rgb}{0.150000,0.150000,0.150000}%
\pgfsetstrokecolor{textcolor}%
\pgfsetfillcolor{textcolor}%
\pgftext[x=2.942078in,y=1.554926in,,]{\color{textcolor}\sffamily\fontsize{12.000000}{14.400000}\selectfont 29.4\%}%
\end{pgfscope}%
\begin{pgfscope}%
\definecolor{textcolor}{rgb}{0.150000,0.150000,0.150000}%
\pgfsetstrokecolor{textcolor}%
\pgfsetfillcolor{textcolor}%
\pgftext[x=3.710146in,y=1.852657in,,]{\color{textcolor}\sffamily\fontsize{12.000000}{14.400000}\selectfont 0.0\%}%
\end{pgfscope}%
\begin{pgfscope}%
\definecolor{textcolor}{rgb}{0.150000,0.150000,0.150000}%
\pgfsetstrokecolor{textcolor}%
\pgfsetfillcolor{textcolor}%
\pgftext[x=3.888805in,y=2.141309in,,]{\color{textcolor}\sffamily\fontsize{12.000000}{14.400000}\selectfont 11.8\%}%
\end{pgfscope}%
\begin{pgfscope}%
\definecolor{textcolor}{rgb}{0.150000,0.150000,0.150000}%
\pgfsetstrokecolor{textcolor}%
\pgfsetfillcolor{textcolor}%
\pgftext[x=3.951163in,y=2.475000in,,]{\color{textcolor}\sffamily\fontsize{12.000000}{14.400000}\selectfont 0.0\%}%
\end{pgfscope}%
\begin{pgfscope}%
\pgfsetbuttcap%
\pgfsetmiterjoin%
\definecolor{currentfill}{rgb}{1.000000,1.000000,1.000000}%
\pgfsetfillcolor{currentfill}%
\pgfsetfillopacity{0.800000}%
\pgfsetlinewidth{1.003750pt}%
\definecolor{currentstroke}{rgb}{0.800000,0.800000,0.800000}%
\pgfsetstrokecolor{currentstroke}%
\pgfsetstrokeopacity{0.800000}%
\pgfsetdash{}{0pt}%
\pgfpathmoveto{\pgfqpoint{3.271524in}{0.458500in}}%
\pgfpathlineto{\pgfqpoint{5.827163in}{0.458500in}}%
\pgfpathquadraticcurveto{\pgfqpoint{5.852163in}{0.458500in}}{\pgfqpoint{5.852163in}{0.483500in}}%
\pgfpathlineto{\pgfqpoint{5.852163in}{1.571829in}}%
\pgfpathquadraticcurveto{\pgfqpoint{5.852163in}{1.596829in}}{\pgfqpoint{5.827163in}{1.596829in}}%
\pgfpathlineto{\pgfqpoint{3.271524in}{1.596829in}}%
\pgfpathquadraticcurveto{\pgfqpoint{3.246524in}{1.596829in}}{\pgfqpoint{3.246524in}{1.571829in}}%
\pgfpathlineto{\pgfqpoint{3.246524in}{0.483500in}}%
\pgfpathquadraticcurveto{\pgfqpoint{3.246524in}{0.458500in}}{\pgfqpoint{3.271524in}{0.458500in}}%
\pgfpathlineto{\pgfqpoint{3.271524in}{0.458500in}}%
\pgfpathclose%
\pgfusepath{stroke,fill}%
\end{pgfscope}%
\begin{pgfscope}%
\pgfsetbuttcap%
\pgfsetmiterjoin%
\definecolor{currentfill}{rgb}{0.552941,0.827451,0.780392}%
\pgfsetfillcolor{currentfill}%
\pgfsetlinewidth{1.003750pt}%
\definecolor{currentstroke}{rgb}{1.000000,1.000000,1.000000}%
\pgfsetstrokecolor{currentstroke}%
\pgfsetdash{}{0pt}%
\pgfpathmoveto{\pgfqpoint{3.296524in}{1.451858in}}%
\pgfpathlineto{\pgfqpoint{3.546524in}{1.451858in}}%
\pgfpathlineto{\pgfqpoint{3.546524in}{1.539358in}}%
\pgfpathlineto{\pgfqpoint{3.296524in}{1.539358in}}%
\pgfpathlineto{\pgfqpoint{3.296524in}{1.451858in}}%
\pgfpathclose%
\pgfusepath{stroke,fill}%
\end{pgfscope}%
\begin{pgfscope}%
\definecolor{textcolor}{rgb}{0.150000,0.150000,0.150000}%
\pgfsetstrokecolor{textcolor}%
\pgfsetfillcolor{textcolor}%
\pgftext[x=3.646524in,y=1.451858in,left,base]{\color{textcolor}\sffamily\fontsize{9.000000}{10.800000}\selectfont No CS Degree, 29.4 \%}%
\end{pgfscope}%
\begin{pgfscope}%
\pgfsetbuttcap%
\pgfsetmiterjoin%
\definecolor{currentfill}{rgb}{1.000000,1.000000,0.701961}%
\pgfsetfillcolor{currentfill}%
\pgfsetlinewidth{1.003750pt}%
\definecolor{currentstroke}{rgb}{1.000000,1.000000,1.000000}%
\pgfsetstrokecolor{currentstroke}%
\pgfsetdash{}{0pt}%
\pgfpathmoveto{\pgfqpoint{3.296524in}{1.268387in}}%
\pgfpathlineto{\pgfqpoint{3.546524in}{1.268387in}}%
\pgfpathlineto{\pgfqpoint{3.546524in}{1.355887in}}%
\pgfpathlineto{\pgfqpoint{3.296524in}{1.355887in}}%
\pgfpathlineto{\pgfqpoint{3.296524in}{1.268387in}}%
\pgfpathclose%
\pgfusepath{stroke,fill}%
\end{pgfscope}%
\begin{pgfscope}%
\definecolor{textcolor}{rgb}{0.150000,0.150000,0.150000}%
\pgfsetstrokecolor{textcolor}%
\pgfsetfillcolor{textcolor}%
\pgftext[x=3.646524in,y=1.268387in,left,base]{\color{textcolor}\sffamily\fontsize{9.000000}{10.800000}\selectfont Bachelors's Degree, 29.4 \%}%
\end{pgfscope}%
\begin{pgfscope}%
\pgfsetbuttcap%
\pgfsetmiterjoin%
\definecolor{currentfill}{rgb}{0.745098,0.729412,0.854902}%
\pgfsetfillcolor{currentfill}%
\pgfsetlinewidth{1.003750pt}%
\definecolor{currentstroke}{rgb}{1.000000,1.000000,1.000000}%
\pgfsetstrokecolor{currentstroke}%
\pgfsetdash{}{0pt}%
\pgfpathmoveto{\pgfqpoint{3.296524in}{1.084915in}}%
\pgfpathlineto{\pgfqpoint{3.546524in}{1.084915in}}%
\pgfpathlineto{\pgfqpoint{3.546524in}{1.172415in}}%
\pgfpathlineto{\pgfqpoint{3.296524in}{1.172415in}}%
\pgfpathlineto{\pgfqpoint{3.296524in}{1.084915in}}%
\pgfpathclose%
\pgfusepath{stroke,fill}%
\end{pgfscope}%
\begin{pgfscope}%
\definecolor{textcolor}{rgb}{0.150000,0.150000,0.150000}%
\pgfsetstrokecolor{textcolor}%
\pgfsetfillcolor{textcolor}%
\pgftext[x=3.646524in,y=1.084915in,left,base]{\color{textcolor}\sffamily\fontsize{9.000000}{10.800000}\selectfont Master's Degree, 29.4 \%}%
\end{pgfscope}%
\begin{pgfscope}%
\pgfsetbuttcap%
\pgfsetmiterjoin%
\definecolor{currentfill}{rgb}{0.984314,0.501961,0.447059}%
\pgfsetfillcolor{currentfill}%
\pgfsetlinewidth{1.003750pt}%
\definecolor{currentstroke}{rgb}{1.000000,1.000000,1.000000}%
\pgfsetstrokecolor{currentstroke}%
\pgfsetdash{}{0pt}%
\pgfpathmoveto{\pgfqpoint{3.296524in}{0.901444in}}%
\pgfpathlineto{\pgfqpoint{3.546524in}{0.901444in}}%
\pgfpathlineto{\pgfqpoint{3.546524in}{0.988944in}}%
\pgfpathlineto{\pgfqpoint{3.296524in}{0.988944in}}%
\pgfpathlineto{\pgfqpoint{3.296524in}{0.901444in}}%
\pgfpathclose%
\pgfusepath{stroke,fill}%
\end{pgfscope}%
\begin{pgfscope}%
\definecolor{textcolor}{rgb}{0.150000,0.150000,0.150000}%
\pgfsetstrokecolor{textcolor}%
\pgfsetfillcolor{textcolor}%
\pgftext[x=3.646524in,y=0.901444in,left,base]{\color{textcolor}\sffamily\fontsize{9.000000}{10.800000}\selectfont Doctorate Degree, 0.0 \%}%
\end{pgfscope}%
\begin{pgfscope}%
\pgfsetbuttcap%
\pgfsetmiterjoin%
\definecolor{currentfill}{rgb}{0.501961,0.694118,0.827451}%
\pgfsetfillcolor{currentfill}%
\pgfsetlinewidth{1.003750pt}%
\definecolor{currentstroke}{rgb}{1.000000,1.000000,1.000000}%
\pgfsetstrokecolor{currentstroke}%
\pgfsetdash{}{0pt}%
\pgfpathmoveto{\pgfqpoint{3.296524in}{0.717972in}}%
\pgfpathlineto{\pgfqpoint{3.546524in}{0.717972in}}%
\pgfpathlineto{\pgfqpoint{3.546524in}{0.805472in}}%
\pgfpathlineto{\pgfqpoint{3.296524in}{0.805472in}}%
\pgfpathlineto{\pgfqpoint{3.296524in}{0.717972in}}%
\pgfpathclose%
\pgfusepath{stroke,fill}%
\end{pgfscope}%
\begin{pgfscope}%
\definecolor{textcolor}{rgb}{0.150000,0.150000,0.150000}%
\pgfsetstrokecolor{textcolor}%
\pgfsetfillcolor{textcolor}%
\pgftext[x=3.646524in,y=0.717972in,left,base]{\color{textcolor}\sffamily\fontsize{9.000000}{10.800000}\selectfont Other Engineering Degree, 11.8 \%}%
\end{pgfscope}%
\begin{pgfscope}%
\pgfsetbuttcap%
\pgfsetmiterjoin%
\definecolor{currentfill}{rgb}{0.992157,0.705882,0.384314}%
\pgfsetfillcolor{currentfill}%
\pgfsetlinewidth{1.003750pt}%
\definecolor{currentstroke}{rgb}{1.000000,1.000000,1.000000}%
\pgfsetstrokecolor{currentstroke}%
\pgfsetdash{}{0pt}%
\pgfpathmoveto{\pgfqpoint{3.296524in}{0.534501in}}%
\pgfpathlineto{\pgfqpoint{3.546524in}{0.534501in}}%
\pgfpathlineto{\pgfqpoint{3.546524in}{0.622001in}}%
\pgfpathlineto{\pgfqpoint{3.296524in}{0.622001in}}%
\pgfpathlineto{\pgfqpoint{3.296524in}{0.534501in}}%
\pgfpathclose%
\pgfusepath{stroke,fill}%
\end{pgfscope}%
\begin{pgfscope}%
\definecolor{textcolor}{rgb}{0.150000,0.150000,0.150000}%
\pgfsetstrokecolor{textcolor}%
\pgfsetfillcolor{textcolor}%
\pgftext[x=3.646524in,y=0.534501in,left,base]{\color{textcolor}\sffamily\fontsize{9.000000}{10.800000}\selectfont Yes but did not finish, 0.0 \%}%
\end{pgfscope}%
\end{pgfpicture}%
\makeatother%
\endgroup%
}
	\caption{previnteractive}
	\label{fig:previnteractive}
\end{figure}

\begin{figure}[H]
	\scalebox{0.72}{%% Creator: Matplotlib, PGF backend
%%
%% To include the figure in your LaTeX document, write
%%   \input{<filename>.pgf}
%%
%% Make sure the required packages are loaded in your preamble
%%   \usepackage{pgf}
%%
%% Also ensure that all the required font packages are loaded; for instance,
%% the lmodern package is sometimes necessary when using math font.
%%   \usepackage{lmodern}
%%
%% Figures using additional raster images can only be included by \input if
%% they are in the same directory as the main LaTeX file. For loading figures
%% from other directories you can use the `import` package
%%   \usepackage{import}
%%
%% and then include the figures with
%%   \import{<path to file>}{<filename>.pgf}
%%
%% Matplotlib used the following preamble
%%   \usepackage{fontspec}
%%   \setmainfont{DejaVuSerif.ttf}[Path=\detokenize{/home/spam/miniconda3/envs/mpl/lib/python3.10/site-packages/matplotlib/mpl-data/fonts/ttf/}]
%%   \setsansfont{DejaVuSans.ttf}[Path=\detokenize{/home/spam/miniconda3/envs/mpl/lib/python3.10/site-packages/matplotlib/mpl-data/fonts/ttf/}]
%%   \setmonofont{DejaVuSansMono.ttf}[Path=\detokenize{/home/spam/miniconda3/envs/mpl/lib/python3.10/site-packages/matplotlib/mpl-data/fonts/ttf/}]
%%
\begingroup%
\makeatletter%
\begin{pgfpicture}%
\pgfpathrectangle{\pgfpointorigin}{\pgfqpoint{5.906660in}{5.000000in}}%
\pgfusepath{use as bounding box, clip}%
\begin{pgfscope}%
\pgfsetbuttcap%
\pgfsetmiterjoin%
\definecolor{currentfill}{rgb}{1.000000,1.000000,1.000000}%
\pgfsetfillcolor{currentfill}%
\pgfsetlinewidth{0.000000pt}%
\definecolor{currentstroke}{rgb}{1.000000,1.000000,1.000000}%
\pgfsetstrokecolor{currentstroke}%
\pgfsetdash{}{0pt}%
\pgfpathmoveto{\pgfqpoint{0.000000in}{0.000000in}}%
\pgfpathlineto{\pgfqpoint{5.906660in}{0.000000in}}%
\pgfpathlineto{\pgfqpoint{5.906660in}{5.000000in}}%
\pgfpathlineto{\pgfqpoint{0.000000in}{5.000000in}}%
\pgfpathlineto{\pgfqpoint{0.000000in}{0.000000in}}%
\pgfpathclose%
\pgfusepath{fill}%
\end{pgfscope}%
\begin{pgfscope}%
\definecolor{textcolor}{rgb}{0.150000,0.150000,0.150000}%
\pgfsetstrokecolor{textcolor}%
\pgfsetfillcolor{textcolor}%
\pgftext[x=3.027163in,y=0.411111in,,top]{\color{textcolor}\sffamily\fontsize{12.000000}{14.400000}\selectfont Computer Science UniversityUniverity Experience}%
\end{pgfscope}%
\begin{pgfscope}%
\pgfsetbuttcap%
\pgfsetmiterjoin%
\definecolor{currentfill}{rgb}{0.552941,0.827451,0.780392}%
\pgfsetfillcolor{currentfill}%
\pgfsetlinewidth{1.003750pt}%
\definecolor{currentstroke}{rgb}{1.000000,1.000000,1.000000}%
\pgfsetstrokecolor{currentstroke}%
\pgfsetdash{}{0pt}%
\pgfpathmoveto{\pgfqpoint{4.567163in}{2.475000in}}%
\pgfpathcurveto{\pgfqpoint{4.567163in}{2.713219in}}{\pgfqpoint{4.511889in}{2.948220in}}{\pgfqpoint{4.405702in}{3.161463in}}%
\pgfpathcurveto{\pgfqpoint{4.299515in}{3.374705in}}{\pgfqpoint{4.145283in}{3.560429in}}{\pgfqpoint{3.955175in}{3.703981in}}%
\pgfpathcurveto{\pgfqpoint{3.765067in}{3.847533in}}{\pgfqpoint{3.544218in}{3.945035in}}{\pgfqpoint{3.310053in}{3.988794in}}%
\pgfpathcurveto{\pgfqpoint{3.075889in}{4.032554in}}{\pgfqpoint{2.834733in}{4.021389in}}{\pgfqpoint{2.605613in}{3.956180in}}%
\pgfpathlineto{\pgfqpoint{3.027163in}{2.475000in}}%
\pgfpathlineto{\pgfqpoint{4.567163in}{2.475000in}}%
\pgfpathlineto{\pgfqpoint{4.567163in}{2.475000in}}%
\pgfpathclose%
\pgfusepath{stroke,fill}%
\end{pgfscope}%
\begin{pgfscope}%
\pgfsetbuttcap%
\pgfsetmiterjoin%
\definecolor{currentfill}{rgb}{1.000000,1.000000,0.701961}%
\pgfsetfillcolor{currentfill}%
\pgfsetlinewidth{1.003750pt}%
\definecolor{currentstroke}{rgb}{1.000000,1.000000,1.000000}%
\pgfsetstrokecolor{currentstroke}%
\pgfsetdash{}{0pt}%
\pgfpathmoveto{\pgfqpoint{2.605613in}{3.956180in}}%
\pgfpathcurveto{\pgfqpoint{2.376493in}{3.890972in}}{\pgfqpoint{2.165597in}{3.773481in}}{\pgfqpoint{1.989567in}{3.612978in}}%
\pgfpathcurveto{\pgfqpoint{1.813536in}{3.452475in}}{\pgfqpoint{1.677124in}{3.253294in}}{\pgfqpoint{1.591094in}{3.031153in}}%
\pgfpathcurveto{\pgfqpoint{1.505064in}{2.809011in}}{\pgfqpoint{1.471740in}{2.569908in}}{\pgfqpoint{1.493751in}{2.332708in}}%
\pgfpathcurveto{\pgfqpoint{1.515762in}{2.095508in}}{\pgfqpoint{1.592513in}{1.866620in}}{\pgfqpoint{1.717949in}{1.664101in}}%
\pgfpathlineto{\pgfqpoint{3.027163in}{2.475000in}}%
\pgfpathlineto{\pgfqpoint{2.605613in}{3.956180in}}%
\pgfpathlineto{\pgfqpoint{2.605613in}{3.956180in}}%
\pgfpathclose%
\pgfusepath{stroke,fill}%
\end{pgfscope}%
\begin{pgfscope}%
\pgfsetbuttcap%
\pgfsetmiterjoin%
\definecolor{currentfill}{rgb}{0.745098,0.729412,0.854902}%
\pgfsetfillcolor{currentfill}%
\pgfsetlinewidth{1.003750pt}%
\definecolor{currentstroke}{rgb}{1.000000,1.000000,1.000000}%
\pgfsetstrokecolor{currentstroke}%
\pgfsetdash{}{0pt}%
\pgfpathmoveto{\pgfqpoint{1.717949in}{1.664101in}}%
\pgfpathcurveto{\pgfqpoint{1.843385in}{1.461582in}}{\pgfqpoint{2.014117in}{1.290903in}}{\pgfqpoint{2.216676in}{1.165531in}}%
\pgfpathcurveto{\pgfqpoint{2.419234in}{1.040158in}}{\pgfqpoint{2.648147in}{0.963479in}}{\pgfqpoint{2.885354in}{0.941543in}}%
\pgfpathcurveto{\pgfqpoint{3.122560in}{0.919607in}}{\pgfqpoint{3.361653in}{0.953006in}}{\pgfqpoint{3.583768in}{1.039106in}}%
\pgfpathcurveto{\pgfqpoint{3.805882in}{1.125206in}}{\pgfqpoint{4.005020in}{1.261680in}}{\pgfqpoint{4.165467in}{1.437762in}}%
\pgfpathlineto{\pgfqpoint{3.027163in}{2.475000in}}%
\pgfpathlineto{\pgfqpoint{1.717949in}{1.664101in}}%
\pgfpathlineto{\pgfqpoint{1.717949in}{1.664101in}}%
\pgfpathclose%
\pgfusepath{stroke,fill}%
\end{pgfscope}%
\begin{pgfscope}%
\pgfsetbuttcap%
\pgfsetmiterjoin%
\definecolor{currentfill}{rgb}{0.984314,0.501961,0.447059}%
\pgfsetfillcolor{currentfill}%
\pgfsetlinewidth{1.003750pt}%
\definecolor{currentstroke}{rgb}{1.000000,1.000000,1.000000}%
\pgfsetstrokecolor{currentstroke}%
\pgfsetdash{}{0pt}%
\pgfpathmoveto{\pgfqpoint{4.165467in}{1.437762in}}%
\pgfpathcurveto{\pgfqpoint{4.165467in}{1.437762in}}{\pgfqpoint{4.165467in}{1.437762in}}{\pgfqpoint{4.165467in}{1.437762in}}%
\pgfpathlineto{\pgfqpoint{3.027163in}{2.475000in}}%
\pgfpathlineto{\pgfqpoint{4.165467in}{1.437762in}}%
\pgfpathlineto{\pgfqpoint{4.165467in}{1.437762in}}%
\pgfpathclose%
\pgfusepath{stroke,fill}%
\end{pgfscope}%
\begin{pgfscope}%
\pgfsetbuttcap%
\pgfsetmiterjoin%
\definecolor{currentfill}{rgb}{0.501961,0.694118,0.827451}%
\pgfsetfillcolor{currentfill}%
\pgfsetlinewidth{1.003750pt}%
\definecolor{currentstroke}{rgb}{1.000000,1.000000,1.000000}%
\pgfsetstrokecolor{currentstroke}%
\pgfsetdash{}{0pt}%
\pgfpathmoveto{\pgfqpoint{4.165467in}{1.437762in}}%
\pgfpathcurveto{\pgfqpoint{4.293579in}{1.578356in}}{\pgfqpoint{4.394541in}{1.741476in}}{\pgfqpoint{4.463232in}{1.918848in}}%
\pgfpathcurveto{\pgfqpoint{4.531924in}{2.096219in}}{\pgfqpoint{4.567163in}{2.284792in}}{\pgfqpoint{4.567163in}{2.475001in}}%
\pgfpathlineto{\pgfqpoint{3.027163in}{2.475000in}}%
\pgfpathlineto{\pgfqpoint{4.165467in}{1.437762in}}%
\pgfpathlineto{\pgfqpoint{4.165467in}{1.437762in}}%
\pgfpathclose%
\pgfusepath{stroke,fill}%
\end{pgfscope}%
\begin{pgfscope}%
\pgfsetbuttcap%
\pgfsetmiterjoin%
\definecolor{currentfill}{rgb}{0.992157,0.705882,0.384314}%
\pgfsetfillcolor{currentfill}%
\pgfsetlinewidth{1.003750pt}%
\definecolor{currentstroke}{rgb}{1.000000,1.000000,1.000000}%
\pgfsetstrokecolor{currentstroke}%
\pgfsetdash{}{0pt}%
\pgfpathmoveto{\pgfqpoint{4.567163in}{2.475001in}}%
\pgfpathcurveto{\pgfqpoint{4.567163in}{2.475001in}}{\pgfqpoint{4.567163in}{2.475001in}}{\pgfqpoint{4.567163in}{2.475001in}}%
\pgfpathlineto{\pgfqpoint{3.027163in}{2.475000in}}%
\pgfpathlineto{\pgfqpoint{4.567163in}{2.475001in}}%
\pgfpathlineto{\pgfqpoint{4.567163in}{2.475001in}}%
\pgfpathclose%
\pgfusepath{stroke,fill}%
\end{pgfscope}%
\begin{pgfscope}%
\definecolor{textcolor}{rgb}{0.150000,0.150000,0.150000}%
\pgfsetstrokecolor{textcolor}%
\pgfsetfillcolor{textcolor}%
\pgftext[x=3.583970in,y=3.212389in,,]{\color{textcolor}\sffamily\fontsize{12.000000}{14.400000}\selectfont 29.4\%}%
\end{pgfscope}%
\begin{pgfscope}%
\definecolor{textcolor}{rgb}{0.150000,0.150000,0.150000}%
\pgfsetstrokecolor{textcolor}%
\pgfsetfillcolor{textcolor}%
\pgftext[x=2.165522in,y=2.808692in,,]{\color{textcolor}\sffamily\fontsize{12.000000}{14.400000}\selectfont 29.4\%}%
\end{pgfscope}%
\begin{pgfscope}%
\definecolor{textcolor}{rgb}{0.150000,0.150000,0.150000}%
\pgfsetstrokecolor{textcolor}%
\pgfsetfillcolor{textcolor}%
\pgftext[x=2.942078in,y=1.554926in,,]{\color{textcolor}\sffamily\fontsize{12.000000}{14.400000}\selectfont 29.4\%}%
\end{pgfscope}%
\begin{pgfscope}%
\definecolor{textcolor}{rgb}{0.150000,0.150000,0.150000}%
\pgfsetstrokecolor{textcolor}%
\pgfsetfillcolor{textcolor}%
\pgftext[x=3.710146in,y=1.852657in,,]{\color{textcolor}\sffamily\fontsize{12.000000}{14.400000}\selectfont 0.0\%}%
\end{pgfscope}%
\begin{pgfscope}%
\definecolor{textcolor}{rgb}{0.150000,0.150000,0.150000}%
\pgfsetstrokecolor{textcolor}%
\pgfsetfillcolor{textcolor}%
\pgftext[x=3.888805in,y=2.141309in,,]{\color{textcolor}\sffamily\fontsize{12.000000}{14.400000}\selectfont 11.8\%}%
\end{pgfscope}%
\begin{pgfscope}%
\definecolor{textcolor}{rgb}{0.150000,0.150000,0.150000}%
\pgfsetstrokecolor{textcolor}%
\pgfsetfillcolor{textcolor}%
\pgftext[x=3.951163in,y=2.475000in,,]{\color{textcolor}\sffamily\fontsize{12.000000}{14.400000}\selectfont 0.0\%}%
\end{pgfscope}%
\begin{pgfscope}%
\pgfsetbuttcap%
\pgfsetmiterjoin%
\definecolor{currentfill}{rgb}{1.000000,1.000000,1.000000}%
\pgfsetfillcolor{currentfill}%
\pgfsetfillopacity{0.800000}%
\pgfsetlinewidth{1.003750pt}%
\definecolor{currentstroke}{rgb}{0.800000,0.800000,0.800000}%
\pgfsetstrokecolor{currentstroke}%
\pgfsetstrokeopacity{0.800000}%
\pgfsetdash{}{0pt}%
\pgfpathmoveto{\pgfqpoint{3.271524in}{0.458500in}}%
\pgfpathlineto{\pgfqpoint{5.827163in}{0.458500in}}%
\pgfpathquadraticcurveto{\pgfqpoint{5.852163in}{0.458500in}}{\pgfqpoint{5.852163in}{0.483500in}}%
\pgfpathlineto{\pgfqpoint{5.852163in}{1.571829in}}%
\pgfpathquadraticcurveto{\pgfqpoint{5.852163in}{1.596829in}}{\pgfqpoint{5.827163in}{1.596829in}}%
\pgfpathlineto{\pgfqpoint{3.271524in}{1.596829in}}%
\pgfpathquadraticcurveto{\pgfqpoint{3.246524in}{1.596829in}}{\pgfqpoint{3.246524in}{1.571829in}}%
\pgfpathlineto{\pgfqpoint{3.246524in}{0.483500in}}%
\pgfpathquadraticcurveto{\pgfqpoint{3.246524in}{0.458500in}}{\pgfqpoint{3.271524in}{0.458500in}}%
\pgfpathlineto{\pgfqpoint{3.271524in}{0.458500in}}%
\pgfpathclose%
\pgfusepath{stroke,fill}%
\end{pgfscope}%
\begin{pgfscope}%
\pgfsetbuttcap%
\pgfsetmiterjoin%
\definecolor{currentfill}{rgb}{0.552941,0.827451,0.780392}%
\pgfsetfillcolor{currentfill}%
\pgfsetlinewidth{1.003750pt}%
\definecolor{currentstroke}{rgb}{1.000000,1.000000,1.000000}%
\pgfsetstrokecolor{currentstroke}%
\pgfsetdash{}{0pt}%
\pgfpathmoveto{\pgfqpoint{3.296524in}{1.451858in}}%
\pgfpathlineto{\pgfqpoint{3.546524in}{1.451858in}}%
\pgfpathlineto{\pgfqpoint{3.546524in}{1.539358in}}%
\pgfpathlineto{\pgfqpoint{3.296524in}{1.539358in}}%
\pgfpathlineto{\pgfqpoint{3.296524in}{1.451858in}}%
\pgfpathclose%
\pgfusepath{stroke,fill}%
\end{pgfscope}%
\begin{pgfscope}%
\definecolor{textcolor}{rgb}{0.150000,0.150000,0.150000}%
\pgfsetstrokecolor{textcolor}%
\pgfsetfillcolor{textcolor}%
\pgftext[x=3.646524in,y=1.451858in,left,base]{\color{textcolor}\sffamily\fontsize{9.000000}{10.800000}\selectfont No CS Degree, 29.4 \%}%
\end{pgfscope}%
\begin{pgfscope}%
\pgfsetbuttcap%
\pgfsetmiterjoin%
\definecolor{currentfill}{rgb}{1.000000,1.000000,0.701961}%
\pgfsetfillcolor{currentfill}%
\pgfsetlinewidth{1.003750pt}%
\definecolor{currentstroke}{rgb}{1.000000,1.000000,1.000000}%
\pgfsetstrokecolor{currentstroke}%
\pgfsetdash{}{0pt}%
\pgfpathmoveto{\pgfqpoint{3.296524in}{1.268387in}}%
\pgfpathlineto{\pgfqpoint{3.546524in}{1.268387in}}%
\pgfpathlineto{\pgfqpoint{3.546524in}{1.355887in}}%
\pgfpathlineto{\pgfqpoint{3.296524in}{1.355887in}}%
\pgfpathlineto{\pgfqpoint{3.296524in}{1.268387in}}%
\pgfpathclose%
\pgfusepath{stroke,fill}%
\end{pgfscope}%
\begin{pgfscope}%
\definecolor{textcolor}{rgb}{0.150000,0.150000,0.150000}%
\pgfsetstrokecolor{textcolor}%
\pgfsetfillcolor{textcolor}%
\pgftext[x=3.646524in,y=1.268387in,left,base]{\color{textcolor}\sffamily\fontsize{9.000000}{10.800000}\selectfont Bachelors's Degree, 29.4 \%}%
\end{pgfscope}%
\begin{pgfscope}%
\pgfsetbuttcap%
\pgfsetmiterjoin%
\definecolor{currentfill}{rgb}{0.745098,0.729412,0.854902}%
\pgfsetfillcolor{currentfill}%
\pgfsetlinewidth{1.003750pt}%
\definecolor{currentstroke}{rgb}{1.000000,1.000000,1.000000}%
\pgfsetstrokecolor{currentstroke}%
\pgfsetdash{}{0pt}%
\pgfpathmoveto{\pgfqpoint{3.296524in}{1.084915in}}%
\pgfpathlineto{\pgfqpoint{3.546524in}{1.084915in}}%
\pgfpathlineto{\pgfqpoint{3.546524in}{1.172415in}}%
\pgfpathlineto{\pgfqpoint{3.296524in}{1.172415in}}%
\pgfpathlineto{\pgfqpoint{3.296524in}{1.084915in}}%
\pgfpathclose%
\pgfusepath{stroke,fill}%
\end{pgfscope}%
\begin{pgfscope}%
\definecolor{textcolor}{rgb}{0.150000,0.150000,0.150000}%
\pgfsetstrokecolor{textcolor}%
\pgfsetfillcolor{textcolor}%
\pgftext[x=3.646524in,y=1.084915in,left,base]{\color{textcolor}\sffamily\fontsize{9.000000}{10.800000}\selectfont Master's Degree, 29.4 \%}%
\end{pgfscope}%
\begin{pgfscope}%
\pgfsetbuttcap%
\pgfsetmiterjoin%
\definecolor{currentfill}{rgb}{0.984314,0.501961,0.447059}%
\pgfsetfillcolor{currentfill}%
\pgfsetlinewidth{1.003750pt}%
\definecolor{currentstroke}{rgb}{1.000000,1.000000,1.000000}%
\pgfsetstrokecolor{currentstroke}%
\pgfsetdash{}{0pt}%
\pgfpathmoveto{\pgfqpoint{3.296524in}{0.901444in}}%
\pgfpathlineto{\pgfqpoint{3.546524in}{0.901444in}}%
\pgfpathlineto{\pgfqpoint{3.546524in}{0.988944in}}%
\pgfpathlineto{\pgfqpoint{3.296524in}{0.988944in}}%
\pgfpathlineto{\pgfqpoint{3.296524in}{0.901444in}}%
\pgfpathclose%
\pgfusepath{stroke,fill}%
\end{pgfscope}%
\begin{pgfscope}%
\definecolor{textcolor}{rgb}{0.150000,0.150000,0.150000}%
\pgfsetstrokecolor{textcolor}%
\pgfsetfillcolor{textcolor}%
\pgftext[x=3.646524in,y=0.901444in,left,base]{\color{textcolor}\sffamily\fontsize{9.000000}{10.800000}\selectfont Doctorate Degree, 0.0 \%}%
\end{pgfscope}%
\begin{pgfscope}%
\pgfsetbuttcap%
\pgfsetmiterjoin%
\definecolor{currentfill}{rgb}{0.501961,0.694118,0.827451}%
\pgfsetfillcolor{currentfill}%
\pgfsetlinewidth{1.003750pt}%
\definecolor{currentstroke}{rgb}{1.000000,1.000000,1.000000}%
\pgfsetstrokecolor{currentstroke}%
\pgfsetdash{}{0pt}%
\pgfpathmoveto{\pgfqpoint{3.296524in}{0.717972in}}%
\pgfpathlineto{\pgfqpoint{3.546524in}{0.717972in}}%
\pgfpathlineto{\pgfqpoint{3.546524in}{0.805472in}}%
\pgfpathlineto{\pgfqpoint{3.296524in}{0.805472in}}%
\pgfpathlineto{\pgfqpoint{3.296524in}{0.717972in}}%
\pgfpathclose%
\pgfusepath{stroke,fill}%
\end{pgfscope}%
\begin{pgfscope}%
\definecolor{textcolor}{rgb}{0.150000,0.150000,0.150000}%
\pgfsetstrokecolor{textcolor}%
\pgfsetfillcolor{textcolor}%
\pgftext[x=3.646524in,y=0.717972in,left,base]{\color{textcolor}\sffamily\fontsize{9.000000}{10.800000}\selectfont Other Engineering Degree, 11.8 \%}%
\end{pgfscope}%
\begin{pgfscope}%
\pgfsetbuttcap%
\pgfsetmiterjoin%
\definecolor{currentfill}{rgb}{0.992157,0.705882,0.384314}%
\pgfsetfillcolor{currentfill}%
\pgfsetlinewidth{1.003750pt}%
\definecolor{currentstroke}{rgb}{1.000000,1.000000,1.000000}%
\pgfsetstrokecolor{currentstroke}%
\pgfsetdash{}{0pt}%
\pgfpathmoveto{\pgfqpoint{3.296524in}{0.534501in}}%
\pgfpathlineto{\pgfqpoint{3.546524in}{0.534501in}}%
\pgfpathlineto{\pgfqpoint{3.546524in}{0.622001in}}%
\pgfpathlineto{\pgfqpoint{3.296524in}{0.622001in}}%
\pgfpathlineto{\pgfqpoint{3.296524in}{0.534501in}}%
\pgfpathclose%
\pgfusepath{stroke,fill}%
\end{pgfscope}%
\begin{pgfscope}%
\definecolor{textcolor}{rgb}{0.150000,0.150000,0.150000}%
\pgfsetstrokecolor{textcolor}%
\pgfsetfillcolor{textcolor}%
\pgftext[x=3.646524in,y=0.534501in,left,base]{\color{textcolor}\sffamily\fontsize{9.000000}{10.800000}\selectfont Yes but did not finish, 0.0 \%}%
\end{pgfscope}%
\end{pgfpicture}%
\makeatother%
\endgroup%
}
	\caption{question test}
	\label{fig:question}
\end{figure}

\begin{figure}[H]
	\centering
	\scalebox{0.67}{%% Creator: Matplotlib, PGF backend
%%
%% To include the figure in your LaTeX document, write
%%   \input{<filename>.pgf}
%%
%% Make sure the required packages are loaded in your preamble
%%   \usepackage{pgf}
%%
%% Also ensure that all the required font packages are loaded; for instance,
%% the lmodern package is sometimes necessary when using math font.
%%   \usepackage{lmodern}
%%
%% Figures using additional raster images can only be included by \input if
%% they are in the same directory as the main LaTeX file. For loading figures
%% from other directories you can use the `import` package
%%   \usepackage{import}
%%
%% and then include the figures with
%%   \import{<path to file>}{<filename>.pgf}
%%
%% Matplotlib used the following preamble
%%   \usepackage{fontspec}
%%   \setmainfont{DejaVuSerif.ttf}[Path=\detokenize{/home/spam/miniconda3/envs/mpl/lib/python3.10/site-packages/matplotlib/mpl-data/fonts/ttf/}]
%%   \setsansfont{DejaVuSans.ttf}[Path=\detokenize{/home/spam/miniconda3/envs/mpl/lib/python3.10/site-packages/matplotlib/mpl-data/fonts/ttf/}]
%%   \setmonofont{DejaVuSansMono.ttf}[Path=\detokenize{/home/spam/miniconda3/envs/mpl/lib/python3.10/site-packages/matplotlib/mpl-data/fonts/ttf/}]
%%
\begingroup%
\makeatletter%
\begin{pgfpicture}%
\pgfpathrectangle{\pgfpointorigin}{\pgfqpoint{5.906660in}{5.000000in}}%
\pgfusepath{use as bounding box, clip}%
\begin{pgfscope}%
\pgfsetbuttcap%
\pgfsetmiterjoin%
\definecolor{currentfill}{rgb}{1.000000,1.000000,1.000000}%
\pgfsetfillcolor{currentfill}%
\pgfsetlinewidth{0.000000pt}%
\definecolor{currentstroke}{rgb}{1.000000,1.000000,1.000000}%
\pgfsetstrokecolor{currentstroke}%
\pgfsetdash{}{0pt}%
\pgfpathmoveto{\pgfqpoint{0.000000in}{0.000000in}}%
\pgfpathlineto{\pgfqpoint{5.906660in}{0.000000in}}%
\pgfpathlineto{\pgfqpoint{5.906660in}{5.000000in}}%
\pgfpathlineto{\pgfqpoint{0.000000in}{5.000000in}}%
\pgfpathlineto{\pgfqpoint{0.000000in}{0.000000in}}%
\pgfpathclose%
\pgfusepath{fill}%
\end{pgfscope}%
\begin{pgfscope}%
\definecolor{textcolor}{rgb}{0.150000,0.150000,0.150000}%
\pgfsetstrokecolor{textcolor}%
\pgfsetfillcolor{textcolor}%
\pgftext[x=3.027163in,y=0.411111in,,top]{\color{textcolor}\sffamily\fontsize{12.000000}{14.400000}\selectfont Computer Science UniversityUniverity Experience}%
\end{pgfscope}%
\begin{pgfscope}%
\pgfsetbuttcap%
\pgfsetmiterjoin%
\definecolor{currentfill}{rgb}{0.552941,0.827451,0.780392}%
\pgfsetfillcolor{currentfill}%
\pgfsetlinewidth{1.003750pt}%
\definecolor{currentstroke}{rgb}{1.000000,1.000000,1.000000}%
\pgfsetstrokecolor{currentstroke}%
\pgfsetdash{}{0pt}%
\pgfpathmoveto{\pgfqpoint{4.567163in}{2.475000in}}%
\pgfpathcurveto{\pgfqpoint{4.567163in}{2.713219in}}{\pgfqpoint{4.511889in}{2.948220in}}{\pgfqpoint{4.405702in}{3.161463in}}%
\pgfpathcurveto{\pgfqpoint{4.299515in}{3.374705in}}{\pgfqpoint{4.145283in}{3.560429in}}{\pgfqpoint{3.955175in}{3.703981in}}%
\pgfpathcurveto{\pgfqpoint{3.765067in}{3.847533in}}{\pgfqpoint{3.544218in}{3.945035in}}{\pgfqpoint{3.310053in}{3.988794in}}%
\pgfpathcurveto{\pgfqpoint{3.075889in}{4.032554in}}{\pgfqpoint{2.834733in}{4.021389in}}{\pgfqpoint{2.605613in}{3.956180in}}%
\pgfpathlineto{\pgfqpoint{3.027163in}{2.475000in}}%
\pgfpathlineto{\pgfqpoint{4.567163in}{2.475000in}}%
\pgfpathlineto{\pgfqpoint{4.567163in}{2.475000in}}%
\pgfpathclose%
\pgfusepath{stroke,fill}%
\end{pgfscope}%
\begin{pgfscope}%
\pgfsetbuttcap%
\pgfsetmiterjoin%
\definecolor{currentfill}{rgb}{1.000000,1.000000,0.701961}%
\pgfsetfillcolor{currentfill}%
\pgfsetlinewidth{1.003750pt}%
\definecolor{currentstroke}{rgb}{1.000000,1.000000,1.000000}%
\pgfsetstrokecolor{currentstroke}%
\pgfsetdash{}{0pt}%
\pgfpathmoveto{\pgfqpoint{2.605613in}{3.956180in}}%
\pgfpathcurveto{\pgfqpoint{2.376493in}{3.890972in}}{\pgfqpoint{2.165597in}{3.773481in}}{\pgfqpoint{1.989567in}{3.612978in}}%
\pgfpathcurveto{\pgfqpoint{1.813536in}{3.452475in}}{\pgfqpoint{1.677124in}{3.253294in}}{\pgfqpoint{1.591094in}{3.031153in}}%
\pgfpathcurveto{\pgfqpoint{1.505064in}{2.809011in}}{\pgfqpoint{1.471740in}{2.569908in}}{\pgfqpoint{1.493751in}{2.332708in}}%
\pgfpathcurveto{\pgfqpoint{1.515762in}{2.095508in}}{\pgfqpoint{1.592513in}{1.866620in}}{\pgfqpoint{1.717949in}{1.664101in}}%
\pgfpathlineto{\pgfqpoint{3.027163in}{2.475000in}}%
\pgfpathlineto{\pgfqpoint{2.605613in}{3.956180in}}%
\pgfpathlineto{\pgfqpoint{2.605613in}{3.956180in}}%
\pgfpathclose%
\pgfusepath{stroke,fill}%
\end{pgfscope}%
\begin{pgfscope}%
\pgfsetbuttcap%
\pgfsetmiterjoin%
\definecolor{currentfill}{rgb}{0.745098,0.729412,0.854902}%
\pgfsetfillcolor{currentfill}%
\pgfsetlinewidth{1.003750pt}%
\definecolor{currentstroke}{rgb}{1.000000,1.000000,1.000000}%
\pgfsetstrokecolor{currentstroke}%
\pgfsetdash{}{0pt}%
\pgfpathmoveto{\pgfqpoint{1.717949in}{1.664101in}}%
\pgfpathcurveto{\pgfqpoint{1.843385in}{1.461582in}}{\pgfqpoint{2.014117in}{1.290903in}}{\pgfqpoint{2.216676in}{1.165531in}}%
\pgfpathcurveto{\pgfqpoint{2.419234in}{1.040158in}}{\pgfqpoint{2.648147in}{0.963479in}}{\pgfqpoint{2.885354in}{0.941543in}}%
\pgfpathcurveto{\pgfqpoint{3.122560in}{0.919607in}}{\pgfqpoint{3.361653in}{0.953006in}}{\pgfqpoint{3.583768in}{1.039106in}}%
\pgfpathcurveto{\pgfqpoint{3.805882in}{1.125206in}}{\pgfqpoint{4.005020in}{1.261680in}}{\pgfqpoint{4.165467in}{1.437762in}}%
\pgfpathlineto{\pgfqpoint{3.027163in}{2.475000in}}%
\pgfpathlineto{\pgfqpoint{1.717949in}{1.664101in}}%
\pgfpathlineto{\pgfqpoint{1.717949in}{1.664101in}}%
\pgfpathclose%
\pgfusepath{stroke,fill}%
\end{pgfscope}%
\begin{pgfscope}%
\pgfsetbuttcap%
\pgfsetmiterjoin%
\definecolor{currentfill}{rgb}{0.984314,0.501961,0.447059}%
\pgfsetfillcolor{currentfill}%
\pgfsetlinewidth{1.003750pt}%
\definecolor{currentstroke}{rgb}{1.000000,1.000000,1.000000}%
\pgfsetstrokecolor{currentstroke}%
\pgfsetdash{}{0pt}%
\pgfpathmoveto{\pgfqpoint{4.165467in}{1.437762in}}%
\pgfpathcurveto{\pgfqpoint{4.165467in}{1.437762in}}{\pgfqpoint{4.165467in}{1.437762in}}{\pgfqpoint{4.165467in}{1.437762in}}%
\pgfpathlineto{\pgfqpoint{3.027163in}{2.475000in}}%
\pgfpathlineto{\pgfqpoint{4.165467in}{1.437762in}}%
\pgfpathlineto{\pgfqpoint{4.165467in}{1.437762in}}%
\pgfpathclose%
\pgfusepath{stroke,fill}%
\end{pgfscope}%
\begin{pgfscope}%
\pgfsetbuttcap%
\pgfsetmiterjoin%
\definecolor{currentfill}{rgb}{0.501961,0.694118,0.827451}%
\pgfsetfillcolor{currentfill}%
\pgfsetlinewidth{1.003750pt}%
\definecolor{currentstroke}{rgb}{1.000000,1.000000,1.000000}%
\pgfsetstrokecolor{currentstroke}%
\pgfsetdash{}{0pt}%
\pgfpathmoveto{\pgfqpoint{4.165467in}{1.437762in}}%
\pgfpathcurveto{\pgfqpoint{4.293579in}{1.578356in}}{\pgfqpoint{4.394541in}{1.741476in}}{\pgfqpoint{4.463232in}{1.918848in}}%
\pgfpathcurveto{\pgfqpoint{4.531924in}{2.096219in}}{\pgfqpoint{4.567163in}{2.284792in}}{\pgfqpoint{4.567163in}{2.475001in}}%
\pgfpathlineto{\pgfqpoint{3.027163in}{2.475000in}}%
\pgfpathlineto{\pgfqpoint{4.165467in}{1.437762in}}%
\pgfpathlineto{\pgfqpoint{4.165467in}{1.437762in}}%
\pgfpathclose%
\pgfusepath{stroke,fill}%
\end{pgfscope}%
\begin{pgfscope}%
\pgfsetbuttcap%
\pgfsetmiterjoin%
\definecolor{currentfill}{rgb}{0.992157,0.705882,0.384314}%
\pgfsetfillcolor{currentfill}%
\pgfsetlinewidth{1.003750pt}%
\definecolor{currentstroke}{rgb}{1.000000,1.000000,1.000000}%
\pgfsetstrokecolor{currentstroke}%
\pgfsetdash{}{0pt}%
\pgfpathmoveto{\pgfqpoint{4.567163in}{2.475001in}}%
\pgfpathcurveto{\pgfqpoint{4.567163in}{2.475001in}}{\pgfqpoint{4.567163in}{2.475001in}}{\pgfqpoint{4.567163in}{2.475001in}}%
\pgfpathlineto{\pgfqpoint{3.027163in}{2.475000in}}%
\pgfpathlineto{\pgfqpoint{4.567163in}{2.475001in}}%
\pgfpathlineto{\pgfqpoint{4.567163in}{2.475001in}}%
\pgfpathclose%
\pgfusepath{stroke,fill}%
\end{pgfscope}%
\begin{pgfscope}%
\definecolor{textcolor}{rgb}{0.150000,0.150000,0.150000}%
\pgfsetstrokecolor{textcolor}%
\pgfsetfillcolor{textcolor}%
\pgftext[x=3.583970in,y=3.212389in,,]{\color{textcolor}\sffamily\fontsize{12.000000}{14.400000}\selectfont 29.4\%}%
\end{pgfscope}%
\begin{pgfscope}%
\definecolor{textcolor}{rgb}{0.150000,0.150000,0.150000}%
\pgfsetstrokecolor{textcolor}%
\pgfsetfillcolor{textcolor}%
\pgftext[x=2.165522in,y=2.808692in,,]{\color{textcolor}\sffamily\fontsize{12.000000}{14.400000}\selectfont 29.4\%}%
\end{pgfscope}%
\begin{pgfscope}%
\definecolor{textcolor}{rgb}{0.150000,0.150000,0.150000}%
\pgfsetstrokecolor{textcolor}%
\pgfsetfillcolor{textcolor}%
\pgftext[x=2.942078in,y=1.554926in,,]{\color{textcolor}\sffamily\fontsize{12.000000}{14.400000}\selectfont 29.4\%}%
\end{pgfscope}%
\begin{pgfscope}%
\definecolor{textcolor}{rgb}{0.150000,0.150000,0.150000}%
\pgfsetstrokecolor{textcolor}%
\pgfsetfillcolor{textcolor}%
\pgftext[x=3.710146in,y=1.852657in,,]{\color{textcolor}\sffamily\fontsize{12.000000}{14.400000}\selectfont 0.0\%}%
\end{pgfscope}%
\begin{pgfscope}%
\definecolor{textcolor}{rgb}{0.150000,0.150000,0.150000}%
\pgfsetstrokecolor{textcolor}%
\pgfsetfillcolor{textcolor}%
\pgftext[x=3.888805in,y=2.141309in,,]{\color{textcolor}\sffamily\fontsize{12.000000}{14.400000}\selectfont 11.8\%}%
\end{pgfscope}%
\begin{pgfscope}%
\definecolor{textcolor}{rgb}{0.150000,0.150000,0.150000}%
\pgfsetstrokecolor{textcolor}%
\pgfsetfillcolor{textcolor}%
\pgftext[x=3.951163in,y=2.475000in,,]{\color{textcolor}\sffamily\fontsize{12.000000}{14.400000}\selectfont 0.0\%}%
\end{pgfscope}%
\begin{pgfscope}%
\pgfsetbuttcap%
\pgfsetmiterjoin%
\definecolor{currentfill}{rgb}{1.000000,1.000000,1.000000}%
\pgfsetfillcolor{currentfill}%
\pgfsetfillopacity{0.800000}%
\pgfsetlinewidth{1.003750pt}%
\definecolor{currentstroke}{rgb}{0.800000,0.800000,0.800000}%
\pgfsetstrokecolor{currentstroke}%
\pgfsetstrokeopacity{0.800000}%
\pgfsetdash{}{0pt}%
\pgfpathmoveto{\pgfqpoint{3.271524in}{0.458500in}}%
\pgfpathlineto{\pgfqpoint{5.827163in}{0.458500in}}%
\pgfpathquadraticcurveto{\pgfqpoint{5.852163in}{0.458500in}}{\pgfqpoint{5.852163in}{0.483500in}}%
\pgfpathlineto{\pgfqpoint{5.852163in}{1.571829in}}%
\pgfpathquadraticcurveto{\pgfqpoint{5.852163in}{1.596829in}}{\pgfqpoint{5.827163in}{1.596829in}}%
\pgfpathlineto{\pgfqpoint{3.271524in}{1.596829in}}%
\pgfpathquadraticcurveto{\pgfqpoint{3.246524in}{1.596829in}}{\pgfqpoint{3.246524in}{1.571829in}}%
\pgfpathlineto{\pgfqpoint{3.246524in}{0.483500in}}%
\pgfpathquadraticcurveto{\pgfqpoint{3.246524in}{0.458500in}}{\pgfqpoint{3.271524in}{0.458500in}}%
\pgfpathlineto{\pgfqpoint{3.271524in}{0.458500in}}%
\pgfpathclose%
\pgfusepath{stroke,fill}%
\end{pgfscope}%
\begin{pgfscope}%
\pgfsetbuttcap%
\pgfsetmiterjoin%
\definecolor{currentfill}{rgb}{0.552941,0.827451,0.780392}%
\pgfsetfillcolor{currentfill}%
\pgfsetlinewidth{1.003750pt}%
\definecolor{currentstroke}{rgb}{1.000000,1.000000,1.000000}%
\pgfsetstrokecolor{currentstroke}%
\pgfsetdash{}{0pt}%
\pgfpathmoveto{\pgfqpoint{3.296524in}{1.451858in}}%
\pgfpathlineto{\pgfqpoint{3.546524in}{1.451858in}}%
\pgfpathlineto{\pgfqpoint{3.546524in}{1.539358in}}%
\pgfpathlineto{\pgfqpoint{3.296524in}{1.539358in}}%
\pgfpathlineto{\pgfqpoint{3.296524in}{1.451858in}}%
\pgfpathclose%
\pgfusepath{stroke,fill}%
\end{pgfscope}%
\begin{pgfscope}%
\definecolor{textcolor}{rgb}{0.150000,0.150000,0.150000}%
\pgfsetstrokecolor{textcolor}%
\pgfsetfillcolor{textcolor}%
\pgftext[x=3.646524in,y=1.451858in,left,base]{\color{textcolor}\sffamily\fontsize{9.000000}{10.800000}\selectfont No CS Degree, 29.4 \%}%
\end{pgfscope}%
\begin{pgfscope}%
\pgfsetbuttcap%
\pgfsetmiterjoin%
\definecolor{currentfill}{rgb}{1.000000,1.000000,0.701961}%
\pgfsetfillcolor{currentfill}%
\pgfsetlinewidth{1.003750pt}%
\definecolor{currentstroke}{rgb}{1.000000,1.000000,1.000000}%
\pgfsetstrokecolor{currentstroke}%
\pgfsetdash{}{0pt}%
\pgfpathmoveto{\pgfqpoint{3.296524in}{1.268387in}}%
\pgfpathlineto{\pgfqpoint{3.546524in}{1.268387in}}%
\pgfpathlineto{\pgfqpoint{3.546524in}{1.355887in}}%
\pgfpathlineto{\pgfqpoint{3.296524in}{1.355887in}}%
\pgfpathlineto{\pgfqpoint{3.296524in}{1.268387in}}%
\pgfpathclose%
\pgfusepath{stroke,fill}%
\end{pgfscope}%
\begin{pgfscope}%
\definecolor{textcolor}{rgb}{0.150000,0.150000,0.150000}%
\pgfsetstrokecolor{textcolor}%
\pgfsetfillcolor{textcolor}%
\pgftext[x=3.646524in,y=1.268387in,left,base]{\color{textcolor}\sffamily\fontsize{9.000000}{10.800000}\selectfont Bachelors's Degree, 29.4 \%}%
\end{pgfscope}%
\begin{pgfscope}%
\pgfsetbuttcap%
\pgfsetmiterjoin%
\definecolor{currentfill}{rgb}{0.745098,0.729412,0.854902}%
\pgfsetfillcolor{currentfill}%
\pgfsetlinewidth{1.003750pt}%
\definecolor{currentstroke}{rgb}{1.000000,1.000000,1.000000}%
\pgfsetstrokecolor{currentstroke}%
\pgfsetdash{}{0pt}%
\pgfpathmoveto{\pgfqpoint{3.296524in}{1.084915in}}%
\pgfpathlineto{\pgfqpoint{3.546524in}{1.084915in}}%
\pgfpathlineto{\pgfqpoint{3.546524in}{1.172415in}}%
\pgfpathlineto{\pgfqpoint{3.296524in}{1.172415in}}%
\pgfpathlineto{\pgfqpoint{3.296524in}{1.084915in}}%
\pgfpathclose%
\pgfusepath{stroke,fill}%
\end{pgfscope}%
\begin{pgfscope}%
\definecolor{textcolor}{rgb}{0.150000,0.150000,0.150000}%
\pgfsetstrokecolor{textcolor}%
\pgfsetfillcolor{textcolor}%
\pgftext[x=3.646524in,y=1.084915in,left,base]{\color{textcolor}\sffamily\fontsize{9.000000}{10.800000}\selectfont Master's Degree, 29.4 \%}%
\end{pgfscope}%
\begin{pgfscope}%
\pgfsetbuttcap%
\pgfsetmiterjoin%
\definecolor{currentfill}{rgb}{0.984314,0.501961,0.447059}%
\pgfsetfillcolor{currentfill}%
\pgfsetlinewidth{1.003750pt}%
\definecolor{currentstroke}{rgb}{1.000000,1.000000,1.000000}%
\pgfsetstrokecolor{currentstroke}%
\pgfsetdash{}{0pt}%
\pgfpathmoveto{\pgfqpoint{3.296524in}{0.901444in}}%
\pgfpathlineto{\pgfqpoint{3.546524in}{0.901444in}}%
\pgfpathlineto{\pgfqpoint{3.546524in}{0.988944in}}%
\pgfpathlineto{\pgfqpoint{3.296524in}{0.988944in}}%
\pgfpathlineto{\pgfqpoint{3.296524in}{0.901444in}}%
\pgfpathclose%
\pgfusepath{stroke,fill}%
\end{pgfscope}%
\begin{pgfscope}%
\definecolor{textcolor}{rgb}{0.150000,0.150000,0.150000}%
\pgfsetstrokecolor{textcolor}%
\pgfsetfillcolor{textcolor}%
\pgftext[x=3.646524in,y=0.901444in,left,base]{\color{textcolor}\sffamily\fontsize{9.000000}{10.800000}\selectfont Doctorate Degree, 0.0 \%}%
\end{pgfscope}%
\begin{pgfscope}%
\pgfsetbuttcap%
\pgfsetmiterjoin%
\definecolor{currentfill}{rgb}{0.501961,0.694118,0.827451}%
\pgfsetfillcolor{currentfill}%
\pgfsetlinewidth{1.003750pt}%
\definecolor{currentstroke}{rgb}{1.000000,1.000000,1.000000}%
\pgfsetstrokecolor{currentstroke}%
\pgfsetdash{}{0pt}%
\pgfpathmoveto{\pgfqpoint{3.296524in}{0.717972in}}%
\pgfpathlineto{\pgfqpoint{3.546524in}{0.717972in}}%
\pgfpathlineto{\pgfqpoint{3.546524in}{0.805472in}}%
\pgfpathlineto{\pgfqpoint{3.296524in}{0.805472in}}%
\pgfpathlineto{\pgfqpoint{3.296524in}{0.717972in}}%
\pgfpathclose%
\pgfusepath{stroke,fill}%
\end{pgfscope}%
\begin{pgfscope}%
\definecolor{textcolor}{rgb}{0.150000,0.150000,0.150000}%
\pgfsetstrokecolor{textcolor}%
\pgfsetfillcolor{textcolor}%
\pgftext[x=3.646524in,y=0.717972in,left,base]{\color{textcolor}\sffamily\fontsize{9.000000}{10.800000}\selectfont Other Engineering Degree, 11.8 \%}%
\end{pgfscope}%
\begin{pgfscope}%
\pgfsetbuttcap%
\pgfsetmiterjoin%
\definecolor{currentfill}{rgb}{0.992157,0.705882,0.384314}%
\pgfsetfillcolor{currentfill}%
\pgfsetlinewidth{1.003750pt}%
\definecolor{currentstroke}{rgb}{1.000000,1.000000,1.000000}%
\pgfsetstrokecolor{currentstroke}%
\pgfsetdash{}{0pt}%
\pgfpathmoveto{\pgfqpoint{3.296524in}{0.534501in}}%
\pgfpathlineto{\pgfqpoint{3.546524in}{0.534501in}}%
\pgfpathlineto{\pgfqpoint{3.546524in}{0.622001in}}%
\pgfpathlineto{\pgfqpoint{3.296524in}{0.622001in}}%
\pgfpathlineto{\pgfqpoint{3.296524in}{0.534501in}}%
\pgfpathclose%
\pgfusepath{stroke,fill}%
\end{pgfscope}%
\begin{pgfscope}%
\definecolor{textcolor}{rgb}{0.150000,0.150000,0.150000}%
\pgfsetstrokecolor{textcolor}%
\pgfsetfillcolor{textcolor}%
\pgftext[x=3.646524in,y=0.534501in,left,base]{\color{textcolor}\sffamily\fontsize{9.000000}{10.800000}\selectfont Yes but did not finish, 0.0 \%}%
\end{pgfscope}%
\end{pgfpicture}%
\makeatother%
\endgroup%
}
	\vspace{-4em}
	\caption{confidence}
	\label{fig:question}
\end{figure}

\begin{figure}[H]
	\centering
	\scalebox{0.67}{%% Creator: Matplotlib, PGF backend
%%
%% To include the figure in your LaTeX document, write
%%   \input{<filename>.pgf}
%%
%% Make sure the required packages are loaded in your preamble
%%   \usepackage{pgf}
%%
%% Also ensure that all the required font packages are loaded; for instance,
%% the lmodern package is sometimes necessary when using math font.
%%   \usepackage{lmodern}
%%
%% Figures using additional raster images can only be included by \input if
%% they are in the same directory as the main LaTeX file. For loading figures
%% from other directories you can use the `import` package
%%   \usepackage{import}
%%
%% and then include the figures with
%%   \import{<path to file>}{<filename>.pgf}
%%
%% Matplotlib used the following preamble
%%   \usepackage{fontspec}
%%   \setmainfont{DejaVuSerif.ttf}[Path=\detokenize{/home/spam/miniconda3/envs/mpl/lib/python3.10/site-packages/matplotlib/mpl-data/fonts/ttf/}]
%%   \setsansfont{DejaVuSans.ttf}[Path=\detokenize{/home/spam/miniconda3/envs/mpl/lib/python3.10/site-packages/matplotlib/mpl-data/fonts/ttf/}]
%%   \setmonofont{DejaVuSansMono.ttf}[Path=\detokenize{/home/spam/miniconda3/envs/mpl/lib/python3.10/site-packages/matplotlib/mpl-data/fonts/ttf/}]
%%
\begingroup%
\makeatletter%
\begin{pgfpicture}%
\pgfpathrectangle{\pgfpointorigin}{\pgfqpoint{5.906660in}{5.000000in}}%
\pgfusepath{use as bounding box, clip}%
\begin{pgfscope}%
\pgfsetbuttcap%
\pgfsetmiterjoin%
\definecolor{currentfill}{rgb}{1.000000,1.000000,1.000000}%
\pgfsetfillcolor{currentfill}%
\pgfsetlinewidth{0.000000pt}%
\definecolor{currentstroke}{rgb}{1.000000,1.000000,1.000000}%
\pgfsetstrokecolor{currentstroke}%
\pgfsetdash{}{0pt}%
\pgfpathmoveto{\pgfqpoint{0.000000in}{0.000000in}}%
\pgfpathlineto{\pgfqpoint{5.906660in}{0.000000in}}%
\pgfpathlineto{\pgfqpoint{5.906660in}{5.000000in}}%
\pgfpathlineto{\pgfqpoint{0.000000in}{5.000000in}}%
\pgfpathlineto{\pgfqpoint{0.000000in}{0.000000in}}%
\pgfpathclose%
\pgfusepath{fill}%
\end{pgfscope}%
\begin{pgfscope}%
\definecolor{textcolor}{rgb}{0.150000,0.150000,0.150000}%
\pgfsetstrokecolor{textcolor}%
\pgfsetfillcolor{textcolor}%
\pgftext[x=3.027163in,y=0.411111in,,top]{\color{textcolor}\sffamily\fontsize{12.000000}{14.400000}\selectfont Computer Science UniversityUniverity Experience}%
\end{pgfscope}%
\begin{pgfscope}%
\pgfsetbuttcap%
\pgfsetmiterjoin%
\definecolor{currentfill}{rgb}{0.552941,0.827451,0.780392}%
\pgfsetfillcolor{currentfill}%
\pgfsetlinewidth{1.003750pt}%
\definecolor{currentstroke}{rgb}{1.000000,1.000000,1.000000}%
\pgfsetstrokecolor{currentstroke}%
\pgfsetdash{}{0pt}%
\pgfpathmoveto{\pgfqpoint{4.567163in}{2.475000in}}%
\pgfpathcurveto{\pgfqpoint{4.567163in}{2.713219in}}{\pgfqpoint{4.511889in}{2.948220in}}{\pgfqpoint{4.405702in}{3.161463in}}%
\pgfpathcurveto{\pgfqpoint{4.299515in}{3.374705in}}{\pgfqpoint{4.145283in}{3.560429in}}{\pgfqpoint{3.955175in}{3.703981in}}%
\pgfpathcurveto{\pgfqpoint{3.765067in}{3.847533in}}{\pgfqpoint{3.544218in}{3.945035in}}{\pgfqpoint{3.310053in}{3.988794in}}%
\pgfpathcurveto{\pgfqpoint{3.075889in}{4.032554in}}{\pgfqpoint{2.834733in}{4.021389in}}{\pgfqpoint{2.605613in}{3.956180in}}%
\pgfpathlineto{\pgfqpoint{3.027163in}{2.475000in}}%
\pgfpathlineto{\pgfqpoint{4.567163in}{2.475000in}}%
\pgfpathlineto{\pgfqpoint{4.567163in}{2.475000in}}%
\pgfpathclose%
\pgfusepath{stroke,fill}%
\end{pgfscope}%
\begin{pgfscope}%
\pgfsetbuttcap%
\pgfsetmiterjoin%
\definecolor{currentfill}{rgb}{1.000000,1.000000,0.701961}%
\pgfsetfillcolor{currentfill}%
\pgfsetlinewidth{1.003750pt}%
\definecolor{currentstroke}{rgb}{1.000000,1.000000,1.000000}%
\pgfsetstrokecolor{currentstroke}%
\pgfsetdash{}{0pt}%
\pgfpathmoveto{\pgfqpoint{2.605613in}{3.956180in}}%
\pgfpathcurveto{\pgfqpoint{2.376493in}{3.890972in}}{\pgfqpoint{2.165597in}{3.773481in}}{\pgfqpoint{1.989567in}{3.612978in}}%
\pgfpathcurveto{\pgfqpoint{1.813536in}{3.452475in}}{\pgfqpoint{1.677124in}{3.253294in}}{\pgfqpoint{1.591094in}{3.031153in}}%
\pgfpathcurveto{\pgfqpoint{1.505064in}{2.809011in}}{\pgfqpoint{1.471740in}{2.569908in}}{\pgfqpoint{1.493751in}{2.332708in}}%
\pgfpathcurveto{\pgfqpoint{1.515762in}{2.095508in}}{\pgfqpoint{1.592513in}{1.866620in}}{\pgfqpoint{1.717949in}{1.664101in}}%
\pgfpathlineto{\pgfqpoint{3.027163in}{2.475000in}}%
\pgfpathlineto{\pgfqpoint{2.605613in}{3.956180in}}%
\pgfpathlineto{\pgfqpoint{2.605613in}{3.956180in}}%
\pgfpathclose%
\pgfusepath{stroke,fill}%
\end{pgfscope}%
\begin{pgfscope}%
\pgfsetbuttcap%
\pgfsetmiterjoin%
\definecolor{currentfill}{rgb}{0.745098,0.729412,0.854902}%
\pgfsetfillcolor{currentfill}%
\pgfsetlinewidth{1.003750pt}%
\definecolor{currentstroke}{rgb}{1.000000,1.000000,1.000000}%
\pgfsetstrokecolor{currentstroke}%
\pgfsetdash{}{0pt}%
\pgfpathmoveto{\pgfqpoint{1.717949in}{1.664101in}}%
\pgfpathcurveto{\pgfqpoint{1.843385in}{1.461582in}}{\pgfqpoint{2.014117in}{1.290903in}}{\pgfqpoint{2.216676in}{1.165531in}}%
\pgfpathcurveto{\pgfqpoint{2.419234in}{1.040158in}}{\pgfqpoint{2.648147in}{0.963479in}}{\pgfqpoint{2.885354in}{0.941543in}}%
\pgfpathcurveto{\pgfqpoint{3.122560in}{0.919607in}}{\pgfqpoint{3.361653in}{0.953006in}}{\pgfqpoint{3.583768in}{1.039106in}}%
\pgfpathcurveto{\pgfqpoint{3.805882in}{1.125206in}}{\pgfqpoint{4.005020in}{1.261680in}}{\pgfqpoint{4.165467in}{1.437762in}}%
\pgfpathlineto{\pgfqpoint{3.027163in}{2.475000in}}%
\pgfpathlineto{\pgfqpoint{1.717949in}{1.664101in}}%
\pgfpathlineto{\pgfqpoint{1.717949in}{1.664101in}}%
\pgfpathclose%
\pgfusepath{stroke,fill}%
\end{pgfscope}%
\begin{pgfscope}%
\pgfsetbuttcap%
\pgfsetmiterjoin%
\definecolor{currentfill}{rgb}{0.984314,0.501961,0.447059}%
\pgfsetfillcolor{currentfill}%
\pgfsetlinewidth{1.003750pt}%
\definecolor{currentstroke}{rgb}{1.000000,1.000000,1.000000}%
\pgfsetstrokecolor{currentstroke}%
\pgfsetdash{}{0pt}%
\pgfpathmoveto{\pgfqpoint{4.165467in}{1.437762in}}%
\pgfpathcurveto{\pgfqpoint{4.165467in}{1.437762in}}{\pgfqpoint{4.165467in}{1.437762in}}{\pgfqpoint{4.165467in}{1.437762in}}%
\pgfpathlineto{\pgfqpoint{3.027163in}{2.475000in}}%
\pgfpathlineto{\pgfqpoint{4.165467in}{1.437762in}}%
\pgfpathlineto{\pgfqpoint{4.165467in}{1.437762in}}%
\pgfpathclose%
\pgfusepath{stroke,fill}%
\end{pgfscope}%
\begin{pgfscope}%
\pgfsetbuttcap%
\pgfsetmiterjoin%
\definecolor{currentfill}{rgb}{0.501961,0.694118,0.827451}%
\pgfsetfillcolor{currentfill}%
\pgfsetlinewidth{1.003750pt}%
\definecolor{currentstroke}{rgb}{1.000000,1.000000,1.000000}%
\pgfsetstrokecolor{currentstroke}%
\pgfsetdash{}{0pt}%
\pgfpathmoveto{\pgfqpoint{4.165467in}{1.437762in}}%
\pgfpathcurveto{\pgfqpoint{4.293579in}{1.578356in}}{\pgfqpoint{4.394541in}{1.741476in}}{\pgfqpoint{4.463232in}{1.918848in}}%
\pgfpathcurveto{\pgfqpoint{4.531924in}{2.096219in}}{\pgfqpoint{4.567163in}{2.284792in}}{\pgfqpoint{4.567163in}{2.475001in}}%
\pgfpathlineto{\pgfqpoint{3.027163in}{2.475000in}}%
\pgfpathlineto{\pgfqpoint{4.165467in}{1.437762in}}%
\pgfpathlineto{\pgfqpoint{4.165467in}{1.437762in}}%
\pgfpathclose%
\pgfusepath{stroke,fill}%
\end{pgfscope}%
\begin{pgfscope}%
\pgfsetbuttcap%
\pgfsetmiterjoin%
\definecolor{currentfill}{rgb}{0.992157,0.705882,0.384314}%
\pgfsetfillcolor{currentfill}%
\pgfsetlinewidth{1.003750pt}%
\definecolor{currentstroke}{rgb}{1.000000,1.000000,1.000000}%
\pgfsetstrokecolor{currentstroke}%
\pgfsetdash{}{0pt}%
\pgfpathmoveto{\pgfqpoint{4.567163in}{2.475001in}}%
\pgfpathcurveto{\pgfqpoint{4.567163in}{2.475001in}}{\pgfqpoint{4.567163in}{2.475001in}}{\pgfqpoint{4.567163in}{2.475001in}}%
\pgfpathlineto{\pgfqpoint{3.027163in}{2.475000in}}%
\pgfpathlineto{\pgfqpoint{4.567163in}{2.475001in}}%
\pgfpathlineto{\pgfqpoint{4.567163in}{2.475001in}}%
\pgfpathclose%
\pgfusepath{stroke,fill}%
\end{pgfscope}%
\begin{pgfscope}%
\definecolor{textcolor}{rgb}{0.150000,0.150000,0.150000}%
\pgfsetstrokecolor{textcolor}%
\pgfsetfillcolor{textcolor}%
\pgftext[x=3.583970in,y=3.212389in,,]{\color{textcolor}\sffamily\fontsize{12.000000}{14.400000}\selectfont 29.4\%}%
\end{pgfscope}%
\begin{pgfscope}%
\definecolor{textcolor}{rgb}{0.150000,0.150000,0.150000}%
\pgfsetstrokecolor{textcolor}%
\pgfsetfillcolor{textcolor}%
\pgftext[x=2.165522in,y=2.808692in,,]{\color{textcolor}\sffamily\fontsize{12.000000}{14.400000}\selectfont 29.4\%}%
\end{pgfscope}%
\begin{pgfscope}%
\definecolor{textcolor}{rgb}{0.150000,0.150000,0.150000}%
\pgfsetstrokecolor{textcolor}%
\pgfsetfillcolor{textcolor}%
\pgftext[x=2.942078in,y=1.554926in,,]{\color{textcolor}\sffamily\fontsize{12.000000}{14.400000}\selectfont 29.4\%}%
\end{pgfscope}%
\begin{pgfscope}%
\definecolor{textcolor}{rgb}{0.150000,0.150000,0.150000}%
\pgfsetstrokecolor{textcolor}%
\pgfsetfillcolor{textcolor}%
\pgftext[x=3.710146in,y=1.852657in,,]{\color{textcolor}\sffamily\fontsize{12.000000}{14.400000}\selectfont 0.0\%}%
\end{pgfscope}%
\begin{pgfscope}%
\definecolor{textcolor}{rgb}{0.150000,0.150000,0.150000}%
\pgfsetstrokecolor{textcolor}%
\pgfsetfillcolor{textcolor}%
\pgftext[x=3.888805in,y=2.141309in,,]{\color{textcolor}\sffamily\fontsize{12.000000}{14.400000}\selectfont 11.8\%}%
\end{pgfscope}%
\begin{pgfscope}%
\definecolor{textcolor}{rgb}{0.150000,0.150000,0.150000}%
\pgfsetstrokecolor{textcolor}%
\pgfsetfillcolor{textcolor}%
\pgftext[x=3.951163in,y=2.475000in,,]{\color{textcolor}\sffamily\fontsize{12.000000}{14.400000}\selectfont 0.0\%}%
\end{pgfscope}%
\begin{pgfscope}%
\pgfsetbuttcap%
\pgfsetmiterjoin%
\definecolor{currentfill}{rgb}{1.000000,1.000000,1.000000}%
\pgfsetfillcolor{currentfill}%
\pgfsetfillopacity{0.800000}%
\pgfsetlinewidth{1.003750pt}%
\definecolor{currentstroke}{rgb}{0.800000,0.800000,0.800000}%
\pgfsetstrokecolor{currentstroke}%
\pgfsetstrokeopacity{0.800000}%
\pgfsetdash{}{0pt}%
\pgfpathmoveto{\pgfqpoint{3.271524in}{0.458500in}}%
\pgfpathlineto{\pgfqpoint{5.827163in}{0.458500in}}%
\pgfpathquadraticcurveto{\pgfqpoint{5.852163in}{0.458500in}}{\pgfqpoint{5.852163in}{0.483500in}}%
\pgfpathlineto{\pgfqpoint{5.852163in}{1.571829in}}%
\pgfpathquadraticcurveto{\pgfqpoint{5.852163in}{1.596829in}}{\pgfqpoint{5.827163in}{1.596829in}}%
\pgfpathlineto{\pgfqpoint{3.271524in}{1.596829in}}%
\pgfpathquadraticcurveto{\pgfqpoint{3.246524in}{1.596829in}}{\pgfqpoint{3.246524in}{1.571829in}}%
\pgfpathlineto{\pgfqpoint{3.246524in}{0.483500in}}%
\pgfpathquadraticcurveto{\pgfqpoint{3.246524in}{0.458500in}}{\pgfqpoint{3.271524in}{0.458500in}}%
\pgfpathlineto{\pgfqpoint{3.271524in}{0.458500in}}%
\pgfpathclose%
\pgfusepath{stroke,fill}%
\end{pgfscope}%
\begin{pgfscope}%
\pgfsetbuttcap%
\pgfsetmiterjoin%
\definecolor{currentfill}{rgb}{0.552941,0.827451,0.780392}%
\pgfsetfillcolor{currentfill}%
\pgfsetlinewidth{1.003750pt}%
\definecolor{currentstroke}{rgb}{1.000000,1.000000,1.000000}%
\pgfsetstrokecolor{currentstroke}%
\pgfsetdash{}{0pt}%
\pgfpathmoveto{\pgfqpoint{3.296524in}{1.451858in}}%
\pgfpathlineto{\pgfqpoint{3.546524in}{1.451858in}}%
\pgfpathlineto{\pgfqpoint{3.546524in}{1.539358in}}%
\pgfpathlineto{\pgfqpoint{3.296524in}{1.539358in}}%
\pgfpathlineto{\pgfqpoint{3.296524in}{1.451858in}}%
\pgfpathclose%
\pgfusepath{stroke,fill}%
\end{pgfscope}%
\begin{pgfscope}%
\definecolor{textcolor}{rgb}{0.150000,0.150000,0.150000}%
\pgfsetstrokecolor{textcolor}%
\pgfsetfillcolor{textcolor}%
\pgftext[x=3.646524in,y=1.451858in,left,base]{\color{textcolor}\sffamily\fontsize{9.000000}{10.800000}\selectfont No CS Degree, 29.4 \%}%
\end{pgfscope}%
\begin{pgfscope}%
\pgfsetbuttcap%
\pgfsetmiterjoin%
\definecolor{currentfill}{rgb}{1.000000,1.000000,0.701961}%
\pgfsetfillcolor{currentfill}%
\pgfsetlinewidth{1.003750pt}%
\definecolor{currentstroke}{rgb}{1.000000,1.000000,1.000000}%
\pgfsetstrokecolor{currentstroke}%
\pgfsetdash{}{0pt}%
\pgfpathmoveto{\pgfqpoint{3.296524in}{1.268387in}}%
\pgfpathlineto{\pgfqpoint{3.546524in}{1.268387in}}%
\pgfpathlineto{\pgfqpoint{3.546524in}{1.355887in}}%
\pgfpathlineto{\pgfqpoint{3.296524in}{1.355887in}}%
\pgfpathlineto{\pgfqpoint{3.296524in}{1.268387in}}%
\pgfpathclose%
\pgfusepath{stroke,fill}%
\end{pgfscope}%
\begin{pgfscope}%
\definecolor{textcolor}{rgb}{0.150000,0.150000,0.150000}%
\pgfsetstrokecolor{textcolor}%
\pgfsetfillcolor{textcolor}%
\pgftext[x=3.646524in,y=1.268387in,left,base]{\color{textcolor}\sffamily\fontsize{9.000000}{10.800000}\selectfont Bachelors's Degree, 29.4 \%}%
\end{pgfscope}%
\begin{pgfscope}%
\pgfsetbuttcap%
\pgfsetmiterjoin%
\definecolor{currentfill}{rgb}{0.745098,0.729412,0.854902}%
\pgfsetfillcolor{currentfill}%
\pgfsetlinewidth{1.003750pt}%
\definecolor{currentstroke}{rgb}{1.000000,1.000000,1.000000}%
\pgfsetstrokecolor{currentstroke}%
\pgfsetdash{}{0pt}%
\pgfpathmoveto{\pgfqpoint{3.296524in}{1.084915in}}%
\pgfpathlineto{\pgfqpoint{3.546524in}{1.084915in}}%
\pgfpathlineto{\pgfqpoint{3.546524in}{1.172415in}}%
\pgfpathlineto{\pgfqpoint{3.296524in}{1.172415in}}%
\pgfpathlineto{\pgfqpoint{3.296524in}{1.084915in}}%
\pgfpathclose%
\pgfusepath{stroke,fill}%
\end{pgfscope}%
\begin{pgfscope}%
\definecolor{textcolor}{rgb}{0.150000,0.150000,0.150000}%
\pgfsetstrokecolor{textcolor}%
\pgfsetfillcolor{textcolor}%
\pgftext[x=3.646524in,y=1.084915in,left,base]{\color{textcolor}\sffamily\fontsize{9.000000}{10.800000}\selectfont Master's Degree, 29.4 \%}%
\end{pgfscope}%
\begin{pgfscope}%
\pgfsetbuttcap%
\pgfsetmiterjoin%
\definecolor{currentfill}{rgb}{0.984314,0.501961,0.447059}%
\pgfsetfillcolor{currentfill}%
\pgfsetlinewidth{1.003750pt}%
\definecolor{currentstroke}{rgb}{1.000000,1.000000,1.000000}%
\pgfsetstrokecolor{currentstroke}%
\pgfsetdash{}{0pt}%
\pgfpathmoveto{\pgfqpoint{3.296524in}{0.901444in}}%
\pgfpathlineto{\pgfqpoint{3.546524in}{0.901444in}}%
\pgfpathlineto{\pgfqpoint{3.546524in}{0.988944in}}%
\pgfpathlineto{\pgfqpoint{3.296524in}{0.988944in}}%
\pgfpathlineto{\pgfqpoint{3.296524in}{0.901444in}}%
\pgfpathclose%
\pgfusepath{stroke,fill}%
\end{pgfscope}%
\begin{pgfscope}%
\definecolor{textcolor}{rgb}{0.150000,0.150000,0.150000}%
\pgfsetstrokecolor{textcolor}%
\pgfsetfillcolor{textcolor}%
\pgftext[x=3.646524in,y=0.901444in,left,base]{\color{textcolor}\sffamily\fontsize{9.000000}{10.800000}\selectfont Doctorate Degree, 0.0 \%}%
\end{pgfscope}%
\begin{pgfscope}%
\pgfsetbuttcap%
\pgfsetmiterjoin%
\definecolor{currentfill}{rgb}{0.501961,0.694118,0.827451}%
\pgfsetfillcolor{currentfill}%
\pgfsetlinewidth{1.003750pt}%
\definecolor{currentstroke}{rgb}{1.000000,1.000000,1.000000}%
\pgfsetstrokecolor{currentstroke}%
\pgfsetdash{}{0pt}%
\pgfpathmoveto{\pgfqpoint{3.296524in}{0.717972in}}%
\pgfpathlineto{\pgfqpoint{3.546524in}{0.717972in}}%
\pgfpathlineto{\pgfqpoint{3.546524in}{0.805472in}}%
\pgfpathlineto{\pgfqpoint{3.296524in}{0.805472in}}%
\pgfpathlineto{\pgfqpoint{3.296524in}{0.717972in}}%
\pgfpathclose%
\pgfusepath{stroke,fill}%
\end{pgfscope}%
\begin{pgfscope}%
\definecolor{textcolor}{rgb}{0.150000,0.150000,0.150000}%
\pgfsetstrokecolor{textcolor}%
\pgfsetfillcolor{textcolor}%
\pgftext[x=3.646524in,y=0.717972in,left,base]{\color{textcolor}\sffamily\fontsize{9.000000}{10.800000}\selectfont Other Engineering Degree, 11.8 \%}%
\end{pgfscope}%
\begin{pgfscope}%
\pgfsetbuttcap%
\pgfsetmiterjoin%
\definecolor{currentfill}{rgb}{0.992157,0.705882,0.384314}%
\pgfsetfillcolor{currentfill}%
\pgfsetlinewidth{1.003750pt}%
\definecolor{currentstroke}{rgb}{1.000000,1.000000,1.000000}%
\pgfsetstrokecolor{currentstroke}%
\pgfsetdash{}{0pt}%
\pgfpathmoveto{\pgfqpoint{3.296524in}{0.534501in}}%
\pgfpathlineto{\pgfqpoint{3.546524in}{0.534501in}}%
\pgfpathlineto{\pgfqpoint{3.546524in}{0.622001in}}%
\pgfpathlineto{\pgfqpoint{3.296524in}{0.622001in}}%
\pgfpathlineto{\pgfqpoint{3.296524in}{0.534501in}}%
\pgfpathclose%
\pgfusepath{stroke,fill}%
\end{pgfscope}%
\begin{pgfscope}%
\definecolor{textcolor}{rgb}{0.150000,0.150000,0.150000}%
\pgfsetstrokecolor{textcolor}%
\pgfsetfillcolor{textcolor}%
\pgftext[x=3.646524in,y=0.534501in,left,base]{\color{textcolor}\sffamily\fontsize{9.000000}{10.800000}\selectfont Yes but did not finish, 0.0 \%}%
\end{pgfscope}%
\end{pgfpicture}%
\makeatother%
\endgroup%
}
	\caption{On average how often do you use command line applications or terminal based tools?}
	\label{fig:question}
\end{figure}

\begin{figure}[H]
	\centering
	\scalebox{0.67}{%% Creator: Matplotlib, PGF backend
%%
%% To include the figure in your LaTeX document, write
%%   \input{<filename>.pgf}
%%
%% Make sure the required packages are loaded in your preamble
%%   \usepackage{pgf}
%%
%% Also ensure that all the required font packages are loaded; for instance,
%% the lmodern package is sometimes necessary when using math font.
%%   \usepackage{lmodern}
%%
%% Figures using additional raster images can only be included by \input if
%% they are in the same directory as the main LaTeX file. For loading figures
%% from other directories you can use the `import` package
%%   \usepackage{import}
%%
%% and then include the figures with
%%   \import{<path to file>}{<filename>.pgf}
%%
%% Matplotlib used the following preamble
%%   \usepackage{fontspec}
%%   \setmainfont{DejaVuSerif.ttf}[Path=\detokenize{/home/spam/miniconda3/envs/mpl/lib/python3.10/site-packages/matplotlib/mpl-data/fonts/ttf/}]
%%   \setsansfont{DejaVuSans.ttf}[Path=\detokenize{/home/spam/miniconda3/envs/mpl/lib/python3.10/site-packages/matplotlib/mpl-data/fonts/ttf/}]
%%   \setmonofont{DejaVuSansMono.ttf}[Path=\detokenize{/home/spam/miniconda3/envs/mpl/lib/python3.10/site-packages/matplotlib/mpl-data/fonts/ttf/}]
%%
\begingroup%
\makeatletter%
\begin{pgfpicture}%
\pgfpathrectangle{\pgfpointorigin}{\pgfqpoint{5.906660in}{5.000000in}}%
\pgfusepath{use as bounding box, clip}%
\begin{pgfscope}%
\pgfsetbuttcap%
\pgfsetmiterjoin%
\definecolor{currentfill}{rgb}{1.000000,1.000000,1.000000}%
\pgfsetfillcolor{currentfill}%
\pgfsetlinewidth{0.000000pt}%
\definecolor{currentstroke}{rgb}{1.000000,1.000000,1.000000}%
\pgfsetstrokecolor{currentstroke}%
\pgfsetdash{}{0pt}%
\pgfpathmoveto{\pgfqpoint{0.000000in}{0.000000in}}%
\pgfpathlineto{\pgfqpoint{5.906660in}{0.000000in}}%
\pgfpathlineto{\pgfqpoint{5.906660in}{5.000000in}}%
\pgfpathlineto{\pgfqpoint{0.000000in}{5.000000in}}%
\pgfpathlineto{\pgfqpoint{0.000000in}{0.000000in}}%
\pgfpathclose%
\pgfusepath{fill}%
\end{pgfscope}%
\begin{pgfscope}%
\definecolor{textcolor}{rgb}{0.150000,0.150000,0.150000}%
\pgfsetstrokecolor{textcolor}%
\pgfsetfillcolor{textcolor}%
\pgftext[x=3.027163in,y=0.411111in,,top]{\color{textcolor}\sffamily\fontsize{12.000000}{14.400000}\selectfont Computer Science UniversityUniverity Experience}%
\end{pgfscope}%
\begin{pgfscope}%
\pgfsetbuttcap%
\pgfsetmiterjoin%
\definecolor{currentfill}{rgb}{0.552941,0.827451,0.780392}%
\pgfsetfillcolor{currentfill}%
\pgfsetlinewidth{1.003750pt}%
\definecolor{currentstroke}{rgb}{1.000000,1.000000,1.000000}%
\pgfsetstrokecolor{currentstroke}%
\pgfsetdash{}{0pt}%
\pgfpathmoveto{\pgfqpoint{4.567163in}{2.475000in}}%
\pgfpathcurveto{\pgfqpoint{4.567163in}{2.713219in}}{\pgfqpoint{4.511889in}{2.948220in}}{\pgfqpoint{4.405702in}{3.161463in}}%
\pgfpathcurveto{\pgfqpoint{4.299515in}{3.374705in}}{\pgfqpoint{4.145283in}{3.560429in}}{\pgfqpoint{3.955175in}{3.703981in}}%
\pgfpathcurveto{\pgfqpoint{3.765067in}{3.847533in}}{\pgfqpoint{3.544218in}{3.945035in}}{\pgfqpoint{3.310053in}{3.988794in}}%
\pgfpathcurveto{\pgfqpoint{3.075889in}{4.032554in}}{\pgfqpoint{2.834733in}{4.021389in}}{\pgfqpoint{2.605613in}{3.956180in}}%
\pgfpathlineto{\pgfqpoint{3.027163in}{2.475000in}}%
\pgfpathlineto{\pgfqpoint{4.567163in}{2.475000in}}%
\pgfpathlineto{\pgfqpoint{4.567163in}{2.475000in}}%
\pgfpathclose%
\pgfusepath{stroke,fill}%
\end{pgfscope}%
\begin{pgfscope}%
\pgfsetbuttcap%
\pgfsetmiterjoin%
\definecolor{currentfill}{rgb}{1.000000,1.000000,0.701961}%
\pgfsetfillcolor{currentfill}%
\pgfsetlinewidth{1.003750pt}%
\definecolor{currentstroke}{rgb}{1.000000,1.000000,1.000000}%
\pgfsetstrokecolor{currentstroke}%
\pgfsetdash{}{0pt}%
\pgfpathmoveto{\pgfqpoint{2.605613in}{3.956180in}}%
\pgfpathcurveto{\pgfqpoint{2.376493in}{3.890972in}}{\pgfqpoint{2.165597in}{3.773481in}}{\pgfqpoint{1.989567in}{3.612978in}}%
\pgfpathcurveto{\pgfqpoint{1.813536in}{3.452475in}}{\pgfqpoint{1.677124in}{3.253294in}}{\pgfqpoint{1.591094in}{3.031153in}}%
\pgfpathcurveto{\pgfqpoint{1.505064in}{2.809011in}}{\pgfqpoint{1.471740in}{2.569908in}}{\pgfqpoint{1.493751in}{2.332708in}}%
\pgfpathcurveto{\pgfqpoint{1.515762in}{2.095508in}}{\pgfqpoint{1.592513in}{1.866620in}}{\pgfqpoint{1.717949in}{1.664101in}}%
\pgfpathlineto{\pgfqpoint{3.027163in}{2.475000in}}%
\pgfpathlineto{\pgfqpoint{2.605613in}{3.956180in}}%
\pgfpathlineto{\pgfqpoint{2.605613in}{3.956180in}}%
\pgfpathclose%
\pgfusepath{stroke,fill}%
\end{pgfscope}%
\begin{pgfscope}%
\pgfsetbuttcap%
\pgfsetmiterjoin%
\definecolor{currentfill}{rgb}{0.745098,0.729412,0.854902}%
\pgfsetfillcolor{currentfill}%
\pgfsetlinewidth{1.003750pt}%
\definecolor{currentstroke}{rgb}{1.000000,1.000000,1.000000}%
\pgfsetstrokecolor{currentstroke}%
\pgfsetdash{}{0pt}%
\pgfpathmoveto{\pgfqpoint{1.717949in}{1.664101in}}%
\pgfpathcurveto{\pgfqpoint{1.843385in}{1.461582in}}{\pgfqpoint{2.014117in}{1.290903in}}{\pgfqpoint{2.216676in}{1.165531in}}%
\pgfpathcurveto{\pgfqpoint{2.419234in}{1.040158in}}{\pgfqpoint{2.648147in}{0.963479in}}{\pgfqpoint{2.885354in}{0.941543in}}%
\pgfpathcurveto{\pgfqpoint{3.122560in}{0.919607in}}{\pgfqpoint{3.361653in}{0.953006in}}{\pgfqpoint{3.583768in}{1.039106in}}%
\pgfpathcurveto{\pgfqpoint{3.805882in}{1.125206in}}{\pgfqpoint{4.005020in}{1.261680in}}{\pgfqpoint{4.165467in}{1.437762in}}%
\pgfpathlineto{\pgfqpoint{3.027163in}{2.475000in}}%
\pgfpathlineto{\pgfqpoint{1.717949in}{1.664101in}}%
\pgfpathlineto{\pgfqpoint{1.717949in}{1.664101in}}%
\pgfpathclose%
\pgfusepath{stroke,fill}%
\end{pgfscope}%
\begin{pgfscope}%
\pgfsetbuttcap%
\pgfsetmiterjoin%
\definecolor{currentfill}{rgb}{0.984314,0.501961,0.447059}%
\pgfsetfillcolor{currentfill}%
\pgfsetlinewidth{1.003750pt}%
\definecolor{currentstroke}{rgb}{1.000000,1.000000,1.000000}%
\pgfsetstrokecolor{currentstroke}%
\pgfsetdash{}{0pt}%
\pgfpathmoveto{\pgfqpoint{4.165467in}{1.437762in}}%
\pgfpathcurveto{\pgfqpoint{4.165467in}{1.437762in}}{\pgfqpoint{4.165467in}{1.437762in}}{\pgfqpoint{4.165467in}{1.437762in}}%
\pgfpathlineto{\pgfqpoint{3.027163in}{2.475000in}}%
\pgfpathlineto{\pgfqpoint{4.165467in}{1.437762in}}%
\pgfpathlineto{\pgfqpoint{4.165467in}{1.437762in}}%
\pgfpathclose%
\pgfusepath{stroke,fill}%
\end{pgfscope}%
\begin{pgfscope}%
\pgfsetbuttcap%
\pgfsetmiterjoin%
\definecolor{currentfill}{rgb}{0.501961,0.694118,0.827451}%
\pgfsetfillcolor{currentfill}%
\pgfsetlinewidth{1.003750pt}%
\definecolor{currentstroke}{rgb}{1.000000,1.000000,1.000000}%
\pgfsetstrokecolor{currentstroke}%
\pgfsetdash{}{0pt}%
\pgfpathmoveto{\pgfqpoint{4.165467in}{1.437762in}}%
\pgfpathcurveto{\pgfqpoint{4.293579in}{1.578356in}}{\pgfqpoint{4.394541in}{1.741476in}}{\pgfqpoint{4.463232in}{1.918848in}}%
\pgfpathcurveto{\pgfqpoint{4.531924in}{2.096219in}}{\pgfqpoint{4.567163in}{2.284792in}}{\pgfqpoint{4.567163in}{2.475001in}}%
\pgfpathlineto{\pgfqpoint{3.027163in}{2.475000in}}%
\pgfpathlineto{\pgfqpoint{4.165467in}{1.437762in}}%
\pgfpathlineto{\pgfqpoint{4.165467in}{1.437762in}}%
\pgfpathclose%
\pgfusepath{stroke,fill}%
\end{pgfscope}%
\begin{pgfscope}%
\pgfsetbuttcap%
\pgfsetmiterjoin%
\definecolor{currentfill}{rgb}{0.992157,0.705882,0.384314}%
\pgfsetfillcolor{currentfill}%
\pgfsetlinewidth{1.003750pt}%
\definecolor{currentstroke}{rgb}{1.000000,1.000000,1.000000}%
\pgfsetstrokecolor{currentstroke}%
\pgfsetdash{}{0pt}%
\pgfpathmoveto{\pgfqpoint{4.567163in}{2.475001in}}%
\pgfpathcurveto{\pgfqpoint{4.567163in}{2.475001in}}{\pgfqpoint{4.567163in}{2.475001in}}{\pgfqpoint{4.567163in}{2.475001in}}%
\pgfpathlineto{\pgfqpoint{3.027163in}{2.475000in}}%
\pgfpathlineto{\pgfqpoint{4.567163in}{2.475001in}}%
\pgfpathlineto{\pgfqpoint{4.567163in}{2.475001in}}%
\pgfpathclose%
\pgfusepath{stroke,fill}%
\end{pgfscope}%
\begin{pgfscope}%
\definecolor{textcolor}{rgb}{0.150000,0.150000,0.150000}%
\pgfsetstrokecolor{textcolor}%
\pgfsetfillcolor{textcolor}%
\pgftext[x=3.583970in,y=3.212389in,,]{\color{textcolor}\sffamily\fontsize{12.000000}{14.400000}\selectfont 29.4\%}%
\end{pgfscope}%
\begin{pgfscope}%
\definecolor{textcolor}{rgb}{0.150000,0.150000,0.150000}%
\pgfsetstrokecolor{textcolor}%
\pgfsetfillcolor{textcolor}%
\pgftext[x=2.165522in,y=2.808692in,,]{\color{textcolor}\sffamily\fontsize{12.000000}{14.400000}\selectfont 29.4\%}%
\end{pgfscope}%
\begin{pgfscope}%
\definecolor{textcolor}{rgb}{0.150000,0.150000,0.150000}%
\pgfsetstrokecolor{textcolor}%
\pgfsetfillcolor{textcolor}%
\pgftext[x=2.942078in,y=1.554926in,,]{\color{textcolor}\sffamily\fontsize{12.000000}{14.400000}\selectfont 29.4\%}%
\end{pgfscope}%
\begin{pgfscope}%
\definecolor{textcolor}{rgb}{0.150000,0.150000,0.150000}%
\pgfsetstrokecolor{textcolor}%
\pgfsetfillcolor{textcolor}%
\pgftext[x=3.710146in,y=1.852657in,,]{\color{textcolor}\sffamily\fontsize{12.000000}{14.400000}\selectfont 0.0\%}%
\end{pgfscope}%
\begin{pgfscope}%
\definecolor{textcolor}{rgb}{0.150000,0.150000,0.150000}%
\pgfsetstrokecolor{textcolor}%
\pgfsetfillcolor{textcolor}%
\pgftext[x=3.888805in,y=2.141309in,,]{\color{textcolor}\sffamily\fontsize{12.000000}{14.400000}\selectfont 11.8\%}%
\end{pgfscope}%
\begin{pgfscope}%
\definecolor{textcolor}{rgb}{0.150000,0.150000,0.150000}%
\pgfsetstrokecolor{textcolor}%
\pgfsetfillcolor{textcolor}%
\pgftext[x=3.951163in,y=2.475000in,,]{\color{textcolor}\sffamily\fontsize{12.000000}{14.400000}\selectfont 0.0\%}%
\end{pgfscope}%
\begin{pgfscope}%
\pgfsetbuttcap%
\pgfsetmiterjoin%
\definecolor{currentfill}{rgb}{1.000000,1.000000,1.000000}%
\pgfsetfillcolor{currentfill}%
\pgfsetfillopacity{0.800000}%
\pgfsetlinewidth{1.003750pt}%
\definecolor{currentstroke}{rgb}{0.800000,0.800000,0.800000}%
\pgfsetstrokecolor{currentstroke}%
\pgfsetstrokeopacity{0.800000}%
\pgfsetdash{}{0pt}%
\pgfpathmoveto{\pgfqpoint{3.271524in}{0.458500in}}%
\pgfpathlineto{\pgfqpoint{5.827163in}{0.458500in}}%
\pgfpathquadraticcurveto{\pgfqpoint{5.852163in}{0.458500in}}{\pgfqpoint{5.852163in}{0.483500in}}%
\pgfpathlineto{\pgfqpoint{5.852163in}{1.571829in}}%
\pgfpathquadraticcurveto{\pgfqpoint{5.852163in}{1.596829in}}{\pgfqpoint{5.827163in}{1.596829in}}%
\pgfpathlineto{\pgfqpoint{3.271524in}{1.596829in}}%
\pgfpathquadraticcurveto{\pgfqpoint{3.246524in}{1.596829in}}{\pgfqpoint{3.246524in}{1.571829in}}%
\pgfpathlineto{\pgfqpoint{3.246524in}{0.483500in}}%
\pgfpathquadraticcurveto{\pgfqpoint{3.246524in}{0.458500in}}{\pgfqpoint{3.271524in}{0.458500in}}%
\pgfpathlineto{\pgfqpoint{3.271524in}{0.458500in}}%
\pgfpathclose%
\pgfusepath{stroke,fill}%
\end{pgfscope}%
\begin{pgfscope}%
\pgfsetbuttcap%
\pgfsetmiterjoin%
\definecolor{currentfill}{rgb}{0.552941,0.827451,0.780392}%
\pgfsetfillcolor{currentfill}%
\pgfsetlinewidth{1.003750pt}%
\definecolor{currentstroke}{rgb}{1.000000,1.000000,1.000000}%
\pgfsetstrokecolor{currentstroke}%
\pgfsetdash{}{0pt}%
\pgfpathmoveto{\pgfqpoint{3.296524in}{1.451858in}}%
\pgfpathlineto{\pgfqpoint{3.546524in}{1.451858in}}%
\pgfpathlineto{\pgfqpoint{3.546524in}{1.539358in}}%
\pgfpathlineto{\pgfqpoint{3.296524in}{1.539358in}}%
\pgfpathlineto{\pgfqpoint{3.296524in}{1.451858in}}%
\pgfpathclose%
\pgfusepath{stroke,fill}%
\end{pgfscope}%
\begin{pgfscope}%
\definecolor{textcolor}{rgb}{0.150000,0.150000,0.150000}%
\pgfsetstrokecolor{textcolor}%
\pgfsetfillcolor{textcolor}%
\pgftext[x=3.646524in,y=1.451858in,left,base]{\color{textcolor}\sffamily\fontsize{9.000000}{10.800000}\selectfont No CS Degree, 29.4 \%}%
\end{pgfscope}%
\begin{pgfscope}%
\pgfsetbuttcap%
\pgfsetmiterjoin%
\definecolor{currentfill}{rgb}{1.000000,1.000000,0.701961}%
\pgfsetfillcolor{currentfill}%
\pgfsetlinewidth{1.003750pt}%
\definecolor{currentstroke}{rgb}{1.000000,1.000000,1.000000}%
\pgfsetstrokecolor{currentstroke}%
\pgfsetdash{}{0pt}%
\pgfpathmoveto{\pgfqpoint{3.296524in}{1.268387in}}%
\pgfpathlineto{\pgfqpoint{3.546524in}{1.268387in}}%
\pgfpathlineto{\pgfqpoint{3.546524in}{1.355887in}}%
\pgfpathlineto{\pgfqpoint{3.296524in}{1.355887in}}%
\pgfpathlineto{\pgfqpoint{3.296524in}{1.268387in}}%
\pgfpathclose%
\pgfusepath{stroke,fill}%
\end{pgfscope}%
\begin{pgfscope}%
\definecolor{textcolor}{rgb}{0.150000,0.150000,0.150000}%
\pgfsetstrokecolor{textcolor}%
\pgfsetfillcolor{textcolor}%
\pgftext[x=3.646524in,y=1.268387in,left,base]{\color{textcolor}\sffamily\fontsize{9.000000}{10.800000}\selectfont Bachelors's Degree, 29.4 \%}%
\end{pgfscope}%
\begin{pgfscope}%
\pgfsetbuttcap%
\pgfsetmiterjoin%
\definecolor{currentfill}{rgb}{0.745098,0.729412,0.854902}%
\pgfsetfillcolor{currentfill}%
\pgfsetlinewidth{1.003750pt}%
\definecolor{currentstroke}{rgb}{1.000000,1.000000,1.000000}%
\pgfsetstrokecolor{currentstroke}%
\pgfsetdash{}{0pt}%
\pgfpathmoveto{\pgfqpoint{3.296524in}{1.084915in}}%
\pgfpathlineto{\pgfqpoint{3.546524in}{1.084915in}}%
\pgfpathlineto{\pgfqpoint{3.546524in}{1.172415in}}%
\pgfpathlineto{\pgfqpoint{3.296524in}{1.172415in}}%
\pgfpathlineto{\pgfqpoint{3.296524in}{1.084915in}}%
\pgfpathclose%
\pgfusepath{stroke,fill}%
\end{pgfscope}%
\begin{pgfscope}%
\definecolor{textcolor}{rgb}{0.150000,0.150000,0.150000}%
\pgfsetstrokecolor{textcolor}%
\pgfsetfillcolor{textcolor}%
\pgftext[x=3.646524in,y=1.084915in,left,base]{\color{textcolor}\sffamily\fontsize{9.000000}{10.800000}\selectfont Master's Degree, 29.4 \%}%
\end{pgfscope}%
\begin{pgfscope}%
\pgfsetbuttcap%
\pgfsetmiterjoin%
\definecolor{currentfill}{rgb}{0.984314,0.501961,0.447059}%
\pgfsetfillcolor{currentfill}%
\pgfsetlinewidth{1.003750pt}%
\definecolor{currentstroke}{rgb}{1.000000,1.000000,1.000000}%
\pgfsetstrokecolor{currentstroke}%
\pgfsetdash{}{0pt}%
\pgfpathmoveto{\pgfqpoint{3.296524in}{0.901444in}}%
\pgfpathlineto{\pgfqpoint{3.546524in}{0.901444in}}%
\pgfpathlineto{\pgfqpoint{3.546524in}{0.988944in}}%
\pgfpathlineto{\pgfqpoint{3.296524in}{0.988944in}}%
\pgfpathlineto{\pgfqpoint{3.296524in}{0.901444in}}%
\pgfpathclose%
\pgfusepath{stroke,fill}%
\end{pgfscope}%
\begin{pgfscope}%
\definecolor{textcolor}{rgb}{0.150000,0.150000,0.150000}%
\pgfsetstrokecolor{textcolor}%
\pgfsetfillcolor{textcolor}%
\pgftext[x=3.646524in,y=0.901444in,left,base]{\color{textcolor}\sffamily\fontsize{9.000000}{10.800000}\selectfont Doctorate Degree, 0.0 \%}%
\end{pgfscope}%
\begin{pgfscope}%
\pgfsetbuttcap%
\pgfsetmiterjoin%
\definecolor{currentfill}{rgb}{0.501961,0.694118,0.827451}%
\pgfsetfillcolor{currentfill}%
\pgfsetlinewidth{1.003750pt}%
\definecolor{currentstroke}{rgb}{1.000000,1.000000,1.000000}%
\pgfsetstrokecolor{currentstroke}%
\pgfsetdash{}{0pt}%
\pgfpathmoveto{\pgfqpoint{3.296524in}{0.717972in}}%
\pgfpathlineto{\pgfqpoint{3.546524in}{0.717972in}}%
\pgfpathlineto{\pgfqpoint{3.546524in}{0.805472in}}%
\pgfpathlineto{\pgfqpoint{3.296524in}{0.805472in}}%
\pgfpathlineto{\pgfqpoint{3.296524in}{0.717972in}}%
\pgfpathclose%
\pgfusepath{stroke,fill}%
\end{pgfscope}%
\begin{pgfscope}%
\definecolor{textcolor}{rgb}{0.150000,0.150000,0.150000}%
\pgfsetstrokecolor{textcolor}%
\pgfsetfillcolor{textcolor}%
\pgftext[x=3.646524in,y=0.717972in,left,base]{\color{textcolor}\sffamily\fontsize{9.000000}{10.800000}\selectfont Other Engineering Degree, 11.8 \%}%
\end{pgfscope}%
\begin{pgfscope}%
\pgfsetbuttcap%
\pgfsetmiterjoin%
\definecolor{currentfill}{rgb}{0.992157,0.705882,0.384314}%
\pgfsetfillcolor{currentfill}%
\pgfsetlinewidth{1.003750pt}%
\definecolor{currentstroke}{rgb}{1.000000,1.000000,1.000000}%
\pgfsetstrokecolor{currentstroke}%
\pgfsetdash{}{0pt}%
\pgfpathmoveto{\pgfqpoint{3.296524in}{0.534501in}}%
\pgfpathlineto{\pgfqpoint{3.546524in}{0.534501in}}%
\pgfpathlineto{\pgfqpoint{3.546524in}{0.622001in}}%
\pgfpathlineto{\pgfqpoint{3.296524in}{0.622001in}}%
\pgfpathlineto{\pgfqpoint{3.296524in}{0.534501in}}%
\pgfpathclose%
\pgfusepath{stroke,fill}%
\end{pgfscope}%
\begin{pgfscope}%
\definecolor{textcolor}{rgb}{0.150000,0.150000,0.150000}%
\pgfsetstrokecolor{textcolor}%
\pgfsetfillcolor{textcolor}%
\pgftext[x=3.646524in,y=0.534501in,left,base]{\color{textcolor}\sffamily\fontsize{9.000000}{10.800000}\selectfont Yes but did not finish, 0.0 \%}%
\end{pgfscope}%
\end{pgfpicture}%
\makeatother%
\endgroup%
}
	\caption{On average how often do you use command line applications or terminal based tools?}
	\label{fig:question}
\end{figure}

\begin{table}[htbp]\centering
\begin{tabular}{|c|c|c|}
\hline
\textbf{Question}                                                                                                                                                      & \begin{tabular}[c]{@{}c@{}}Interactive\\ Correct \%\end{tabular} & \begin{tabular}[c]{@{}c@{}}Non Interactive \\ Correct \%\end{tabular} \\ \hline
\textbf{What does CLI stand for?}                                                                                                                                      & 100.00\%                                                         & 100.00\%                                                              \\ \hline
\textbf{\begin{tabular}[c]{@{}c@{}}Which one of the following best describes \\ the role of flags\\  when issuing a command?\end{tabular}}                             & 63.16\%                                                          & 46.67\%                                                               \\ \hline
\textbf{\begin{tabular}[c]{@{}c@{}}In your own words can you describe \\ what the shell is?\end{tabular}}                                                              & 94.28\%                                                          & 73.34\%                                                               \\ \hline
\textbf{\begin{tabular}[c]{@{}c@{}}How would you count the number of lines in\\  a given file with the word count utility?\end{tabular}}                               & 100.00\%                                                         & 93.34\%                                                               \\ \hline
\textbf{\begin{tabular}[c]{@{}c@{}}Which one of the following flow diagrams\\  best describes textual interaction \\ with the operating system\end{tabular}}           & 63.16\%                                                          & 60.00\%                                                               \\ \hline
\textbf{\begin{tabular}[c]{@{}c@{}}Which of the following is an\\  incorrect usage of flags?\end{tabular}}                                                             & 52.64\%                                                          & 20.00\%                                                               \\ \hline
\textbf{What is the role of the prompt?}                                                                                                                               & 52.63\%                                                          & 60.00\%                                                               \\ \hline
\textbf{\begin{tabular}[c]{@{}c@{}}What is the command to see where you are \\ on your file system and\\  what does the abbreviation stand for?\end{tabular}}          & 94.28\%                                                          & 66.67\%                                                               \\ \hline
\textbf{\begin{tabular}[c]{@{}c@{}}Which of the following structures\\  describes the file system best?\end{tabular}}                                                  & 94.28\%                                                          & 80.00\%                                                               \\ \hline
\textbf{\begin{tabular}[c]{@{}c@{}}What command do you use to \\ list the contents of your current directory?\end{tabular}}                                            & 94.28\%                                                          & 100.00\%                                                              \\ \hline
\textbf{\begin{tabular}[c]{@{}c@{}}Can you explain in what situations\\  the rmdir command\\  will not delete a directory?\end{tabular}}                               & 78.95\%                                                          & 86.67\%                                                               \\ \hline
\textbf{\begin{tabular}[c]{@{}c@{}}What shell keyboard shortcut \\ cancels a command or input?\end{tabular}}                                                           & 89.47\%                                                          & 93.34\%                                                               \\ \hline
\textbf{\begin{tabular}[c]{@{}c@{}}What is the name of the command that brings up \\ documentation about a command? \\ What does this command stand for?\end{tabular}} & 89.47\%                                                          & 80.47\%                                                               \\ \hline
\textbf{Overall Score}                                                                                                                                                 & 82.19\%                                                          & 72.85\%                                                               \\ \hline
\end{tabular}
\caption{Summary of questions answered correctly by method during the evaluation phase.}
\label{tab:evaluation}
\end{table}


\chapter{Reflections and Related Work}
\label{chap:reflection}
\section{Section}
% sandboxing
\subsubsection{Subsubsection}

\paragraph{Paragraph.} Always with a point.


\chapter{Conclusion}
\label{chap:conclusion}

In this work,


\backmatter

% TODO: ASK ABOUT THIS STYLE
% \bibliographystyle{alpha}
\bibliographystyle{ieeetr}
\bibliography{thesis}

% appendix 
\chapter{Appendix}
\section{Section}
%
\subsubsection{Subsubsection}

\paragraph{Paragraph.} Always with a point.

\begin{lstlisting}[caption=An example code snippet]
/**
 * Javadoc comment
 */
public class Foo {
	// line comment
	public void bar(int number) {
		if (number < 0) {
			return; /* block comment */
		}
	}
}
\end{lstlisting}



\end{document}
