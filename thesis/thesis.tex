\documentclass{seal_thesis}

\thesisType{Master's Thesis}
\date{\today}
\title{CLI Tutor}
\subtitle{Can interactive learning make the command line more approachable?}
\author{Qasim Warraich}
\home{Lahore} % Geburtsort
\country{Pakistan}
\legi{18-787-796}
\prof{Prof. Dr. Harald C. Gall}
\assistent{Dr. Carol V. Alexandru-Funakoshi}
\email{qasim.warraich@uzh.ch}
\url{gitlab.com/qasimwarraich/clitutor}
\begindate{2022-03-14}
\enddate{2022-03-14}

\begin{document}
\maketitle

\frontmatter

%\begin{acknowledgements}
%\end{acknowledgements}

\begin{abstract}
Despite the arguably dated appearance, difficult learning curve and practical
non-existence in the general personal computing space, Command Line Interfaces
(CLIs) have more than stood the test of time in the software development world.
There are a multitude of extremely popular tools and applications that
primarily focus on the command line as an interaction medium. Some examples
include version control software like `git', compilers and interpreters for
programming languages, package managers and various core utilities that are
popular in areas such as software development, scripting and system
administration.

As mentioned before, the use of the command line as an interaction paradigm has
effectively disappeared from a mainstream personal computer usage perspective.
This contributes greatly to the intimidation factor and learning difficulty for
those interested in getting into software engineering or system administration.
This unfamiliarity, paired with the inevitability of usage of CLIs in the
development space highlights a need to make the command line more accessible to
new users for whom text-based interaction with their computer is an alien
concept. In recent years interactive learning utilising tools such as sandboxed
environments have been gaining in popularity and have the potential to be a
suitable medium for learning command line basics through actual usage, examples
and practice.

\end{abstract}

% NOTE: Does this need to be in German. Also is a summary really necessary?

\begin{zusammenfassung}
\end{zusammenfassung}

\tableofcontents
\listoffigures
\listoftables
\lstlistoflistings

\mainmatter
\chapter{Introduction}
\section{Section}
%
\subsubsection{Subsubsection}
\fig[.5\textwidth]{seal_blue}{seal logo}{logo}

\subsection{Subsection}
%

% NOTE: What are use cases for paragraphs like this or are they in place of
% list items as in the proposal.

% NOTE: Paragraphs titles should always have a point (.) after the title.
\paragraph{Paragraph.} Always with a point.

\begin{lstlisting}[caption=An example code snippet]
/**
 * Javadoc comment
 */
public class Foo {
	// line comment
	public void bar(int number) {
		if (number < 0) {
			return; /* block comment */
		}
	}
}
\end{lstlisting}


\section{Curriculum}

The Curriculum is an important part of defining such a tool 

\backmatter

% NOTE: this bibstyle seems weird. 
\bibliographystyle{alpha}
\bibliography{thesis}

\end{document}
