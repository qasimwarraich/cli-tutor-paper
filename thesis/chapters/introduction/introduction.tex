% - Introduction
% - RQs
\chapter{Introduction}
\label{chap:intro}

Command line interfaces are still widely employed and being developed for a
variety of uses. Benefitting from their simplicity, Command line interfaces
allow for comparatively rapid development and the implementation of features
when compared to their graphical counterparts, which would require a much more
detailed and complicated implementation and more challenging user interface
considerations. Command line programs also further benefit from simplicity of
their textual nature when it comes to their ability to work with other
programs. The simple notion of textual input and output being the main forms of
interaction allow for an extremely flexible interface\footnote{In this case
    "interface" describes a medium through which programs can communicate, not
a user interface.} for chaining tools together. This composition of programs is
referred to as piping and is a hallmark element of what is considered to be the
\textit{UNIX Philosophy} \cite{mcilroy1978unix}. Since most command line
applications share this philosophy of textual input and output there isn't the
need for bespoke mediums of communication to be developed and features can be
more effectively shared across programs.

Since CLI programs are run within a shell, which has direct access to the
underlying operating system, a tight integration between program and system is
formed. This tight coupling, paired with the low system resource requirements,
the ability to specify inputs and arguments via text and modify program
behaviour upon instantiation with flags opens the room for powerful batch
processing capabilities, easy automation and the ability to deal with large
amounts of input and output efficiently.

The learning curve of such command line interfaces is however a strong barrier
to widespread enjoyment of these benefits. For new users, this learning curve
is made steeper by the paradigm shift in the personal computing space towards
graphical user interfaces. In this chapter, we will look at the interactive
learning tool, \textit{CLI-Tutor}, developed as a potential solution that
addresses the steep learning curve of the command line through interactive
lessons and a sandboxed shell environment.


\section{Interactive Learning Tools}

Interactive learning tools are not a new area of research in the area of
software development. Tutorial software for programming languages have existed
for decades \cite{anderson1985lisp, anderson1986automatedtutoring,
gerdes2012interactive, permpool2019interactive, lee2005intelligent,
jeuring2011programming,   holland2009j}.
%TODO: FINISH THIS SECTION
To validate the tool and answer our research questions, a user study will be
conducted, most likely with bachelor's students at the University of Zurich. A
secondary goal is to embed the learning tool into a prototypical web
application in order to make it more accessible and portable. This tool aims to
determine whether an interactive learning method may ease the introduction into
command line interfaces for novice users, particularly mitigating the 'scare
factor' experienced by first-time users.

\section{Requirements}

operating system compatability, distributability, ease of use. 


\subsection{Research Questions}
\label{subsec:rqs}

From the perspective of command line interfaces and how effective potential so..

We define the following research questions:

\begin{enumerate}[label=\textbf{RQ\arabic*}, leftmargin=*]
	\item Are there identifiable patterns of difficulty when it
	      comes to adopting CLIs? Can the `indimidation factor' be pinned down? \label{rq:1}
	\item How should an interactive learning tool be designed to mitigate
	      the difficulty and indimidation factor of learning CLIs? \label{rq:2}
	\item How can a `forgiving' shell be implemented on top of an existing
	      shell to enable the transition from learning to real-world usage? \label{rq:3}
	\item Is the interactive tool more effective than text based learning methods? \label{rq:4}
	\item Are novice CLI users more likely to continue
	      using CLI interfaces after using such a tool? \label{rq:5}
\end{enumerate}

