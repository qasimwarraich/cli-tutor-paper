\chapter{Introduction}
\label{chap:intro}

\section{Problem Description}

Command line interfaces have been with us since the 1950s
\cite{raymond2004art}. The rise of command line interfaces was closely coupled
with the rise of time-sharing computer systems. Command line interfaces and
time-sharing systems greatly shortened the feedback loop of earlier batch
computing systems and allowed programmers to be able to modify their programs
in near real time. Time-sharing systems also introduced the concept of a
\textit{shell}. A \textit{shell}, a term coined by Louis Pouzin \cite{pouzin},
is a program that allows for interaction with the operating system via textual
commands\cite{mashey1976using}. The \textit{Multics}\cite{corbato1965introduction} operating system
pioneered the concept of operating system shells. The concept proved to be
extremely influential and directly influenced the creation of the
\textit{Thompson shell} for the \textit{UNIX} \cite{ritchie1974unix} operating
system. The \textit{UNIX} shell has arguably had the greatest influence on
command line interfaces as we know them today, as most modern shells are
descendants of the original \textit{UNIX} shell \cite{raymond2004art}.

The command line is still a popular interaction medium for tooling in the areas
of software engineering and system administration
\cite{hultstrand2015git, takayama2006trust}. However, the usage of the command
line in the general personal computing space has practically disappeared
\cite{reimer2005history}. This is especially true for younger individuals,
whose very first exposure to computers has been working with Graphical User
Interfaces (GUIs). This schism in interaction mediums causes issues when the
same younger individuals strive to enter technical fields such as software
development or system administration. The issue is further propagated by the
fact that the command line is its own distinct interaction paradigm based
entirely on writing and reading text. This can mean that the usage of
traditional learning resources such as documentation and manuals might prove
difficult to translate into effective usage without active textual interaction
practice on the part of the learner. On the other hand, jumping straight into
practising on the command line comes with its caveats. The shell can be an
unforgiving tool for a novice user as it is very sensitive to syntax and often
provides feedback that is difficult to understand for novice users. The shell
also interacts directly with the operating system and does not require
confirmation for certain destructive tasks such as file deletion.

\subsection{Solving The Problem} The goal of this Master's thesis is to develop
an interactive learning tool that simultaneously aims to address the previously
introduced issues of lack of exposure to textual interaction, the novice
unfriendly environment of the shell and the difficulties of traditional
learning methodologies. The tool should provide a more forgiving experience
than using the shell directly whilst still being a faithful representation of a
system shell. Concepts learned during the interactive tutorials should be
directly transferrable to a standard Unix-like shell. Furthermore, the tool
should be easy to use and encourage experimentation in order to augment the
learning experience and in order to better exploit the advantages offered by
interactive learning systems.
%TODO: Expand a bit

\subsection{Research Questions}
\label{subsec:rqs}

In this work, we set out to address the following research questions:

\begin{enumerate}[label=\textbf{RQ\arabic*}, leftmargin=*]
	\item Are there identifiable patterns of difficulty when it
	      comes to adopting CLIs? Can the `intimidation factor' be pinned down? \label{rq:1}
	\item How should an interactive learning tool be designed to mitigate
	      the difficulty and intimidation factor of learning CLIs? \label{rq:2}
	\item How can a `forgiving' shell be implemented on top of an existing
	      shell to enable the transition from learning to real-world usage? \label{rq:3}
	\item Is the interactive tool more effective than text based learning methods? \label{rq:4}
	\item Are novice CLI users more likely to continue
	      using CLI interfaces after using such a tool? \label{rq:5}
\end{enumerate}

\subsection{Why CLIs?}
Before introducing our solution to the stated problem, we would like to discuss
the motivations behind using command line interfaces and lowering the barrier of
entry to their use.

Command line interfaces are still widely employed and being developed for a
variety of uses. Benefitting from their simplicity, Command line interfaces
allow for comparatively rapid development and the implementation of features
when compared to their graphical counterparts, which would require a much more
detailed and complicated implementation and more challenging user interface
considerations. Command line programs also further benefit from simplicity of
their textual nature when it comes to their ability to work with other
programs. The simple notion of textual input and output being the main forms of
interaction allow for an extremely flexible interface\footnote{In this case
    "interface" describes a medium through which programs can communicate, not
a user interface.} for chaining tools together. This composition of programs is
referred to as piping and is a hallmark element of what is considered to be the
\textit{UNIX Philosophy} \cite{mcilroy1978unix}. Since most command line
applications share this philosophy of textual input and output there isn't the
need for bespoke mediums of communication to be developed and features can be
more effectively shared across programs.

Since CLI programs are run within a shell, which has direct access to the
underlying operating system, a tight integration between program and system is
formed. This tight coupling, paired with the low system resource requirements,
the ability to specify inputs and arguments via text and modify program
behaviour upon instantiation with flags opens the room for powerful batch
processing capabilities, easy automation and the ability to deal with large
amounts of input and output efficiently.

The learning curve of such command line interfaces is however a strong barrier
to widespread enjoyment of these benefits. For new users, this learning curve
is made steeper by the paradigm shift in the personal computing space towards
graphical user interfaces. In this chapter, we will look at the interactive
learning tool, \textit{CLI-Tutor}, developed as a potential solution that
addresses the steep learning curve of the command line through interactive
lessons and a sandboxed shell environment.


% \fig[0.75\textwidth]{img/vimtutor}{Screenshot of vimtutor}{fig:vimtutor}
\section{Introducing "CLI-Tutor"}
\begin{figure}[htbp]
	\centering
	\includegraphics[width=0.75\textwidth]{img/vimtutor}
	\caption{Screenshot of \textit{vimtutor}}
	\label{fig:vimtutor}
\end{figure}

The \textit{CLI-Tutor} tool is an interactive shell-like interactive tutorial program for
the command line. We draw inspiration from the
\textit{vimtutor}\cite{pierce_ware_smith_moolenaar_2019} (see: Figure
\ref{fig:vimtutor}) utility shipping alongside the popular terminal-based text
editor \textit{Vim}. \textit{Vimtutor} is an interactive tutorial for the Vim text
editor. It is one long guided lesson presented as a text file. The file itself
contains the basic instructions for editing and creating text within Vim and is
intended to be modified while proceeding through the lesson. This "learning by
doing" approach presented itself as a very suitable choice for teaching the
command line. Much like with Vim, where one must learn a new paradigm of modal
editing when novices learn the command line they are not only taking on new
information about shell usage but also learning an entirely unfamiliar
(textual) interaction paradigm. Newer versions of \textit{vimtutor} are even more
interactive. In the version included in the popular fork of Vim called \textit{Neovim}
\cite{neovimHomeNeovim}, lines intended to be edited by the user also provide
feedback regarding correctness in the form of green arrows and red crosses.

The \textit{CLI-Tutor} tool introduces users to topics such as shell basics and
Unix-like core utility usage through a series of interactive examples. The tool
aims to relax the steep learning curve associated with the command line by
leveraging interactive examples and a feedback mechanism to instruct and
educate the user about the current stage of the lesson and provide some
feedback based on the inputs of the user. The core of \textit{CLI-Tutor} is
contained within a command line application written in golang\footnote{Go
	programming language: \href{https://go.dev/}{https://go.dev/}} and serves as a
standalone application. However, in order to make the learning experience safer
for the user and to encourage exploration, the CLI application has been wrapped
into a web application to form a sandboxed environment for the user that
exposes a terminal over the web. This means the user can use the tutorial
application without fear of causing their own system any harm.

The \textit{CLI-Tutor} application is fully open source and also comes with its
own modified parser and structure for specifying lessons based on Markdown
documents. This means that the material covered by the lessons can be easily
contributed to and distributed, making the tool easily extensible.


% \fig[0.75\textwidth]{img/clitutor}{Screenshot of CLI-Tutor}{fig:clitutor}
% \begin{figure}[htbp]
\begin{figure}[H]
	\centering
	\includegraphics[width=0.75\textwidth]{img/clitutor}
	\caption{Screenshot of \textit{CLI-Tutor}}
	\label{fig:clitutor}
\end{figure}

\section{Thesis Outline}

Over the next few chapters, we will discuss and show the design, implementation
and overall goals of this solution. \autoref{chap:intro} will discuss the
problem space and explain the motivations behind the interactive approach
employed by \textit{CLI-Tutor}. In \autoref{chap:clitutor}, the semantic
aspects CLI application and associated web application will be discussed. We
will discuss the design and implementation of the solution in
\autoref{chap:design}. \autoref{chap:userstudy} will discuss the methodology
and findings of the user study conducted during this Master's thesis work. In
\autoref{chap:reflection}, we will evaluate and reflect upon the solution. We
will consider existing approaches and make suggestions regarding building upon
and refining the \textit{CLI-Tutor}, before, concluding in
\autoref{chap:conclusion}.


