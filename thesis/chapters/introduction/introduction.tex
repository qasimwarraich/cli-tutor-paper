% - Introduction
% - RQs
\chapter{Introduction}
\label{chap:intro}

Command line interfaces are still widely employed and created for a variety of
reasons. Command line interfaces allow for rapid development and the
implementation of a very large amount of features that would require entail
much more detailed and complicated implementation decisions if they were to be
integrated into a graphical interface. Command line programs also benefit a lot
from their inherent simplicity. The simple notion of text input and output
being the main forms of interactions allow for an extremely flexible interface
for chaining tools together. This composition of programs is referred to as
piping ( SOME UNIX PHILOSOPHY MEME STUFF HERE ). This compositional simplicity
paired with the ability to specify arguments and modify program behaviour on
instantiation opens the room for powerful batch processing capabilities,
automation and the ability to deal with large amounts of input and output
efficiently.

The learning curve to such systems is however a strong barrier towards
employments of these benefits. This learning curve is made steeper by the
paradigm shift in the personal computing space towards graphical user
interfaces.



\section{Interactive Learning Tools}
To validate the tool and answer our
research questions, a user study will be conducted, most likely with bachelor's
students at the University of Zurich. A secondary goal is to embed the learning
tool into a prototypical web application in order to make it more accessible
and portable.
This tool aims to determine whether an interactive learning method may ease the
introduction into command line interfaces for novice users, particularly
mitigating the 'scare factor' experienced by first-time users. 

\section{Requirements}

% NOTE: Paragraphs titles should always have a point (.) after the title.
\paragraph{Paragraph.} Always with a point.

