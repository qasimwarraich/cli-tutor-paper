\chapter{Introduction}
This tool aims to determine whether an interactive learning method may ease the
introduction into command line interfaces for novice users, particularly
mitigating the 'scare factor' experienced by first-time users. We do so by
creating a forgiving CLI with the goal of teaching topics such as shell
scripting basics and Unix-like core utility usage through the use of
interactive examples. We draw inspiration from the
\cite{pierce_ware_smith_moolenaar_2019} utility shipping alongside the popular
terminal based text editor Vim. The proposed tool shall allow for opt-in
analytics that are sent back to a data collection service for the purpose of
learning which mistakes are most commonly made, and to improve the tool
accordingly. To validate the tool and answer our research questions, a user
study will be conducted, most likely with bachelor students at the University
of Zurich. A secondary goal is to embed the learning tool into a prototypical
web application in order to make it more accessible and portable.
\section{Section}
%
\subsubsection{Subsubsection}
\fig[.5\textwidth]{seal_blue}{seal logo}{logo}

\subsection{Subsection}
%

% NOTE: What are use cases for paragraphs like this or are they in place of
% list items as in the proposal.

% NOTE: Paragraphs titles should always have a point (.) after the title.
\paragraph{Paragraph.} Always with a point.

\begin{lstlisting}[caption=An example code snippet]
/**
 * Javadoc comment
 */
public class Foo {
	// line comment
	public void bar(int number) {
		if (number < 0) {
			return; /* block comment */
		}
	}
}
\end{lstlisting}


\section{Curriculum}

The Curriculum is an important part of defining such a tool 

\section{Interactive Learning Tool}

The Curriculum is an important part of defining such a tool 

