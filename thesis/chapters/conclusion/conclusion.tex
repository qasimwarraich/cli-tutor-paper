\chapter{Conclusion}
\label{chap:conclusion}

In this work, we presented \textit{CLI-Tutor}, an interactive tutorial tool for
introducing the command line to beginners. We demonstrated that an interactive
approach to teaching the concepts of command line interaction has the
potential to be more effective than the traditional documentation-based
approaches that exist. The \textit{CLI-Tutor} succeeds in its intended goals of
offering an interactive and low-barrier to access tutorial tool for
the command line.

As a result of a user study comparing \textit{CLI-Tutor} to
a state-of-the-art documentation tool, we were able to produce encouraging
results. Our findings indicate that tutorials that incorporate interactive
elements have the potential to not only produce better learning results but
also provide an engaging and safe environment for experimentation and
exploration. The research performed as part of this work also brings to light
some difficulties related specifically to the command line that could prove
beneficial to future research in this area. Mitigation of these identified
intimidation factors offers a framework for understanding the difficulties and
issues regarding command line interfaces. This has the potential not only to
produce better tutorials and tutorial software but can also be used to improve
the design of command line interfaces in general. Finally, this work makes
suggestions and outlines potential improvements for building future interactive
learning tools for the command line and beyond.
