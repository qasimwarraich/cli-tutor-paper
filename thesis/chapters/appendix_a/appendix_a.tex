\chapter{Appendix A: User Study Survey Questions}
\label{chap:appendixa}
\section{Survey Questions}
%

\subsection{User Familiarity Questions}
\begin{itemize}
	\item  \textbf{Please provide a name or identify yourself}
	      \begin{itemize}
		      \item  \textit{user textual input}

	      \end{itemize}

	\item  \textbf{How many years of programming experience do you have (if any)?}
	      \begin{itemize}
		      \item  \textit{user numerical input}

	      \end{itemize}
	\item  \textbf{Are you currently enrolled in or did you ever participate in a Computer Science degree or a similar computation/information theory degree at a University level?}
	      \begin{itemize}
		      \item  No.
		      \item  Yes, at a Bachelor's level.
		      \item  Yes, at a Master's level.
		      \item  Yes, at a Doctorate level.
		      \item  Yes, but I did not complete my studies.
		      \item  Other
	      \end{itemize}

	\item  \textbf{What is your preferred method of learning related to technical topics?}
	      \begin{itemize}
		      \item Books/ Online Documentation
		      \item Interactive Tutorials
		      \item Video Tutorials
		      \item University Lectures
		      \item Forums/ Online Groups \\(e.g. Discord Servers)
		      \item  Other
	      \end{itemize}

	\item  \textbf{How effective would you rate reading books and documentation as a learning medium for you? } \\(1 is extremely ineffective and 5 is extremely effective. This includes online resources such as documentation, blogs and tutorials.)
	      \begin{itemize}
		      \item  1
		      \item 2
		      \item 3
		      \item 4
		      \item 5
	      \end{itemize}
	\item  \textbf{Have you ever previously used an interactive learning resource?} \\(An interactive learning resource must allow some sort of input and verification from the user during the learning process. Youtube or non interactive online video based courses do not satisfy this requirement.)
	      \begin{itemize}
		      \item  Yes
		      \item No
	      \end{itemize}
	\item  \textbf{How  comfortable would you describe yourself with command line applications?} \\(1 = Very uncomfortable 5 = Extremely comfortable)
	      \begin{itemize}
		      \item  1
		      \item 2
		      \item 3
		      \item 4
		      \item 5
	      \end{itemize}

	\item  \textbf{On average how often do you use command line applications or terminal based tools?}
	      \begin{itemize}
		      \item Never
		      \item A few times a year
		      \item At least once a year
		      \item At least once a month
		      \item Almost every day
	      \end{itemize}

	\item  \textbf{On average, How often do you perform regular computing tasks such as file management from the command line?}
	      \begin{itemize}
		      \item Never
		      \item A few times a year
		      \item At least once a year
		      \item At least once a month
		      \item Almost every day
	      \end{itemize}

	\item  \textbf{Are you interested in integrating the command line more into your day-to-day computer use? If so, why?}
	      \begin{itemize}
		      \item  Yes
		      \item No
	      \end{itemize}
\end{itemize}

\subsection{Evaluation}

\begin{itemize}
	\item  \textbf{What does CLI stand for?}
	      \begin{itemize}
		      \item Command Line Interface
		      \item Command Line Interaction
		      \item Command Line Instrument
		      \item Cool Linux Interaction
	      \end{itemize}

	\item  \textbf{Which one of the following best describes the role of flags when issuing a command?}
	      \begin{itemize}
		      \item They modify a program's behaviour
		      \item They modify the input given to a program
		      \item They are the input to a program
		      \item They add functionality to a program
		      \item I don't know
	      \end{itemize}

	\item  \textbf{In your own words can you describe what the shell is?}
	      \begin{itemize}
		      \item \textit{user textual input}
	      \end{itemize}

	\item  \textbf{How would you count the number of lines in a given file with the word count utility?} (Please choose all that apply)
	      \begin{itemize}
		      \item wc -l file.txt
		      \item wc lines file.txt
		      \item wc file.txt --lines
		      \item wc file.txt
		      \item lines file.txt
	      \end{itemize}

	\item  \textbf{Which one of the following flow diagrams best describes textual interaction with the operating system using a shell?}
	      \begin{itemize}
		      \item user <--> shell <--> os
		      \item user <--> operating system
		      \item user --> shell --> os --> user
		      \item user <--> shell
		      \item I don't know
	      \end{itemize}

	\item  \textbf{Which of the following is an incorrect usage of flags?}
	      \begin{itemize}
		      \item wc --lineswords file.txt
		      \item wc -lw file.txt
		      \item wc --lines --words file.txt
		      \item wc file.txt --words -l
		      \item I don't know
	      \end{itemize}

	\item  \textbf{What is the role of the prompt?}
	      \begin{itemize}
		      \item \textit{user textual input}
	      \end{itemize}

	\item  \textbf{What is the command to see where you are on your file system, and what does the abbreviation stand for?}
	      \begin{itemize}
		      \item \textit{user textual input}
	      \end{itemize}

	\item  \textbf{Which of the following structures describes the file system best?}
	      \begin{itemize}
		      \item Tree
		      \item Folder
		      \item Database
		      \item Chain
		      \item I don't know
	      \end{itemize}
	\item  \textbf{What command do you use to list the contents of your current directory?}
	      \begin{itemize}
		      \item ls
		      \item wc
		      \item pwd
		      \item list
		      \item I don't know
	      \end{itemize}

	\item  \textbf{Can you explain in what situations the rmdir command will not delete a directory?}
	      \begin{itemize}
		      \item \textit{user textual input}
	      \end{itemize}
	\item  \textbf{What shell keyboard shortcut cancels a command or input?}
	      \begin{itemize}
		      \item Control + c
		      \item Control + Alt + Delete
		      \item Escape
		      \item Alt + F4
		      \item I don't know
	      \end{itemize}
	\item  \textbf{What is the name of the command that brings up documentation about a command? What does this command stand for?}
	      \begin{itemize}
		      \item \textit{user textual input}
	      \end{itemize}
\end{itemize}

\subsection{Feedback}
Feedback was provided by a combination of two choice (e.g. \textit{More} or \textit{Less}) questions with a comment
field where participants could expand on their answers.
\subsubsection{Interactive Group}

\begin{itemize}

	\item  \textbf{Do you feel more or less intimidated by the command line after this interactive tutor?}
	      \begin{itemize}
		      \item More
		      \item Less
		      \item \textit{user textual input}
	      \end{itemize}

	\item  \textbf{What intimidation factors or difficulties did you personally experience when it comes to using the command line?}
	      \begin{itemize}
		      \item \textit{user textual input}
	      \end{itemize}

	\item  \textbf{Are you more or less likely to use the command line more than you do currently after using this interactive tutor application?}
	      \begin{itemize}
		      \item More
		      \item Less
		      \item \textit{user textual input}
	      \end{itemize}

	\item  \textbf{What are your feelings regarding interactive learning tools after this experience?}
	      \begin{itemize}
		      \item \textit{user textual input}
	      \end{itemize}

	\item  \textbf{Did you learn anything new? If so, what was the most interesting or useful thing you learnt}
	      \begin{itemize}
		      \item \textit{user textual input}
	      \end{itemize}
	\item  \textbf{We would love to hear some of your general feedback or reflections about the cli-tutor and the interactive learning process?}
	      \begin{itemize}
		      \item \textit{user textual input}
	      \end{itemize}
\end{itemize}
\subsubsection{Non-Interactive Group}

\begin{itemize}

	\item  \textbf{Do you feel more or less intimidated by the command line after taking these lessons?}
	      \begin{itemize}
		      \item More
		      \item Less
		      \item \textit{user textual input}
	      \end{itemize}

	\item  \textbf{What intimidation factors or difficulties did you personally experience when it comes to using the command line?}
	      \begin{itemize}
		      \item \textit{user textual input}
	      \end{itemize}

	\item  \textbf{Are you more or less likely to use the command line more than you do currently after taking these lessons?}
	      \begin{itemize}
		      \item More
		      \item Less
		      \item \textit{user textual input}
	      \end{itemize}

	\item  \textbf{What are your feelings regarding learning technical topics by reading documentation after this experience?}
	      \begin{itemize}
		      \item \textit{user textual input}
	      \end{itemize}

	\item  \textbf{Do you think an interactive command line tutorial application would improve the learning process?}
	      \begin{itemize}
		      \item Yes
		      \item No
		      \item It would make no difference
		      \item \textit{user textual input}
	      \end{itemize}

	\item  \textbf{Did you learn anything new? If so, what was the most interesting or useful thing you learnt}
	      \begin{itemize}
		      \item \textit{user textual input}
	      \end{itemize}

	\item  \textbf{We would love to hear some of your general feedback or reflections about the cli-tutor and the interactive learning process?}
	      \begin{itemize}
		      \item \textit{user textual input}
	      \end{itemize}
\end{itemize}

\subsubsection{Non-Interactive Group}
Users were also encouraged to optionally try the alternative tool to which they were assigned and asked to provide feedback.
\begin{itemize}
	\item  \textbf{Please provide some thoughts or feedback about your experience trying out the alternative version of the tutorial tool.}(e.g. Is there something you prefer or disfavour about one approach or the other? Is there a difference in levels of intimidation or difficulty with one approach or the other?)
	      \begin{itemize}
		      \item \textit{user textual input}
	      \end{itemize}
\end{itemize}
