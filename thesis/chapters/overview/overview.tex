\chapter{Overview}
% - Overview - problem description - introduce tool - outline thesis
\section{Problem Description}

The command line is still a popular interaction medium for tooling in the areas
of software engineering and system administration. However, the usage of the
command line in the general personal computing space has practically
disappeared. This is especially true for younger individuals who from their
very first exposure to computers have been working with Graphical User
Interfaces (GUIs). This schism in interaction mediums causes issues when the
same younger individuals strive to enter technical fields such as software
development or system administration. The issue is further propagated by the
fact that because the command line is its own distinct interaction paradigm and
the usage of traditional learning resources such as documentation and manuals
might prove difficult to translate into effective usage without active command
line usage practice on the part of the learner. Jumping straight into the
command line, however, comes with its caveats. The shell can be an unforgiving
tool for a novice user as it is very sensitive to syntax and often provides
feedback that is difficult to understand for novice users.

The goal of this Master's thesis is to develop a tool that simultaneously aims
to address the previously introduced issues of lack of exposure to textual
interaction and the difficulties of traditional learning methodologies. The
tool should provide a more forgiving experience than using the shell directly
whilst still being an honest representation of shell interaction. Concepts
learned during the interactive tutorials should be directly transferrable to a
standard Unix-like shell. Furthermore, the tool should easy to use and
encourage experimentation in order to augment the learning experience and in
order to better exploit the advantages offered by interactive systems.

%
\fig[1\textwidth]{img/vimtutor}{Screen shot of vimtutor}{vimtutor}

\section{Introducing "CLI-Tutor"}

The \textit{CLI-Tutor} tool is an interactive shell like tutorial program for
the command line.

This tool aims to determine whether an interactive learning method may ease the
introduction into command line interfaces for novice users, particularly
mitigating the 'scare factor' experienced by first-time users. We do so by
creating a forgiving shell-like command line tutor application with the goal of
teaching topics such as shell  basics and Unix-like core utility usage through
the use of interactive examples. We draw inspiration from the
`vimtutor'\cite{pierce_ware_smith_moolenaar_2019} (see: Figure \ref{vimtutor})
utility shipping alongside the popular terminal-based text editor Vim.

The proposed tool shall allow for opt-in analytics that are sent back to a data
collection service for the purpose of learning which mistakes are most commonly
made, and to improve the tool accordingly. To validate the tool and answer our
research questions, a user study will be conducted, most likely with bachelor's
students at the University of Zurich. A secondary goal is to embed the learning
tool into a prototypical web application in order to make it more accessible
and portable.

\fig[1\textwidth]{img/clitutor}{Screen shot of CLI-Tutor}{clitutor}

\section{Thesis Outline}

\paragraph{Paragraph.} Always with a point. {}
