\chapter{User Study}
% - User Study
%    - methodology
%       - assignment
%       - survey structure
%    - participants
% DATAPOINTS: Interest in interactive learnign or percentage of people who find
% it more effective than reading
\label{chap:userstudy}

In order to test and validate the effectiveness of our solution, a user study
was conducted. The goal of this user study was two part. Firstly, we were
interested in assessing the usability and response to \textit{CLI-Tutor}.
Secondly, we wanted to ascertain if interactive learning would be a more
effective medium to teach command line interaction than the traditional means
such as online documentation or books. In this chapter, we will describe our
user study in detail.

\section{Methodology}

The user study for this thesis work was conducted remotely and asynchronously.
We designed an online survey using the \textit{LimeSurvey}\cite{schmitzlime}
tool made available to us by the University of Zurich.

The focus of the user study was primarily on the comparison between interactive
learning approaches such as that of \textit{CLI-Tutor} and traditional ones,
which are mostly reading based. In the modern software development space,
online documentation is the status quo and the medium we choose to compare our
solution against. As discussed in \autoref{chap:design}, \textit{CLI-Tutor}
uses \textit{Markdown} to specify lessons. Many static documentation generation
websites use Markdown files to generate documentation from. This is also true
for our chosen generator. This enables us to objectively compare the
interaction medium rather than the lesson content, since the exact same lessons
can be used in both tools. Furthermore, due to the popularity of
\textit{MkDocs} it is a very realistic representation of documentation that
individuals may encounter in the wild.


\subsection{Interactive versus Non-interactive}

We designed our user study with the intention to perform an A/B style testing
comparing learning mediums. To support this we used the pseudorandom number
generation feature of \textit{LimeSurvey} to assign our participants to one of
two groups, interactive and non-interactive.

\subsection{Structure}

In this section, we provide a structural overview of our online survey. The
entire survey is available in \autoref{chap:appendixa}.

Our online survey was divided into the following sections:

\begin{itemize}

	\item User Familiarity: In this section, users answered questions relating
	      to their experience, interest and preferences to provide us with some
	      insights on each individual participant.

	\item Assignment: All participants where divided into one of two groups,
	      interactive and non-interactive. The assignment value is unknown to the
	      participant at the time of starting the survey. Our survey tool then
	      conditionally rendered a URL for the participants to follow to the next
	      section of the survey.

	\item Tutorial: At this stage, once the participants have been assigned to
	      one of the two groups, they will either be sent to our web application
	      running \textit{CLI-Tutor} or to our documentation website. If assigned
	      to the non-interactive group participants were also given the option of
	      navigating to the  \textit{CLI-only} version of our tool, in case they
	      did not have access to a Unix-like terminal (see:
	      \autoref{fig:cliversion}).

	\item Evaluation: The evaluation stage is where participants were asked a
	      series of basic questions relating to the lessons they took in the
	      previous stage. Questions were a mixture of multiple choice and free
	      text questions. This section was identical for both user groups as the
	      lessons were also identical.

	\item Feedback: In this section participants were able to provide feedback
	      regarding their experience. All but one of the questions in this
	      section were identical for both user groups. The non-interactive group
	      were asked one additional question regarding interactive learning:
	      \textit{Do you think an interactive command line tutorial application
		      would improve the learning process?}

	\item Feedback Opposite (optional): This section was optional and included
	      only one quick feedback question. At the end of the survey,
	      participants were then given a chance to try out the opposite tool to
	      which they were assigned for the bulk of the user study. No evaluation
	      or tutorial was mandated here and participants were given one free text
	      response to report on their feelings using the alternative tool.
\end{itemize}


\section{Participants}

In this section we will share insights we gathered from the first section of
the survey, where we asked experience, familiarity and more general questions.

In total, 34 participants took part in our user study. 19 of whom were assigned
to the interactive group and 15 to the non-interactive group. Recruitment was
primarily done through University channels though some participants were also
sourced through work email and word of mouth. While not limited to individuals
in software related fields, over 60\% (see: \autoref{fig:uniexp}) of the
participants were from technical backgrounds or currently computer science
students.

\subsection{Technical Experience}

Given the goals of \textit{CLI-Tutor}, It comes as no surprise that most of the
individuals who participated in our user study were of not very highly
experienced, though there were two outliers with over twenty years of
programming experience. Most of the participants were far less experienced with
a median experience of 3 years (see: \autoref{fig:programmingexp}). While not
very highly experienced in programming, most of our participants did come from
some sort of Computer Science or engineering background (see:
\autoref{fig:uniexp}).

\begin{figure}[H]
	\centering
	\scalebox{0.75}{%% Creator: Matplotlib, PGF backend
%%
%% To include the figure in your LaTeX document, write
%%   \input{<filename>.pgf}
%%
%% Make sure the required packages are loaded in your preamble
%%   \usepackage{pgf}
%%
%% Also ensure that all the required font packages are loaded; for instance,
%% the lmodern package is sometimes necessary when using math font.
%%   \usepackage{lmodern}
%%
%% Figures using additional raster images can only be included by \input if
%% they are in the same directory as the main LaTeX file. For loading figures
%% from other directories you can use the `import` package
%%   \usepackage{import}
%%
%% and then include the figures with
%%   \import{<path to file>}{<filename>.pgf}
%%
%% Matplotlib used the following preamble
%%   \usepackage{fontspec}
%%   \setmainfont{DejaVuSerif.ttf}[Path=\detokenize{/home/spam/miniconda3/envs/mpl/lib/python3.10/site-packages/matplotlib/mpl-data/fonts/ttf/}]
%%   \setsansfont{DejaVuSans.ttf}[Path=\detokenize{/home/spam/miniconda3/envs/mpl/lib/python3.10/site-packages/matplotlib/mpl-data/fonts/ttf/}]
%%   \setmonofont{DejaVuSansMono.ttf}[Path=\detokenize{/home/spam/miniconda3/envs/mpl/lib/python3.10/site-packages/matplotlib/mpl-data/fonts/ttf/}]
%%
\begingroup%
\makeatletter%
\begin{pgfpicture}%
\pgfpathrectangle{\pgfpointorigin}{\pgfqpoint{8.799314in}{3.116660in}}%
\pgfusepath{use as bounding box, clip}%
\begin{pgfscope}%
\pgfsetbuttcap%
\pgfsetmiterjoin%
\pgfsetlinewidth{0.000000pt}%
\definecolor{currentstroke}{rgb}{0.000000,0.000000,0.000000}%
\pgfsetstrokecolor{currentstroke}%
\pgfsetstrokeopacity{0.000000}%
\pgfsetdash{}{0pt}%
\pgfpathmoveto{\pgfqpoint{0.000000in}{0.000000in}}%
\pgfpathlineto{\pgfqpoint{8.799314in}{0.000000in}}%
\pgfpathlineto{\pgfqpoint{8.799314in}{3.116660in}}%
\pgfpathlineto{\pgfqpoint{0.000000in}{3.116660in}}%
\pgfpathlineto{\pgfqpoint{0.000000in}{0.000000in}}%
\pgfpathclose%
\pgfusepath{}%
\end{pgfscope}%
\begin{pgfscope}%
\pgfsetbuttcap%
\pgfsetmiterjoin%
\definecolor{currentfill}{rgb}{0.501961,0.694118,0.827451}%
\pgfsetfillcolor{currentfill}%
\pgfsetlinewidth{1.003750pt}%
\definecolor{currentstroke}{rgb}{1.000000,1.000000,1.000000}%
\pgfsetstrokecolor{currentstroke}%
\pgfsetdash{}{0pt}%
\pgfpathmoveto{\pgfqpoint{3.894914in}{1.314902in}}%
\pgfpathcurveto{\pgfqpoint{3.894914in}{1.442534in}}{\pgfqpoint{3.869773in}{1.568924in}}{\pgfqpoint{3.820931in}{1.686841in}}%
\pgfpathcurveto{\pgfqpoint{3.772088in}{1.804757in}}{\pgfqpoint{3.700494in}{1.911906in}}{\pgfqpoint{3.610245in}{2.002155in}}%
\pgfpathcurveto{\pgfqpoint{3.519995in}{2.092404in}}{\pgfqpoint{3.412847in}{2.163998in}}{\pgfqpoint{3.294931in}{2.212841in}}%
\pgfpathcurveto{\pgfqpoint{3.177014in}{2.261683in}}{\pgfqpoint{3.050624in}{2.286824in}}{\pgfqpoint{2.922992in}{2.286824in}}%
\pgfpathcurveto{\pgfqpoint{2.795361in}{2.286824in}}{\pgfqpoint{2.668970in}{2.261683in}}{\pgfqpoint{2.551054in}{2.212841in}}%
\pgfpathcurveto{\pgfqpoint{2.433138in}{2.163998in}}{\pgfqpoint{2.325989in}{2.092404in}}{\pgfqpoint{2.235740in}{2.002155in}}%
\pgfpathcurveto{\pgfqpoint{2.145491in}{1.911906in}}{\pgfqpoint{2.073896in}{1.804757in}}{\pgfqpoint{2.025054in}{1.686841in}}%
\pgfpathcurveto{\pgfqpoint{1.976211in}{1.568924in}}{\pgfqpoint{1.951070in}{1.442534in}}{\pgfqpoint{1.951070in}{1.314902in}}%
\pgfpathcurveto{\pgfqpoint{1.951070in}{1.187271in}}{\pgfqpoint{1.976211in}{1.060880in}}{\pgfqpoint{2.025054in}{0.942964in}}%
\pgfpathcurveto{\pgfqpoint{2.073896in}{0.825048in}}{\pgfqpoint{2.145491in}{0.717899in}}{\pgfqpoint{2.235740in}{0.627650in}}%
\pgfpathcurveto{\pgfqpoint{2.325989in}{0.537401in}}{\pgfqpoint{2.433138in}{0.465806in}}{\pgfqpoint{2.551054in}{0.416964in}}%
\pgfpathcurveto{\pgfqpoint{2.668970in}{0.368121in}}{\pgfqpoint{2.795361in}{0.342980in}}{\pgfqpoint{2.922992in}{0.342980in}}%
\pgfpathcurveto{\pgfqpoint{3.050624in}{0.342980in}}{\pgfqpoint{3.177014in}{0.368121in}}{\pgfqpoint{3.294931in}{0.416964in}}%
\pgfpathcurveto{\pgfqpoint{3.412847in}{0.465806in}}{\pgfqpoint{3.519995in}{0.537401in}}{\pgfqpoint{3.610245in}{0.627650in}}%
\pgfpathcurveto{\pgfqpoint{3.700494in}{0.717899in}}{\pgfqpoint{3.772088in}{0.825048in}}{\pgfqpoint{3.820931in}{0.942964in}}%
\pgfpathcurveto{\pgfqpoint{3.869773in}{1.060880in}}{\pgfqpoint{3.894914in}{1.187271in}}{\pgfqpoint{3.894914in}{1.314902in}}%
\pgfpathmoveto{\pgfqpoint{2.922992in}{1.314902in}}%
\pgfpathmoveto{\pgfqpoint{3.894914in}{1.314902in}}%
\pgfpathlineto{\pgfqpoint{3.894914in}{1.314902in}}%
\pgfpathclose%
\pgfusepath{stroke,fill}%
\end{pgfscope}%
\begin{pgfscope}%
\pgfsetbuttcap%
\pgfsetmiterjoin%
\definecolor{currentfill}{rgb}{0.992157,0.705882,0.384314}%
\pgfsetfillcolor{currentfill}%
\pgfsetlinewidth{1.003750pt}%
\definecolor{currentstroke}{rgb}{1.000000,1.000000,1.000000}%
\pgfsetstrokecolor{currentstroke}%
\pgfsetdash{}{0pt}%
\pgfpathmoveto{\pgfqpoint{2.922992in}{2.286824in}}%
\pgfpathcurveto{\pgfqpoint{2.922992in}{2.286824in}}{\pgfqpoint{2.922992in}{2.286824in}}{\pgfqpoint{2.922992in}{2.286824in}}%
\pgfpathlineto{\pgfqpoint{2.922992in}{1.314902in}}%
\pgfpathlineto{\pgfqpoint{2.922992in}{2.286824in}}%
\pgfpathlineto{\pgfqpoint{2.922992in}{2.286824in}}%
\pgfpathclose%
\pgfusepath{stroke,fill}%
\end{pgfscope}%
\begin{pgfscope}%
\definecolor{textcolor}{rgb}{0.150000,0.150000,0.150000}%
\pgfsetstrokecolor{textcolor}%
\pgfsetfillcolor{textcolor}%
\pgftext[x=2.922992in,y=0.731749in,,]{\color{textcolor}\sffamily\fontsize{12.000000}{14.400000}\selectfont 100.00\%}%
\end{pgfscope}%
\begin{pgfscope}%
\definecolor{textcolor}{rgb}{0.150000,0.150000,0.150000}%
\pgfsetstrokecolor{textcolor}%
\pgfsetfillcolor{textcolor}%
\pgftext[x=2.922992in,y=1.898055in,,]{\color{textcolor}\sffamily\fontsize{12.000000}{14.400000}\selectfont 0.00\%}%
\end{pgfscope}%
\begin{pgfscope}%
\definecolor{textcolor}{rgb}{0.000000,0.000000,0.000000}%
\pgfsetstrokecolor{textcolor}%
\pgfsetfillcolor{textcolor}%
\pgftext[x=2.922992in,y=2.613138in,,base]{\color{textcolor}\sffamily\fontsize{12.000000}{14.400000}\selectfont CLI-Tutor}%
\end{pgfscope}%
\begin{pgfscope}%
\pgfsetbuttcap%
\pgfsetmiterjoin%
\definecolor{currentfill}{rgb}{0.501961,0.694118,0.827451}%
\pgfsetfillcolor{currentfill}%
\pgfsetlinewidth{1.003750pt}%
\definecolor{currentstroke}{rgb}{1.000000,1.000000,1.000000}%
\pgfsetstrokecolor{currentstroke}%
\pgfsetdash{}{0pt}%
\pgfpathmoveto{\pgfqpoint{5.876322in}{2.286824in}}%
\pgfpathcurveto{\pgfqpoint{5.725984in}{2.286824in}}{\pgfqpoint{5.577676in}{2.251942in}}{\pgfqpoint{5.443099in}{2.184931in}}%
\pgfpathcurveto{\pgfqpoint{5.308522in}{2.117920in}}{\pgfqpoint{5.191311in}{2.020588in}}{\pgfqpoint{5.100712in}{1.900616in}}%
\pgfpathcurveto{\pgfqpoint{5.010113in}{1.780644in}}{\pgfqpoint{4.948574in}{1.641271in}}{\pgfqpoint{4.920949in}{1.493492in}}%
\pgfpathcurveto{\pgfqpoint{4.893325in}{1.345714in}}{\pgfqpoint{4.900361in}{1.193522in}}{\pgfqpoint{4.941503in}{1.048923in}}%
\pgfpathcurveto{\pgfqpoint{4.982645in}{0.904324in}}{\pgfqpoint{5.056781in}{0.771224in}}{\pgfqpoint{5.158063in}{0.660123in}}%
\pgfpathcurveto{\pgfqpoint{5.259345in}{0.549022in}}{\pgfqpoint{5.385037in}{0.462921in}}{\pgfqpoint{5.525223in}{0.408612in}}%
\pgfpathcurveto{\pgfqpoint{5.665409in}{0.354304in}}{\pgfqpoint{5.816303in}{0.333255in}}{\pgfqpoint{5.966000in}{0.347126in}}%
\pgfpathcurveto{\pgfqpoint{6.115696in}{0.360998in}}{\pgfqpoint{6.260153in}{0.409415in}}{\pgfqpoint{6.387973in}{0.488558in}}%
\pgfpathlineto{\pgfqpoint{5.876322in}{1.314902in}}%
\pgfpathlineto{\pgfqpoint{5.876322in}{2.286824in}}%
\pgfpathlineto{\pgfqpoint{5.876322in}{2.286824in}}%
\pgfpathclose%
\pgfusepath{stroke,fill}%
\end{pgfscope}%
\begin{pgfscope}%
\pgfsetbuttcap%
\pgfsetmiterjoin%
\definecolor{currentfill}{rgb}{0.992157,0.705882,0.384314}%
\pgfsetfillcolor{currentfill}%
\pgfsetlinewidth{1.003750pt}%
\definecolor{currentstroke}{rgb}{1.000000,1.000000,1.000000}%
\pgfsetstrokecolor{currentstroke}%
\pgfsetdash{}{0pt}%
\pgfpathmoveto{\pgfqpoint{6.387973in}{0.488558in}}%
\pgfpathcurveto{\pgfqpoint{6.567670in}{0.599821in}}{\pgfqpoint{6.706262in}{0.766721in}}{\pgfqpoint{6.782612in}{0.963803in}}%
\pgfpathcurveto{\pgfqpoint{6.858962in}{1.160886in}}{\pgfqpoint{6.868981in}{1.377595in}}{\pgfqpoint{6.811141in}{1.580881in}}%
\pgfpathcurveto{\pgfqpoint{6.753302in}{1.784167in}}{\pgfqpoint{6.630701in}{1.963143in}}{\pgfqpoint{6.462036in}{2.090512in}}%
\pgfpathcurveto{\pgfqpoint{6.293371in}{2.217882in}}{\pgfqpoint{6.087677in}{2.286824in}}{\pgfqpoint{5.876322in}{2.286824in}}%
\pgfpathlineto{\pgfqpoint{5.876322in}{1.314902in}}%
\pgfpathlineto{\pgfqpoint{6.387973in}{0.488558in}}%
\pgfpathlineto{\pgfqpoint{6.387973in}{0.488558in}}%
\pgfpathclose%
\pgfusepath{stroke,fill}%
\end{pgfscope}%
\begin{pgfscope}%
\definecolor{textcolor}{rgb}{0.150000,0.150000,0.150000}%
\pgfsetstrokecolor{textcolor}%
\pgfsetfillcolor{textcolor}%
\pgftext[x=5.315431in,y=1.155315in,,]{\color{textcolor}\sffamily\fontsize{12.000000}{14.400000}\selectfont 58.82\%}%
\end{pgfscope}%
\begin{pgfscope}%
\definecolor{textcolor}{rgb}{0.150000,0.150000,0.150000}%
\pgfsetstrokecolor{textcolor}%
\pgfsetfillcolor{textcolor}%
\pgftext[x=6.437214in,y=1.474490in,,]{\color{textcolor}\sffamily\fontsize{12.000000}{14.400000}\selectfont 41.18\%}%
\end{pgfscope}%
\begin{pgfscope}%
\definecolor{textcolor}{rgb}{0.000000,0.000000,0.000000}%
\pgfsetstrokecolor{textcolor}%
\pgfsetfillcolor{textcolor}%
\pgftext[x=5.876322in,y=2.613138in,,base]{\color{textcolor}\sffamily\fontsize{12.000000}{14.400000}\selectfont Non Interactive Tutor}%
\end{pgfscope}%
\begin{pgfscope}%
\definecolor{textcolor}{rgb}{0.150000,0.150000,0.150000}%
\pgfsetstrokecolor{textcolor}%
\pgfsetfillcolor{textcolor}%
\pgftext[x=4.399657in,y=3.016660in,,top]{\color{textcolor}\sffamily\fontsize{14.400000}{17.280000}\selectfont Do you feel more or less intimidated by the command line after this interactive tutor?}%
\end{pgfscope}%
\begin{pgfscope}%
\pgfsetbuttcap%
\pgfsetmiterjoin%
\definecolor{currentfill}{rgb}{1.000000,1.000000,1.000000}%
\pgfsetfillcolor{currentfill}%
\pgfsetfillopacity{0.800000}%
\pgfsetlinewidth{1.003750pt}%
\definecolor{currentstroke}{rgb}{0.800000,0.800000,0.800000}%
\pgfsetstrokecolor{currentstroke}%
\pgfsetstrokeopacity{0.800000}%
\pgfsetdash{}{0pt}%
\pgfpathmoveto{\pgfqpoint{4.031901in}{1.341726in}}%
\pgfpathlineto{\pgfqpoint{4.767413in}{1.341726in}}%
\pgfpathquadraticcurveto{\pgfqpoint{4.797969in}{1.341726in}}{\pgfqpoint{4.797969in}{1.372282in}}%
\pgfpathlineto{\pgfqpoint{4.797969in}{1.744378in}}%
\pgfpathquadraticcurveto{\pgfqpoint{4.797969in}{1.774934in}}{\pgfqpoint{4.767413in}{1.774934in}}%
\pgfpathlineto{\pgfqpoint{4.031901in}{1.774934in}}%
\pgfpathquadraticcurveto{\pgfqpoint{4.001346in}{1.774934in}}{\pgfqpoint{4.001346in}{1.744378in}}%
\pgfpathlineto{\pgfqpoint{4.001346in}{1.372282in}}%
\pgfpathquadraticcurveto{\pgfqpoint{4.001346in}{1.341726in}}{\pgfqpoint{4.031901in}{1.341726in}}%
\pgfpathlineto{\pgfqpoint{4.031901in}{1.341726in}}%
\pgfpathclose%
\pgfusepath{stroke,fill}%
\end{pgfscope}%
\begin{pgfscope}%
\pgfsetbuttcap%
\pgfsetmiterjoin%
\definecolor{currentfill}{rgb}{0.501961,0.694118,0.827451}%
\pgfsetfillcolor{currentfill}%
\pgfsetlinewidth{1.003750pt}%
\definecolor{currentstroke}{rgb}{1.000000,1.000000,1.000000}%
\pgfsetstrokecolor{currentstroke}%
\pgfsetdash{}{0pt}%
\pgfpathmoveto{\pgfqpoint{4.062457in}{1.597748in}}%
\pgfpathlineto{\pgfqpoint{4.368012in}{1.597748in}}%
\pgfpathlineto{\pgfqpoint{4.368012in}{1.704692in}}%
\pgfpathlineto{\pgfqpoint{4.062457in}{1.704692in}}%
\pgfpathlineto{\pgfqpoint{4.062457in}{1.597748in}}%
\pgfpathclose%
\pgfusepath{stroke,fill}%
\end{pgfscope}%
\begin{pgfscope}%
\definecolor{textcolor}{rgb}{0.150000,0.150000,0.150000}%
\pgfsetstrokecolor{textcolor}%
\pgfsetfillcolor{textcolor}%
\pgftext[x=4.490235in,y=1.597748in,left,base]{\color{textcolor}\sffamily\fontsize{11.000000}{13.200000}\selectfont Yes}%
\end{pgfscope}%
\begin{pgfscope}%
\pgfsetbuttcap%
\pgfsetmiterjoin%
\definecolor{currentfill}{rgb}{0.992157,0.705882,0.384314}%
\pgfsetfillcolor{currentfill}%
\pgfsetlinewidth{1.003750pt}%
\definecolor{currentstroke}{rgb}{1.000000,1.000000,1.000000}%
\pgfsetstrokecolor{currentstroke}%
\pgfsetdash{}{0pt}%
\pgfpathmoveto{\pgfqpoint{4.062457in}{1.434616in}}%
\pgfpathlineto{\pgfqpoint{4.368012in}{1.434616in}}%
\pgfpathlineto{\pgfqpoint{4.368012in}{1.541560in}}%
\pgfpathlineto{\pgfqpoint{4.062457in}{1.541560in}}%
\pgfpathlineto{\pgfqpoint{4.062457in}{1.434616in}}%
\pgfpathclose%
\pgfusepath{stroke,fill}%
\end{pgfscope}%
\begin{pgfscope}%
\definecolor{textcolor}{rgb}{0.150000,0.150000,0.150000}%
\pgfsetstrokecolor{textcolor}%
\pgfsetfillcolor{textcolor}%
\pgftext[x=4.490235in,y=1.434616in,left,base]{\color{textcolor}\sffamily\fontsize{11.000000}{13.200000}\selectfont No}%
\end{pgfscope}%
\end{pgfpicture}%
\makeatother%
\endgroup%
}
	\caption{The distribution of programming experience amongst study participants.}
	\label{fig:programmingexp}
\end{figure}

In the above figure, the programming experience of all the participants is
presented as a histogram. Programming experience is not a metric that can
reliably predict command line proficiency but can indicate the levels of
exposure to command line interfaces a participant may have had. In general,
most participants have been programming for 5 years or less and the most common
group was individuals who have been programming for 1 year. This makes sense as
a significant portion of participants were sourced through a recruiting email
distributed in the University of Zurich.

\begin{figure}[h]
	\centering
	\scalebox{0.8}{%% Creator: Matplotlib, PGF backend
%%
%% To include the figure in your LaTeX document, write
%%   \input{<filename>.pgf}
%%
%% Make sure the required packages are loaded in your preamble
%%   \usepackage{pgf}
%%
%% Also ensure that all the required font packages are loaded; for instance,
%% the lmodern package is sometimes necessary when using math font.
%%   \usepackage{lmodern}
%%
%% Figures using additional raster images can only be included by \input if
%% they are in the same directory as the main LaTeX file. For loading figures
%% from other directories you can use the `import` package
%%   \usepackage{import}
%%
%% and then include the figures with
%%   \import{<path to file>}{<filename>.pgf}
%%
%% Matplotlib used the following preamble
%%   \usepackage{fontspec}
%%   \setmainfont{DejaVuSerif.ttf}[Path=\detokenize{/home/spam/miniconda3/envs/mpl/lib/python3.10/site-packages/matplotlib/mpl-data/fonts/ttf/}]
%%   \setsansfont{DejaVuSans.ttf}[Path=\detokenize{/home/spam/miniconda3/envs/mpl/lib/python3.10/site-packages/matplotlib/mpl-data/fonts/ttf/}]
%%   \setmonofont{DejaVuSansMono.ttf}[Path=\detokenize{/home/spam/miniconda3/envs/mpl/lib/python3.10/site-packages/matplotlib/mpl-data/fonts/ttf/}]
%%
\begingroup%
\makeatletter%
\begin{pgfpicture}%
\pgfpathrectangle{\pgfpointorigin}{\pgfqpoint{8.799314in}{3.116660in}}%
\pgfusepath{use as bounding box, clip}%
\begin{pgfscope}%
\pgfsetbuttcap%
\pgfsetmiterjoin%
\pgfsetlinewidth{0.000000pt}%
\definecolor{currentstroke}{rgb}{0.000000,0.000000,0.000000}%
\pgfsetstrokecolor{currentstroke}%
\pgfsetstrokeopacity{0.000000}%
\pgfsetdash{}{0pt}%
\pgfpathmoveto{\pgfqpoint{0.000000in}{0.000000in}}%
\pgfpathlineto{\pgfqpoint{8.799314in}{0.000000in}}%
\pgfpathlineto{\pgfqpoint{8.799314in}{3.116660in}}%
\pgfpathlineto{\pgfqpoint{0.000000in}{3.116660in}}%
\pgfpathlineto{\pgfqpoint{0.000000in}{0.000000in}}%
\pgfpathclose%
\pgfusepath{}%
\end{pgfscope}%
\begin{pgfscope}%
\pgfsetbuttcap%
\pgfsetmiterjoin%
\definecolor{currentfill}{rgb}{0.501961,0.694118,0.827451}%
\pgfsetfillcolor{currentfill}%
\pgfsetlinewidth{1.003750pt}%
\definecolor{currentstroke}{rgb}{1.000000,1.000000,1.000000}%
\pgfsetstrokecolor{currentstroke}%
\pgfsetdash{}{0pt}%
\pgfpathmoveto{\pgfqpoint{3.894914in}{1.314902in}}%
\pgfpathcurveto{\pgfqpoint{3.894914in}{1.442534in}}{\pgfqpoint{3.869773in}{1.568924in}}{\pgfqpoint{3.820931in}{1.686841in}}%
\pgfpathcurveto{\pgfqpoint{3.772088in}{1.804757in}}{\pgfqpoint{3.700494in}{1.911906in}}{\pgfqpoint{3.610245in}{2.002155in}}%
\pgfpathcurveto{\pgfqpoint{3.519995in}{2.092404in}}{\pgfqpoint{3.412847in}{2.163998in}}{\pgfqpoint{3.294931in}{2.212841in}}%
\pgfpathcurveto{\pgfqpoint{3.177014in}{2.261683in}}{\pgfqpoint{3.050624in}{2.286824in}}{\pgfqpoint{2.922992in}{2.286824in}}%
\pgfpathcurveto{\pgfqpoint{2.795361in}{2.286824in}}{\pgfqpoint{2.668970in}{2.261683in}}{\pgfqpoint{2.551054in}{2.212841in}}%
\pgfpathcurveto{\pgfqpoint{2.433138in}{2.163998in}}{\pgfqpoint{2.325989in}{2.092404in}}{\pgfqpoint{2.235740in}{2.002155in}}%
\pgfpathcurveto{\pgfqpoint{2.145491in}{1.911906in}}{\pgfqpoint{2.073896in}{1.804757in}}{\pgfqpoint{2.025054in}{1.686841in}}%
\pgfpathcurveto{\pgfqpoint{1.976211in}{1.568924in}}{\pgfqpoint{1.951070in}{1.442534in}}{\pgfqpoint{1.951070in}{1.314902in}}%
\pgfpathcurveto{\pgfqpoint{1.951070in}{1.187271in}}{\pgfqpoint{1.976211in}{1.060880in}}{\pgfqpoint{2.025054in}{0.942964in}}%
\pgfpathcurveto{\pgfqpoint{2.073896in}{0.825048in}}{\pgfqpoint{2.145491in}{0.717899in}}{\pgfqpoint{2.235740in}{0.627650in}}%
\pgfpathcurveto{\pgfqpoint{2.325989in}{0.537401in}}{\pgfqpoint{2.433138in}{0.465806in}}{\pgfqpoint{2.551054in}{0.416964in}}%
\pgfpathcurveto{\pgfqpoint{2.668970in}{0.368121in}}{\pgfqpoint{2.795361in}{0.342980in}}{\pgfqpoint{2.922992in}{0.342980in}}%
\pgfpathcurveto{\pgfqpoint{3.050624in}{0.342980in}}{\pgfqpoint{3.177014in}{0.368121in}}{\pgfqpoint{3.294931in}{0.416964in}}%
\pgfpathcurveto{\pgfqpoint{3.412847in}{0.465806in}}{\pgfqpoint{3.519995in}{0.537401in}}{\pgfqpoint{3.610245in}{0.627650in}}%
\pgfpathcurveto{\pgfqpoint{3.700494in}{0.717899in}}{\pgfqpoint{3.772088in}{0.825048in}}{\pgfqpoint{3.820931in}{0.942964in}}%
\pgfpathcurveto{\pgfqpoint{3.869773in}{1.060880in}}{\pgfqpoint{3.894914in}{1.187271in}}{\pgfqpoint{3.894914in}{1.314902in}}%
\pgfpathmoveto{\pgfqpoint{2.922992in}{1.314902in}}%
\pgfpathmoveto{\pgfqpoint{3.894914in}{1.314902in}}%
\pgfpathlineto{\pgfqpoint{3.894914in}{1.314902in}}%
\pgfpathclose%
\pgfusepath{stroke,fill}%
\end{pgfscope}%
\begin{pgfscope}%
\pgfsetbuttcap%
\pgfsetmiterjoin%
\definecolor{currentfill}{rgb}{0.992157,0.705882,0.384314}%
\pgfsetfillcolor{currentfill}%
\pgfsetlinewidth{1.003750pt}%
\definecolor{currentstroke}{rgb}{1.000000,1.000000,1.000000}%
\pgfsetstrokecolor{currentstroke}%
\pgfsetdash{}{0pt}%
\pgfpathmoveto{\pgfqpoint{2.922992in}{2.286824in}}%
\pgfpathcurveto{\pgfqpoint{2.922992in}{2.286824in}}{\pgfqpoint{2.922992in}{2.286824in}}{\pgfqpoint{2.922992in}{2.286824in}}%
\pgfpathlineto{\pgfqpoint{2.922992in}{1.314902in}}%
\pgfpathlineto{\pgfqpoint{2.922992in}{2.286824in}}%
\pgfpathlineto{\pgfqpoint{2.922992in}{2.286824in}}%
\pgfpathclose%
\pgfusepath{stroke,fill}%
\end{pgfscope}%
\begin{pgfscope}%
\definecolor{textcolor}{rgb}{0.150000,0.150000,0.150000}%
\pgfsetstrokecolor{textcolor}%
\pgfsetfillcolor{textcolor}%
\pgftext[x=2.922992in,y=0.731749in,,]{\color{textcolor}\sffamily\fontsize{12.000000}{14.400000}\selectfont 100.00\%}%
\end{pgfscope}%
\begin{pgfscope}%
\definecolor{textcolor}{rgb}{0.150000,0.150000,0.150000}%
\pgfsetstrokecolor{textcolor}%
\pgfsetfillcolor{textcolor}%
\pgftext[x=2.922992in,y=1.898055in,,]{\color{textcolor}\sffamily\fontsize{12.000000}{14.400000}\selectfont 0.00\%}%
\end{pgfscope}%
\begin{pgfscope}%
\definecolor{textcolor}{rgb}{0.000000,0.000000,0.000000}%
\pgfsetstrokecolor{textcolor}%
\pgfsetfillcolor{textcolor}%
\pgftext[x=2.922992in,y=2.613138in,,base]{\color{textcolor}\sffamily\fontsize{12.000000}{14.400000}\selectfont CLI-Tutor}%
\end{pgfscope}%
\begin{pgfscope}%
\pgfsetbuttcap%
\pgfsetmiterjoin%
\definecolor{currentfill}{rgb}{0.501961,0.694118,0.827451}%
\pgfsetfillcolor{currentfill}%
\pgfsetlinewidth{1.003750pt}%
\definecolor{currentstroke}{rgb}{1.000000,1.000000,1.000000}%
\pgfsetstrokecolor{currentstroke}%
\pgfsetdash{}{0pt}%
\pgfpathmoveto{\pgfqpoint{5.876322in}{2.286824in}}%
\pgfpathcurveto{\pgfqpoint{5.725984in}{2.286824in}}{\pgfqpoint{5.577676in}{2.251942in}}{\pgfqpoint{5.443099in}{2.184931in}}%
\pgfpathcurveto{\pgfqpoint{5.308522in}{2.117920in}}{\pgfqpoint{5.191311in}{2.020588in}}{\pgfqpoint{5.100712in}{1.900616in}}%
\pgfpathcurveto{\pgfqpoint{5.010113in}{1.780644in}}{\pgfqpoint{4.948574in}{1.641271in}}{\pgfqpoint{4.920949in}{1.493492in}}%
\pgfpathcurveto{\pgfqpoint{4.893325in}{1.345714in}}{\pgfqpoint{4.900361in}{1.193522in}}{\pgfqpoint{4.941503in}{1.048923in}}%
\pgfpathcurveto{\pgfqpoint{4.982645in}{0.904324in}}{\pgfqpoint{5.056781in}{0.771224in}}{\pgfqpoint{5.158063in}{0.660123in}}%
\pgfpathcurveto{\pgfqpoint{5.259345in}{0.549022in}}{\pgfqpoint{5.385037in}{0.462921in}}{\pgfqpoint{5.525223in}{0.408612in}}%
\pgfpathcurveto{\pgfqpoint{5.665409in}{0.354304in}}{\pgfqpoint{5.816303in}{0.333255in}}{\pgfqpoint{5.966000in}{0.347126in}}%
\pgfpathcurveto{\pgfqpoint{6.115696in}{0.360998in}}{\pgfqpoint{6.260153in}{0.409415in}}{\pgfqpoint{6.387973in}{0.488558in}}%
\pgfpathlineto{\pgfqpoint{5.876322in}{1.314902in}}%
\pgfpathlineto{\pgfqpoint{5.876322in}{2.286824in}}%
\pgfpathlineto{\pgfqpoint{5.876322in}{2.286824in}}%
\pgfpathclose%
\pgfusepath{stroke,fill}%
\end{pgfscope}%
\begin{pgfscope}%
\pgfsetbuttcap%
\pgfsetmiterjoin%
\definecolor{currentfill}{rgb}{0.992157,0.705882,0.384314}%
\pgfsetfillcolor{currentfill}%
\pgfsetlinewidth{1.003750pt}%
\definecolor{currentstroke}{rgb}{1.000000,1.000000,1.000000}%
\pgfsetstrokecolor{currentstroke}%
\pgfsetdash{}{0pt}%
\pgfpathmoveto{\pgfqpoint{6.387973in}{0.488558in}}%
\pgfpathcurveto{\pgfqpoint{6.567670in}{0.599821in}}{\pgfqpoint{6.706262in}{0.766721in}}{\pgfqpoint{6.782612in}{0.963803in}}%
\pgfpathcurveto{\pgfqpoint{6.858962in}{1.160886in}}{\pgfqpoint{6.868981in}{1.377595in}}{\pgfqpoint{6.811141in}{1.580881in}}%
\pgfpathcurveto{\pgfqpoint{6.753302in}{1.784167in}}{\pgfqpoint{6.630701in}{1.963143in}}{\pgfqpoint{6.462036in}{2.090512in}}%
\pgfpathcurveto{\pgfqpoint{6.293371in}{2.217882in}}{\pgfqpoint{6.087677in}{2.286824in}}{\pgfqpoint{5.876322in}{2.286824in}}%
\pgfpathlineto{\pgfqpoint{5.876322in}{1.314902in}}%
\pgfpathlineto{\pgfqpoint{6.387973in}{0.488558in}}%
\pgfpathlineto{\pgfqpoint{6.387973in}{0.488558in}}%
\pgfpathclose%
\pgfusepath{stroke,fill}%
\end{pgfscope}%
\begin{pgfscope}%
\definecolor{textcolor}{rgb}{0.150000,0.150000,0.150000}%
\pgfsetstrokecolor{textcolor}%
\pgfsetfillcolor{textcolor}%
\pgftext[x=5.315431in,y=1.155315in,,]{\color{textcolor}\sffamily\fontsize{12.000000}{14.400000}\selectfont 58.82\%}%
\end{pgfscope}%
\begin{pgfscope}%
\definecolor{textcolor}{rgb}{0.150000,0.150000,0.150000}%
\pgfsetstrokecolor{textcolor}%
\pgfsetfillcolor{textcolor}%
\pgftext[x=6.437214in,y=1.474490in,,]{\color{textcolor}\sffamily\fontsize{12.000000}{14.400000}\selectfont 41.18\%}%
\end{pgfscope}%
\begin{pgfscope}%
\definecolor{textcolor}{rgb}{0.000000,0.000000,0.000000}%
\pgfsetstrokecolor{textcolor}%
\pgfsetfillcolor{textcolor}%
\pgftext[x=5.876322in,y=2.613138in,,base]{\color{textcolor}\sffamily\fontsize{12.000000}{14.400000}\selectfont Non Interactive Tutor}%
\end{pgfscope}%
\begin{pgfscope}%
\definecolor{textcolor}{rgb}{0.150000,0.150000,0.150000}%
\pgfsetstrokecolor{textcolor}%
\pgfsetfillcolor{textcolor}%
\pgftext[x=4.399657in,y=3.016660in,,top]{\color{textcolor}\sffamily\fontsize{14.400000}{17.280000}\selectfont Do you feel more or less intimidated by the command line after this interactive tutor?}%
\end{pgfscope}%
\begin{pgfscope}%
\pgfsetbuttcap%
\pgfsetmiterjoin%
\definecolor{currentfill}{rgb}{1.000000,1.000000,1.000000}%
\pgfsetfillcolor{currentfill}%
\pgfsetfillopacity{0.800000}%
\pgfsetlinewidth{1.003750pt}%
\definecolor{currentstroke}{rgb}{0.800000,0.800000,0.800000}%
\pgfsetstrokecolor{currentstroke}%
\pgfsetstrokeopacity{0.800000}%
\pgfsetdash{}{0pt}%
\pgfpathmoveto{\pgfqpoint{4.031901in}{1.341726in}}%
\pgfpathlineto{\pgfqpoint{4.767413in}{1.341726in}}%
\pgfpathquadraticcurveto{\pgfqpoint{4.797969in}{1.341726in}}{\pgfqpoint{4.797969in}{1.372282in}}%
\pgfpathlineto{\pgfqpoint{4.797969in}{1.744378in}}%
\pgfpathquadraticcurveto{\pgfqpoint{4.797969in}{1.774934in}}{\pgfqpoint{4.767413in}{1.774934in}}%
\pgfpathlineto{\pgfqpoint{4.031901in}{1.774934in}}%
\pgfpathquadraticcurveto{\pgfqpoint{4.001346in}{1.774934in}}{\pgfqpoint{4.001346in}{1.744378in}}%
\pgfpathlineto{\pgfqpoint{4.001346in}{1.372282in}}%
\pgfpathquadraticcurveto{\pgfqpoint{4.001346in}{1.341726in}}{\pgfqpoint{4.031901in}{1.341726in}}%
\pgfpathlineto{\pgfqpoint{4.031901in}{1.341726in}}%
\pgfpathclose%
\pgfusepath{stroke,fill}%
\end{pgfscope}%
\begin{pgfscope}%
\pgfsetbuttcap%
\pgfsetmiterjoin%
\definecolor{currentfill}{rgb}{0.501961,0.694118,0.827451}%
\pgfsetfillcolor{currentfill}%
\pgfsetlinewidth{1.003750pt}%
\definecolor{currentstroke}{rgb}{1.000000,1.000000,1.000000}%
\pgfsetstrokecolor{currentstroke}%
\pgfsetdash{}{0pt}%
\pgfpathmoveto{\pgfqpoint{4.062457in}{1.597748in}}%
\pgfpathlineto{\pgfqpoint{4.368012in}{1.597748in}}%
\pgfpathlineto{\pgfqpoint{4.368012in}{1.704692in}}%
\pgfpathlineto{\pgfqpoint{4.062457in}{1.704692in}}%
\pgfpathlineto{\pgfqpoint{4.062457in}{1.597748in}}%
\pgfpathclose%
\pgfusepath{stroke,fill}%
\end{pgfscope}%
\begin{pgfscope}%
\definecolor{textcolor}{rgb}{0.150000,0.150000,0.150000}%
\pgfsetstrokecolor{textcolor}%
\pgfsetfillcolor{textcolor}%
\pgftext[x=4.490235in,y=1.597748in,left,base]{\color{textcolor}\sffamily\fontsize{11.000000}{13.200000}\selectfont Yes}%
\end{pgfscope}%
\begin{pgfscope}%
\pgfsetbuttcap%
\pgfsetmiterjoin%
\definecolor{currentfill}{rgb}{0.992157,0.705882,0.384314}%
\pgfsetfillcolor{currentfill}%
\pgfsetlinewidth{1.003750pt}%
\definecolor{currentstroke}{rgb}{1.000000,1.000000,1.000000}%
\pgfsetstrokecolor{currentstroke}%
\pgfsetdash{}{0pt}%
\pgfpathmoveto{\pgfqpoint{4.062457in}{1.434616in}}%
\pgfpathlineto{\pgfqpoint{4.368012in}{1.434616in}}%
\pgfpathlineto{\pgfqpoint{4.368012in}{1.541560in}}%
\pgfpathlineto{\pgfqpoint{4.062457in}{1.541560in}}%
\pgfpathlineto{\pgfqpoint{4.062457in}{1.434616in}}%
\pgfpathclose%
\pgfusepath{stroke,fill}%
\end{pgfscope}%
\begin{pgfscope}%
\definecolor{textcolor}{rgb}{0.150000,0.150000,0.150000}%
\pgfsetstrokecolor{textcolor}%
\pgfsetfillcolor{textcolor}%
\pgftext[x=4.490235in,y=1.434616in,left,base]{\color{textcolor}\sffamily\fontsize{11.000000}{13.200000}\selectfont No}%
\end{pgfscope}%
\end{pgfpicture}%
\makeatother%
\endgroup%
}
	\caption{University level Computer Science experience amongst study participants.}
	\label{fig:uniexp}
\end{figure}

%TODO: SEE IF FOLLOWING PAGE THING HOLDS
Participants were also asked about whether they had any University experience
in Computer Science or related fields. Most participants had some degree of
experience at the Bachelor's and Master's level but despite this 29.32\%
reported not having any university experience in Computer Science. The complete
distribution across all levels of university experience can be found on the following page in
\autoref{fig:uniexp}.

\FloatBarrier %TODO: SEE IF THIS IS NECESSARY


\subsection{Feelings about the CLIs}

To gauge interest and feelings, questions regarding interest and comfort level
with CLIs were also asked. In general the interest in integrating the command
line more into day to day computer use was high. With 79.41\% (see:
\autoref{fig:daytoday}) of respondents reporting that they were interested in
doing so. Participants were also asked to explain their motivations.
Motivations were varied, but there were some commonalities in the motivations
to integrate CLIs more in daily computer usage. For some, the motivation was a
curiosity about getting to know their computers better:

\begin{quotes}
	"It is an additional tool, where one can learn and do a myriad of things with while also
	improving one's understanding of computer systems as a whole."
\end{quotes}

\begin{quotes}
	"to build up a better understanding of how things work behind the scenes."
\end{quotes}

For some respondents, this sentiment was even a bit altruistic.

\begin{quotes}
	"For using to fix ppl's and my computer if it breaks/bugs out"
\end{quotes}

Other sentiments expressed interest in integrating CLIs more in the work
environment as a productivity booster or as an important skill.


\begin{quotes}
	"Due to my field of work, more familiarity and proficiency with any CLI would be helpful"
\end{quotes}

\begin{quotes}
	"to optimize the workflow"
\end{quotes}

\begin{quotes}
	"It't much easier to replicate/reproduce the results than using GUI"
\end{quotes}

\begin{figure}[htbp]
	\scalebox{0.65}{%% Creator: Matplotlib, PGF backend
%%
%% To include the figure in your LaTeX document, write
%%   \input{<filename>.pgf}
%%
%% Make sure the required packages are loaded in your preamble
%%   \usepackage{pgf}
%%
%% Also ensure that all the required font packages are loaded; for instance,
%% the lmodern package is sometimes necessary when using math font.
%%   \usepackage{lmodern}
%%
%% Figures using additional raster images can only be included by \input if
%% they are in the same directory as the main LaTeX file. For loading figures
%% from other directories you can use the `import` package
%%   \usepackage{import}
%%
%% and then include the figures with
%%   \import{<path to file>}{<filename>.pgf}
%%
%% Matplotlib used the following preamble
%%   \usepackage{fontspec}
%%   \setmainfont{DejaVuSerif.ttf}[Path=\detokenize{/home/spam/miniconda3/envs/mpl/lib/python3.10/site-packages/matplotlib/mpl-data/fonts/ttf/}]
%%   \setsansfont{DejaVuSans.ttf}[Path=\detokenize{/home/spam/miniconda3/envs/mpl/lib/python3.10/site-packages/matplotlib/mpl-data/fonts/ttf/}]
%%   \setmonofont{DejaVuSansMono.ttf}[Path=\detokenize{/home/spam/miniconda3/envs/mpl/lib/python3.10/site-packages/matplotlib/mpl-data/fonts/ttf/}]
%%
\begingroup%
\makeatletter%
\begin{pgfpicture}%
\pgfpathrectangle{\pgfpointorigin}{\pgfqpoint{8.799314in}{3.116660in}}%
\pgfusepath{use as bounding box, clip}%
\begin{pgfscope}%
\pgfsetbuttcap%
\pgfsetmiterjoin%
\pgfsetlinewidth{0.000000pt}%
\definecolor{currentstroke}{rgb}{0.000000,0.000000,0.000000}%
\pgfsetstrokecolor{currentstroke}%
\pgfsetstrokeopacity{0.000000}%
\pgfsetdash{}{0pt}%
\pgfpathmoveto{\pgfqpoint{0.000000in}{0.000000in}}%
\pgfpathlineto{\pgfqpoint{8.799314in}{0.000000in}}%
\pgfpathlineto{\pgfqpoint{8.799314in}{3.116660in}}%
\pgfpathlineto{\pgfqpoint{0.000000in}{3.116660in}}%
\pgfpathlineto{\pgfqpoint{0.000000in}{0.000000in}}%
\pgfpathclose%
\pgfusepath{}%
\end{pgfscope}%
\begin{pgfscope}%
\pgfsetbuttcap%
\pgfsetmiterjoin%
\definecolor{currentfill}{rgb}{0.501961,0.694118,0.827451}%
\pgfsetfillcolor{currentfill}%
\pgfsetlinewidth{1.003750pt}%
\definecolor{currentstroke}{rgb}{1.000000,1.000000,1.000000}%
\pgfsetstrokecolor{currentstroke}%
\pgfsetdash{}{0pt}%
\pgfpathmoveto{\pgfqpoint{3.894914in}{1.314902in}}%
\pgfpathcurveto{\pgfqpoint{3.894914in}{1.442534in}}{\pgfqpoint{3.869773in}{1.568924in}}{\pgfqpoint{3.820931in}{1.686841in}}%
\pgfpathcurveto{\pgfqpoint{3.772088in}{1.804757in}}{\pgfqpoint{3.700494in}{1.911906in}}{\pgfqpoint{3.610245in}{2.002155in}}%
\pgfpathcurveto{\pgfqpoint{3.519995in}{2.092404in}}{\pgfqpoint{3.412847in}{2.163998in}}{\pgfqpoint{3.294931in}{2.212841in}}%
\pgfpathcurveto{\pgfqpoint{3.177014in}{2.261683in}}{\pgfqpoint{3.050624in}{2.286824in}}{\pgfqpoint{2.922992in}{2.286824in}}%
\pgfpathcurveto{\pgfqpoint{2.795361in}{2.286824in}}{\pgfqpoint{2.668970in}{2.261683in}}{\pgfqpoint{2.551054in}{2.212841in}}%
\pgfpathcurveto{\pgfqpoint{2.433138in}{2.163998in}}{\pgfqpoint{2.325989in}{2.092404in}}{\pgfqpoint{2.235740in}{2.002155in}}%
\pgfpathcurveto{\pgfqpoint{2.145491in}{1.911906in}}{\pgfqpoint{2.073896in}{1.804757in}}{\pgfqpoint{2.025054in}{1.686841in}}%
\pgfpathcurveto{\pgfqpoint{1.976211in}{1.568924in}}{\pgfqpoint{1.951070in}{1.442534in}}{\pgfqpoint{1.951070in}{1.314902in}}%
\pgfpathcurveto{\pgfqpoint{1.951070in}{1.187271in}}{\pgfqpoint{1.976211in}{1.060880in}}{\pgfqpoint{2.025054in}{0.942964in}}%
\pgfpathcurveto{\pgfqpoint{2.073896in}{0.825048in}}{\pgfqpoint{2.145491in}{0.717899in}}{\pgfqpoint{2.235740in}{0.627650in}}%
\pgfpathcurveto{\pgfqpoint{2.325989in}{0.537401in}}{\pgfqpoint{2.433138in}{0.465806in}}{\pgfqpoint{2.551054in}{0.416964in}}%
\pgfpathcurveto{\pgfqpoint{2.668970in}{0.368121in}}{\pgfqpoint{2.795361in}{0.342980in}}{\pgfqpoint{2.922992in}{0.342980in}}%
\pgfpathcurveto{\pgfqpoint{3.050624in}{0.342980in}}{\pgfqpoint{3.177014in}{0.368121in}}{\pgfqpoint{3.294931in}{0.416964in}}%
\pgfpathcurveto{\pgfqpoint{3.412847in}{0.465806in}}{\pgfqpoint{3.519995in}{0.537401in}}{\pgfqpoint{3.610245in}{0.627650in}}%
\pgfpathcurveto{\pgfqpoint{3.700494in}{0.717899in}}{\pgfqpoint{3.772088in}{0.825048in}}{\pgfqpoint{3.820931in}{0.942964in}}%
\pgfpathcurveto{\pgfqpoint{3.869773in}{1.060880in}}{\pgfqpoint{3.894914in}{1.187271in}}{\pgfqpoint{3.894914in}{1.314902in}}%
\pgfpathmoveto{\pgfqpoint{2.922992in}{1.314902in}}%
\pgfpathmoveto{\pgfqpoint{3.894914in}{1.314902in}}%
\pgfpathlineto{\pgfqpoint{3.894914in}{1.314902in}}%
\pgfpathclose%
\pgfusepath{stroke,fill}%
\end{pgfscope}%
\begin{pgfscope}%
\pgfsetbuttcap%
\pgfsetmiterjoin%
\definecolor{currentfill}{rgb}{0.992157,0.705882,0.384314}%
\pgfsetfillcolor{currentfill}%
\pgfsetlinewidth{1.003750pt}%
\definecolor{currentstroke}{rgb}{1.000000,1.000000,1.000000}%
\pgfsetstrokecolor{currentstroke}%
\pgfsetdash{}{0pt}%
\pgfpathmoveto{\pgfqpoint{2.922992in}{2.286824in}}%
\pgfpathcurveto{\pgfqpoint{2.922992in}{2.286824in}}{\pgfqpoint{2.922992in}{2.286824in}}{\pgfqpoint{2.922992in}{2.286824in}}%
\pgfpathlineto{\pgfqpoint{2.922992in}{1.314902in}}%
\pgfpathlineto{\pgfqpoint{2.922992in}{2.286824in}}%
\pgfpathlineto{\pgfqpoint{2.922992in}{2.286824in}}%
\pgfpathclose%
\pgfusepath{stroke,fill}%
\end{pgfscope}%
\begin{pgfscope}%
\definecolor{textcolor}{rgb}{0.150000,0.150000,0.150000}%
\pgfsetstrokecolor{textcolor}%
\pgfsetfillcolor{textcolor}%
\pgftext[x=2.922992in,y=0.731749in,,]{\color{textcolor}\sffamily\fontsize{12.000000}{14.400000}\selectfont 100.00\%}%
\end{pgfscope}%
\begin{pgfscope}%
\definecolor{textcolor}{rgb}{0.150000,0.150000,0.150000}%
\pgfsetstrokecolor{textcolor}%
\pgfsetfillcolor{textcolor}%
\pgftext[x=2.922992in,y=1.898055in,,]{\color{textcolor}\sffamily\fontsize{12.000000}{14.400000}\selectfont 0.00\%}%
\end{pgfscope}%
\begin{pgfscope}%
\definecolor{textcolor}{rgb}{0.000000,0.000000,0.000000}%
\pgfsetstrokecolor{textcolor}%
\pgfsetfillcolor{textcolor}%
\pgftext[x=2.922992in,y=2.613138in,,base]{\color{textcolor}\sffamily\fontsize{12.000000}{14.400000}\selectfont CLI-Tutor}%
\end{pgfscope}%
\begin{pgfscope}%
\pgfsetbuttcap%
\pgfsetmiterjoin%
\definecolor{currentfill}{rgb}{0.501961,0.694118,0.827451}%
\pgfsetfillcolor{currentfill}%
\pgfsetlinewidth{1.003750pt}%
\definecolor{currentstroke}{rgb}{1.000000,1.000000,1.000000}%
\pgfsetstrokecolor{currentstroke}%
\pgfsetdash{}{0pt}%
\pgfpathmoveto{\pgfqpoint{5.876322in}{2.286824in}}%
\pgfpathcurveto{\pgfqpoint{5.725984in}{2.286824in}}{\pgfqpoint{5.577676in}{2.251942in}}{\pgfqpoint{5.443099in}{2.184931in}}%
\pgfpathcurveto{\pgfqpoint{5.308522in}{2.117920in}}{\pgfqpoint{5.191311in}{2.020588in}}{\pgfqpoint{5.100712in}{1.900616in}}%
\pgfpathcurveto{\pgfqpoint{5.010113in}{1.780644in}}{\pgfqpoint{4.948574in}{1.641271in}}{\pgfqpoint{4.920949in}{1.493492in}}%
\pgfpathcurveto{\pgfqpoint{4.893325in}{1.345714in}}{\pgfqpoint{4.900361in}{1.193522in}}{\pgfqpoint{4.941503in}{1.048923in}}%
\pgfpathcurveto{\pgfqpoint{4.982645in}{0.904324in}}{\pgfqpoint{5.056781in}{0.771224in}}{\pgfqpoint{5.158063in}{0.660123in}}%
\pgfpathcurveto{\pgfqpoint{5.259345in}{0.549022in}}{\pgfqpoint{5.385037in}{0.462921in}}{\pgfqpoint{5.525223in}{0.408612in}}%
\pgfpathcurveto{\pgfqpoint{5.665409in}{0.354304in}}{\pgfqpoint{5.816303in}{0.333255in}}{\pgfqpoint{5.966000in}{0.347126in}}%
\pgfpathcurveto{\pgfqpoint{6.115696in}{0.360998in}}{\pgfqpoint{6.260153in}{0.409415in}}{\pgfqpoint{6.387973in}{0.488558in}}%
\pgfpathlineto{\pgfqpoint{5.876322in}{1.314902in}}%
\pgfpathlineto{\pgfqpoint{5.876322in}{2.286824in}}%
\pgfpathlineto{\pgfqpoint{5.876322in}{2.286824in}}%
\pgfpathclose%
\pgfusepath{stroke,fill}%
\end{pgfscope}%
\begin{pgfscope}%
\pgfsetbuttcap%
\pgfsetmiterjoin%
\definecolor{currentfill}{rgb}{0.992157,0.705882,0.384314}%
\pgfsetfillcolor{currentfill}%
\pgfsetlinewidth{1.003750pt}%
\definecolor{currentstroke}{rgb}{1.000000,1.000000,1.000000}%
\pgfsetstrokecolor{currentstroke}%
\pgfsetdash{}{0pt}%
\pgfpathmoveto{\pgfqpoint{6.387973in}{0.488558in}}%
\pgfpathcurveto{\pgfqpoint{6.567670in}{0.599821in}}{\pgfqpoint{6.706262in}{0.766721in}}{\pgfqpoint{6.782612in}{0.963803in}}%
\pgfpathcurveto{\pgfqpoint{6.858962in}{1.160886in}}{\pgfqpoint{6.868981in}{1.377595in}}{\pgfqpoint{6.811141in}{1.580881in}}%
\pgfpathcurveto{\pgfqpoint{6.753302in}{1.784167in}}{\pgfqpoint{6.630701in}{1.963143in}}{\pgfqpoint{6.462036in}{2.090512in}}%
\pgfpathcurveto{\pgfqpoint{6.293371in}{2.217882in}}{\pgfqpoint{6.087677in}{2.286824in}}{\pgfqpoint{5.876322in}{2.286824in}}%
\pgfpathlineto{\pgfqpoint{5.876322in}{1.314902in}}%
\pgfpathlineto{\pgfqpoint{6.387973in}{0.488558in}}%
\pgfpathlineto{\pgfqpoint{6.387973in}{0.488558in}}%
\pgfpathclose%
\pgfusepath{stroke,fill}%
\end{pgfscope}%
\begin{pgfscope}%
\definecolor{textcolor}{rgb}{0.150000,0.150000,0.150000}%
\pgfsetstrokecolor{textcolor}%
\pgfsetfillcolor{textcolor}%
\pgftext[x=5.315431in,y=1.155315in,,]{\color{textcolor}\sffamily\fontsize{12.000000}{14.400000}\selectfont 58.82\%}%
\end{pgfscope}%
\begin{pgfscope}%
\definecolor{textcolor}{rgb}{0.150000,0.150000,0.150000}%
\pgfsetstrokecolor{textcolor}%
\pgfsetfillcolor{textcolor}%
\pgftext[x=6.437214in,y=1.474490in,,]{\color{textcolor}\sffamily\fontsize{12.000000}{14.400000}\selectfont 41.18\%}%
\end{pgfscope}%
\begin{pgfscope}%
\definecolor{textcolor}{rgb}{0.000000,0.000000,0.000000}%
\pgfsetstrokecolor{textcolor}%
\pgfsetfillcolor{textcolor}%
\pgftext[x=5.876322in,y=2.613138in,,base]{\color{textcolor}\sffamily\fontsize{12.000000}{14.400000}\selectfont Non Interactive Tutor}%
\end{pgfscope}%
\begin{pgfscope}%
\definecolor{textcolor}{rgb}{0.150000,0.150000,0.150000}%
\pgfsetstrokecolor{textcolor}%
\pgfsetfillcolor{textcolor}%
\pgftext[x=4.399657in,y=3.016660in,,top]{\color{textcolor}\sffamily\fontsize{14.400000}{17.280000}\selectfont Do you feel more or less intimidated by the command line after this interactive tutor?}%
\end{pgfscope}%
\begin{pgfscope}%
\pgfsetbuttcap%
\pgfsetmiterjoin%
\definecolor{currentfill}{rgb}{1.000000,1.000000,1.000000}%
\pgfsetfillcolor{currentfill}%
\pgfsetfillopacity{0.800000}%
\pgfsetlinewidth{1.003750pt}%
\definecolor{currentstroke}{rgb}{0.800000,0.800000,0.800000}%
\pgfsetstrokecolor{currentstroke}%
\pgfsetstrokeopacity{0.800000}%
\pgfsetdash{}{0pt}%
\pgfpathmoveto{\pgfqpoint{4.031901in}{1.341726in}}%
\pgfpathlineto{\pgfqpoint{4.767413in}{1.341726in}}%
\pgfpathquadraticcurveto{\pgfqpoint{4.797969in}{1.341726in}}{\pgfqpoint{4.797969in}{1.372282in}}%
\pgfpathlineto{\pgfqpoint{4.797969in}{1.744378in}}%
\pgfpathquadraticcurveto{\pgfqpoint{4.797969in}{1.774934in}}{\pgfqpoint{4.767413in}{1.774934in}}%
\pgfpathlineto{\pgfqpoint{4.031901in}{1.774934in}}%
\pgfpathquadraticcurveto{\pgfqpoint{4.001346in}{1.774934in}}{\pgfqpoint{4.001346in}{1.744378in}}%
\pgfpathlineto{\pgfqpoint{4.001346in}{1.372282in}}%
\pgfpathquadraticcurveto{\pgfqpoint{4.001346in}{1.341726in}}{\pgfqpoint{4.031901in}{1.341726in}}%
\pgfpathlineto{\pgfqpoint{4.031901in}{1.341726in}}%
\pgfpathclose%
\pgfusepath{stroke,fill}%
\end{pgfscope}%
\begin{pgfscope}%
\pgfsetbuttcap%
\pgfsetmiterjoin%
\definecolor{currentfill}{rgb}{0.501961,0.694118,0.827451}%
\pgfsetfillcolor{currentfill}%
\pgfsetlinewidth{1.003750pt}%
\definecolor{currentstroke}{rgb}{1.000000,1.000000,1.000000}%
\pgfsetstrokecolor{currentstroke}%
\pgfsetdash{}{0pt}%
\pgfpathmoveto{\pgfqpoint{4.062457in}{1.597748in}}%
\pgfpathlineto{\pgfqpoint{4.368012in}{1.597748in}}%
\pgfpathlineto{\pgfqpoint{4.368012in}{1.704692in}}%
\pgfpathlineto{\pgfqpoint{4.062457in}{1.704692in}}%
\pgfpathlineto{\pgfqpoint{4.062457in}{1.597748in}}%
\pgfpathclose%
\pgfusepath{stroke,fill}%
\end{pgfscope}%
\begin{pgfscope}%
\definecolor{textcolor}{rgb}{0.150000,0.150000,0.150000}%
\pgfsetstrokecolor{textcolor}%
\pgfsetfillcolor{textcolor}%
\pgftext[x=4.490235in,y=1.597748in,left,base]{\color{textcolor}\sffamily\fontsize{11.000000}{13.200000}\selectfont Yes}%
\end{pgfscope}%
\begin{pgfscope}%
\pgfsetbuttcap%
\pgfsetmiterjoin%
\definecolor{currentfill}{rgb}{0.992157,0.705882,0.384314}%
\pgfsetfillcolor{currentfill}%
\pgfsetlinewidth{1.003750pt}%
\definecolor{currentstroke}{rgb}{1.000000,1.000000,1.000000}%
\pgfsetstrokecolor{currentstroke}%
\pgfsetdash{}{0pt}%
\pgfpathmoveto{\pgfqpoint{4.062457in}{1.434616in}}%
\pgfpathlineto{\pgfqpoint{4.368012in}{1.434616in}}%
\pgfpathlineto{\pgfqpoint{4.368012in}{1.541560in}}%
\pgfpathlineto{\pgfqpoint{4.062457in}{1.541560in}}%
\pgfpathlineto{\pgfqpoint{4.062457in}{1.434616in}}%
\pgfpathclose%
\pgfusepath{stroke,fill}%
\end{pgfscope}%
\begin{pgfscope}%
\definecolor{textcolor}{rgb}{0.150000,0.150000,0.150000}%
\pgfsetstrokecolor{textcolor}%
\pgfsetfillcolor{textcolor}%
\pgftext[x=4.490235in,y=1.434616in,left,base]{\color{textcolor}\sffamily\fontsize{11.000000}{13.200000}\selectfont No}%
\end{pgfscope}%
\end{pgfpicture}%
\makeatother%
\endgroup%
}
	\caption{Chart depicting interest in integrating CLIs more into day to day computer use.}
	\label{fig:daytoday}
\end{figure}

Another factor the participants were questioned about was their existing
comfort level with command line interfaces. This question was presented in a
Likert-style fashion with answers ranging from "Extremely Uncomfortable" to
"Extremely Comfortable" (see: \autoref{fig:comfortlevel}).

\begin{figure}[H]
	\scalebox{0.72}{%% Creator: Matplotlib, PGF backend
%%
%% To include the figure in your LaTeX document, write
%%   \input{<filename>.pgf}
%%
%% Make sure the required packages are loaded in your preamble
%%   \usepackage{pgf}
%%
%% Also ensure that all the required font packages are loaded; for instance,
%% the lmodern package is sometimes necessary when using math font.
%%   \usepackage{lmodern}
%%
%% Figures using additional raster images can only be included by \input if
%% they are in the same directory as the main LaTeX file. For loading figures
%% from other directories you can use the `import` package
%%   \usepackage{import}
%%
%% and then include the figures with
%%   \import{<path to file>}{<filename>.pgf}
%%
%% Matplotlib used the following preamble
%%   \usepackage{fontspec}
%%   \setmainfont{DejaVuSerif.ttf}[Path=\detokenize{/home/spam/miniconda3/envs/mpl/lib/python3.10/site-packages/matplotlib/mpl-data/fonts/ttf/}]
%%   \setsansfont{DejaVuSans.ttf}[Path=\detokenize{/home/spam/miniconda3/envs/mpl/lib/python3.10/site-packages/matplotlib/mpl-data/fonts/ttf/}]
%%   \setmonofont{DejaVuSansMono.ttf}[Path=\detokenize{/home/spam/miniconda3/envs/mpl/lib/python3.10/site-packages/matplotlib/mpl-data/fonts/ttf/}]
%%
\begingroup%
\makeatletter%
\begin{pgfpicture}%
\pgfpathrectangle{\pgfpointorigin}{\pgfqpoint{8.799314in}{3.116660in}}%
\pgfusepath{use as bounding box, clip}%
\begin{pgfscope}%
\pgfsetbuttcap%
\pgfsetmiterjoin%
\pgfsetlinewidth{0.000000pt}%
\definecolor{currentstroke}{rgb}{0.000000,0.000000,0.000000}%
\pgfsetstrokecolor{currentstroke}%
\pgfsetstrokeopacity{0.000000}%
\pgfsetdash{}{0pt}%
\pgfpathmoveto{\pgfqpoint{0.000000in}{0.000000in}}%
\pgfpathlineto{\pgfqpoint{8.799314in}{0.000000in}}%
\pgfpathlineto{\pgfqpoint{8.799314in}{3.116660in}}%
\pgfpathlineto{\pgfqpoint{0.000000in}{3.116660in}}%
\pgfpathlineto{\pgfqpoint{0.000000in}{0.000000in}}%
\pgfpathclose%
\pgfusepath{}%
\end{pgfscope}%
\begin{pgfscope}%
\pgfsetbuttcap%
\pgfsetmiterjoin%
\definecolor{currentfill}{rgb}{0.501961,0.694118,0.827451}%
\pgfsetfillcolor{currentfill}%
\pgfsetlinewidth{1.003750pt}%
\definecolor{currentstroke}{rgb}{1.000000,1.000000,1.000000}%
\pgfsetstrokecolor{currentstroke}%
\pgfsetdash{}{0pt}%
\pgfpathmoveto{\pgfqpoint{3.894914in}{1.314902in}}%
\pgfpathcurveto{\pgfqpoint{3.894914in}{1.442534in}}{\pgfqpoint{3.869773in}{1.568924in}}{\pgfqpoint{3.820931in}{1.686841in}}%
\pgfpathcurveto{\pgfqpoint{3.772088in}{1.804757in}}{\pgfqpoint{3.700494in}{1.911906in}}{\pgfqpoint{3.610245in}{2.002155in}}%
\pgfpathcurveto{\pgfqpoint{3.519995in}{2.092404in}}{\pgfqpoint{3.412847in}{2.163998in}}{\pgfqpoint{3.294931in}{2.212841in}}%
\pgfpathcurveto{\pgfqpoint{3.177014in}{2.261683in}}{\pgfqpoint{3.050624in}{2.286824in}}{\pgfqpoint{2.922992in}{2.286824in}}%
\pgfpathcurveto{\pgfqpoint{2.795361in}{2.286824in}}{\pgfqpoint{2.668970in}{2.261683in}}{\pgfqpoint{2.551054in}{2.212841in}}%
\pgfpathcurveto{\pgfqpoint{2.433138in}{2.163998in}}{\pgfqpoint{2.325989in}{2.092404in}}{\pgfqpoint{2.235740in}{2.002155in}}%
\pgfpathcurveto{\pgfqpoint{2.145491in}{1.911906in}}{\pgfqpoint{2.073896in}{1.804757in}}{\pgfqpoint{2.025054in}{1.686841in}}%
\pgfpathcurveto{\pgfqpoint{1.976211in}{1.568924in}}{\pgfqpoint{1.951070in}{1.442534in}}{\pgfqpoint{1.951070in}{1.314902in}}%
\pgfpathcurveto{\pgfqpoint{1.951070in}{1.187271in}}{\pgfqpoint{1.976211in}{1.060880in}}{\pgfqpoint{2.025054in}{0.942964in}}%
\pgfpathcurveto{\pgfqpoint{2.073896in}{0.825048in}}{\pgfqpoint{2.145491in}{0.717899in}}{\pgfqpoint{2.235740in}{0.627650in}}%
\pgfpathcurveto{\pgfqpoint{2.325989in}{0.537401in}}{\pgfqpoint{2.433138in}{0.465806in}}{\pgfqpoint{2.551054in}{0.416964in}}%
\pgfpathcurveto{\pgfqpoint{2.668970in}{0.368121in}}{\pgfqpoint{2.795361in}{0.342980in}}{\pgfqpoint{2.922992in}{0.342980in}}%
\pgfpathcurveto{\pgfqpoint{3.050624in}{0.342980in}}{\pgfqpoint{3.177014in}{0.368121in}}{\pgfqpoint{3.294931in}{0.416964in}}%
\pgfpathcurveto{\pgfqpoint{3.412847in}{0.465806in}}{\pgfqpoint{3.519995in}{0.537401in}}{\pgfqpoint{3.610245in}{0.627650in}}%
\pgfpathcurveto{\pgfqpoint{3.700494in}{0.717899in}}{\pgfqpoint{3.772088in}{0.825048in}}{\pgfqpoint{3.820931in}{0.942964in}}%
\pgfpathcurveto{\pgfqpoint{3.869773in}{1.060880in}}{\pgfqpoint{3.894914in}{1.187271in}}{\pgfqpoint{3.894914in}{1.314902in}}%
\pgfpathmoveto{\pgfqpoint{2.922992in}{1.314902in}}%
\pgfpathmoveto{\pgfqpoint{3.894914in}{1.314902in}}%
\pgfpathlineto{\pgfqpoint{3.894914in}{1.314902in}}%
\pgfpathclose%
\pgfusepath{stroke,fill}%
\end{pgfscope}%
\begin{pgfscope}%
\pgfsetbuttcap%
\pgfsetmiterjoin%
\definecolor{currentfill}{rgb}{0.992157,0.705882,0.384314}%
\pgfsetfillcolor{currentfill}%
\pgfsetlinewidth{1.003750pt}%
\definecolor{currentstroke}{rgb}{1.000000,1.000000,1.000000}%
\pgfsetstrokecolor{currentstroke}%
\pgfsetdash{}{0pt}%
\pgfpathmoveto{\pgfqpoint{2.922992in}{2.286824in}}%
\pgfpathcurveto{\pgfqpoint{2.922992in}{2.286824in}}{\pgfqpoint{2.922992in}{2.286824in}}{\pgfqpoint{2.922992in}{2.286824in}}%
\pgfpathlineto{\pgfqpoint{2.922992in}{1.314902in}}%
\pgfpathlineto{\pgfqpoint{2.922992in}{2.286824in}}%
\pgfpathlineto{\pgfqpoint{2.922992in}{2.286824in}}%
\pgfpathclose%
\pgfusepath{stroke,fill}%
\end{pgfscope}%
\begin{pgfscope}%
\definecolor{textcolor}{rgb}{0.150000,0.150000,0.150000}%
\pgfsetstrokecolor{textcolor}%
\pgfsetfillcolor{textcolor}%
\pgftext[x=2.922992in,y=0.731749in,,]{\color{textcolor}\sffamily\fontsize{12.000000}{14.400000}\selectfont 100.00\%}%
\end{pgfscope}%
\begin{pgfscope}%
\definecolor{textcolor}{rgb}{0.150000,0.150000,0.150000}%
\pgfsetstrokecolor{textcolor}%
\pgfsetfillcolor{textcolor}%
\pgftext[x=2.922992in,y=1.898055in,,]{\color{textcolor}\sffamily\fontsize{12.000000}{14.400000}\selectfont 0.00\%}%
\end{pgfscope}%
\begin{pgfscope}%
\definecolor{textcolor}{rgb}{0.000000,0.000000,0.000000}%
\pgfsetstrokecolor{textcolor}%
\pgfsetfillcolor{textcolor}%
\pgftext[x=2.922992in,y=2.613138in,,base]{\color{textcolor}\sffamily\fontsize{12.000000}{14.400000}\selectfont CLI-Tutor}%
\end{pgfscope}%
\begin{pgfscope}%
\pgfsetbuttcap%
\pgfsetmiterjoin%
\definecolor{currentfill}{rgb}{0.501961,0.694118,0.827451}%
\pgfsetfillcolor{currentfill}%
\pgfsetlinewidth{1.003750pt}%
\definecolor{currentstroke}{rgb}{1.000000,1.000000,1.000000}%
\pgfsetstrokecolor{currentstroke}%
\pgfsetdash{}{0pt}%
\pgfpathmoveto{\pgfqpoint{5.876322in}{2.286824in}}%
\pgfpathcurveto{\pgfqpoint{5.725984in}{2.286824in}}{\pgfqpoint{5.577676in}{2.251942in}}{\pgfqpoint{5.443099in}{2.184931in}}%
\pgfpathcurveto{\pgfqpoint{5.308522in}{2.117920in}}{\pgfqpoint{5.191311in}{2.020588in}}{\pgfqpoint{5.100712in}{1.900616in}}%
\pgfpathcurveto{\pgfqpoint{5.010113in}{1.780644in}}{\pgfqpoint{4.948574in}{1.641271in}}{\pgfqpoint{4.920949in}{1.493492in}}%
\pgfpathcurveto{\pgfqpoint{4.893325in}{1.345714in}}{\pgfqpoint{4.900361in}{1.193522in}}{\pgfqpoint{4.941503in}{1.048923in}}%
\pgfpathcurveto{\pgfqpoint{4.982645in}{0.904324in}}{\pgfqpoint{5.056781in}{0.771224in}}{\pgfqpoint{5.158063in}{0.660123in}}%
\pgfpathcurveto{\pgfqpoint{5.259345in}{0.549022in}}{\pgfqpoint{5.385037in}{0.462921in}}{\pgfqpoint{5.525223in}{0.408612in}}%
\pgfpathcurveto{\pgfqpoint{5.665409in}{0.354304in}}{\pgfqpoint{5.816303in}{0.333255in}}{\pgfqpoint{5.966000in}{0.347126in}}%
\pgfpathcurveto{\pgfqpoint{6.115696in}{0.360998in}}{\pgfqpoint{6.260153in}{0.409415in}}{\pgfqpoint{6.387973in}{0.488558in}}%
\pgfpathlineto{\pgfqpoint{5.876322in}{1.314902in}}%
\pgfpathlineto{\pgfqpoint{5.876322in}{2.286824in}}%
\pgfpathlineto{\pgfqpoint{5.876322in}{2.286824in}}%
\pgfpathclose%
\pgfusepath{stroke,fill}%
\end{pgfscope}%
\begin{pgfscope}%
\pgfsetbuttcap%
\pgfsetmiterjoin%
\definecolor{currentfill}{rgb}{0.992157,0.705882,0.384314}%
\pgfsetfillcolor{currentfill}%
\pgfsetlinewidth{1.003750pt}%
\definecolor{currentstroke}{rgb}{1.000000,1.000000,1.000000}%
\pgfsetstrokecolor{currentstroke}%
\pgfsetdash{}{0pt}%
\pgfpathmoveto{\pgfqpoint{6.387973in}{0.488558in}}%
\pgfpathcurveto{\pgfqpoint{6.567670in}{0.599821in}}{\pgfqpoint{6.706262in}{0.766721in}}{\pgfqpoint{6.782612in}{0.963803in}}%
\pgfpathcurveto{\pgfqpoint{6.858962in}{1.160886in}}{\pgfqpoint{6.868981in}{1.377595in}}{\pgfqpoint{6.811141in}{1.580881in}}%
\pgfpathcurveto{\pgfqpoint{6.753302in}{1.784167in}}{\pgfqpoint{6.630701in}{1.963143in}}{\pgfqpoint{6.462036in}{2.090512in}}%
\pgfpathcurveto{\pgfqpoint{6.293371in}{2.217882in}}{\pgfqpoint{6.087677in}{2.286824in}}{\pgfqpoint{5.876322in}{2.286824in}}%
\pgfpathlineto{\pgfqpoint{5.876322in}{1.314902in}}%
\pgfpathlineto{\pgfqpoint{6.387973in}{0.488558in}}%
\pgfpathlineto{\pgfqpoint{6.387973in}{0.488558in}}%
\pgfpathclose%
\pgfusepath{stroke,fill}%
\end{pgfscope}%
\begin{pgfscope}%
\definecolor{textcolor}{rgb}{0.150000,0.150000,0.150000}%
\pgfsetstrokecolor{textcolor}%
\pgfsetfillcolor{textcolor}%
\pgftext[x=5.315431in,y=1.155315in,,]{\color{textcolor}\sffamily\fontsize{12.000000}{14.400000}\selectfont 58.82\%}%
\end{pgfscope}%
\begin{pgfscope}%
\definecolor{textcolor}{rgb}{0.150000,0.150000,0.150000}%
\pgfsetstrokecolor{textcolor}%
\pgfsetfillcolor{textcolor}%
\pgftext[x=6.437214in,y=1.474490in,,]{\color{textcolor}\sffamily\fontsize{12.000000}{14.400000}\selectfont 41.18\%}%
\end{pgfscope}%
\begin{pgfscope}%
\definecolor{textcolor}{rgb}{0.000000,0.000000,0.000000}%
\pgfsetstrokecolor{textcolor}%
\pgfsetfillcolor{textcolor}%
\pgftext[x=5.876322in,y=2.613138in,,base]{\color{textcolor}\sffamily\fontsize{12.000000}{14.400000}\selectfont Non Interactive Tutor}%
\end{pgfscope}%
\begin{pgfscope}%
\definecolor{textcolor}{rgb}{0.150000,0.150000,0.150000}%
\pgfsetstrokecolor{textcolor}%
\pgfsetfillcolor{textcolor}%
\pgftext[x=4.399657in,y=3.016660in,,top]{\color{textcolor}\sffamily\fontsize{14.400000}{17.280000}\selectfont Do you feel more or less intimidated by the command line after this interactive tutor?}%
\end{pgfscope}%
\begin{pgfscope}%
\pgfsetbuttcap%
\pgfsetmiterjoin%
\definecolor{currentfill}{rgb}{1.000000,1.000000,1.000000}%
\pgfsetfillcolor{currentfill}%
\pgfsetfillopacity{0.800000}%
\pgfsetlinewidth{1.003750pt}%
\definecolor{currentstroke}{rgb}{0.800000,0.800000,0.800000}%
\pgfsetstrokecolor{currentstroke}%
\pgfsetstrokeopacity{0.800000}%
\pgfsetdash{}{0pt}%
\pgfpathmoveto{\pgfqpoint{4.031901in}{1.341726in}}%
\pgfpathlineto{\pgfqpoint{4.767413in}{1.341726in}}%
\pgfpathquadraticcurveto{\pgfqpoint{4.797969in}{1.341726in}}{\pgfqpoint{4.797969in}{1.372282in}}%
\pgfpathlineto{\pgfqpoint{4.797969in}{1.744378in}}%
\pgfpathquadraticcurveto{\pgfqpoint{4.797969in}{1.774934in}}{\pgfqpoint{4.767413in}{1.774934in}}%
\pgfpathlineto{\pgfqpoint{4.031901in}{1.774934in}}%
\pgfpathquadraticcurveto{\pgfqpoint{4.001346in}{1.774934in}}{\pgfqpoint{4.001346in}{1.744378in}}%
\pgfpathlineto{\pgfqpoint{4.001346in}{1.372282in}}%
\pgfpathquadraticcurveto{\pgfqpoint{4.001346in}{1.341726in}}{\pgfqpoint{4.031901in}{1.341726in}}%
\pgfpathlineto{\pgfqpoint{4.031901in}{1.341726in}}%
\pgfpathclose%
\pgfusepath{stroke,fill}%
\end{pgfscope}%
\begin{pgfscope}%
\pgfsetbuttcap%
\pgfsetmiterjoin%
\definecolor{currentfill}{rgb}{0.501961,0.694118,0.827451}%
\pgfsetfillcolor{currentfill}%
\pgfsetlinewidth{1.003750pt}%
\definecolor{currentstroke}{rgb}{1.000000,1.000000,1.000000}%
\pgfsetstrokecolor{currentstroke}%
\pgfsetdash{}{0pt}%
\pgfpathmoveto{\pgfqpoint{4.062457in}{1.597748in}}%
\pgfpathlineto{\pgfqpoint{4.368012in}{1.597748in}}%
\pgfpathlineto{\pgfqpoint{4.368012in}{1.704692in}}%
\pgfpathlineto{\pgfqpoint{4.062457in}{1.704692in}}%
\pgfpathlineto{\pgfqpoint{4.062457in}{1.597748in}}%
\pgfpathclose%
\pgfusepath{stroke,fill}%
\end{pgfscope}%
\begin{pgfscope}%
\definecolor{textcolor}{rgb}{0.150000,0.150000,0.150000}%
\pgfsetstrokecolor{textcolor}%
\pgfsetfillcolor{textcolor}%
\pgftext[x=4.490235in,y=1.597748in,left,base]{\color{textcolor}\sffamily\fontsize{11.000000}{13.200000}\selectfont Yes}%
\end{pgfscope}%
\begin{pgfscope}%
\pgfsetbuttcap%
\pgfsetmiterjoin%
\definecolor{currentfill}{rgb}{0.992157,0.705882,0.384314}%
\pgfsetfillcolor{currentfill}%
\pgfsetlinewidth{1.003750pt}%
\definecolor{currentstroke}{rgb}{1.000000,1.000000,1.000000}%
\pgfsetstrokecolor{currentstroke}%
\pgfsetdash{}{0pt}%
\pgfpathmoveto{\pgfqpoint{4.062457in}{1.434616in}}%
\pgfpathlineto{\pgfqpoint{4.368012in}{1.434616in}}%
\pgfpathlineto{\pgfqpoint{4.368012in}{1.541560in}}%
\pgfpathlineto{\pgfqpoint{4.062457in}{1.541560in}}%
\pgfpathlineto{\pgfqpoint{4.062457in}{1.434616in}}%
\pgfpathclose%
\pgfusepath{stroke,fill}%
\end{pgfscope}%
\begin{pgfscope}%
\definecolor{textcolor}{rgb}{0.150000,0.150000,0.150000}%
\pgfsetstrokecolor{textcolor}%
\pgfsetfillcolor{textcolor}%
\pgftext[x=4.490235in,y=1.434616in,left,base]{\color{textcolor}\sffamily\fontsize{11.000000}{13.200000}\selectfont No}%
\end{pgfscope}%
\end{pgfpicture}%
\makeatother%
\endgroup%
}
	\caption{Chart depicting the self-reported comfort level with CLIs in a Likert-style question.}
	\label{fig:comfortlevel}
\end{figure}

More than half of the participants responded that they were extremely or at
least uncomfortable with using the command line. The percentage of individuals
who identified as at least comfortable was 23.53\%. This large amount of
discomfort aligns with the higher amount of low experience participants.
Another potential contributing factor is the fact that most people were
university students who generally have only ever worked with GUIs in their
personal computer usage.

\FloatBarrier %TODO: SEE IF THIS IS NECESSARY

\subsection{CLI Usage}

Beyond comfort and interest, the amount existing experience specifically with
command line interfaces is another interesting factor. Participants were
questioned about their existing usage of command line interfaces. These
questions were presented in a multiple choice and single answer form. Two
questions were asked of the participants, and they were intentionally divided
in order to highlight the differences between general experience  (see:
\autoref{fig:often}) and individuals who also use CLIs in their personal time
(see: \autoref{fig:often2}). Furthermore, this intentional division was
intended to balance the fact that a lot of individuals participating in the
user study were computer science students and might have had some experience
that did not reflect their personal affinity towards CLIs.

The usage of command line interfaces amongst our participant groups with higher
than anticipated. Over half of respondents stated they use command line
applications or tools at least once a month, with 35.29\% stating that they
almost use them every day. This finding, strongly highlights the continuing
relevance of command line interfaces, especially in the software development
space. Fascinatingly, when asked about usage of command line applications or
tools in day to day personal computing the statistics are almost mirrored
between the previous question. With almost 38.24\% stating that they never use
command line applications for personal tasks. This indicates that CLIs are
perceived and used mostly as work related tools and their qualities in
personal computing are not highlighted. This point could be related to the fact
that since CLIs are no longer the default paradigm of interaction with the
computer they are only presented to individuals in environments where they are
considered the De Facto \textit{tool  for the job}, such as with
\textit{git}\cite{hultstrand2015git}.

\begin{figure}[htbp]
	\centering
	\scalebox{0.67}{%% Creator: Matplotlib, PGF backend
%%
%% To include the figure in your LaTeX document, write
%%   \input{<filename>.pgf}
%%
%% Make sure the required packages are loaded in your preamble
%%   \usepackage{pgf}
%%
%% Also ensure that all the required font packages are loaded; for instance,
%% the lmodern package is sometimes necessary when using math font.
%%   \usepackage{lmodern}
%%
%% Figures using additional raster images can only be included by \input if
%% they are in the same directory as the main LaTeX file. For loading figures
%% from other directories you can use the `import` package
%%   \usepackage{import}
%%
%% and then include the figures with
%%   \import{<path to file>}{<filename>.pgf}
%%
%% Matplotlib used the following preamble
%%   \usepackage{fontspec}
%%   \setmainfont{DejaVuSerif.ttf}[Path=\detokenize{/home/spam/miniconda3/envs/mpl/lib/python3.10/site-packages/matplotlib/mpl-data/fonts/ttf/}]
%%   \setsansfont{DejaVuSans.ttf}[Path=\detokenize{/home/spam/miniconda3/envs/mpl/lib/python3.10/site-packages/matplotlib/mpl-data/fonts/ttf/}]
%%   \setmonofont{DejaVuSansMono.ttf}[Path=\detokenize{/home/spam/miniconda3/envs/mpl/lib/python3.10/site-packages/matplotlib/mpl-data/fonts/ttf/}]
%%
\begingroup%
\makeatletter%
\begin{pgfpicture}%
\pgfpathrectangle{\pgfpointorigin}{\pgfqpoint{8.799314in}{3.116660in}}%
\pgfusepath{use as bounding box, clip}%
\begin{pgfscope}%
\pgfsetbuttcap%
\pgfsetmiterjoin%
\pgfsetlinewidth{0.000000pt}%
\definecolor{currentstroke}{rgb}{0.000000,0.000000,0.000000}%
\pgfsetstrokecolor{currentstroke}%
\pgfsetstrokeopacity{0.000000}%
\pgfsetdash{}{0pt}%
\pgfpathmoveto{\pgfqpoint{0.000000in}{0.000000in}}%
\pgfpathlineto{\pgfqpoint{8.799314in}{0.000000in}}%
\pgfpathlineto{\pgfqpoint{8.799314in}{3.116660in}}%
\pgfpathlineto{\pgfqpoint{0.000000in}{3.116660in}}%
\pgfpathlineto{\pgfqpoint{0.000000in}{0.000000in}}%
\pgfpathclose%
\pgfusepath{}%
\end{pgfscope}%
\begin{pgfscope}%
\pgfsetbuttcap%
\pgfsetmiterjoin%
\definecolor{currentfill}{rgb}{0.501961,0.694118,0.827451}%
\pgfsetfillcolor{currentfill}%
\pgfsetlinewidth{1.003750pt}%
\definecolor{currentstroke}{rgb}{1.000000,1.000000,1.000000}%
\pgfsetstrokecolor{currentstroke}%
\pgfsetdash{}{0pt}%
\pgfpathmoveto{\pgfqpoint{3.894914in}{1.314902in}}%
\pgfpathcurveto{\pgfqpoint{3.894914in}{1.442534in}}{\pgfqpoint{3.869773in}{1.568924in}}{\pgfqpoint{3.820931in}{1.686841in}}%
\pgfpathcurveto{\pgfqpoint{3.772088in}{1.804757in}}{\pgfqpoint{3.700494in}{1.911906in}}{\pgfqpoint{3.610245in}{2.002155in}}%
\pgfpathcurveto{\pgfqpoint{3.519995in}{2.092404in}}{\pgfqpoint{3.412847in}{2.163998in}}{\pgfqpoint{3.294931in}{2.212841in}}%
\pgfpathcurveto{\pgfqpoint{3.177014in}{2.261683in}}{\pgfqpoint{3.050624in}{2.286824in}}{\pgfqpoint{2.922992in}{2.286824in}}%
\pgfpathcurveto{\pgfqpoint{2.795361in}{2.286824in}}{\pgfqpoint{2.668970in}{2.261683in}}{\pgfqpoint{2.551054in}{2.212841in}}%
\pgfpathcurveto{\pgfqpoint{2.433138in}{2.163998in}}{\pgfqpoint{2.325989in}{2.092404in}}{\pgfqpoint{2.235740in}{2.002155in}}%
\pgfpathcurveto{\pgfqpoint{2.145491in}{1.911906in}}{\pgfqpoint{2.073896in}{1.804757in}}{\pgfqpoint{2.025054in}{1.686841in}}%
\pgfpathcurveto{\pgfqpoint{1.976211in}{1.568924in}}{\pgfqpoint{1.951070in}{1.442534in}}{\pgfqpoint{1.951070in}{1.314902in}}%
\pgfpathcurveto{\pgfqpoint{1.951070in}{1.187271in}}{\pgfqpoint{1.976211in}{1.060880in}}{\pgfqpoint{2.025054in}{0.942964in}}%
\pgfpathcurveto{\pgfqpoint{2.073896in}{0.825048in}}{\pgfqpoint{2.145491in}{0.717899in}}{\pgfqpoint{2.235740in}{0.627650in}}%
\pgfpathcurveto{\pgfqpoint{2.325989in}{0.537401in}}{\pgfqpoint{2.433138in}{0.465806in}}{\pgfqpoint{2.551054in}{0.416964in}}%
\pgfpathcurveto{\pgfqpoint{2.668970in}{0.368121in}}{\pgfqpoint{2.795361in}{0.342980in}}{\pgfqpoint{2.922992in}{0.342980in}}%
\pgfpathcurveto{\pgfqpoint{3.050624in}{0.342980in}}{\pgfqpoint{3.177014in}{0.368121in}}{\pgfqpoint{3.294931in}{0.416964in}}%
\pgfpathcurveto{\pgfqpoint{3.412847in}{0.465806in}}{\pgfqpoint{3.519995in}{0.537401in}}{\pgfqpoint{3.610245in}{0.627650in}}%
\pgfpathcurveto{\pgfqpoint{3.700494in}{0.717899in}}{\pgfqpoint{3.772088in}{0.825048in}}{\pgfqpoint{3.820931in}{0.942964in}}%
\pgfpathcurveto{\pgfqpoint{3.869773in}{1.060880in}}{\pgfqpoint{3.894914in}{1.187271in}}{\pgfqpoint{3.894914in}{1.314902in}}%
\pgfpathmoveto{\pgfqpoint{2.922992in}{1.314902in}}%
\pgfpathmoveto{\pgfqpoint{3.894914in}{1.314902in}}%
\pgfpathlineto{\pgfqpoint{3.894914in}{1.314902in}}%
\pgfpathclose%
\pgfusepath{stroke,fill}%
\end{pgfscope}%
\begin{pgfscope}%
\pgfsetbuttcap%
\pgfsetmiterjoin%
\definecolor{currentfill}{rgb}{0.992157,0.705882,0.384314}%
\pgfsetfillcolor{currentfill}%
\pgfsetlinewidth{1.003750pt}%
\definecolor{currentstroke}{rgb}{1.000000,1.000000,1.000000}%
\pgfsetstrokecolor{currentstroke}%
\pgfsetdash{}{0pt}%
\pgfpathmoveto{\pgfqpoint{2.922992in}{2.286824in}}%
\pgfpathcurveto{\pgfqpoint{2.922992in}{2.286824in}}{\pgfqpoint{2.922992in}{2.286824in}}{\pgfqpoint{2.922992in}{2.286824in}}%
\pgfpathlineto{\pgfqpoint{2.922992in}{1.314902in}}%
\pgfpathlineto{\pgfqpoint{2.922992in}{2.286824in}}%
\pgfpathlineto{\pgfqpoint{2.922992in}{2.286824in}}%
\pgfpathclose%
\pgfusepath{stroke,fill}%
\end{pgfscope}%
\begin{pgfscope}%
\definecolor{textcolor}{rgb}{0.150000,0.150000,0.150000}%
\pgfsetstrokecolor{textcolor}%
\pgfsetfillcolor{textcolor}%
\pgftext[x=2.922992in,y=0.731749in,,]{\color{textcolor}\sffamily\fontsize{12.000000}{14.400000}\selectfont 100.00\%}%
\end{pgfscope}%
\begin{pgfscope}%
\definecolor{textcolor}{rgb}{0.150000,0.150000,0.150000}%
\pgfsetstrokecolor{textcolor}%
\pgfsetfillcolor{textcolor}%
\pgftext[x=2.922992in,y=1.898055in,,]{\color{textcolor}\sffamily\fontsize{12.000000}{14.400000}\selectfont 0.00\%}%
\end{pgfscope}%
\begin{pgfscope}%
\definecolor{textcolor}{rgb}{0.000000,0.000000,0.000000}%
\pgfsetstrokecolor{textcolor}%
\pgfsetfillcolor{textcolor}%
\pgftext[x=2.922992in,y=2.613138in,,base]{\color{textcolor}\sffamily\fontsize{12.000000}{14.400000}\selectfont CLI-Tutor}%
\end{pgfscope}%
\begin{pgfscope}%
\pgfsetbuttcap%
\pgfsetmiterjoin%
\definecolor{currentfill}{rgb}{0.501961,0.694118,0.827451}%
\pgfsetfillcolor{currentfill}%
\pgfsetlinewidth{1.003750pt}%
\definecolor{currentstroke}{rgb}{1.000000,1.000000,1.000000}%
\pgfsetstrokecolor{currentstroke}%
\pgfsetdash{}{0pt}%
\pgfpathmoveto{\pgfqpoint{5.876322in}{2.286824in}}%
\pgfpathcurveto{\pgfqpoint{5.725984in}{2.286824in}}{\pgfqpoint{5.577676in}{2.251942in}}{\pgfqpoint{5.443099in}{2.184931in}}%
\pgfpathcurveto{\pgfqpoint{5.308522in}{2.117920in}}{\pgfqpoint{5.191311in}{2.020588in}}{\pgfqpoint{5.100712in}{1.900616in}}%
\pgfpathcurveto{\pgfqpoint{5.010113in}{1.780644in}}{\pgfqpoint{4.948574in}{1.641271in}}{\pgfqpoint{4.920949in}{1.493492in}}%
\pgfpathcurveto{\pgfqpoint{4.893325in}{1.345714in}}{\pgfqpoint{4.900361in}{1.193522in}}{\pgfqpoint{4.941503in}{1.048923in}}%
\pgfpathcurveto{\pgfqpoint{4.982645in}{0.904324in}}{\pgfqpoint{5.056781in}{0.771224in}}{\pgfqpoint{5.158063in}{0.660123in}}%
\pgfpathcurveto{\pgfqpoint{5.259345in}{0.549022in}}{\pgfqpoint{5.385037in}{0.462921in}}{\pgfqpoint{5.525223in}{0.408612in}}%
\pgfpathcurveto{\pgfqpoint{5.665409in}{0.354304in}}{\pgfqpoint{5.816303in}{0.333255in}}{\pgfqpoint{5.966000in}{0.347126in}}%
\pgfpathcurveto{\pgfqpoint{6.115696in}{0.360998in}}{\pgfqpoint{6.260153in}{0.409415in}}{\pgfqpoint{6.387973in}{0.488558in}}%
\pgfpathlineto{\pgfqpoint{5.876322in}{1.314902in}}%
\pgfpathlineto{\pgfqpoint{5.876322in}{2.286824in}}%
\pgfpathlineto{\pgfqpoint{5.876322in}{2.286824in}}%
\pgfpathclose%
\pgfusepath{stroke,fill}%
\end{pgfscope}%
\begin{pgfscope}%
\pgfsetbuttcap%
\pgfsetmiterjoin%
\definecolor{currentfill}{rgb}{0.992157,0.705882,0.384314}%
\pgfsetfillcolor{currentfill}%
\pgfsetlinewidth{1.003750pt}%
\definecolor{currentstroke}{rgb}{1.000000,1.000000,1.000000}%
\pgfsetstrokecolor{currentstroke}%
\pgfsetdash{}{0pt}%
\pgfpathmoveto{\pgfqpoint{6.387973in}{0.488558in}}%
\pgfpathcurveto{\pgfqpoint{6.567670in}{0.599821in}}{\pgfqpoint{6.706262in}{0.766721in}}{\pgfqpoint{6.782612in}{0.963803in}}%
\pgfpathcurveto{\pgfqpoint{6.858962in}{1.160886in}}{\pgfqpoint{6.868981in}{1.377595in}}{\pgfqpoint{6.811141in}{1.580881in}}%
\pgfpathcurveto{\pgfqpoint{6.753302in}{1.784167in}}{\pgfqpoint{6.630701in}{1.963143in}}{\pgfqpoint{6.462036in}{2.090512in}}%
\pgfpathcurveto{\pgfqpoint{6.293371in}{2.217882in}}{\pgfqpoint{6.087677in}{2.286824in}}{\pgfqpoint{5.876322in}{2.286824in}}%
\pgfpathlineto{\pgfqpoint{5.876322in}{1.314902in}}%
\pgfpathlineto{\pgfqpoint{6.387973in}{0.488558in}}%
\pgfpathlineto{\pgfqpoint{6.387973in}{0.488558in}}%
\pgfpathclose%
\pgfusepath{stroke,fill}%
\end{pgfscope}%
\begin{pgfscope}%
\definecolor{textcolor}{rgb}{0.150000,0.150000,0.150000}%
\pgfsetstrokecolor{textcolor}%
\pgfsetfillcolor{textcolor}%
\pgftext[x=5.315431in,y=1.155315in,,]{\color{textcolor}\sffamily\fontsize{12.000000}{14.400000}\selectfont 58.82\%}%
\end{pgfscope}%
\begin{pgfscope}%
\definecolor{textcolor}{rgb}{0.150000,0.150000,0.150000}%
\pgfsetstrokecolor{textcolor}%
\pgfsetfillcolor{textcolor}%
\pgftext[x=6.437214in,y=1.474490in,,]{\color{textcolor}\sffamily\fontsize{12.000000}{14.400000}\selectfont 41.18\%}%
\end{pgfscope}%
\begin{pgfscope}%
\definecolor{textcolor}{rgb}{0.000000,0.000000,0.000000}%
\pgfsetstrokecolor{textcolor}%
\pgfsetfillcolor{textcolor}%
\pgftext[x=5.876322in,y=2.613138in,,base]{\color{textcolor}\sffamily\fontsize{12.000000}{14.400000}\selectfont Non Interactive Tutor}%
\end{pgfscope}%
\begin{pgfscope}%
\definecolor{textcolor}{rgb}{0.150000,0.150000,0.150000}%
\pgfsetstrokecolor{textcolor}%
\pgfsetfillcolor{textcolor}%
\pgftext[x=4.399657in,y=3.016660in,,top]{\color{textcolor}\sffamily\fontsize{14.400000}{17.280000}\selectfont Do you feel more or less intimidated by the command line after this interactive tutor?}%
\end{pgfscope}%
\begin{pgfscope}%
\pgfsetbuttcap%
\pgfsetmiterjoin%
\definecolor{currentfill}{rgb}{1.000000,1.000000,1.000000}%
\pgfsetfillcolor{currentfill}%
\pgfsetfillopacity{0.800000}%
\pgfsetlinewidth{1.003750pt}%
\definecolor{currentstroke}{rgb}{0.800000,0.800000,0.800000}%
\pgfsetstrokecolor{currentstroke}%
\pgfsetstrokeopacity{0.800000}%
\pgfsetdash{}{0pt}%
\pgfpathmoveto{\pgfqpoint{4.031901in}{1.341726in}}%
\pgfpathlineto{\pgfqpoint{4.767413in}{1.341726in}}%
\pgfpathquadraticcurveto{\pgfqpoint{4.797969in}{1.341726in}}{\pgfqpoint{4.797969in}{1.372282in}}%
\pgfpathlineto{\pgfqpoint{4.797969in}{1.744378in}}%
\pgfpathquadraticcurveto{\pgfqpoint{4.797969in}{1.774934in}}{\pgfqpoint{4.767413in}{1.774934in}}%
\pgfpathlineto{\pgfqpoint{4.031901in}{1.774934in}}%
\pgfpathquadraticcurveto{\pgfqpoint{4.001346in}{1.774934in}}{\pgfqpoint{4.001346in}{1.744378in}}%
\pgfpathlineto{\pgfqpoint{4.001346in}{1.372282in}}%
\pgfpathquadraticcurveto{\pgfqpoint{4.001346in}{1.341726in}}{\pgfqpoint{4.031901in}{1.341726in}}%
\pgfpathlineto{\pgfqpoint{4.031901in}{1.341726in}}%
\pgfpathclose%
\pgfusepath{stroke,fill}%
\end{pgfscope}%
\begin{pgfscope}%
\pgfsetbuttcap%
\pgfsetmiterjoin%
\definecolor{currentfill}{rgb}{0.501961,0.694118,0.827451}%
\pgfsetfillcolor{currentfill}%
\pgfsetlinewidth{1.003750pt}%
\definecolor{currentstroke}{rgb}{1.000000,1.000000,1.000000}%
\pgfsetstrokecolor{currentstroke}%
\pgfsetdash{}{0pt}%
\pgfpathmoveto{\pgfqpoint{4.062457in}{1.597748in}}%
\pgfpathlineto{\pgfqpoint{4.368012in}{1.597748in}}%
\pgfpathlineto{\pgfqpoint{4.368012in}{1.704692in}}%
\pgfpathlineto{\pgfqpoint{4.062457in}{1.704692in}}%
\pgfpathlineto{\pgfqpoint{4.062457in}{1.597748in}}%
\pgfpathclose%
\pgfusepath{stroke,fill}%
\end{pgfscope}%
\begin{pgfscope}%
\definecolor{textcolor}{rgb}{0.150000,0.150000,0.150000}%
\pgfsetstrokecolor{textcolor}%
\pgfsetfillcolor{textcolor}%
\pgftext[x=4.490235in,y=1.597748in,left,base]{\color{textcolor}\sffamily\fontsize{11.000000}{13.200000}\selectfont Yes}%
\end{pgfscope}%
\begin{pgfscope}%
\pgfsetbuttcap%
\pgfsetmiterjoin%
\definecolor{currentfill}{rgb}{0.992157,0.705882,0.384314}%
\pgfsetfillcolor{currentfill}%
\pgfsetlinewidth{1.003750pt}%
\definecolor{currentstroke}{rgb}{1.000000,1.000000,1.000000}%
\pgfsetstrokecolor{currentstroke}%
\pgfsetdash{}{0pt}%
\pgfpathmoveto{\pgfqpoint{4.062457in}{1.434616in}}%
\pgfpathlineto{\pgfqpoint{4.368012in}{1.434616in}}%
\pgfpathlineto{\pgfqpoint{4.368012in}{1.541560in}}%
\pgfpathlineto{\pgfqpoint{4.062457in}{1.541560in}}%
\pgfpathlineto{\pgfqpoint{4.062457in}{1.434616in}}%
\pgfpathclose%
\pgfusepath{stroke,fill}%
\end{pgfscope}%
\begin{pgfscope}%
\definecolor{textcolor}{rgb}{0.150000,0.150000,0.150000}%
\pgfsetstrokecolor{textcolor}%
\pgfsetfillcolor{textcolor}%
\pgftext[x=4.490235in,y=1.434616in,left,base]{\color{textcolor}\sffamily\fontsize{11.000000}{13.200000}\selectfont No}%
\end{pgfscope}%
\end{pgfpicture}%
\makeatother%
\endgroup%
}
	\caption{Chart depicting the frequency of command line usage amongst participants.}
	\label{fig:often}
\end{figure}

\begin{figure}[htbp]
	\centering
	\scalebox{0.67}{%% Creator: Matplotlib, PGF backend
%%
%% To include the figure in your LaTeX document, write
%%   \input{<filename>.pgf}
%%
%% Make sure the required packages are loaded in your preamble
%%   \usepackage{pgf}
%%
%% Also ensure that all the required font packages are loaded; for instance,
%% the lmodern package is sometimes necessary when using math font.
%%   \usepackage{lmodern}
%%
%% Figures using additional raster images can only be included by \input if
%% they are in the same directory as the main LaTeX file. For loading figures
%% from other directories you can use the `import` package
%%   \usepackage{import}
%%
%% and then include the figures with
%%   \import{<path to file>}{<filename>.pgf}
%%
%% Matplotlib used the following preamble
%%   \usepackage{fontspec}
%%   \setmainfont{DejaVuSerif.ttf}[Path=\detokenize{/home/spam/miniconda3/envs/mpl/lib/python3.10/site-packages/matplotlib/mpl-data/fonts/ttf/}]
%%   \setsansfont{DejaVuSans.ttf}[Path=\detokenize{/home/spam/miniconda3/envs/mpl/lib/python3.10/site-packages/matplotlib/mpl-data/fonts/ttf/}]
%%   \setmonofont{DejaVuSansMono.ttf}[Path=\detokenize{/home/spam/miniconda3/envs/mpl/lib/python3.10/site-packages/matplotlib/mpl-data/fonts/ttf/}]
%%
\begingroup%
\makeatletter%
\begin{pgfpicture}%
\pgfpathrectangle{\pgfpointorigin}{\pgfqpoint{8.799314in}{3.116660in}}%
\pgfusepath{use as bounding box, clip}%
\begin{pgfscope}%
\pgfsetbuttcap%
\pgfsetmiterjoin%
\pgfsetlinewidth{0.000000pt}%
\definecolor{currentstroke}{rgb}{0.000000,0.000000,0.000000}%
\pgfsetstrokecolor{currentstroke}%
\pgfsetstrokeopacity{0.000000}%
\pgfsetdash{}{0pt}%
\pgfpathmoveto{\pgfqpoint{0.000000in}{0.000000in}}%
\pgfpathlineto{\pgfqpoint{8.799314in}{0.000000in}}%
\pgfpathlineto{\pgfqpoint{8.799314in}{3.116660in}}%
\pgfpathlineto{\pgfqpoint{0.000000in}{3.116660in}}%
\pgfpathlineto{\pgfqpoint{0.000000in}{0.000000in}}%
\pgfpathclose%
\pgfusepath{}%
\end{pgfscope}%
\begin{pgfscope}%
\pgfsetbuttcap%
\pgfsetmiterjoin%
\definecolor{currentfill}{rgb}{0.501961,0.694118,0.827451}%
\pgfsetfillcolor{currentfill}%
\pgfsetlinewidth{1.003750pt}%
\definecolor{currentstroke}{rgb}{1.000000,1.000000,1.000000}%
\pgfsetstrokecolor{currentstroke}%
\pgfsetdash{}{0pt}%
\pgfpathmoveto{\pgfqpoint{3.894914in}{1.314902in}}%
\pgfpathcurveto{\pgfqpoint{3.894914in}{1.442534in}}{\pgfqpoint{3.869773in}{1.568924in}}{\pgfqpoint{3.820931in}{1.686841in}}%
\pgfpathcurveto{\pgfqpoint{3.772088in}{1.804757in}}{\pgfqpoint{3.700494in}{1.911906in}}{\pgfqpoint{3.610245in}{2.002155in}}%
\pgfpathcurveto{\pgfqpoint{3.519995in}{2.092404in}}{\pgfqpoint{3.412847in}{2.163998in}}{\pgfqpoint{3.294931in}{2.212841in}}%
\pgfpathcurveto{\pgfqpoint{3.177014in}{2.261683in}}{\pgfqpoint{3.050624in}{2.286824in}}{\pgfqpoint{2.922992in}{2.286824in}}%
\pgfpathcurveto{\pgfqpoint{2.795361in}{2.286824in}}{\pgfqpoint{2.668970in}{2.261683in}}{\pgfqpoint{2.551054in}{2.212841in}}%
\pgfpathcurveto{\pgfqpoint{2.433138in}{2.163998in}}{\pgfqpoint{2.325989in}{2.092404in}}{\pgfqpoint{2.235740in}{2.002155in}}%
\pgfpathcurveto{\pgfqpoint{2.145491in}{1.911906in}}{\pgfqpoint{2.073896in}{1.804757in}}{\pgfqpoint{2.025054in}{1.686841in}}%
\pgfpathcurveto{\pgfqpoint{1.976211in}{1.568924in}}{\pgfqpoint{1.951070in}{1.442534in}}{\pgfqpoint{1.951070in}{1.314902in}}%
\pgfpathcurveto{\pgfqpoint{1.951070in}{1.187271in}}{\pgfqpoint{1.976211in}{1.060880in}}{\pgfqpoint{2.025054in}{0.942964in}}%
\pgfpathcurveto{\pgfqpoint{2.073896in}{0.825048in}}{\pgfqpoint{2.145491in}{0.717899in}}{\pgfqpoint{2.235740in}{0.627650in}}%
\pgfpathcurveto{\pgfqpoint{2.325989in}{0.537401in}}{\pgfqpoint{2.433138in}{0.465806in}}{\pgfqpoint{2.551054in}{0.416964in}}%
\pgfpathcurveto{\pgfqpoint{2.668970in}{0.368121in}}{\pgfqpoint{2.795361in}{0.342980in}}{\pgfqpoint{2.922992in}{0.342980in}}%
\pgfpathcurveto{\pgfqpoint{3.050624in}{0.342980in}}{\pgfqpoint{3.177014in}{0.368121in}}{\pgfqpoint{3.294931in}{0.416964in}}%
\pgfpathcurveto{\pgfqpoint{3.412847in}{0.465806in}}{\pgfqpoint{3.519995in}{0.537401in}}{\pgfqpoint{3.610245in}{0.627650in}}%
\pgfpathcurveto{\pgfqpoint{3.700494in}{0.717899in}}{\pgfqpoint{3.772088in}{0.825048in}}{\pgfqpoint{3.820931in}{0.942964in}}%
\pgfpathcurveto{\pgfqpoint{3.869773in}{1.060880in}}{\pgfqpoint{3.894914in}{1.187271in}}{\pgfqpoint{3.894914in}{1.314902in}}%
\pgfpathmoveto{\pgfqpoint{2.922992in}{1.314902in}}%
\pgfpathmoveto{\pgfqpoint{3.894914in}{1.314902in}}%
\pgfpathlineto{\pgfqpoint{3.894914in}{1.314902in}}%
\pgfpathclose%
\pgfusepath{stroke,fill}%
\end{pgfscope}%
\begin{pgfscope}%
\pgfsetbuttcap%
\pgfsetmiterjoin%
\definecolor{currentfill}{rgb}{0.992157,0.705882,0.384314}%
\pgfsetfillcolor{currentfill}%
\pgfsetlinewidth{1.003750pt}%
\definecolor{currentstroke}{rgb}{1.000000,1.000000,1.000000}%
\pgfsetstrokecolor{currentstroke}%
\pgfsetdash{}{0pt}%
\pgfpathmoveto{\pgfqpoint{2.922992in}{2.286824in}}%
\pgfpathcurveto{\pgfqpoint{2.922992in}{2.286824in}}{\pgfqpoint{2.922992in}{2.286824in}}{\pgfqpoint{2.922992in}{2.286824in}}%
\pgfpathlineto{\pgfqpoint{2.922992in}{1.314902in}}%
\pgfpathlineto{\pgfqpoint{2.922992in}{2.286824in}}%
\pgfpathlineto{\pgfqpoint{2.922992in}{2.286824in}}%
\pgfpathclose%
\pgfusepath{stroke,fill}%
\end{pgfscope}%
\begin{pgfscope}%
\definecolor{textcolor}{rgb}{0.150000,0.150000,0.150000}%
\pgfsetstrokecolor{textcolor}%
\pgfsetfillcolor{textcolor}%
\pgftext[x=2.922992in,y=0.731749in,,]{\color{textcolor}\sffamily\fontsize{12.000000}{14.400000}\selectfont 100.00\%}%
\end{pgfscope}%
\begin{pgfscope}%
\definecolor{textcolor}{rgb}{0.150000,0.150000,0.150000}%
\pgfsetstrokecolor{textcolor}%
\pgfsetfillcolor{textcolor}%
\pgftext[x=2.922992in,y=1.898055in,,]{\color{textcolor}\sffamily\fontsize{12.000000}{14.400000}\selectfont 0.00\%}%
\end{pgfscope}%
\begin{pgfscope}%
\definecolor{textcolor}{rgb}{0.000000,0.000000,0.000000}%
\pgfsetstrokecolor{textcolor}%
\pgfsetfillcolor{textcolor}%
\pgftext[x=2.922992in,y=2.613138in,,base]{\color{textcolor}\sffamily\fontsize{12.000000}{14.400000}\selectfont CLI-Tutor}%
\end{pgfscope}%
\begin{pgfscope}%
\pgfsetbuttcap%
\pgfsetmiterjoin%
\definecolor{currentfill}{rgb}{0.501961,0.694118,0.827451}%
\pgfsetfillcolor{currentfill}%
\pgfsetlinewidth{1.003750pt}%
\definecolor{currentstroke}{rgb}{1.000000,1.000000,1.000000}%
\pgfsetstrokecolor{currentstroke}%
\pgfsetdash{}{0pt}%
\pgfpathmoveto{\pgfqpoint{5.876322in}{2.286824in}}%
\pgfpathcurveto{\pgfqpoint{5.725984in}{2.286824in}}{\pgfqpoint{5.577676in}{2.251942in}}{\pgfqpoint{5.443099in}{2.184931in}}%
\pgfpathcurveto{\pgfqpoint{5.308522in}{2.117920in}}{\pgfqpoint{5.191311in}{2.020588in}}{\pgfqpoint{5.100712in}{1.900616in}}%
\pgfpathcurveto{\pgfqpoint{5.010113in}{1.780644in}}{\pgfqpoint{4.948574in}{1.641271in}}{\pgfqpoint{4.920949in}{1.493492in}}%
\pgfpathcurveto{\pgfqpoint{4.893325in}{1.345714in}}{\pgfqpoint{4.900361in}{1.193522in}}{\pgfqpoint{4.941503in}{1.048923in}}%
\pgfpathcurveto{\pgfqpoint{4.982645in}{0.904324in}}{\pgfqpoint{5.056781in}{0.771224in}}{\pgfqpoint{5.158063in}{0.660123in}}%
\pgfpathcurveto{\pgfqpoint{5.259345in}{0.549022in}}{\pgfqpoint{5.385037in}{0.462921in}}{\pgfqpoint{5.525223in}{0.408612in}}%
\pgfpathcurveto{\pgfqpoint{5.665409in}{0.354304in}}{\pgfqpoint{5.816303in}{0.333255in}}{\pgfqpoint{5.966000in}{0.347126in}}%
\pgfpathcurveto{\pgfqpoint{6.115696in}{0.360998in}}{\pgfqpoint{6.260153in}{0.409415in}}{\pgfqpoint{6.387973in}{0.488558in}}%
\pgfpathlineto{\pgfqpoint{5.876322in}{1.314902in}}%
\pgfpathlineto{\pgfqpoint{5.876322in}{2.286824in}}%
\pgfpathlineto{\pgfqpoint{5.876322in}{2.286824in}}%
\pgfpathclose%
\pgfusepath{stroke,fill}%
\end{pgfscope}%
\begin{pgfscope}%
\pgfsetbuttcap%
\pgfsetmiterjoin%
\definecolor{currentfill}{rgb}{0.992157,0.705882,0.384314}%
\pgfsetfillcolor{currentfill}%
\pgfsetlinewidth{1.003750pt}%
\definecolor{currentstroke}{rgb}{1.000000,1.000000,1.000000}%
\pgfsetstrokecolor{currentstroke}%
\pgfsetdash{}{0pt}%
\pgfpathmoveto{\pgfqpoint{6.387973in}{0.488558in}}%
\pgfpathcurveto{\pgfqpoint{6.567670in}{0.599821in}}{\pgfqpoint{6.706262in}{0.766721in}}{\pgfqpoint{6.782612in}{0.963803in}}%
\pgfpathcurveto{\pgfqpoint{6.858962in}{1.160886in}}{\pgfqpoint{6.868981in}{1.377595in}}{\pgfqpoint{6.811141in}{1.580881in}}%
\pgfpathcurveto{\pgfqpoint{6.753302in}{1.784167in}}{\pgfqpoint{6.630701in}{1.963143in}}{\pgfqpoint{6.462036in}{2.090512in}}%
\pgfpathcurveto{\pgfqpoint{6.293371in}{2.217882in}}{\pgfqpoint{6.087677in}{2.286824in}}{\pgfqpoint{5.876322in}{2.286824in}}%
\pgfpathlineto{\pgfqpoint{5.876322in}{1.314902in}}%
\pgfpathlineto{\pgfqpoint{6.387973in}{0.488558in}}%
\pgfpathlineto{\pgfqpoint{6.387973in}{0.488558in}}%
\pgfpathclose%
\pgfusepath{stroke,fill}%
\end{pgfscope}%
\begin{pgfscope}%
\definecolor{textcolor}{rgb}{0.150000,0.150000,0.150000}%
\pgfsetstrokecolor{textcolor}%
\pgfsetfillcolor{textcolor}%
\pgftext[x=5.315431in,y=1.155315in,,]{\color{textcolor}\sffamily\fontsize{12.000000}{14.400000}\selectfont 58.82\%}%
\end{pgfscope}%
\begin{pgfscope}%
\definecolor{textcolor}{rgb}{0.150000,0.150000,0.150000}%
\pgfsetstrokecolor{textcolor}%
\pgfsetfillcolor{textcolor}%
\pgftext[x=6.437214in,y=1.474490in,,]{\color{textcolor}\sffamily\fontsize{12.000000}{14.400000}\selectfont 41.18\%}%
\end{pgfscope}%
\begin{pgfscope}%
\definecolor{textcolor}{rgb}{0.000000,0.000000,0.000000}%
\pgfsetstrokecolor{textcolor}%
\pgfsetfillcolor{textcolor}%
\pgftext[x=5.876322in,y=2.613138in,,base]{\color{textcolor}\sffamily\fontsize{12.000000}{14.400000}\selectfont Non Interactive Tutor}%
\end{pgfscope}%
\begin{pgfscope}%
\definecolor{textcolor}{rgb}{0.150000,0.150000,0.150000}%
\pgfsetstrokecolor{textcolor}%
\pgfsetfillcolor{textcolor}%
\pgftext[x=4.399657in,y=3.016660in,,top]{\color{textcolor}\sffamily\fontsize{14.400000}{17.280000}\selectfont Do you feel more or less intimidated by the command line after this interactive tutor?}%
\end{pgfscope}%
\begin{pgfscope}%
\pgfsetbuttcap%
\pgfsetmiterjoin%
\definecolor{currentfill}{rgb}{1.000000,1.000000,1.000000}%
\pgfsetfillcolor{currentfill}%
\pgfsetfillopacity{0.800000}%
\pgfsetlinewidth{1.003750pt}%
\definecolor{currentstroke}{rgb}{0.800000,0.800000,0.800000}%
\pgfsetstrokecolor{currentstroke}%
\pgfsetstrokeopacity{0.800000}%
\pgfsetdash{}{0pt}%
\pgfpathmoveto{\pgfqpoint{4.031901in}{1.341726in}}%
\pgfpathlineto{\pgfqpoint{4.767413in}{1.341726in}}%
\pgfpathquadraticcurveto{\pgfqpoint{4.797969in}{1.341726in}}{\pgfqpoint{4.797969in}{1.372282in}}%
\pgfpathlineto{\pgfqpoint{4.797969in}{1.744378in}}%
\pgfpathquadraticcurveto{\pgfqpoint{4.797969in}{1.774934in}}{\pgfqpoint{4.767413in}{1.774934in}}%
\pgfpathlineto{\pgfqpoint{4.031901in}{1.774934in}}%
\pgfpathquadraticcurveto{\pgfqpoint{4.001346in}{1.774934in}}{\pgfqpoint{4.001346in}{1.744378in}}%
\pgfpathlineto{\pgfqpoint{4.001346in}{1.372282in}}%
\pgfpathquadraticcurveto{\pgfqpoint{4.001346in}{1.341726in}}{\pgfqpoint{4.031901in}{1.341726in}}%
\pgfpathlineto{\pgfqpoint{4.031901in}{1.341726in}}%
\pgfpathclose%
\pgfusepath{stroke,fill}%
\end{pgfscope}%
\begin{pgfscope}%
\pgfsetbuttcap%
\pgfsetmiterjoin%
\definecolor{currentfill}{rgb}{0.501961,0.694118,0.827451}%
\pgfsetfillcolor{currentfill}%
\pgfsetlinewidth{1.003750pt}%
\definecolor{currentstroke}{rgb}{1.000000,1.000000,1.000000}%
\pgfsetstrokecolor{currentstroke}%
\pgfsetdash{}{0pt}%
\pgfpathmoveto{\pgfqpoint{4.062457in}{1.597748in}}%
\pgfpathlineto{\pgfqpoint{4.368012in}{1.597748in}}%
\pgfpathlineto{\pgfqpoint{4.368012in}{1.704692in}}%
\pgfpathlineto{\pgfqpoint{4.062457in}{1.704692in}}%
\pgfpathlineto{\pgfqpoint{4.062457in}{1.597748in}}%
\pgfpathclose%
\pgfusepath{stroke,fill}%
\end{pgfscope}%
\begin{pgfscope}%
\definecolor{textcolor}{rgb}{0.150000,0.150000,0.150000}%
\pgfsetstrokecolor{textcolor}%
\pgfsetfillcolor{textcolor}%
\pgftext[x=4.490235in,y=1.597748in,left,base]{\color{textcolor}\sffamily\fontsize{11.000000}{13.200000}\selectfont Yes}%
\end{pgfscope}%
\begin{pgfscope}%
\pgfsetbuttcap%
\pgfsetmiterjoin%
\definecolor{currentfill}{rgb}{0.992157,0.705882,0.384314}%
\pgfsetfillcolor{currentfill}%
\pgfsetlinewidth{1.003750pt}%
\definecolor{currentstroke}{rgb}{1.000000,1.000000,1.000000}%
\pgfsetstrokecolor{currentstroke}%
\pgfsetdash{}{0pt}%
\pgfpathmoveto{\pgfqpoint{4.062457in}{1.434616in}}%
\pgfpathlineto{\pgfqpoint{4.368012in}{1.434616in}}%
\pgfpathlineto{\pgfqpoint{4.368012in}{1.541560in}}%
\pgfpathlineto{\pgfqpoint{4.062457in}{1.541560in}}%
\pgfpathlineto{\pgfqpoint{4.062457in}{1.434616in}}%
\pgfpathclose%
\pgfusepath{stroke,fill}%
\end{pgfscope}%
\begin{pgfscope}%
\definecolor{textcolor}{rgb}{0.150000,0.150000,0.150000}%
\pgfsetstrokecolor{textcolor}%
\pgfsetfillcolor{textcolor}%
\pgftext[x=4.490235in,y=1.434616in,left,base]{\color{textcolor}\sffamily\fontsize{11.000000}{13.200000}\selectfont No}%
\end{pgfscope}%
\end{pgfpicture}%
\makeatother%
\endgroup%
}
	\caption{Chart depicting the frequency of command line usage amongst participants for personal tasks.}
	\label{fig:often2}
\end{figure}


\FloatBarrier %TODO: SEE IF THIS IS NECESSARY

\subsection{Learning Preferences}

The preferences regarding learning mediums is another important factor relevant
to this study. As also mentioned in our research questions (see:
\autoref{subsec:rqs}), analysing the effectiveness of interactive learning
tools is part of the goals of this work.

To get an idea of the existing preferences of our study participants, they were
asked to select what mediums they preferred when learning technical topics.
Encouragingly, over 50\% of participants indicated their preferred method of
learning was at least partially interactive, with most indicating that their
preferred method was video tutorials. This is most likely due to the vast
amount of video content, especially technical content, available on the web.
This large proportion of individuals preferring video and interactive tutorials
is a good indicator for the potential utility of interactive learning tools
such as \textit{CLI-Tutor}. 26.37\% of participants stated their preferred
method of learning was via books or online documentation.

\begin{figure}[htbp]
	\centering
	\scalebox{0.67}{%% Creator: Matplotlib, PGF backend
%%
%% To include the figure in your LaTeX document, write
%%   \input{<filename>.pgf}
%%
%% Make sure the required packages are loaded in your preamble
%%   \usepackage{pgf}
%%
%% Also ensure that all the required font packages are loaded; for instance,
%% the lmodern package is sometimes necessary when using math font.
%%   \usepackage{lmodern}
%%
%% Figures using additional raster images can only be included by \input if
%% they are in the same directory as the main LaTeX file. For loading figures
%% from other directories you can use the `import` package
%%   \usepackage{import}
%%
%% and then include the figures with
%%   \import{<path to file>}{<filename>.pgf}
%%
%% Matplotlib used the following preamble
%%   \usepackage{fontspec}
%%   \setmainfont{DejaVuSerif.ttf}[Path=\detokenize{/home/spam/miniconda3/envs/mpl/lib/python3.10/site-packages/matplotlib/mpl-data/fonts/ttf/}]
%%   \setsansfont{DejaVuSans.ttf}[Path=\detokenize{/home/spam/miniconda3/envs/mpl/lib/python3.10/site-packages/matplotlib/mpl-data/fonts/ttf/}]
%%   \setmonofont{DejaVuSansMono.ttf}[Path=\detokenize{/home/spam/miniconda3/envs/mpl/lib/python3.10/site-packages/matplotlib/mpl-data/fonts/ttf/}]
%%
\begingroup%
\makeatletter%
\begin{pgfpicture}%
\pgfpathrectangle{\pgfpointorigin}{\pgfqpoint{8.799314in}{3.116660in}}%
\pgfusepath{use as bounding box, clip}%
\begin{pgfscope}%
\pgfsetbuttcap%
\pgfsetmiterjoin%
\pgfsetlinewidth{0.000000pt}%
\definecolor{currentstroke}{rgb}{0.000000,0.000000,0.000000}%
\pgfsetstrokecolor{currentstroke}%
\pgfsetstrokeopacity{0.000000}%
\pgfsetdash{}{0pt}%
\pgfpathmoveto{\pgfqpoint{0.000000in}{0.000000in}}%
\pgfpathlineto{\pgfqpoint{8.799314in}{0.000000in}}%
\pgfpathlineto{\pgfqpoint{8.799314in}{3.116660in}}%
\pgfpathlineto{\pgfqpoint{0.000000in}{3.116660in}}%
\pgfpathlineto{\pgfqpoint{0.000000in}{0.000000in}}%
\pgfpathclose%
\pgfusepath{}%
\end{pgfscope}%
\begin{pgfscope}%
\pgfsetbuttcap%
\pgfsetmiterjoin%
\definecolor{currentfill}{rgb}{0.501961,0.694118,0.827451}%
\pgfsetfillcolor{currentfill}%
\pgfsetlinewidth{1.003750pt}%
\definecolor{currentstroke}{rgb}{1.000000,1.000000,1.000000}%
\pgfsetstrokecolor{currentstroke}%
\pgfsetdash{}{0pt}%
\pgfpathmoveto{\pgfqpoint{3.894914in}{1.314902in}}%
\pgfpathcurveto{\pgfqpoint{3.894914in}{1.442534in}}{\pgfqpoint{3.869773in}{1.568924in}}{\pgfqpoint{3.820931in}{1.686841in}}%
\pgfpathcurveto{\pgfqpoint{3.772088in}{1.804757in}}{\pgfqpoint{3.700494in}{1.911906in}}{\pgfqpoint{3.610245in}{2.002155in}}%
\pgfpathcurveto{\pgfqpoint{3.519995in}{2.092404in}}{\pgfqpoint{3.412847in}{2.163998in}}{\pgfqpoint{3.294931in}{2.212841in}}%
\pgfpathcurveto{\pgfqpoint{3.177014in}{2.261683in}}{\pgfqpoint{3.050624in}{2.286824in}}{\pgfqpoint{2.922992in}{2.286824in}}%
\pgfpathcurveto{\pgfqpoint{2.795361in}{2.286824in}}{\pgfqpoint{2.668970in}{2.261683in}}{\pgfqpoint{2.551054in}{2.212841in}}%
\pgfpathcurveto{\pgfqpoint{2.433138in}{2.163998in}}{\pgfqpoint{2.325989in}{2.092404in}}{\pgfqpoint{2.235740in}{2.002155in}}%
\pgfpathcurveto{\pgfqpoint{2.145491in}{1.911906in}}{\pgfqpoint{2.073896in}{1.804757in}}{\pgfqpoint{2.025054in}{1.686841in}}%
\pgfpathcurveto{\pgfqpoint{1.976211in}{1.568924in}}{\pgfqpoint{1.951070in}{1.442534in}}{\pgfqpoint{1.951070in}{1.314902in}}%
\pgfpathcurveto{\pgfqpoint{1.951070in}{1.187271in}}{\pgfqpoint{1.976211in}{1.060880in}}{\pgfqpoint{2.025054in}{0.942964in}}%
\pgfpathcurveto{\pgfqpoint{2.073896in}{0.825048in}}{\pgfqpoint{2.145491in}{0.717899in}}{\pgfqpoint{2.235740in}{0.627650in}}%
\pgfpathcurveto{\pgfqpoint{2.325989in}{0.537401in}}{\pgfqpoint{2.433138in}{0.465806in}}{\pgfqpoint{2.551054in}{0.416964in}}%
\pgfpathcurveto{\pgfqpoint{2.668970in}{0.368121in}}{\pgfqpoint{2.795361in}{0.342980in}}{\pgfqpoint{2.922992in}{0.342980in}}%
\pgfpathcurveto{\pgfqpoint{3.050624in}{0.342980in}}{\pgfqpoint{3.177014in}{0.368121in}}{\pgfqpoint{3.294931in}{0.416964in}}%
\pgfpathcurveto{\pgfqpoint{3.412847in}{0.465806in}}{\pgfqpoint{3.519995in}{0.537401in}}{\pgfqpoint{3.610245in}{0.627650in}}%
\pgfpathcurveto{\pgfqpoint{3.700494in}{0.717899in}}{\pgfqpoint{3.772088in}{0.825048in}}{\pgfqpoint{3.820931in}{0.942964in}}%
\pgfpathcurveto{\pgfqpoint{3.869773in}{1.060880in}}{\pgfqpoint{3.894914in}{1.187271in}}{\pgfqpoint{3.894914in}{1.314902in}}%
\pgfpathmoveto{\pgfqpoint{2.922992in}{1.314902in}}%
\pgfpathmoveto{\pgfqpoint{3.894914in}{1.314902in}}%
\pgfpathlineto{\pgfqpoint{3.894914in}{1.314902in}}%
\pgfpathclose%
\pgfusepath{stroke,fill}%
\end{pgfscope}%
\begin{pgfscope}%
\pgfsetbuttcap%
\pgfsetmiterjoin%
\definecolor{currentfill}{rgb}{0.992157,0.705882,0.384314}%
\pgfsetfillcolor{currentfill}%
\pgfsetlinewidth{1.003750pt}%
\definecolor{currentstroke}{rgb}{1.000000,1.000000,1.000000}%
\pgfsetstrokecolor{currentstroke}%
\pgfsetdash{}{0pt}%
\pgfpathmoveto{\pgfqpoint{2.922992in}{2.286824in}}%
\pgfpathcurveto{\pgfqpoint{2.922992in}{2.286824in}}{\pgfqpoint{2.922992in}{2.286824in}}{\pgfqpoint{2.922992in}{2.286824in}}%
\pgfpathlineto{\pgfqpoint{2.922992in}{1.314902in}}%
\pgfpathlineto{\pgfqpoint{2.922992in}{2.286824in}}%
\pgfpathlineto{\pgfqpoint{2.922992in}{2.286824in}}%
\pgfpathclose%
\pgfusepath{stroke,fill}%
\end{pgfscope}%
\begin{pgfscope}%
\definecolor{textcolor}{rgb}{0.150000,0.150000,0.150000}%
\pgfsetstrokecolor{textcolor}%
\pgfsetfillcolor{textcolor}%
\pgftext[x=2.922992in,y=0.731749in,,]{\color{textcolor}\sffamily\fontsize{12.000000}{14.400000}\selectfont 100.00\%}%
\end{pgfscope}%
\begin{pgfscope}%
\definecolor{textcolor}{rgb}{0.150000,0.150000,0.150000}%
\pgfsetstrokecolor{textcolor}%
\pgfsetfillcolor{textcolor}%
\pgftext[x=2.922992in,y=1.898055in,,]{\color{textcolor}\sffamily\fontsize{12.000000}{14.400000}\selectfont 0.00\%}%
\end{pgfscope}%
\begin{pgfscope}%
\definecolor{textcolor}{rgb}{0.000000,0.000000,0.000000}%
\pgfsetstrokecolor{textcolor}%
\pgfsetfillcolor{textcolor}%
\pgftext[x=2.922992in,y=2.613138in,,base]{\color{textcolor}\sffamily\fontsize{12.000000}{14.400000}\selectfont CLI-Tutor}%
\end{pgfscope}%
\begin{pgfscope}%
\pgfsetbuttcap%
\pgfsetmiterjoin%
\definecolor{currentfill}{rgb}{0.501961,0.694118,0.827451}%
\pgfsetfillcolor{currentfill}%
\pgfsetlinewidth{1.003750pt}%
\definecolor{currentstroke}{rgb}{1.000000,1.000000,1.000000}%
\pgfsetstrokecolor{currentstroke}%
\pgfsetdash{}{0pt}%
\pgfpathmoveto{\pgfqpoint{5.876322in}{2.286824in}}%
\pgfpathcurveto{\pgfqpoint{5.725984in}{2.286824in}}{\pgfqpoint{5.577676in}{2.251942in}}{\pgfqpoint{5.443099in}{2.184931in}}%
\pgfpathcurveto{\pgfqpoint{5.308522in}{2.117920in}}{\pgfqpoint{5.191311in}{2.020588in}}{\pgfqpoint{5.100712in}{1.900616in}}%
\pgfpathcurveto{\pgfqpoint{5.010113in}{1.780644in}}{\pgfqpoint{4.948574in}{1.641271in}}{\pgfqpoint{4.920949in}{1.493492in}}%
\pgfpathcurveto{\pgfqpoint{4.893325in}{1.345714in}}{\pgfqpoint{4.900361in}{1.193522in}}{\pgfqpoint{4.941503in}{1.048923in}}%
\pgfpathcurveto{\pgfqpoint{4.982645in}{0.904324in}}{\pgfqpoint{5.056781in}{0.771224in}}{\pgfqpoint{5.158063in}{0.660123in}}%
\pgfpathcurveto{\pgfqpoint{5.259345in}{0.549022in}}{\pgfqpoint{5.385037in}{0.462921in}}{\pgfqpoint{5.525223in}{0.408612in}}%
\pgfpathcurveto{\pgfqpoint{5.665409in}{0.354304in}}{\pgfqpoint{5.816303in}{0.333255in}}{\pgfqpoint{5.966000in}{0.347126in}}%
\pgfpathcurveto{\pgfqpoint{6.115696in}{0.360998in}}{\pgfqpoint{6.260153in}{0.409415in}}{\pgfqpoint{6.387973in}{0.488558in}}%
\pgfpathlineto{\pgfqpoint{5.876322in}{1.314902in}}%
\pgfpathlineto{\pgfqpoint{5.876322in}{2.286824in}}%
\pgfpathlineto{\pgfqpoint{5.876322in}{2.286824in}}%
\pgfpathclose%
\pgfusepath{stroke,fill}%
\end{pgfscope}%
\begin{pgfscope}%
\pgfsetbuttcap%
\pgfsetmiterjoin%
\definecolor{currentfill}{rgb}{0.992157,0.705882,0.384314}%
\pgfsetfillcolor{currentfill}%
\pgfsetlinewidth{1.003750pt}%
\definecolor{currentstroke}{rgb}{1.000000,1.000000,1.000000}%
\pgfsetstrokecolor{currentstroke}%
\pgfsetdash{}{0pt}%
\pgfpathmoveto{\pgfqpoint{6.387973in}{0.488558in}}%
\pgfpathcurveto{\pgfqpoint{6.567670in}{0.599821in}}{\pgfqpoint{6.706262in}{0.766721in}}{\pgfqpoint{6.782612in}{0.963803in}}%
\pgfpathcurveto{\pgfqpoint{6.858962in}{1.160886in}}{\pgfqpoint{6.868981in}{1.377595in}}{\pgfqpoint{6.811141in}{1.580881in}}%
\pgfpathcurveto{\pgfqpoint{6.753302in}{1.784167in}}{\pgfqpoint{6.630701in}{1.963143in}}{\pgfqpoint{6.462036in}{2.090512in}}%
\pgfpathcurveto{\pgfqpoint{6.293371in}{2.217882in}}{\pgfqpoint{6.087677in}{2.286824in}}{\pgfqpoint{5.876322in}{2.286824in}}%
\pgfpathlineto{\pgfqpoint{5.876322in}{1.314902in}}%
\pgfpathlineto{\pgfqpoint{6.387973in}{0.488558in}}%
\pgfpathlineto{\pgfqpoint{6.387973in}{0.488558in}}%
\pgfpathclose%
\pgfusepath{stroke,fill}%
\end{pgfscope}%
\begin{pgfscope}%
\definecolor{textcolor}{rgb}{0.150000,0.150000,0.150000}%
\pgfsetstrokecolor{textcolor}%
\pgfsetfillcolor{textcolor}%
\pgftext[x=5.315431in,y=1.155315in,,]{\color{textcolor}\sffamily\fontsize{12.000000}{14.400000}\selectfont 58.82\%}%
\end{pgfscope}%
\begin{pgfscope}%
\definecolor{textcolor}{rgb}{0.150000,0.150000,0.150000}%
\pgfsetstrokecolor{textcolor}%
\pgfsetfillcolor{textcolor}%
\pgftext[x=6.437214in,y=1.474490in,,]{\color{textcolor}\sffamily\fontsize{12.000000}{14.400000}\selectfont 41.18\%}%
\end{pgfscope}%
\begin{pgfscope}%
\definecolor{textcolor}{rgb}{0.000000,0.000000,0.000000}%
\pgfsetstrokecolor{textcolor}%
\pgfsetfillcolor{textcolor}%
\pgftext[x=5.876322in,y=2.613138in,,base]{\color{textcolor}\sffamily\fontsize{12.000000}{14.400000}\selectfont Non Interactive Tutor}%
\end{pgfscope}%
\begin{pgfscope}%
\definecolor{textcolor}{rgb}{0.150000,0.150000,0.150000}%
\pgfsetstrokecolor{textcolor}%
\pgfsetfillcolor{textcolor}%
\pgftext[x=4.399657in,y=3.016660in,,top]{\color{textcolor}\sffamily\fontsize{14.400000}{17.280000}\selectfont Do you feel more or less intimidated by the command line after this interactive tutor?}%
\end{pgfscope}%
\begin{pgfscope}%
\pgfsetbuttcap%
\pgfsetmiterjoin%
\definecolor{currentfill}{rgb}{1.000000,1.000000,1.000000}%
\pgfsetfillcolor{currentfill}%
\pgfsetfillopacity{0.800000}%
\pgfsetlinewidth{1.003750pt}%
\definecolor{currentstroke}{rgb}{0.800000,0.800000,0.800000}%
\pgfsetstrokecolor{currentstroke}%
\pgfsetstrokeopacity{0.800000}%
\pgfsetdash{}{0pt}%
\pgfpathmoveto{\pgfqpoint{4.031901in}{1.341726in}}%
\pgfpathlineto{\pgfqpoint{4.767413in}{1.341726in}}%
\pgfpathquadraticcurveto{\pgfqpoint{4.797969in}{1.341726in}}{\pgfqpoint{4.797969in}{1.372282in}}%
\pgfpathlineto{\pgfqpoint{4.797969in}{1.744378in}}%
\pgfpathquadraticcurveto{\pgfqpoint{4.797969in}{1.774934in}}{\pgfqpoint{4.767413in}{1.774934in}}%
\pgfpathlineto{\pgfqpoint{4.031901in}{1.774934in}}%
\pgfpathquadraticcurveto{\pgfqpoint{4.001346in}{1.774934in}}{\pgfqpoint{4.001346in}{1.744378in}}%
\pgfpathlineto{\pgfqpoint{4.001346in}{1.372282in}}%
\pgfpathquadraticcurveto{\pgfqpoint{4.001346in}{1.341726in}}{\pgfqpoint{4.031901in}{1.341726in}}%
\pgfpathlineto{\pgfqpoint{4.031901in}{1.341726in}}%
\pgfpathclose%
\pgfusepath{stroke,fill}%
\end{pgfscope}%
\begin{pgfscope}%
\pgfsetbuttcap%
\pgfsetmiterjoin%
\definecolor{currentfill}{rgb}{0.501961,0.694118,0.827451}%
\pgfsetfillcolor{currentfill}%
\pgfsetlinewidth{1.003750pt}%
\definecolor{currentstroke}{rgb}{1.000000,1.000000,1.000000}%
\pgfsetstrokecolor{currentstroke}%
\pgfsetdash{}{0pt}%
\pgfpathmoveto{\pgfqpoint{4.062457in}{1.597748in}}%
\pgfpathlineto{\pgfqpoint{4.368012in}{1.597748in}}%
\pgfpathlineto{\pgfqpoint{4.368012in}{1.704692in}}%
\pgfpathlineto{\pgfqpoint{4.062457in}{1.704692in}}%
\pgfpathlineto{\pgfqpoint{4.062457in}{1.597748in}}%
\pgfpathclose%
\pgfusepath{stroke,fill}%
\end{pgfscope}%
\begin{pgfscope}%
\definecolor{textcolor}{rgb}{0.150000,0.150000,0.150000}%
\pgfsetstrokecolor{textcolor}%
\pgfsetfillcolor{textcolor}%
\pgftext[x=4.490235in,y=1.597748in,left,base]{\color{textcolor}\sffamily\fontsize{11.000000}{13.200000}\selectfont Yes}%
\end{pgfscope}%
\begin{pgfscope}%
\pgfsetbuttcap%
\pgfsetmiterjoin%
\definecolor{currentfill}{rgb}{0.992157,0.705882,0.384314}%
\pgfsetfillcolor{currentfill}%
\pgfsetlinewidth{1.003750pt}%
\definecolor{currentstroke}{rgb}{1.000000,1.000000,1.000000}%
\pgfsetstrokecolor{currentstroke}%
\pgfsetdash{}{0pt}%
\pgfpathmoveto{\pgfqpoint{4.062457in}{1.434616in}}%
\pgfpathlineto{\pgfqpoint{4.368012in}{1.434616in}}%
\pgfpathlineto{\pgfqpoint{4.368012in}{1.541560in}}%
\pgfpathlineto{\pgfqpoint{4.062457in}{1.541560in}}%
\pgfpathlineto{\pgfqpoint{4.062457in}{1.434616in}}%
\pgfpathclose%
\pgfusepath{stroke,fill}%
\end{pgfscope}%
\begin{pgfscope}%
\definecolor{textcolor}{rgb}{0.150000,0.150000,0.150000}%
\pgfsetstrokecolor{textcolor}%
\pgfsetfillcolor{textcolor}%
\pgftext[x=4.490235in,y=1.434616in,left,base]{\color{textcolor}\sffamily\fontsize{11.000000}{13.200000}\selectfont No}%
\end{pgfscope}%
\end{pgfpicture}%
\makeatother%
\endgroup%
}
	\caption{Chart depicting the preferences in learning mediums amongst participants.}
	\label{fig:question}
\end{figure}


From the above question, the verbatim responses for \textit{Other} were:


\begin{itemize}[label={}, leftmargin=-2pt]
	\item \begin{quotes}
		      "All of the above"
	      \end{quotes}
	\item \begin{quotes}
		      "Work (good mix between pressure to get it right and being paid for it)"
	      \end{quotes}
	\item \begin{quotes}
		      "Starting projects and learning from tutorials/documentation as I go"
	      \end{quotes}
	\item \begin{quotes}
		      "testing curiosities"
	      \end{quotes}
	\item \begin{quotes}
		      "DIY"
	      \end{quotes}
\end{itemize}

\begin{figure}[htbp]
	\scalebox{0.72}{%% Creator: Matplotlib, PGF backend
%%
%% To include the figure in your LaTeX document, write
%%   \input{<filename>.pgf}
%%
%% Make sure the required packages are loaded in your preamble
%%   \usepackage{pgf}
%%
%% Also ensure that all the required font packages are loaded; for instance,
%% the lmodern package is sometimes necessary when using math font.
%%   \usepackage{lmodern}
%%
%% Figures using additional raster images can only be included by \input if
%% they are in the same directory as the main LaTeX file. For loading figures
%% from other directories you can use the `import` package
%%   \usepackage{import}
%%
%% and then include the figures with
%%   \import{<path to file>}{<filename>.pgf}
%%
%% Matplotlib used the following preamble
%%   \usepackage{fontspec}
%%   \setmainfont{DejaVuSerif.ttf}[Path=\detokenize{/home/spam/miniconda3/envs/mpl/lib/python3.10/site-packages/matplotlib/mpl-data/fonts/ttf/}]
%%   \setsansfont{DejaVuSans.ttf}[Path=\detokenize{/home/spam/miniconda3/envs/mpl/lib/python3.10/site-packages/matplotlib/mpl-data/fonts/ttf/}]
%%   \setmonofont{DejaVuSansMono.ttf}[Path=\detokenize{/home/spam/miniconda3/envs/mpl/lib/python3.10/site-packages/matplotlib/mpl-data/fonts/ttf/}]
%%
\begingroup%
\makeatletter%
\begin{pgfpicture}%
\pgfpathrectangle{\pgfpointorigin}{\pgfqpoint{8.799314in}{3.116660in}}%
\pgfusepath{use as bounding box, clip}%
\begin{pgfscope}%
\pgfsetbuttcap%
\pgfsetmiterjoin%
\pgfsetlinewidth{0.000000pt}%
\definecolor{currentstroke}{rgb}{0.000000,0.000000,0.000000}%
\pgfsetstrokecolor{currentstroke}%
\pgfsetstrokeopacity{0.000000}%
\pgfsetdash{}{0pt}%
\pgfpathmoveto{\pgfqpoint{0.000000in}{0.000000in}}%
\pgfpathlineto{\pgfqpoint{8.799314in}{0.000000in}}%
\pgfpathlineto{\pgfqpoint{8.799314in}{3.116660in}}%
\pgfpathlineto{\pgfqpoint{0.000000in}{3.116660in}}%
\pgfpathlineto{\pgfqpoint{0.000000in}{0.000000in}}%
\pgfpathclose%
\pgfusepath{}%
\end{pgfscope}%
\begin{pgfscope}%
\pgfsetbuttcap%
\pgfsetmiterjoin%
\definecolor{currentfill}{rgb}{0.501961,0.694118,0.827451}%
\pgfsetfillcolor{currentfill}%
\pgfsetlinewidth{1.003750pt}%
\definecolor{currentstroke}{rgb}{1.000000,1.000000,1.000000}%
\pgfsetstrokecolor{currentstroke}%
\pgfsetdash{}{0pt}%
\pgfpathmoveto{\pgfqpoint{3.894914in}{1.314902in}}%
\pgfpathcurveto{\pgfqpoint{3.894914in}{1.442534in}}{\pgfqpoint{3.869773in}{1.568924in}}{\pgfqpoint{3.820931in}{1.686841in}}%
\pgfpathcurveto{\pgfqpoint{3.772088in}{1.804757in}}{\pgfqpoint{3.700494in}{1.911906in}}{\pgfqpoint{3.610245in}{2.002155in}}%
\pgfpathcurveto{\pgfqpoint{3.519995in}{2.092404in}}{\pgfqpoint{3.412847in}{2.163998in}}{\pgfqpoint{3.294931in}{2.212841in}}%
\pgfpathcurveto{\pgfqpoint{3.177014in}{2.261683in}}{\pgfqpoint{3.050624in}{2.286824in}}{\pgfqpoint{2.922992in}{2.286824in}}%
\pgfpathcurveto{\pgfqpoint{2.795361in}{2.286824in}}{\pgfqpoint{2.668970in}{2.261683in}}{\pgfqpoint{2.551054in}{2.212841in}}%
\pgfpathcurveto{\pgfqpoint{2.433138in}{2.163998in}}{\pgfqpoint{2.325989in}{2.092404in}}{\pgfqpoint{2.235740in}{2.002155in}}%
\pgfpathcurveto{\pgfqpoint{2.145491in}{1.911906in}}{\pgfqpoint{2.073896in}{1.804757in}}{\pgfqpoint{2.025054in}{1.686841in}}%
\pgfpathcurveto{\pgfqpoint{1.976211in}{1.568924in}}{\pgfqpoint{1.951070in}{1.442534in}}{\pgfqpoint{1.951070in}{1.314902in}}%
\pgfpathcurveto{\pgfqpoint{1.951070in}{1.187271in}}{\pgfqpoint{1.976211in}{1.060880in}}{\pgfqpoint{2.025054in}{0.942964in}}%
\pgfpathcurveto{\pgfqpoint{2.073896in}{0.825048in}}{\pgfqpoint{2.145491in}{0.717899in}}{\pgfqpoint{2.235740in}{0.627650in}}%
\pgfpathcurveto{\pgfqpoint{2.325989in}{0.537401in}}{\pgfqpoint{2.433138in}{0.465806in}}{\pgfqpoint{2.551054in}{0.416964in}}%
\pgfpathcurveto{\pgfqpoint{2.668970in}{0.368121in}}{\pgfqpoint{2.795361in}{0.342980in}}{\pgfqpoint{2.922992in}{0.342980in}}%
\pgfpathcurveto{\pgfqpoint{3.050624in}{0.342980in}}{\pgfqpoint{3.177014in}{0.368121in}}{\pgfqpoint{3.294931in}{0.416964in}}%
\pgfpathcurveto{\pgfqpoint{3.412847in}{0.465806in}}{\pgfqpoint{3.519995in}{0.537401in}}{\pgfqpoint{3.610245in}{0.627650in}}%
\pgfpathcurveto{\pgfqpoint{3.700494in}{0.717899in}}{\pgfqpoint{3.772088in}{0.825048in}}{\pgfqpoint{3.820931in}{0.942964in}}%
\pgfpathcurveto{\pgfqpoint{3.869773in}{1.060880in}}{\pgfqpoint{3.894914in}{1.187271in}}{\pgfqpoint{3.894914in}{1.314902in}}%
\pgfpathmoveto{\pgfqpoint{2.922992in}{1.314902in}}%
\pgfpathmoveto{\pgfqpoint{3.894914in}{1.314902in}}%
\pgfpathlineto{\pgfqpoint{3.894914in}{1.314902in}}%
\pgfpathclose%
\pgfusepath{stroke,fill}%
\end{pgfscope}%
\begin{pgfscope}%
\pgfsetbuttcap%
\pgfsetmiterjoin%
\definecolor{currentfill}{rgb}{0.992157,0.705882,0.384314}%
\pgfsetfillcolor{currentfill}%
\pgfsetlinewidth{1.003750pt}%
\definecolor{currentstroke}{rgb}{1.000000,1.000000,1.000000}%
\pgfsetstrokecolor{currentstroke}%
\pgfsetdash{}{0pt}%
\pgfpathmoveto{\pgfqpoint{2.922992in}{2.286824in}}%
\pgfpathcurveto{\pgfqpoint{2.922992in}{2.286824in}}{\pgfqpoint{2.922992in}{2.286824in}}{\pgfqpoint{2.922992in}{2.286824in}}%
\pgfpathlineto{\pgfqpoint{2.922992in}{1.314902in}}%
\pgfpathlineto{\pgfqpoint{2.922992in}{2.286824in}}%
\pgfpathlineto{\pgfqpoint{2.922992in}{2.286824in}}%
\pgfpathclose%
\pgfusepath{stroke,fill}%
\end{pgfscope}%
\begin{pgfscope}%
\definecolor{textcolor}{rgb}{0.150000,0.150000,0.150000}%
\pgfsetstrokecolor{textcolor}%
\pgfsetfillcolor{textcolor}%
\pgftext[x=2.922992in,y=0.731749in,,]{\color{textcolor}\sffamily\fontsize{12.000000}{14.400000}\selectfont 100.00\%}%
\end{pgfscope}%
\begin{pgfscope}%
\definecolor{textcolor}{rgb}{0.150000,0.150000,0.150000}%
\pgfsetstrokecolor{textcolor}%
\pgfsetfillcolor{textcolor}%
\pgftext[x=2.922992in,y=1.898055in,,]{\color{textcolor}\sffamily\fontsize{12.000000}{14.400000}\selectfont 0.00\%}%
\end{pgfscope}%
\begin{pgfscope}%
\definecolor{textcolor}{rgb}{0.000000,0.000000,0.000000}%
\pgfsetstrokecolor{textcolor}%
\pgfsetfillcolor{textcolor}%
\pgftext[x=2.922992in,y=2.613138in,,base]{\color{textcolor}\sffamily\fontsize{12.000000}{14.400000}\selectfont CLI-Tutor}%
\end{pgfscope}%
\begin{pgfscope}%
\pgfsetbuttcap%
\pgfsetmiterjoin%
\definecolor{currentfill}{rgb}{0.501961,0.694118,0.827451}%
\pgfsetfillcolor{currentfill}%
\pgfsetlinewidth{1.003750pt}%
\definecolor{currentstroke}{rgb}{1.000000,1.000000,1.000000}%
\pgfsetstrokecolor{currentstroke}%
\pgfsetdash{}{0pt}%
\pgfpathmoveto{\pgfqpoint{5.876322in}{2.286824in}}%
\pgfpathcurveto{\pgfqpoint{5.725984in}{2.286824in}}{\pgfqpoint{5.577676in}{2.251942in}}{\pgfqpoint{5.443099in}{2.184931in}}%
\pgfpathcurveto{\pgfqpoint{5.308522in}{2.117920in}}{\pgfqpoint{5.191311in}{2.020588in}}{\pgfqpoint{5.100712in}{1.900616in}}%
\pgfpathcurveto{\pgfqpoint{5.010113in}{1.780644in}}{\pgfqpoint{4.948574in}{1.641271in}}{\pgfqpoint{4.920949in}{1.493492in}}%
\pgfpathcurveto{\pgfqpoint{4.893325in}{1.345714in}}{\pgfqpoint{4.900361in}{1.193522in}}{\pgfqpoint{4.941503in}{1.048923in}}%
\pgfpathcurveto{\pgfqpoint{4.982645in}{0.904324in}}{\pgfqpoint{5.056781in}{0.771224in}}{\pgfqpoint{5.158063in}{0.660123in}}%
\pgfpathcurveto{\pgfqpoint{5.259345in}{0.549022in}}{\pgfqpoint{5.385037in}{0.462921in}}{\pgfqpoint{5.525223in}{0.408612in}}%
\pgfpathcurveto{\pgfqpoint{5.665409in}{0.354304in}}{\pgfqpoint{5.816303in}{0.333255in}}{\pgfqpoint{5.966000in}{0.347126in}}%
\pgfpathcurveto{\pgfqpoint{6.115696in}{0.360998in}}{\pgfqpoint{6.260153in}{0.409415in}}{\pgfqpoint{6.387973in}{0.488558in}}%
\pgfpathlineto{\pgfqpoint{5.876322in}{1.314902in}}%
\pgfpathlineto{\pgfqpoint{5.876322in}{2.286824in}}%
\pgfpathlineto{\pgfqpoint{5.876322in}{2.286824in}}%
\pgfpathclose%
\pgfusepath{stroke,fill}%
\end{pgfscope}%
\begin{pgfscope}%
\pgfsetbuttcap%
\pgfsetmiterjoin%
\definecolor{currentfill}{rgb}{0.992157,0.705882,0.384314}%
\pgfsetfillcolor{currentfill}%
\pgfsetlinewidth{1.003750pt}%
\definecolor{currentstroke}{rgb}{1.000000,1.000000,1.000000}%
\pgfsetstrokecolor{currentstroke}%
\pgfsetdash{}{0pt}%
\pgfpathmoveto{\pgfqpoint{6.387973in}{0.488558in}}%
\pgfpathcurveto{\pgfqpoint{6.567670in}{0.599821in}}{\pgfqpoint{6.706262in}{0.766721in}}{\pgfqpoint{6.782612in}{0.963803in}}%
\pgfpathcurveto{\pgfqpoint{6.858962in}{1.160886in}}{\pgfqpoint{6.868981in}{1.377595in}}{\pgfqpoint{6.811141in}{1.580881in}}%
\pgfpathcurveto{\pgfqpoint{6.753302in}{1.784167in}}{\pgfqpoint{6.630701in}{1.963143in}}{\pgfqpoint{6.462036in}{2.090512in}}%
\pgfpathcurveto{\pgfqpoint{6.293371in}{2.217882in}}{\pgfqpoint{6.087677in}{2.286824in}}{\pgfqpoint{5.876322in}{2.286824in}}%
\pgfpathlineto{\pgfqpoint{5.876322in}{1.314902in}}%
\pgfpathlineto{\pgfqpoint{6.387973in}{0.488558in}}%
\pgfpathlineto{\pgfqpoint{6.387973in}{0.488558in}}%
\pgfpathclose%
\pgfusepath{stroke,fill}%
\end{pgfscope}%
\begin{pgfscope}%
\definecolor{textcolor}{rgb}{0.150000,0.150000,0.150000}%
\pgfsetstrokecolor{textcolor}%
\pgfsetfillcolor{textcolor}%
\pgftext[x=5.315431in,y=1.155315in,,]{\color{textcolor}\sffamily\fontsize{12.000000}{14.400000}\selectfont 58.82\%}%
\end{pgfscope}%
\begin{pgfscope}%
\definecolor{textcolor}{rgb}{0.150000,0.150000,0.150000}%
\pgfsetstrokecolor{textcolor}%
\pgfsetfillcolor{textcolor}%
\pgftext[x=6.437214in,y=1.474490in,,]{\color{textcolor}\sffamily\fontsize{12.000000}{14.400000}\selectfont 41.18\%}%
\end{pgfscope}%
\begin{pgfscope}%
\definecolor{textcolor}{rgb}{0.000000,0.000000,0.000000}%
\pgfsetstrokecolor{textcolor}%
\pgfsetfillcolor{textcolor}%
\pgftext[x=5.876322in,y=2.613138in,,base]{\color{textcolor}\sffamily\fontsize{12.000000}{14.400000}\selectfont Non Interactive Tutor}%
\end{pgfscope}%
\begin{pgfscope}%
\definecolor{textcolor}{rgb}{0.150000,0.150000,0.150000}%
\pgfsetstrokecolor{textcolor}%
\pgfsetfillcolor{textcolor}%
\pgftext[x=4.399657in,y=3.016660in,,top]{\color{textcolor}\sffamily\fontsize{14.400000}{17.280000}\selectfont Do you feel more or less intimidated by the command line after this interactive tutor?}%
\end{pgfscope}%
\begin{pgfscope}%
\pgfsetbuttcap%
\pgfsetmiterjoin%
\definecolor{currentfill}{rgb}{1.000000,1.000000,1.000000}%
\pgfsetfillcolor{currentfill}%
\pgfsetfillopacity{0.800000}%
\pgfsetlinewidth{1.003750pt}%
\definecolor{currentstroke}{rgb}{0.800000,0.800000,0.800000}%
\pgfsetstrokecolor{currentstroke}%
\pgfsetstrokeopacity{0.800000}%
\pgfsetdash{}{0pt}%
\pgfpathmoveto{\pgfqpoint{4.031901in}{1.341726in}}%
\pgfpathlineto{\pgfqpoint{4.767413in}{1.341726in}}%
\pgfpathquadraticcurveto{\pgfqpoint{4.797969in}{1.341726in}}{\pgfqpoint{4.797969in}{1.372282in}}%
\pgfpathlineto{\pgfqpoint{4.797969in}{1.744378in}}%
\pgfpathquadraticcurveto{\pgfqpoint{4.797969in}{1.774934in}}{\pgfqpoint{4.767413in}{1.774934in}}%
\pgfpathlineto{\pgfqpoint{4.031901in}{1.774934in}}%
\pgfpathquadraticcurveto{\pgfqpoint{4.001346in}{1.774934in}}{\pgfqpoint{4.001346in}{1.744378in}}%
\pgfpathlineto{\pgfqpoint{4.001346in}{1.372282in}}%
\pgfpathquadraticcurveto{\pgfqpoint{4.001346in}{1.341726in}}{\pgfqpoint{4.031901in}{1.341726in}}%
\pgfpathlineto{\pgfqpoint{4.031901in}{1.341726in}}%
\pgfpathclose%
\pgfusepath{stroke,fill}%
\end{pgfscope}%
\begin{pgfscope}%
\pgfsetbuttcap%
\pgfsetmiterjoin%
\definecolor{currentfill}{rgb}{0.501961,0.694118,0.827451}%
\pgfsetfillcolor{currentfill}%
\pgfsetlinewidth{1.003750pt}%
\definecolor{currentstroke}{rgb}{1.000000,1.000000,1.000000}%
\pgfsetstrokecolor{currentstroke}%
\pgfsetdash{}{0pt}%
\pgfpathmoveto{\pgfqpoint{4.062457in}{1.597748in}}%
\pgfpathlineto{\pgfqpoint{4.368012in}{1.597748in}}%
\pgfpathlineto{\pgfqpoint{4.368012in}{1.704692in}}%
\pgfpathlineto{\pgfqpoint{4.062457in}{1.704692in}}%
\pgfpathlineto{\pgfqpoint{4.062457in}{1.597748in}}%
\pgfpathclose%
\pgfusepath{stroke,fill}%
\end{pgfscope}%
\begin{pgfscope}%
\definecolor{textcolor}{rgb}{0.150000,0.150000,0.150000}%
\pgfsetstrokecolor{textcolor}%
\pgfsetfillcolor{textcolor}%
\pgftext[x=4.490235in,y=1.597748in,left,base]{\color{textcolor}\sffamily\fontsize{11.000000}{13.200000}\selectfont Yes}%
\end{pgfscope}%
\begin{pgfscope}%
\pgfsetbuttcap%
\pgfsetmiterjoin%
\definecolor{currentfill}{rgb}{0.992157,0.705882,0.384314}%
\pgfsetfillcolor{currentfill}%
\pgfsetlinewidth{1.003750pt}%
\definecolor{currentstroke}{rgb}{1.000000,1.000000,1.000000}%
\pgfsetstrokecolor{currentstroke}%
\pgfsetdash{}{0pt}%
\pgfpathmoveto{\pgfqpoint{4.062457in}{1.434616in}}%
\pgfpathlineto{\pgfqpoint{4.368012in}{1.434616in}}%
\pgfpathlineto{\pgfqpoint{4.368012in}{1.541560in}}%
\pgfpathlineto{\pgfqpoint{4.062457in}{1.541560in}}%
\pgfpathlineto{\pgfqpoint{4.062457in}{1.434616in}}%
\pgfpathclose%
\pgfusepath{stroke,fill}%
\end{pgfscope}%
\begin{pgfscope}%
\definecolor{textcolor}{rgb}{0.150000,0.150000,0.150000}%
\pgfsetstrokecolor{textcolor}%
\pgfsetfillcolor{textcolor}%
\pgftext[x=4.490235in,y=1.434616in,left,base]{\color{textcolor}\sffamily\fontsize{11.000000}{13.200000}\selectfont No}%
\end{pgfscope}%
\end{pgfpicture}%
\makeatother%
\endgroup%
}
	\caption{Chart depicting the self-reported effectiveness perception of reading.}
	\label{fig:reading}
\end{figure}

Similar to the question regarding comfort with CLIs, the perceived
effectiveness of written documentation was also inquired about in a
Likert-style, with options ranging from extremely effective to extremely
ineffective. Though a majority of individuals considered reading to be an
effective means of learning, a large proportion (29.4\%) reported reading to be
ineffective and a significant 17.65\% were ambivalent about the question.

\begin{figure}[H]
	\centering
	\scalebox{0.65}{%% Creator: Matplotlib, PGF backend
%%
%% To include the figure in your LaTeX document, write
%%   \input{<filename>.pgf}
%%
%% Make sure the required packages are loaded in your preamble
%%   \usepackage{pgf}
%%
%% Also ensure that all the required font packages are loaded; for instance,
%% the lmodern package is sometimes necessary when using math font.
%%   \usepackage{lmodern}
%%
%% Figures using additional raster images can only be included by \input if
%% they are in the same directory as the main LaTeX file. For loading figures
%% from other directories you can use the `import` package
%%   \usepackage{import}
%%
%% and then include the figures with
%%   \import{<path to file>}{<filename>.pgf}
%%
%% Matplotlib used the following preamble
%%   \usepackage{fontspec}
%%   \setmainfont{DejaVuSerif.ttf}[Path=\detokenize{/home/spam/miniconda3/envs/mpl/lib/python3.10/site-packages/matplotlib/mpl-data/fonts/ttf/}]
%%   \setsansfont{DejaVuSans.ttf}[Path=\detokenize{/home/spam/miniconda3/envs/mpl/lib/python3.10/site-packages/matplotlib/mpl-data/fonts/ttf/}]
%%   \setmonofont{DejaVuSansMono.ttf}[Path=\detokenize{/home/spam/miniconda3/envs/mpl/lib/python3.10/site-packages/matplotlib/mpl-data/fonts/ttf/}]
%%
\begingroup%
\makeatletter%
\begin{pgfpicture}%
\pgfpathrectangle{\pgfpointorigin}{\pgfqpoint{8.799314in}{3.116660in}}%
\pgfusepath{use as bounding box, clip}%
\begin{pgfscope}%
\pgfsetbuttcap%
\pgfsetmiterjoin%
\pgfsetlinewidth{0.000000pt}%
\definecolor{currentstroke}{rgb}{0.000000,0.000000,0.000000}%
\pgfsetstrokecolor{currentstroke}%
\pgfsetstrokeopacity{0.000000}%
\pgfsetdash{}{0pt}%
\pgfpathmoveto{\pgfqpoint{0.000000in}{0.000000in}}%
\pgfpathlineto{\pgfqpoint{8.799314in}{0.000000in}}%
\pgfpathlineto{\pgfqpoint{8.799314in}{3.116660in}}%
\pgfpathlineto{\pgfqpoint{0.000000in}{3.116660in}}%
\pgfpathlineto{\pgfqpoint{0.000000in}{0.000000in}}%
\pgfpathclose%
\pgfusepath{}%
\end{pgfscope}%
\begin{pgfscope}%
\pgfsetbuttcap%
\pgfsetmiterjoin%
\definecolor{currentfill}{rgb}{0.501961,0.694118,0.827451}%
\pgfsetfillcolor{currentfill}%
\pgfsetlinewidth{1.003750pt}%
\definecolor{currentstroke}{rgb}{1.000000,1.000000,1.000000}%
\pgfsetstrokecolor{currentstroke}%
\pgfsetdash{}{0pt}%
\pgfpathmoveto{\pgfqpoint{3.894914in}{1.314902in}}%
\pgfpathcurveto{\pgfqpoint{3.894914in}{1.442534in}}{\pgfqpoint{3.869773in}{1.568924in}}{\pgfqpoint{3.820931in}{1.686841in}}%
\pgfpathcurveto{\pgfqpoint{3.772088in}{1.804757in}}{\pgfqpoint{3.700494in}{1.911906in}}{\pgfqpoint{3.610245in}{2.002155in}}%
\pgfpathcurveto{\pgfqpoint{3.519995in}{2.092404in}}{\pgfqpoint{3.412847in}{2.163998in}}{\pgfqpoint{3.294931in}{2.212841in}}%
\pgfpathcurveto{\pgfqpoint{3.177014in}{2.261683in}}{\pgfqpoint{3.050624in}{2.286824in}}{\pgfqpoint{2.922992in}{2.286824in}}%
\pgfpathcurveto{\pgfqpoint{2.795361in}{2.286824in}}{\pgfqpoint{2.668970in}{2.261683in}}{\pgfqpoint{2.551054in}{2.212841in}}%
\pgfpathcurveto{\pgfqpoint{2.433138in}{2.163998in}}{\pgfqpoint{2.325989in}{2.092404in}}{\pgfqpoint{2.235740in}{2.002155in}}%
\pgfpathcurveto{\pgfqpoint{2.145491in}{1.911906in}}{\pgfqpoint{2.073896in}{1.804757in}}{\pgfqpoint{2.025054in}{1.686841in}}%
\pgfpathcurveto{\pgfqpoint{1.976211in}{1.568924in}}{\pgfqpoint{1.951070in}{1.442534in}}{\pgfqpoint{1.951070in}{1.314902in}}%
\pgfpathcurveto{\pgfqpoint{1.951070in}{1.187271in}}{\pgfqpoint{1.976211in}{1.060880in}}{\pgfqpoint{2.025054in}{0.942964in}}%
\pgfpathcurveto{\pgfqpoint{2.073896in}{0.825048in}}{\pgfqpoint{2.145491in}{0.717899in}}{\pgfqpoint{2.235740in}{0.627650in}}%
\pgfpathcurveto{\pgfqpoint{2.325989in}{0.537401in}}{\pgfqpoint{2.433138in}{0.465806in}}{\pgfqpoint{2.551054in}{0.416964in}}%
\pgfpathcurveto{\pgfqpoint{2.668970in}{0.368121in}}{\pgfqpoint{2.795361in}{0.342980in}}{\pgfqpoint{2.922992in}{0.342980in}}%
\pgfpathcurveto{\pgfqpoint{3.050624in}{0.342980in}}{\pgfqpoint{3.177014in}{0.368121in}}{\pgfqpoint{3.294931in}{0.416964in}}%
\pgfpathcurveto{\pgfqpoint{3.412847in}{0.465806in}}{\pgfqpoint{3.519995in}{0.537401in}}{\pgfqpoint{3.610245in}{0.627650in}}%
\pgfpathcurveto{\pgfqpoint{3.700494in}{0.717899in}}{\pgfqpoint{3.772088in}{0.825048in}}{\pgfqpoint{3.820931in}{0.942964in}}%
\pgfpathcurveto{\pgfqpoint{3.869773in}{1.060880in}}{\pgfqpoint{3.894914in}{1.187271in}}{\pgfqpoint{3.894914in}{1.314902in}}%
\pgfpathmoveto{\pgfqpoint{2.922992in}{1.314902in}}%
\pgfpathmoveto{\pgfqpoint{3.894914in}{1.314902in}}%
\pgfpathlineto{\pgfqpoint{3.894914in}{1.314902in}}%
\pgfpathclose%
\pgfusepath{stroke,fill}%
\end{pgfscope}%
\begin{pgfscope}%
\pgfsetbuttcap%
\pgfsetmiterjoin%
\definecolor{currentfill}{rgb}{0.992157,0.705882,0.384314}%
\pgfsetfillcolor{currentfill}%
\pgfsetlinewidth{1.003750pt}%
\definecolor{currentstroke}{rgb}{1.000000,1.000000,1.000000}%
\pgfsetstrokecolor{currentstroke}%
\pgfsetdash{}{0pt}%
\pgfpathmoveto{\pgfqpoint{2.922992in}{2.286824in}}%
\pgfpathcurveto{\pgfqpoint{2.922992in}{2.286824in}}{\pgfqpoint{2.922992in}{2.286824in}}{\pgfqpoint{2.922992in}{2.286824in}}%
\pgfpathlineto{\pgfqpoint{2.922992in}{1.314902in}}%
\pgfpathlineto{\pgfqpoint{2.922992in}{2.286824in}}%
\pgfpathlineto{\pgfqpoint{2.922992in}{2.286824in}}%
\pgfpathclose%
\pgfusepath{stroke,fill}%
\end{pgfscope}%
\begin{pgfscope}%
\definecolor{textcolor}{rgb}{0.150000,0.150000,0.150000}%
\pgfsetstrokecolor{textcolor}%
\pgfsetfillcolor{textcolor}%
\pgftext[x=2.922992in,y=0.731749in,,]{\color{textcolor}\sffamily\fontsize{12.000000}{14.400000}\selectfont 100.00\%}%
\end{pgfscope}%
\begin{pgfscope}%
\definecolor{textcolor}{rgb}{0.150000,0.150000,0.150000}%
\pgfsetstrokecolor{textcolor}%
\pgfsetfillcolor{textcolor}%
\pgftext[x=2.922992in,y=1.898055in,,]{\color{textcolor}\sffamily\fontsize{12.000000}{14.400000}\selectfont 0.00\%}%
\end{pgfscope}%
\begin{pgfscope}%
\definecolor{textcolor}{rgb}{0.000000,0.000000,0.000000}%
\pgfsetstrokecolor{textcolor}%
\pgfsetfillcolor{textcolor}%
\pgftext[x=2.922992in,y=2.613138in,,base]{\color{textcolor}\sffamily\fontsize{12.000000}{14.400000}\selectfont CLI-Tutor}%
\end{pgfscope}%
\begin{pgfscope}%
\pgfsetbuttcap%
\pgfsetmiterjoin%
\definecolor{currentfill}{rgb}{0.501961,0.694118,0.827451}%
\pgfsetfillcolor{currentfill}%
\pgfsetlinewidth{1.003750pt}%
\definecolor{currentstroke}{rgb}{1.000000,1.000000,1.000000}%
\pgfsetstrokecolor{currentstroke}%
\pgfsetdash{}{0pt}%
\pgfpathmoveto{\pgfqpoint{5.876322in}{2.286824in}}%
\pgfpathcurveto{\pgfqpoint{5.725984in}{2.286824in}}{\pgfqpoint{5.577676in}{2.251942in}}{\pgfqpoint{5.443099in}{2.184931in}}%
\pgfpathcurveto{\pgfqpoint{5.308522in}{2.117920in}}{\pgfqpoint{5.191311in}{2.020588in}}{\pgfqpoint{5.100712in}{1.900616in}}%
\pgfpathcurveto{\pgfqpoint{5.010113in}{1.780644in}}{\pgfqpoint{4.948574in}{1.641271in}}{\pgfqpoint{4.920949in}{1.493492in}}%
\pgfpathcurveto{\pgfqpoint{4.893325in}{1.345714in}}{\pgfqpoint{4.900361in}{1.193522in}}{\pgfqpoint{4.941503in}{1.048923in}}%
\pgfpathcurveto{\pgfqpoint{4.982645in}{0.904324in}}{\pgfqpoint{5.056781in}{0.771224in}}{\pgfqpoint{5.158063in}{0.660123in}}%
\pgfpathcurveto{\pgfqpoint{5.259345in}{0.549022in}}{\pgfqpoint{5.385037in}{0.462921in}}{\pgfqpoint{5.525223in}{0.408612in}}%
\pgfpathcurveto{\pgfqpoint{5.665409in}{0.354304in}}{\pgfqpoint{5.816303in}{0.333255in}}{\pgfqpoint{5.966000in}{0.347126in}}%
\pgfpathcurveto{\pgfqpoint{6.115696in}{0.360998in}}{\pgfqpoint{6.260153in}{0.409415in}}{\pgfqpoint{6.387973in}{0.488558in}}%
\pgfpathlineto{\pgfqpoint{5.876322in}{1.314902in}}%
\pgfpathlineto{\pgfqpoint{5.876322in}{2.286824in}}%
\pgfpathlineto{\pgfqpoint{5.876322in}{2.286824in}}%
\pgfpathclose%
\pgfusepath{stroke,fill}%
\end{pgfscope}%
\begin{pgfscope}%
\pgfsetbuttcap%
\pgfsetmiterjoin%
\definecolor{currentfill}{rgb}{0.992157,0.705882,0.384314}%
\pgfsetfillcolor{currentfill}%
\pgfsetlinewidth{1.003750pt}%
\definecolor{currentstroke}{rgb}{1.000000,1.000000,1.000000}%
\pgfsetstrokecolor{currentstroke}%
\pgfsetdash{}{0pt}%
\pgfpathmoveto{\pgfqpoint{6.387973in}{0.488558in}}%
\pgfpathcurveto{\pgfqpoint{6.567670in}{0.599821in}}{\pgfqpoint{6.706262in}{0.766721in}}{\pgfqpoint{6.782612in}{0.963803in}}%
\pgfpathcurveto{\pgfqpoint{6.858962in}{1.160886in}}{\pgfqpoint{6.868981in}{1.377595in}}{\pgfqpoint{6.811141in}{1.580881in}}%
\pgfpathcurveto{\pgfqpoint{6.753302in}{1.784167in}}{\pgfqpoint{6.630701in}{1.963143in}}{\pgfqpoint{6.462036in}{2.090512in}}%
\pgfpathcurveto{\pgfqpoint{6.293371in}{2.217882in}}{\pgfqpoint{6.087677in}{2.286824in}}{\pgfqpoint{5.876322in}{2.286824in}}%
\pgfpathlineto{\pgfqpoint{5.876322in}{1.314902in}}%
\pgfpathlineto{\pgfqpoint{6.387973in}{0.488558in}}%
\pgfpathlineto{\pgfqpoint{6.387973in}{0.488558in}}%
\pgfpathclose%
\pgfusepath{stroke,fill}%
\end{pgfscope}%
\begin{pgfscope}%
\definecolor{textcolor}{rgb}{0.150000,0.150000,0.150000}%
\pgfsetstrokecolor{textcolor}%
\pgfsetfillcolor{textcolor}%
\pgftext[x=5.315431in,y=1.155315in,,]{\color{textcolor}\sffamily\fontsize{12.000000}{14.400000}\selectfont 58.82\%}%
\end{pgfscope}%
\begin{pgfscope}%
\definecolor{textcolor}{rgb}{0.150000,0.150000,0.150000}%
\pgfsetstrokecolor{textcolor}%
\pgfsetfillcolor{textcolor}%
\pgftext[x=6.437214in,y=1.474490in,,]{\color{textcolor}\sffamily\fontsize{12.000000}{14.400000}\selectfont 41.18\%}%
\end{pgfscope}%
\begin{pgfscope}%
\definecolor{textcolor}{rgb}{0.000000,0.000000,0.000000}%
\pgfsetstrokecolor{textcolor}%
\pgfsetfillcolor{textcolor}%
\pgftext[x=5.876322in,y=2.613138in,,base]{\color{textcolor}\sffamily\fontsize{12.000000}{14.400000}\selectfont Non Interactive Tutor}%
\end{pgfscope}%
\begin{pgfscope}%
\definecolor{textcolor}{rgb}{0.150000,0.150000,0.150000}%
\pgfsetstrokecolor{textcolor}%
\pgfsetfillcolor{textcolor}%
\pgftext[x=4.399657in,y=3.016660in,,top]{\color{textcolor}\sffamily\fontsize{14.400000}{17.280000}\selectfont Do you feel more or less intimidated by the command line after this interactive tutor?}%
\end{pgfscope}%
\begin{pgfscope}%
\pgfsetbuttcap%
\pgfsetmiterjoin%
\definecolor{currentfill}{rgb}{1.000000,1.000000,1.000000}%
\pgfsetfillcolor{currentfill}%
\pgfsetfillopacity{0.800000}%
\pgfsetlinewidth{1.003750pt}%
\definecolor{currentstroke}{rgb}{0.800000,0.800000,0.800000}%
\pgfsetstrokecolor{currentstroke}%
\pgfsetstrokeopacity{0.800000}%
\pgfsetdash{}{0pt}%
\pgfpathmoveto{\pgfqpoint{4.031901in}{1.341726in}}%
\pgfpathlineto{\pgfqpoint{4.767413in}{1.341726in}}%
\pgfpathquadraticcurveto{\pgfqpoint{4.797969in}{1.341726in}}{\pgfqpoint{4.797969in}{1.372282in}}%
\pgfpathlineto{\pgfqpoint{4.797969in}{1.744378in}}%
\pgfpathquadraticcurveto{\pgfqpoint{4.797969in}{1.774934in}}{\pgfqpoint{4.767413in}{1.774934in}}%
\pgfpathlineto{\pgfqpoint{4.031901in}{1.774934in}}%
\pgfpathquadraticcurveto{\pgfqpoint{4.001346in}{1.774934in}}{\pgfqpoint{4.001346in}{1.744378in}}%
\pgfpathlineto{\pgfqpoint{4.001346in}{1.372282in}}%
\pgfpathquadraticcurveto{\pgfqpoint{4.001346in}{1.341726in}}{\pgfqpoint{4.031901in}{1.341726in}}%
\pgfpathlineto{\pgfqpoint{4.031901in}{1.341726in}}%
\pgfpathclose%
\pgfusepath{stroke,fill}%
\end{pgfscope}%
\begin{pgfscope}%
\pgfsetbuttcap%
\pgfsetmiterjoin%
\definecolor{currentfill}{rgb}{0.501961,0.694118,0.827451}%
\pgfsetfillcolor{currentfill}%
\pgfsetlinewidth{1.003750pt}%
\definecolor{currentstroke}{rgb}{1.000000,1.000000,1.000000}%
\pgfsetstrokecolor{currentstroke}%
\pgfsetdash{}{0pt}%
\pgfpathmoveto{\pgfqpoint{4.062457in}{1.597748in}}%
\pgfpathlineto{\pgfqpoint{4.368012in}{1.597748in}}%
\pgfpathlineto{\pgfqpoint{4.368012in}{1.704692in}}%
\pgfpathlineto{\pgfqpoint{4.062457in}{1.704692in}}%
\pgfpathlineto{\pgfqpoint{4.062457in}{1.597748in}}%
\pgfpathclose%
\pgfusepath{stroke,fill}%
\end{pgfscope}%
\begin{pgfscope}%
\definecolor{textcolor}{rgb}{0.150000,0.150000,0.150000}%
\pgfsetstrokecolor{textcolor}%
\pgfsetfillcolor{textcolor}%
\pgftext[x=4.490235in,y=1.597748in,left,base]{\color{textcolor}\sffamily\fontsize{11.000000}{13.200000}\selectfont Yes}%
\end{pgfscope}%
\begin{pgfscope}%
\pgfsetbuttcap%
\pgfsetmiterjoin%
\definecolor{currentfill}{rgb}{0.992157,0.705882,0.384314}%
\pgfsetfillcolor{currentfill}%
\pgfsetlinewidth{1.003750pt}%
\definecolor{currentstroke}{rgb}{1.000000,1.000000,1.000000}%
\pgfsetstrokecolor{currentstroke}%
\pgfsetdash{}{0pt}%
\pgfpathmoveto{\pgfqpoint{4.062457in}{1.434616in}}%
\pgfpathlineto{\pgfqpoint{4.368012in}{1.434616in}}%
\pgfpathlineto{\pgfqpoint{4.368012in}{1.541560in}}%
\pgfpathlineto{\pgfqpoint{4.062457in}{1.541560in}}%
\pgfpathlineto{\pgfqpoint{4.062457in}{1.434616in}}%
\pgfpathclose%
\pgfusepath{stroke,fill}%
\end{pgfscope}%
\begin{pgfscope}%
\definecolor{textcolor}{rgb}{0.150000,0.150000,0.150000}%
\pgfsetstrokecolor{textcolor}%
\pgfsetfillcolor{textcolor}%
\pgftext[x=4.490235in,y=1.434616in,left,base]{\color{textcolor}\sffamily\fontsize{11.000000}{13.200000}\selectfont No}%
\end{pgfscope}%
\end{pgfpicture}%
\makeatother%
\endgroup%
}
	\caption{Chart depicting the amount of participants with previous interactive learning experience.}
	\label{fig:previnteractive}
\end{figure}

Participants were also questioned about their existing experience with
interactive learning resources. The vast majority, (73.53\%) indicated that
they had previous experience with interactive learning resources (see:
\autoref{fig:previnteractive}).
