\chapter{User Study}
% - User Study
%    - methodology
%       - assignment
%       - survey structure
%    - participants
% DATAPOINTS: Interest in interactive learnign or percentage of people who find
% it more effective than reading
\label{chap:userstudy}

In order to test and validate the effectiveness of our solution, a user study
was conducted. The goal of this user study was two part. Firstly, we were
interested in assessing the usability and response to \textit{CLI-Tutor}.
Secondly, we wanted to ascertain if interactive learning would be a more
effective medium to teach command line interaction than the traditional means
such as online documentation or books. In this chapter, we will describe our
user study in detail.


\section{Methodology}

The user study for this thesis work was conducted remotely and asynchronously.
We designed an online survey using the \textit{LimeSurvey}\cite{schmitzlime}
tool made available to us by the University of Zurich.

The user study focused on the comparison between interactive learning
approaches such as that of \textit{CLI-Tutor} and traditional ones, which are
mostly reading based. In the modern software development space, online
documentation is the status quo and the medium we choose to compare our
solution against. As discussed in \autoref{chap:design}, \textit{CLI-Tutor}
uses \textit{Markdown} to specify lessons. Many static documentation generation
websites use Markdown files to generate documentation from. This is also true
for our chosen generator. This enables us to make objectively compare the
interaction medium rather than the lesson content, since the exact same lessons
can be used in both tools. Furthermore, due to the popularity of
\textit{MkDocs} it is a very realistic representation of documentation that
individuals may encounter in the wild.


\subsection{Interactive versus Non-interactive}

Our user study intended to perform an A/B testing comparing learning mediums.
To support this we randomly assigned our participants to one of two groups,
interactive and non-interactive. 


\subsection{Structure}

In this section, we provide a structural overview of our online survey. The
entire survey is available in \autoref{chap:appendixa}.

Our online survey was divided into the following sections:

\begin{itemize}

    \item User Familiarity: In this section users answered questions relating
        to their experience, interest and preferences to provide us with some
        information regarding each individual participant.

    \item Assignment: All participants where divided into one of two groups,
        interactive and non-interactive. The assignment value is unknown to the
        participant at the time of starting the survey. Our survey tool then
        conditionally rendered a URL for the participants to follow to the next
        section of the survey.

    \item Tutorial: At this stage, once the participants have been assigned to
        one of the two groups, they will either be sent to our web application
        running \textit{CLI-Tutor} or to our documentation website. If assigned
        to the non-interactive group...

    \item Evaluation: The evaluation stage is where participants were asked a
        series of basic questions relating to the lessons they took in the
        previous stage.

    \item Feedback: In this section participants were able to provide feedback
        regarding their experience. All but one of the questions in this
        section were identical for both user groups. The non-interactive group
        were asked one additional question regarding interactive learning: \textit{Do 
        you think an interactive command line tutorial application would
    improve the learning process?}

    \item Feedback Opposite (optional): This section was optional and included
        only one quick feedback question. At the end of the survey,
        participants were then given a chance to try out the opposite tool to
        which they were assigned for the bulk of the user study. No evaluation
        or tutorial was mandated here and participants were given one free text
        response to report on their feelings using the alternative tool.
\end{itemize}

\subsection{Assignment}

Assignment was achieved through the use of pseudorandom number generation, a
feature of \textit{LimeSurvey}. 19 were assigned to the interactive group and 15 to the non-interactive group.

\section{Participants}

In total, 34 participants took part in our user study. Recruitment was
primarily done through University channels though some participants were also
sourced through work email and word of mouth.


\section{Section} In addition to these two views, there is also a prompt to
enter an identifier when the program is first launched. This was only added to
the program to help identify log files for the user study conducted as a part
of this thesis work.

\paragraph{Paragraph.} Always with a point.


\begin{figure}[H]
	\centering
	\scalebox{0.75}{%% Creator: Matplotlib, PGF backend
%%
%% To include the figure in your LaTeX document, write
%%   \input{<filename>.pgf}
%%
%% Make sure the required packages are loaded in your preamble
%%   \usepackage{pgf}
%%
%% Also ensure that all the required font packages are loaded; for instance,
%% the lmodern package is sometimes necessary when using math font.
%%   \usepackage{lmodern}
%%
%% Figures using additional raster images can only be included by \input if
%% they are in the same directory as the main LaTeX file. For loading figures
%% from other directories you can use the `import` package
%%   \usepackage{import}
%%
%% and then include the figures with
%%   \import{<path to file>}{<filename>.pgf}
%%
%% Matplotlib used the following preamble
%%   \usepackage{fontspec}
%%   \setmainfont{DejaVuSerif.ttf}[Path=\detokenize{/home/spam/miniconda3/envs/mpl/lib/python3.10/site-packages/matplotlib/mpl-data/fonts/ttf/}]
%%   \setsansfont{DejaVuSans.ttf}[Path=\detokenize{/home/spam/miniconda3/envs/mpl/lib/python3.10/site-packages/matplotlib/mpl-data/fonts/ttf/}]
%%   \setmonofont{DejaVuSansMono.ttf}[Path=\detokenize{/home/spam/miniconda3/envs/mpl/lib/python3.10/site-packages/matplotlib/mpl-data/fonts/ttf/}]
%%
\begingroup%
\makeatletter%
\begin{pgfpicture}%
\pgfpathrectangle{\pgfpointorigin}{\pgfqpoint{5.906660in}{5.000000in}}%
\pgfusepath{use as bounding box, clip}%
\begin{pgfscope}%
\pgfsetbuttcap%
\pgfsetmiterjoin%
\definecolor{currentfill}{rgb}{1.000000,1.000000,1.000000}%
\pgfsetfillcolor{currentfill}%
\pgfsetlinewidth{0.000000pt}%
\definecolor{currentstroke}{rgb}{1.000000,1.000000,1.000000}%
\pgfsetstrokecolor{currentstroke}%
\pgfsetdash{}{0pt}%
\pgfpathmoveto{\pgfqpoint{0.000000in}{0.000000in}}%
\pgfpathlineto{\pgfqpoint{5.906660in}{0.000000in}}%
\pgfpathlineto{\pgfqpoint{5.906660in}{5.000000in}}%
\pgfpathlineto{\pgfqpoint{0.000000in}{5.000000in}}%
\pgfpathlineto{\pgfqpoint{0.000000in}{0.000000in}}%
\pgfpathclose%
\pgfusepath{fill}%
\end{pgfscope}%
\begin{pgfscope}%
\definecolor{textcolor}{rgb}{0.150000,0.150000,0.150000}%
\pgfsetstrokecolor{textcolor}%
\pgfsetfillcolor{textcolor}%
\pgftext[x=3.027163in,y=0.411111in,,top]{\color{textcolor}\sffamily\fontsize{12.000000}{14.400000}\selectfont Computer Science UniversityUniverity Experience}%
\end{pgfscope}%
\begin{pgfscope}%
\pgfsetbuttcap%
\pgfsetmiterjoin%
\definecolor{currentfill}{rgb}{0.552941,0.827451,0.780392}%
\pgfsetfillcolor{currentfill}%
\pgfsetlinewidth{1.003750pt}%
\definecolor{currentstroke}{rgb}{1.000000,1.000000,1.000000}%
\pgfsetstrokecolor{currentstroke}%
\pgfsetdash{}{0pt}%
\pgfpathmoveto{\pgfqpoint{4.567163in}{2.475000in}}%
\pgfpathcurveto{\pgfqpoint{4.567163in}{2.713219in}}{\pgfqpoint{4.511889in}{2.948220in}}{\pgfqpoint{4.405702in}{3.161463in}}%
\pgfpathcurveto{\pgfqpoint{4.299515in}{3.374705in}}{\pgfqpoint{4.145283in}{3.560429in}}{\pgfqpoint{3.955175in}{3.703981in}}%
\pgfpathcurveto{\pgfqpoint{3.765067in}{3.847533in}}{\pgfqpoint{3.544218in}{3.945035in}}{\pgfqpoint{3.310053in}{3.988794in}}%
\pgfpathcurveto{\pgfqpoint{3.075889in}{4.032554in}}{\pgfqpoint{2.834733in}{4.021389in}}{\pgfqpoint{2.605613in}{3.956180in}}%
\pgfpathlineto{\pgfqpoint{3.027163in}{2.475000in}}%
\pgfpathlineto{\pgfqpoint{4.567163in}{2.475000in}}%
\pgfpathlineto{\pgfqpoint{4.567163in}{2.475000in}}%
\pgfpathclose%
\pgfusepath{stroke,fill}%
\end{pgfscope}%
\begin{pgfscope}%
\pgfsetbuttcap%
\pgfsetmiterjoin%
\definecolor{currentfill}{rgb}{1.000000,1.000000,0.701961}%
\pgfsetfillcolor{currentfill}%
\pgfsetlinewidth{1.003750pt}%
\definecolor{currentstroke}{rgb}{1.000000,1.000000,1.000000}%
\pgfsetstrokecolor{currentstroke}%
\pgfsetdash{}{0pt}%
\pgfpathmoveto{\pgfqpoint{2.605613in}{3.956180in}}%
\pgfpathcurveto{\pgfqpoint{2.376493in}{3.890972in}}{\pgfqpoint{2.165597in}{3.773481in}}{\pgfqpoint{1.989567in}{3.612978in}}%
\pgfpathcurveto{\pgfqpoint{1.813536in}{3.452475in}}{\pgfqpoint{1.677124in}{3.253294in}}{\pgfqpoint{1.591094in}{3.031153in}}%
\pgfpathcurveto{\pgfqpoint{1.505064in}{2.809011in}}{\pgfqpoint{1.471740in}{2.569908in}}{\pgfqpoint{1.493751in}{2.332708in}}%
\pgfpathcurveto{\pgfqpoint{1.515762in}{2.095508in}}{\pgfqpoint{1.592513in}{1.866620in}}{\pgfqpoint{1.717949in}{1.664101in}}%
\pgfpathlineto{\pgfqpoint{3.027163in}{2.475000in}}%
\pgfpathlineto{\pgfqpoint{2.605613in}{3.956180in}}%
\pgfpathlineto{\pgfqpoint{2.605613in}{3.956180in}}%
\pgfpathclose%
\pgfusepath{stroke,fill}%
\end{pgfscope}%
\begin{pgfscope}%
\pgfsetbuttcap%
\pgfsetmiterjoin%
\definecolor{currentfill}{rgb}{0.745098,0.729412,0.854902}%
\pgfsetfillcolor{currentfill}%
\pgfsetlinewidth{1.003750pt}%
\definecolor{currentstroke}{rgb}{1.000000,1.000000,1.000000}%
\pgfsetstrokecolor{currentstroke}%
\pgfsetdash{}{0pt}%
\pgfpathmoveto{\pgfqpoint{1.717949in}{1.664101in}}%
\pgfpathcurveto{\pgfqpoint{1.843385in}{1.461582in}}{\pgfqpoint{2.014117in}{1.290903in}}{\pgfqpoint{2.216676in}{1.165531in}}%
\pgfpathcurveto{\pgfqpoint{2.419234in}{1.040158in}}{\pgfqpoint{2.648147in}{0.963479in}}{\pgfqpoint{2.885354in}{0.941543in}}%
\pgfpathcurveto{\pgfqpoint{3.122560in}{0.919607in}}{\pgfqpoint{3.361653in}{0.953006in}}{\pgfqpoint{3.583768in}{1.039106in}}%
\pgfpathcurveto{\pgfqpoint{3.805882in}{1.125206in}}{\pgfqpoint{4.005020in}{1.261680in}}{\pgfqpoint{4.165467in}{1.437762in}}%
\pgfpathlineto{\pgfqpoint{3.027163in}{2.475000in}}%
\pgfpathlineto{\pgfqpoint{1.717949in}{1.664101in}}%
\pgfpathlineto{\pgfqpoint{1.717949in}{1.664101in}}%
\pgfpathclose%
\pgfusepath{stroke,fill}%
\end{pgfscope}%
\begin{pgfscope}%
\pgfsetbuttcap%
\pgfsetmiterjoin%
\definecolor{currentfill}{rgb}{0.984314,0.501961,0.447059}%
\pgfsetfillcolor{currentfill}%
\pgfsetlinewidth{1.003750pt}%
\definecolor{currentstroke}{rgb}{1.000000,1.000000,1.000000}%
\pgfsetstrokecolor{currentstroke}%
\pgfsetdash{}{0pt}%
\pgfpathmoveto{\pgfqpoint{4.165467in}{1.437762in}}%
\pgfpathcurveto{\pgfqpoint{4.165467in}{1.437762in}}{\pgfqpoint{4.165467in}{1.437762in}}{\pgfqpoint{4.165467in}{1.437762in}}%
\pgfpathlineto{\pgfqpoint{3.027163in}{2.475000in}}%
\pgfpathlineto{\pgfqpoint{4.165467in}{1.437762in}}%
\pgfpathlineto{\pgfqpoint{4.165467in}{1.437762in}}%
\pgfpathclose%
\pgfusepath{stroke,fill}%
\end{pgfscope}%
\begin{pgfscope}%
\pgfsetbuttcap%
\pgfsetmiterjoin%
\definecolor{currentfill}{rgb}{0.501961,0.694118,0.827451}%
\pgfsetfillcolor{currentfill}%
\pgfsetlinewidth{1.003750pt}%
\definecolor{currentstroke}{rgb}{1.000000,1.000000,1.000000}%
\pgfsetstrokecolor{currentstroke}%
\pgfsetdash{}{0pt}%
\pgfpathmoveto{\pgfqpoint{4.165467in}{1.437762in}}%
\pgfpathcurveto{\pgfqpoint{4.293579in}{1.578356in}}{\pgfqpoint{4.394541in}{1.741476in}}{\pgfqpoint{4.463232in}{1.918848in}}%
\pgfpathcurveto{\pgfqpoint{4.531924in}{2.096219in}}{\pgfqpoint{4.567163in}{2.284792in}}{\pgfqpoint{4.567163in}{2.475001in}}%
\pgfpathlineto{\pgfqpoint{3.027163in}{2.475000in}}%
\pgfpathlineto{\pgfqpoint{4.165467in}{1.437762in}}%
\pgfpathlineto{\pgfqpoint{4.165467in}{1.437762in}}%
\pgfpathclose%
\pgfusepath{stroke,fill}%
\end{pgfscope}%
\begin{pgfscope}%
\pgfsetbuttcap%
\pgfsetmiterjoin%
\definecolor{currentfill}{rgb}{0.992157,0.705882,0.384314}%
\pgfsetfillcolor{currentfill}%
\pgfsetlinewidth{1.003750pt}%
\definecolor{currentstroke}{rgb}{1.000000,1.000000,1.000000}%
\pgfsetstrokecolor{currentstroke}%
\pgfsetdash{}{0pt}%
\pgfpathmoveto{\pgfqpoint{4.567163in}{2.475001in}}%
\pgfpathcurveto{\pgfqpoint{4.567163in}{2.475001in}}{\pgfqpoint{4.567163in}{2.475001in}}{\pgfqpoint{4.567163in}{2.475001in}}%
\pgfpathlineto{\pgfqpoint{3.027163in}{2.475000in}}%
\pgfpathlineto{\pgfqpoint{4.567163in}{2.475001in}}%
\pgfpathlineto{\pgfqpoint{4.567163in}{2.475001in}}%
\pgfpathclose%
\pgfusepath{stroke,fill}%
\end{pgfscope}%
\begin{pgfscope}%
\definecolor{textcolor}{rgb}{0.150000,0.150000,0.150000}%
\pgfsetstrokecolor{textcolor}%
\pgfsetfillcolor{textcolor}%
\pgftext[x=3.583970in,y=3.212389in,,]{\color{textcolor}\sffamily\fontsize{12.000000}{14.400000}\selectfont 29.4\%}%
\end{pgfscope}%
\begin{pgfscope}%
\definecolor{textcolor}{rgb}{0.150000,0.150000,0.150000}%
\pgfsetstrokecolor{textcolor}%
\pgfsetfillcolor{textcolor}%
\pgftext[x=2.165522in,y=2.808692in,,]{\color{textcolor}\sffamily\fontsize{12.000000}{14.400000}\selectfont 29.4\%}%
\end{pgfscope}%
\begin{pgfscope}%
\definecolor{textcolor}{rgb}{0.150000,0.150000,0.150000}%
\pgfsetstrokecolor{textcolor}%
\pgfsetfillcolor{textcolor}%
\pgftext[x=2.942078in,y=1.554926in,,]{\color{textcolor}\sffamily\fontsize{12.000000}{14.400000}\selectfont 29.4\%}%
\end{pgfscope}%
\begin{pgfscope}%
\definecolor{textcolor}{rgb}{0.150000,0.150000,0.150000}%
\pgfsetstrokecolor{textcolor}%
\pgfsetfillcolor{textcolor}%
\pgftext[x=3.710146in,y=1.852657in,,]{\color{textcolor}\sffamily\fontsize{12.000000}{14.400000}\selectfont 0.0\%}%
\end{pgfscope}%
\begin{pgfscope}%
\definecolor{textcolor}{rgb}{0.150000,0.150000,0.150000}%
\pgfsetstrokecolor{textcolor}%
\pgfsetfillcolor{textcolor}%
\pgftext[x=3.888805in,y=2.141309in,,]{\color{textcolor}\sffamily\fontsize{12.000000}{14.400000}\selectfont 11.8\%}%
\end{pgfscope}%
\begin{pgfscope}%
\definecolor{textcolor}{rgb}{0.150000,0.150000,0.150000}%
\pgfsetstrokecolor{textcolor}%
\pgfsetfillcolor{textcolor}%
\pgftext[x=3.951163in,y=2.475000in,,]{\color{textcolor}\sffamily\fontsize{12.000000}{14.400000}\selectfont 0.0\%}%
\end{pgfscope}%
\begin{pgfscope}%
\pgfsetbuttcap%
\pgfsetmiterjoin%
\definecolor{currentfill}{rgb}{1.000000,1.000000,1.000000}%
\pgfsetfillcolor{currentfill}%
\pgfsetfillopacity{0.800000}%
\pgfsetlinewidth{1.003750pt}%
\definecolor{currentstroke}{rgb}{0.800000,0.800000,0.800000}%
\pgfsetstrokecolor{currentstroke}%
\pgfsetstrokeopacity{0.800000}%
\pgfsetdash{}{0pt}%
\pgfpathmoveto{\pgfqpoint{3.271524in}{0.458500in}}%
\pgfpathlineto{\pgfqpoint{5.827163in}{0.458500in}}%
\pgfpathquadraticcurveto{\pgfqpoint{5.852163in}{0.458500in}}{\pgfqpoint{5.852163in}{0.483500in}}%
\pgfpathlineto{\pgfqpoint{5.852163in}{1.571829in}}%
\pgfpathquadraticcurveto{\pgfqpoint{5.852163in}{1.596829in}}{\pgfqpoint{5.827163in}{1.596829in}}%
\pgfpathlineto{\pgfqpoint{3.271524in}{1.596829in}}%
\pgfpathquadraticcurveto{\pgfqpoint{3.246524in}{1.596829in}}{\pgfqpoint{3.246524in}{1.571829in}}%
\pgfpathlineto{\pgfqpoint{3.246524in}{0.483500in}}%
\pgfpathquadraticcurveto{\pgfqpoint{3.246524in}{0.458500in}}{\pgfqpoint{3.271524in}{0.458500in}}%
\pgfpathlineto{\pgfqpoint{3.271524in}{0.458500in}}%
\pgfpathclose%
\pgfusepath{stroke,fill}%
\end{pgfscope}%
\begin{pgfscope}%
\pgfsetbuttcap%
\pgfsetmiterjoin%
\definecolor{currentfill}{rgb}{0.552941,0.827451,0.780392}%
\pgfsetfillcolor{currentfill}%
\pgfsetlinewidth{1.003750pt}%
\definecolor{currentstroke}{rgb}{1.000000,1.000000,1.000000}%
\pgfsetstrokecolor{currentstroke}%
\pgfsetdash{}{0pt}%
\pgfpathmoveto{\pgfqpoint{3.296524in}{1.451858in}}%
\pgfpathlineto{\pgfqpoint{3.546524in}{1.451858in}}%
\pgfpathlineto{\pgfqpoint{3.546524in}{1.539358in}}%
\pgfpathlineto{\pgfqpoint{3.296524in}{1.539358in}}%
\pgfpathlineto{\pgfqpoint{3.296524in}{1.451858in}}%
\pgfpathclose%
\pgfusepath{stroke,fill}%
\end{pgfscope}%
\begin{pgfscope}%
\definecolor{textcolor}{rgb}{0.150000,0.150000,0.150000}%
\pgfsetstrokecolor{textcolor}%
\pgfsetfillcolor{textcolor}%
\pgftext[x=3.646524in,y=1.451858in,left,base]{\color{textcolor}\sffamily\fontsize{9.000000}{10.800000}\selectfont No CS Degree, 29.4 \%}%
\end{pgfscope}%
\begin{pgfscope}%
\pgfsetbuttcap%
\pgfsetmiterjoin%
\definecolor{currentfill}{rgb}{1.000000,1.000000,0.701961}%
\pgfsetfillcolor{currentfill}%
\pgfsetlinewidth{1.003750pt}%
\definecolor{currentstroke}{rgb}{1.000000,1.000000,1.000000}%
\pgfsetstrokecolor{currentstroke}%
\pgfsetdash{}{0pt}%
\pgfpathmoveto{\pgfqpoint{3.296524in}{1.268387in}}%
\pgfpathlineto{\pgfqpoint{3.546524in}{1.268387in}}%
\pgfpathlineto{\pgfqpoint{3.546524in}{1.355887in}}%
\pgfpathlineto{\pgfqpoint{3.296524in}{1.355887in}}%
\pgfpathlineto{\pgfqpoint{3.296524in}{1.268387in}}%
\pgfpathclose%
\pgfusepath{stroke,fill}%
\end{pgfscope}%
\begin{pgfscope}%
\definecolor{textcolor}{rgb}{0.150000,0.150000,0.150000}%
\pgfsetstrokecolor{textcolor}%
\pgfsetfillcolor{textcolor}%
\pgftext[x=3.646524in,y=1.268387in,left,base]{\color{textcolor}\sffamily\fontsize{9.000000}{10.800000}\selectfont Bachelors's Degree, 29.4 \%}%
\end{pgfscope}%
\begin{pgfscope}%
\pgfsetbuttcap%
\pgfsetmiterjoin%
\definecolor{currentfill}{rgb}{0.745098,0.729412,0.854902}%
\pgfsetfillcolor{currentfill}%
\pgfsetlinewidth{1.003750pt}%
\definecolor{currentstroke}{rgb}{1.000000,1.000000,1.000000}%
\pgfsetstrokecolor{currentstroke}%
\pgfsetdash{}{0pt}%
\pgfpathmoveto{\pgfqpoint{3.296524in}{1.084915in}}%
\pgfpathlineto{\pgfqpoint{3.546524in}{1.084915in}}%
\pgfpathlineto{\pgfqpoint{3.546524in}{1.172415in}}%
\pgfpathlineto{\pgfqpoint{3.296524in}{1.172415in}}%
\pgfpathlineto{\pgfqpoint{3.296524in}{1.084915in}}%
\pgfpathclose%
\pgfusepath{stroke,fill}%
\end{pgfscope}%
\begin{pgfscope}%
\definecolor{textcolor}{rgb}{0.150000,0.150000,0.150000}%
\pgfsetstrokecolor{textcolor}%
\pgfsetfillcolor{textcolor}%
\pgftext[x=3.646524in,y=1.084915in,left,base]{\color{textcolor}\sffamily\fontsize{9.000000}{10.800000}\selectfont Master's Degree, 29.4 \%}%
\end{pgfscope}%
\begin{pgfscope}%
\pgfsetbuttcap%
\pgfsetmiterjoin%
\definecolor{currentfill}{rgb}{0.984314,0.501961,0.447059}%
\pgfsetfillcolor{currentfill}%
\pgfsetlinewidth{1.003750pt}%
\definecolor{currentstroke}{rgb}{1.000000,1.000000,1.000000}%
\pgfsetstrokecolor{currentstroke}%
\pgfsetdash{}{0pt}%
\pgfpathmoveto{\pgfqpoint{3.296524in}{0.901444in}}%
\pgfpathlineto{\pgfqpoint{3.546524in}{0.901444in}}%
\pgfpathlineto{\pgfqpoint{3.546524in}{0.988944in}}%
\pgfpathlineto{\pgfqpoint{3.296524in}{0.988944in}}%
\pgfpathlineto{\pgfqpoint{3.296524in}{0.901444in}}%
\pgfpathclose%
\pgfusepath{stroke,fill}%
\end{pgfscope}%
\begin{pgfscope}%
\definecolor{textcolor}{rgb}{0.150000,0.150000,0.150000}%
\pgfsetstrokecolor{textcolor}%
\pgfsetfillcolor{textcolor}%
\pgftext[x=3.646524in,y=0.901444in,left,base]{\color{textcolor}\sffamily\fontsize{9.000000}{10.800000}\selectfont Doctorate Degree, 0.0 \%}%
\end{pgfscope}%
\begin{pgfscope}%
\pgfsetbuttcap%
\pgfsetmiterjoin%
\definecolor{currentfill}{rgb}{0.501961,0.694118,0.827451}%
\pgfsetfillcolor{currentfill}%
\pgfsetlinewidth{1.003750pt}%
\definecolor{currentstroke}{rgb}{1.000000,1.000000,1.000000}%
\pgfsetstrokecolor{currentstroke}%
\pgfsetdash{}{0pt}%
\pgfpathmoveto{\pgfqpoint{3.296524in}{0.717972in}}%
\pgfpathlineto{\pgfqpoint{3.546524in}{0.717972in}}%
\pgfpathlineto{\pgfqpoint{3.546524in}{0.805472in}}%
\pgfpathlineto{\pgfqpoint{3.296524in}{0.805472in}}%
\pgfpathlineto{\pgfqpoint{3.296524in}{0.717972in}}%
\pgfpathclose%
\pgfusepath{stroke,fill}%
\end{pgfscope}%
\begin{pgfscope}%
\definecolor{textcolor}{rgb}{0.150000,0.150000,0.150000}%
\pgfsetstrokecolor{textcolor}%
\pgfsetfillcolor{textcolor}%
\pgftext[x=3.646524in,y=0.717972in,left,base]{\color{textcolor}\sffamily\fontsize{9.000000}{10.800000}\selectfont Other Engineering Degree, 11.8 \%}%
\end{pgfscope}%
\begin{pgfscope}%
\pgfsetbuttcap%
\pgfsetmiterjoin%
\definecolor{currentfill}{rgb}{0.992157,0.705882,0.384314}%
\pgfsetfillcolor{currentfill}%
\pgfsetlinewidth{1.003750pt}%
\definecolor{currentstroke}{rgb}{1.000000,1.000000,1.000000}%
\pgfsetstrokecolor{currentstroke}%
\pgfsetdash{}{0pt}%
\pgfpathmoveto{\pgfqpoint{3.296524in}{0.534501in}}%
\pgfpathlineto{\pgfqpoint{3.546524in}{0.534501in}}%
\pgfpathlineto{\pgfqpoint{3.546524in}{0.622001in}}%
\pgfpathlineto{\pgfqpoint{3.296524in}{0.622001in}}%
\pgfpathlineto{\pgfqpoint{3.296524in}{0.534501in}}%
\pgfpathclose%
\pgfusepath{stroke,fill}%
\end{pgfscope}%
\begin{pgfscope}%
\definecolor{textcolor}{rgb}{0.150000,0.150000,0.150000}%
\pgfsetstrokecolor{textcolor}%
\pgfsetfillcolor{textcolor}%
\pgftext[x=3.646524in,y=0.534501in,left,base]{\color{textcolor}\sffamily\fontsize{9.000000}{10.800000}\selectfont Yes but did not finish, 0.0 \%}%
\end{pgfscope}%
\end{pgfpicture}%
\makeatother%
\endgroup%
}
	\caption{The distribution of programming experience amongst study participants.}
	\label{fig:programmingexp}
\end{figure}

\begin{figure}[H]
	\centering
	\scalebox{0.8}{%% Creator: Matplotlib, PGF backend
%%
%% To include the figure in your LaTeX document, write
%%   \input{<filename>.pgf}
%%
%% Make sure the required packages are loaded in your preamble
%%   \usepackage{pgf}
%%
%% Also ensure that all the required font packages are loaded; for instance,
%% the lmodern package is sometimes necessary when using math font.
%%   \usepackage{lmodern}
%%
%% Figures using additional raster images can only be included by \input if
%% they are in the same directory as the main LaTeX file. For loading figures
%% from other directories you can use the `import` package
%%   \usepackage{import}
%%
%% and then include the figures with
%%   \import{<path to file>}{<filename>.pgf}
%%
%% Matplotlib used the following preamble
%%   \usepackage{fontspec}
%%   \setmainfont{DejaVuSerif.ttf}[Path=\detokenize{/home/spam/miniconda3/envs/mpl/lib/python3.10/site-packages/matplotlib/mpl-data/fonts/ttf/}]
%%   \setsansfont{DejaVuSans.ttf}[Path=\detokenize{/home/spam/miniconda3/envs/mpl/lib/python3.10/site-packages/matplotlib/mpl-data/fonts/ttf/}]
%%   \setmonofont{DejaVuSansMono.ttf}[Path=\detokenize{/home/spam/miniconda3/envs/mpl/lib/python3.10/site-packages/matplotlib/mpl-data/fonts/ttf/}]
%%
\begingroup%
\makeatletter%
\begin{pgfpicture}%
\pgfpathrectangle{\pgfpointorigin}{\pgfqpoint{5.906660in}{5.000000in}}%
\pgfusepath{use as bounding box, clip}%
\begin{pgfscope}%
\pgfsetbuttcap%
\pgfsetmiterjoin%
\definecolor{currentfill}{rgb}{1.000000,1.000000,1.000000}%
\pgfsetfillcolor{currentfill}%
\pgfsetlinewidth{0.000000pt}%
\definecolor{currentstroke}{rgb}{1.000000,1.000000,1.000000}%
\pgfsetstrokecolor{currentstroke}%
\pgfsetdash{}{0pt}%
\pgfpathmoveto{\pgfqpoint{0.000000in}{0.000000in}}%
\pgfpathlineto{\pgfqpoint{5.906660in}{0.000000in}}%
\pgfpathlineto{\pgfqpoint{5.906660in}{5.000000in}}%
\pgfpathlineto{\pgfqpoint{0.000000in}{5.000000in}}%
\pgfpathlineto{\pgfqpoint{0.000000in}{0.000000in}}%
\pgfpathclose%
\pgfusepath{fill}%
\end{pgfscope}%
\begin{pgfscope}%
\definecolor{textcolor}{rgb}{0.150000,0.150000,0.150000}%
\pgfsetstrokecolor{textcolor}%
\pgfsetfillcolor{textcolor}%
\pgftext[x=3.027163in,y=0.411111in,,top]{\color{textcolor}\sffamily\fontsize{12.000000}{14.400000}\selectfont Computer Science UniversityUniverity Experience}%
\end{pgfscope}%
\begin{pgfscope}%
\pgfsetbuttcap%
\pgfsetmiterjoin%
\definecolor{currentfill}{rgb}{0.552941,0.827451,0.780392}%
\pgfsetfillcolor{currentfill}%
\pgfsetlinewidth{1.003750pt}%
\definecolor{currentstroke}{rgb}{1.000000,1.000000,1.000000}%
\pgfsetstrokecolor{currentstroke}%
\pgfsetdash{}{0pt}%
\pgfpathmoveto{\pgfqpoint{4.567163in}{2.475000in}}%
\pgfpathcurveto{\pgfqpoint{4.567163in}{2.713219in}}{\pgfqpoint{4.511889in}{2.948220in}}{\pgfqpoint{4.405702in}{3.161463in}}%
\pgfpathcurveto{\pgfqpoint{4.299515in}{3.374705in}}{\pgfqpoint{4.145283in}{3.560429in}}{\pgfqpoint{3.955175in}{3.703981in}}%
\pgfpathcurveto{\pgfqpoint{3.765067in}{3.847533in}}{\pgfqpoint{3.544218in}{3.945035in}}{\pgfqpoint{3.310053in}{3.988794in}}%
\pgfpathcurveto{\pgfqpoint{3.075889in}{4.032554in}}{\pgfqpoint{2.834733in}{4.021389in}}{\pgfqpoint{2.605613in}{3.956180in}}%
\pgfpathlineto{\pgfqpoint{3.027163in}{2.475000in}}%
\pgfpathlineto{\pgfqpoint{4.567163in}{2.475000in}}%
\pgfpathlineto{\pgfqpoint{4.567163in}{2.475000in}}%
\pgfpathclose%
\pgfusepath{stroke,fill}%
\end{pgfscope}%
\begin{pgfscope}%
\pgfsetbuttcap%
\pgfsetmiterjoin%
\definecolor{currentfill}{rgb}{1.000000,1.000000,0.701961}%
\pgfsetfillcolor{currentfill}%
\pgfsetlinewidth{1.003750pt}%
\definecolor{currentstroke}{rgb}{1.000000,1.000000,1.000000}%
\pgfsetstrokecolor{currentstroke}%
\pgfsetdash{}{0pt}%
\pgfpathmoveto{\pgfqpoint{2.605613in}{3.956180in}}%
\pgfpathcurveto{\pgfqpoint{2.376493in}{3.890972in}}{\pgfqpoint{2.165597in}{3.773481in}}{\pgfqpoint{1.989567in}{3.612978in}}%
\pgfpathcurveto{\pgfqpoint{1.813536in}{3.452475in}}{\pgfqpoint{1.677124in}{3.253294in}}{\pgfqpoint{1.591094in}{3.031153in}}%
\pgfpathcurveto{\pgfqpoint{1.505064in}{2.809011in}}{\pgfqpoint{1.471740in}{2.569908in}}{\pgfqpoint{1.493751in}{2.332708in}}%
\pgfpathcurveto{\pgfqpoint{1.515762in}{2.095508in}}{\pgfqpoint{1.592513in}{1.866620in}}{\pgfqpoint{1.717949in}{1.664101in}}%
\pgfpathlineto{\pgfqpoint{3.027163in}{2.475000in}}%
\pgfpathlineto{\pgfqpoint{2.605613in}{3.956180in}}%
\pgfpathlineto{\pgfqpoint{2.605613in}{3.956180in}}%
\pgfpathclose%
\pgfusepath{stroke,fill}%
\end{pgfscope}%
\begin{pgfscope}%
\pgfsetbuttcap%
\pgfsetmiterjoin%
\definecolor{currentfill}{rgb}{0.745098,0.729412,0.854902}%
\pgfsetfillcolor{currentfill}%
\pgfsetlinewidth{1.003750pt}%
\definecolor{currentstroke}{rgb}{1.000000,1.000000,1.000000}%
\pgfsetstrokecolor{currentstroke}%
\pgfsetdash{}{0pt}%
\pgfpathmoveto{\pgfqpoint{1.717949in}{1.664101in}}%
\pgfpathcurveto{\pgfqpoint{1.843385in}{1.461582in}}{\pgfqpoint{2.014117in}{1.290903in}}{\pgfqpoint{2.216676in}{1.165531in}}%
\pgfpathcurveto{\pgfqpoint{2.419234in}{1.040158in}}{\pgfqpoint{2.648147in}{0.963479in}}{\pgfqpoint{2.885354in}{0.941543in}}%
\pgfpathcurveto{\pgfqpoint{3.122560in}{0.919607in}}{\pgfqpoint{3.361653in}{0.953006in}}{\pgfqpoint{3.583768in}{1.039106in}}%
\pgfpathcurveto{\pgfqpoint{3.805882in}{1.125206in}}{\pgfqpoint{4.005020in}{1.261680in}}{\pgfqpoint{4.165467in}{1.437762in}}%
\pgfpathlineto{\pgfqpoint{3.027163in}{2.475000in}}%
\pgfpathlineto{\pgfqpoint{1.717949in}{1.664101in}}%
\pgfpathlineto{\pgfqpoint{1.717949in}{1.664101in}}%
\pgfpathclose%
\pgfusepath{stroke,fill}%
\end{pgfscope}%
\begin{pgfscope}%
\pgfsetbuttcap%
\pgfsetmiterjoin%
\definecolor{currentfill}{rgb}{0.984314,0.501961,0.447059}%
\pgfsetfillcolor{currentfill}%
\pgfsetlinewidth{1.003750pt}%
\definecolor{currentstroke}{rgb}{1.000000,1.000000,1.000000}%
\pgfsetstrokecolor{currentstroke}%
\pgfsetdash{}{0pt}%
\pgfpathmoveto{\pgfqpoint{4.165467in}{1.437762in}}%
\pgfpathcurveto{\pgfqpoint{4.165467in}{1.437762in}}{\pgfqpoint{4.165467in}{1.437762in}}{\pgfqpoint{4.165467in}{1.437762in}}%
\pgfpathlineto{\pgfqpoint{3.027163in}{2.475000in}}%
\pgfpathlineto{\pgfqpoint{4.165467in}{1.437762in}}%
\pgfpathlineto{\pgfqpoint{4.165467in}{1.437762in}}%
\pgfpathclose%
\pgfusepath{stroke,fill}%
\end{pgfscope}%
\begin{pgfscope}%
\pgfsetbuttcap%
\pgfsetmiterjoin%
\definecolor{currentfill}{rgb}{0.501961,0.694118,0.827451}%
\pgfsetfillcolor{currentfill}%
\pgfsetlinewidth{1.003750pt}%
\definecolor{currentstroke}{rgb}{1.000000,1.000000,1.000000}%
\pgfsetstrokecolor{currentstroke}%
\pgfsetdash{}{0pt}%
\pgfpathmoveto{\pgfqpoint{4.165467in}{1.437762in}}%
\pgfpathcurveto{\pgfqpoint{4.293579in}{1.578356in}}{\pgfqpoint{4.394541in}{1.741476in}}{\pgfqpoint{4.463232in}{1.918848in}}%
\pgfpathcurveto{\pgfqpoint{4.531924in}{2.096219in}}{\pgfqpoint{4.567163in}{2.284792in}}{\pgfqpoint{4.567163in}{2.475001in}}%
\pgfpathlineto{\pgfqpoint{3.027163in}{2.475000in}}%
\pgfpathlineto{\pgfqpoint{4.165467in}{1.437762in}}%
\pgfpathlineto{\pgfqpoint{4.165467in}{1.437762in}}%
\pgfpathclose%
\pgfusepath{stroke,fill}%
\end{pgfscope}%
\begin{pgfscope}%
\pgfsetbuttcap%
\pgfsetmiterjoin%
\definecolor{currentfill}{rgb}{0.992157,0.705882,0.384314}%
\pgfsetfillcolor{currentfill}%
\pgfsetlinewidth{1.003750pt}%
\definecolor{currentstroke}{rgb}{1.000000,1.000000,1.000000}%
\pgfsetstrokecolor{currentstroke}%
\pgfsetdash{}{0pt}%
\pgfpathmoveto{\pgfqpoint{4.567163in}{2.475001in}}%
\pgfpathcurveto{\pgfqpoint{4.567163in}{2.475001in}}{\pgfqpoint{4.567163in}{2.475001in}}{\pgfqpoint{4.567163in}{2.475001in}}%
\pgfpathlineto{\pgfqpoint{3.027163in}{2.475000in}}%
\pgfpathlineto{\pgfqpoint{4.567163in}{2.475001in}}%
\pgfpathlineto{\pgfqpoint{4.567163in}{2.475001in}}%
\pgfpathclose%
\pgfusepath{stroke,fill}%
\end{pgfscope}%
\begin{pgfscope}%
\definecolor{textcolor}{rgb}{0.150000,0.150000,0.150000}%
\pgfsetstrokecolor{textcolor}%
\pgfsetfillcolor{textcolor}%
\pgftext[x=3.583970in,y=3.212389in,,]{\color{textcolor}\sffamily\fontsize{12.000000}{14.400000}\selectfont 29.4\%}%
\end{pgfscope}%
\begin{pgfscope}%
\definecolor{textcolor}{rgb}{0.150000,0.150000,0.150000}%
\pgfsetstrokecolor{textcolor}%
\pgfsetfillcolor{textcolor}%
\pgftext[x=2.165522in,y=2.808692in,,]{\color{textcolor}\sffamily\fontsize{12.000000}{14.400000}\selectfont 29.4\%}%
\end{pgfscope}%
\begin{pgfscope}%
\definecolor{textcolor}{rgb}{0.150000,0.150000,0.150000}%
\pgfsetstrokecolor{textcolor}%
\pgfsetfillcolor{textcolor}%
\pgftext[x=2.942078in,y=1.554926in,,]{\color{textcolor}\sffamily\fontsize{12.000000}{14.400000}\selectfont 29.4\%}%
\end{pgfscope}%
\begin{pgfscope}%
\definecolor{textcolor}{rgb}{0.150000,0.150000,0.150000}%
\pgfsetstrokecolor{textcolor}%
\pgfsetfillcolor{textcolor}%
\pgftext[x=3.710146in,y=1.852657in,,]{\color{textcolor}\sffamily\fontsize{12.000000}{14.400000}\selectfont 0.0\%}%
\end{pgfscope}%
\begin{pgfscope}%
\definecolor{textcolor}{rgb}{0.150000,0.150000,0.150000}%
\pgfsetstrokecolor{textcolor}%
\pgfsetfillcolor{textcolor}%
\pgftext[x=3.888805in,y=2.141309in,,]{\color{textcolor}\sffamily\fontsize{12.000000}{14.400000}\selectfont 11.8\%}%
\end{pgfscope}%
\begin{pgfscope}%
\definecolor{textcolor}{rgb}{0.150000,0.150000,0.150000}%
\pgfsetstrokecolor{textcolor}%
\pgfsetfillcolor{textcolor}%
\pgftext[x=3.951163in,y=2.475000in,,]{\color{textcolor}\sffamily\fontsize{12.000000}{14.400000}\selectfont 0.0\%}%
\end{pgfscope}%
\begin{pgfscope}%
\pgfsetbuttcap%
\pgfsetmiterjoin%
\definecolor{currentfill}{rgb}{1.000000,1.000000,1.000000}%
\pgfsetfillcolor{currentfill}%
\pgfsetfillopacity{0.800000}%
\pgfsetlinewidth{1.003750pt}%
\definecolor{currentstroke}{rgb}{0.800000,0.800000,0.800000}%
\pgfsetstrokecolor{currentstroke}%
\pgfsetstrokeopacity{0.800000}%
\pgfsetdash{}{0pt}%
\pgfpathmoveto{\pgfqpoint{3.271524in}{0.458500in}}%
\pgfpathlineto{\pgfqpoint{5.827163in}{0.458500in}}%
\pgfpathquadraticcurveto{\pgfqpoint{5.852163in}{0.458500in}}{\pgfqpoint{5.852163in}{0.483500in}}%
\pgfpathlineto{\pgfqpoint{5.852163in}{1.571829in}}%
\pgfpathquadraticcurveto{\pgfqpoint{5.852163in}{1.596829in}}{\pgfqpoint{5.827163in}{1.596829in}}%
\pgfpathlineto{\pgfqpoint{3.271524in}{1.596829in}}%
\pgfpathquadraticcurveto{\pgfqpoint{3.246524in}{1.596829in}}{\pgfqpoint{3.246524in}{1.571829in}}%
\pgfpathlineto{\pgfqpoint{3.246524in}{0.483500in}}%
\pgfpathquadraticcurveto{\pgfqpoint{3.246524in}{0.458500in}}{\pgfqpoint{3.271524in}{0.458500in}}%
\pgfpathlineto{\pgfqpoint{3.271524in}{0.458500in}}%
\pgfpathclose%
\pgfusepath{stroke,fill}%
\end{pgfscope}%
\begin{pgfscope}%
\pgfsetbuttcap%
\pgfsetmiterjoin%
\definecolor{currentfill}{rgb}{0.552941,0.827451,0.780392}%
\pgfsetfillcolor{currentfill}%
\pgfsetlinewidth{1.003750pt}%
\definecolor{currentstroke}{rgb}{1.000000,1.000000,1.000000}%
\pgfsetstrokecolor{currentstroke}%
\pgfsetdash{}{0pt}%
\pgfpathmoveto{\pgfqpoint{3.296524in}{1.451858in}}%
\pgfpathlineto{\pgfqpoint{3.546524in}{1.451858in}}%
\pgfpathlineto{\pgfqpoint{3.546524in}{1.539358in}}%
\pgfpathlineto{\pgfqpoint{3.296524in}{1.539358in}}%
\pgfpathlineto{\pgfqpoint{3.296524in}{1.451858in}}%
\pgfpathclose%
\pgfusepath{stroke,fill}%
\end{pgfscope}%
\begin{pgfscope}%
\definecolor{textcolor}{rgb}{0.150000,0.150000,0.150000}%
\pgfsetstrokecolor{textcolor}%
\pgfsetfillcolor{textcolor}%
\pgftext[x=3.646524in,y=1.451858in,left,base]{\color{textcolor}\sffamily\fontsize{9.000000}{10.800000}\selectfont No CS Degree, 29.4 \%}%
\end{pgfscope}%
\begin{pgfscope}%
\pgfsetbuttcap%
\pgfsetmiterjoin%
\definecolor{currentfill}{rgb}{1.000000,1.000000,0.701961}%
\pgfsetfillcolor{currentfill}%
\pgfsetlinewidth{1.003750pt}%
\definecolor{currentstroke}{rgb}{1.000000,1.000000,1.000000}%
\pgfsetstrokecolor{currentstroke}%
\pgfsetdash{}{0pt}%
\pgfpathmoveto{\pgfqpoint{3.296524in}{1.268387in}}%
\pgfpathlineto{\pgfqpoint{3.546524in}{1.268387in}}%
\pgfpathlineto{\pgfqpoint{3.546524in}{1.355887in}}%
\pgfpathlineto{\pgfqpoint{3.296524in}{1.355887in}}%
\pgfpathlineto{\pgfqpoint{3.296524in}{1.268387in}}%
\pgfpathclose%
\pgfusepath{stroke,fill}%
\end{pgfscope}%
\begin{pgfscope}%
\definecolor{textcolor}{rgb}{0.150000,0.150000,0.150000}%
\pgfsetstrokecolor{textcolor}%
\pgfsetfillcolor{textcolor}%
\pgftext[x=3.646524in,y=1.268387in,left,base]{\color{textcolor}\sffamily\fontsize{9.000000}{10.800000}\selectfont Bachelors's Degree, 29.4 \%}%
\end{pgfscope}%
\begin{pgfscope}%
\pgfsetbuttcap%
\pgfsetmiterjoin%
\definecolor{currentfill}{rgb}{0.745098,0.729412,0.854902}%
\pgfsetfillcolor{currentfill}%
\pgfsetlinewidth{1.003750pt}%
\definecolor{currentstroke}{rgb}{1.000000,1.000000,1.000000}%
\pgfsetstrokecolor{currentstroke}%
\pgfsetdash{}{0pt}%
\pgfpathmoveto{\pgfqpoint{3.296524in}{1.084915in}}%
\pgfpathlineto{\pgfqpoint{3.546524in}{1.084915in}}%
\pgfpathlineto{\pgfqpoint{3.546524in}{1.172415in}}%
\pgfpathlineto{\pgfqpoint{3.296524in}{1.172415in}}%
\pgfpathlineto{\pgfqpoint{3.296524in}{1.084915in}}%
\pgfpathclose%
\pgfusepath{stroke,fill}%
\end{pgfscope}%
\begin{pgfscope}%
\definecolor{textcolor}{rgb}{0.150000,0.150000,0.150000}%
\pgfsetstrokecolor{textcolor}%
\pgfsetfillcolor{textcolor}%
\pgftext[x=3.646524in,y=1.084915in,left,base]{\color{textcolor}\sffamily\fontsize{9.000000}{10.800000}\selectfont Master's Degree, 29.4 \%}%
\end{pgfscope}%
\begin{pgfscope}%
\pgfsetbuttcap%
\pgfsetmiterjoin%
\definecolor{currentfill}{rgb}{0.984314,0.501961,0.447059}%
\pgfsetfillcolor{currentfill}%
\pgfsetlinewidth{1.003750pt}%
\definecolor{currentstroke}{rgb}{1.000000,1.000000,1.000000}%
\pgfsetstrokecolor{currentstroke}%
\pgfsetdash{}{0pt}%
\pgfpathmoveto{\pgfqpoint{3.296524in}{0.901444in}}%
\pgfpathlineto{\pgfqpoint{3.546524in}{0.901444in}}%
\pgfpathlineto{\pgfqpoint{3.546524in}{0.988944in}}%
\pgfpathlineto{\pgfqpoint{3.296524in}{0.988944in}}%
\pgfpathlineto{\pgfqpoint{3.296524in}{0.901444in}}%
\pgfpathclose%
\pgfusepath{stroke,fill}%
\end{pgfscope}%
\begin{pgfscope}%
\definecolor{textcolor}{rgb}{0.150000,0.150000,0.150000}%
\pgfsetstrokecolor{textcolor}%
\pgfsetfillcolor{textcolor}%
\pgftext[x=3.646524in,y=0.901444in,left,base]{\color{textcolor}\sffamily\fontsize{9.000000}{10.800000}\selectfont Doctorate Degree, 0.0 \%}%
\end{pgfscope}%
\begin{pgfscope}%
\pgfsetbuttcap%
\pgfsetmiterjoin%
\definecolor{currentfill}{rgb}{0.501961,0.694118,0.827451}%
\pgfsetfillcolor{currentfill}%
\pgfsetlinewidth{1.003750pt}%
\definecolor{currentstroke}{rgb}{1.000000,1.000000,1.000000}%
\pgfsetstrokecolor{currentstroke}%
\pgfsetdash{}{0pt}%
\pgfpathmoveto{\pgfqpoint{3.296524in}{0.717972in}}%
\pgfpathlineto{\pgfqpoint{3.546524in}{0.717972in}}%
\pgfpathlineto{\pgfqpoint{3.546524in}{0.805472in}}%
\pgfpathlineto{\pgfqpoint{3.296524in}{0.805472in}}%
\pgfpathlineto{\pgfqpoint{3.296524in}{0.717972in}}%
\pgfpathclose%
\pgfusepath{stroke,fill}%
\end{pgfscope}%
\begin{pgfscope}%
\definecolor{textcolor}{rgb}{0.150000,0.150000,0.150000}%
\pgfsetstrokecolor{textcolor}%
\pgfsetfillcolor{textcolor}%
\pgftext[x=3.646524in,y=0.717972in,left,base]{\color{textcolor}\sffamily\fontsize{9.000000}{10.800000}\selectfont Other Engineering Degree, 11.8 \%}%
\end{pgfscope}%
\begin{pgfscope}%
\pgfsetbuttcap%
\pgfsetmiterjoin%
\definecolor{currentfill}{rgb}{0.992157,0.705882,0.384314}%
\pgfsetfillcolor{currentfill}%
\pgfsetlinewidth{1.003750pt}%
\definecolor{currentstroke}{rgb}{1.000000,1.000000,1.000000}%
\pgfsetstrokecolor{currentstroke}%
\pgfsetdash{}{0pt}%
\pgfpathmoveto{\pgfqpoint{3.296524in}{0.534501in}}%
\pgfpathlineto{\pgfqpoint{3.546524in}{0.534501in}}%
\pgfpathlineto{\pgfqpoint{3.546524in}{0.622001in}}%
\pgfpathlineto{\pgfqpoint{3.296524in}{0.622001in}}%
\pgfpathlineto{\pgfqpoint{3.296524in}{0.534501in}}%
\pgfpathclose%
\pgfusepath{stroke,fill}%
\end{pgfscope}%
\begin{pgfscope}%
\definecolor{textcolor}{rgb}{0.150000,0.150000,0.150000}%
\pgfsetstrokecolor{textcolor}%
\pgfsetfillcolor{textcolor}%
\pgftext[x=3.646524in,y=0.534501in,left,base]{\color{textcolor}\sffamily\fontsize{9.000000}{10.800000}\selectfont Yes but did not finish, 0.0 \%}%
\end{pgfscope}%
\end{pgfpicture}%
\makeatother%
\endgroup%
}
	\caption{University level Computer Science experience amongst study participants.}
	\label{fig:uniexp}
\end{figure}

\begin{figure}[H]
	\centering
	\begin{minipage}{0.45\textwidth}
		\centering
		\scalebox{0.7}{%% Creator: Matplotlib, PGF backend
%%
%% To include the figure in your LaTeX document, write
%%   \input{<filename>.pgf}
%%
%% Make sure the required packages are loaded in your preamble
%%   \usepackage{pgf}
%%
%% Also ensure that all the required font packages are loaded; for instance,
%% the lmodern package is sometimes necessary when using math font.
%%   \usepackage{lmodern}
%%
%% Figures using additional raster images can only be included by \input if
%% they are in the same directory as the main LaTeX file. For loading figures
%% from other directories you can use the `import` package
%%   \usepackage{import}
%%
%% and then include the figures with
%%   \import{<path to file>}{<filename>.pgf}
%%
%% Matplotlib used the following preamble
%%   \usepackage{fontspec}
%%   \setmainfont{DejaVuSerif.ttf}[Path=\detokenize{/home/spam/miniconda3/envs/mpl/lib/python3.10/site-packages/matplotlib/mpl-data/fonts/ttf/}]
%%   \setsansfont{DejaVuSans.ttf}[Path=\detokenize{/home/spam/miniconda3/envs/mpl/lib/python3.10/site-packages/matplotlib/mpl-data/fonts/ttf/}]
%%   \setmonofont{DejaVuSansMono.ttf}[Path=\detokenize{/home/spam/miniconda3/envs/mpl/lib/python3.10/site-packages/matplotlib/mpl-data/fonts/ttf/}]
%%
\begingroup%
\makeatletter%
\begin{pgfpicture}%
\pgfpathrectangle{\pgfpointorigin}{\pgfqpoint{5.906660in}{5.000000in}}%
\pgfusepath{use as bounding box, clip}%
\begin{pgfscope}%
\pgfsetbuttcap%
\pgfsetmiterjoin%
\definecolor{currentfill}{rgb}{1.000000,1.000000,1.000000}%
\pgfsetfillcolor{currentfill}%
\pgfsetlinewidth{0.000000pt}%
\definecolor{currentstroke}{rgb}{1.000000,1.000000,1.000000}%
\pgfsetstrokecolor{currentstroke}%
\pgfsetdash{}{0pt}%
\pgfpathmoveto{\pgfqpoint{0.000000in}{0.000000in}}%
\pgfpathlineto{\pgfqpoint{5.906660in}{0.000000in}}%
\pgfpathlineto{\pgfqpoint{5.906660in}{5.000000in}}%
\pgfpathlineto{\pgfqpoint{0.000000in}{5.000000in}}%
\pgfpathlineto{\pgfqpoint{0.000000in}{0.000000in}}%
\pgfpathclose%
\pgfusepath{fill}%
\end{pgfscope}%
\begin{pgfscope}%
\definecolor{textcolor}{rgb}{0.150000,0.150000,0.150000}%
\pgfsetstrokecolor{textcolor}%
\pgfsetfillcolor{textcolor}%
\pgftext[x=3.027163in,y=0.411111in,,top]{\color{textcolor}\sffamily\fontsize{12.000000}{14.400000}\selectfont Computer Science UniversityUniverity Experience}%
\end{pgfscope}%
\begin{pgfscope}%
\pgfsetbuttcap%
\pgfsetmiterjoin%
\definecolor{currentfill}{rgb}{0.552941,0.827451,0.780392}%
\pgfsetfillcolor{currentfill}%
\pgfsetlinewidth{1.003750pt}%
\definecolor{currentstroke}{rgb}{1.000000,1.000000,1.000000}%
\pgfsetstrokecolor{currentstroke}%
\pgfsetdash{}{0pt}%
\pgfpathmoveto{\pgfqpoint{4.567163in}{2.475000in}}%
\pgfpathcurveto{\pgfqpoint{4.567163in}{2.713219in}}{\pgfqpoint{4.511889in}{2.948220in}}{\pgfqpoint{4.405702in}{3.161463in}}%
\pgfpathcurveto{\pgfqpoint{4.299515in}{3.374705in}}{\pgfqpoint{4.145283in}{3.560429in}}{\pgfqpoint{3.955175in}{3.703981in}}%
\pgfpathcurveto{\pgfqpoint{3.765067in}{3.847533in}}{\pgfqpoint{3.544218in}{3.945035in}}{\pgfqpoint{3.310053in}{3.988794in}}%
\pgfpathcurveto{\pgfqpoint{3.075889in}{4.032554in}}{\pgfqpoint{2.834733in}{4.021389in}}{\pgfqpoint{2.605613in}{3.956180in}}%
\pgfpathlineto{\pgfqpoint{3.027163in}{2.475000in}}%
\pgfpathlineto{\pgfqpoint{4.567163in}{2.475000in}}%
\pgfpathlineto{\pgfqpoint{4.567163in}{2.475000in}}%
\pgfpathclose%
\pgfusepath{stroke,fill}%
\end{pgfscope}%
\begin{pgfscope}%
\pgfsetbuttcap%
\pgfsetmiterjoin%
\definecolor{currentfill}{rgb}{1.000000,1.000000,0.701961}%
\pgfsetfillcolor{currentfill}%
\pgfsetlinewidth{1.003750pt}%
\definecolor{currentstroke}{rgb}{1.000000,1.000000,1.000000}%
\pgfsetstrokecolor{currentstroke}%
\pgfsetdash{}{0pt}%
\pgfpathmoveto{\pgfqpoint{2.605613in}{3.956180in}}%
\pgfpathcurveto{\pgfqpoint{2.376493in}{3.890972in}}{\pgfqpoint{2.165597in}{3.773481in}}{\pgfqpoint{1.989567in}{3.612978in}}%
\pgfpathcurveto{\pgfqpoint{1.813536in}{3.452475in}}{\pgfqpoint{1.677124in}{3.253294in}}{\pgfqpoint{1.591094in}{3.031153in}}%
\pgfpathcurveto{\pgfqpoint{1.505064in}{2.809011in}}{\pgfqpoint{1.471740in}{2.569908in}}{\pgfqpoint{1.493751in}{2.332708in}}%
\pgfpathcurveto{\pgfqpoint{1.515762in}{2.095508in}}{\pgfqpoint{1.592513in}{1.866620in}}{\pgfqpoint{1.717949in}{1.664101in}}%
\pgfpathlineto{\pgfqpoint{3.027163in}{2.475000in}}%
\pgfpathlineto{\pgfqpoint{2.605613in}{3.956180in}}%
\pgfpathlineto{\pgfqpoint{2.605613in}{3.956180in}}%
\pgfpathclose%
\pgfusepath{stroke,fill}%
\end{pgfscope}%
\begin{pgfscope}%
\pgfsetbuttcap%
\pgfsetmiterjoin%
\definecolor{currentfill}{rgb}{0.745098,0.729412,0.854902}%
\pgfsetfillcolor{currentfill}%
\pgfsetlinewidth{1.003750pt}%
\definecolor{currentstroke}{rgb}{1.000000,1.000000,1.000000}%
\pgfsetstrokecolor{currentstroke}%
\pgfsetdash{}{0pt}%
\pgfpathmoveto{\pgfqpoint{1.717949in}{1.664101in}}%
\pgfpathcurveto{\pgfqpoint{1.843385in}{1.461582in}}{\pgfqpoint{2.014117in}{1.290903in}}{\pgfqpoint{2.216676in}{1.165531in}}%
\pgfpathcurveto{\pgfqpoint{2.419234in}{1.040158in}}{\pgfqpoint{2.648147in}{0.963479in}}{\pgfqpoint{2.885354in}{0.941543in}}%
\pgfpathcurveto{\pgfqpoint{3.122560in}{0.919607in}}{\pgfqpoint{3.361653in}{0.953006in}}{\pgfqpoint{3.583768in}{1.039106in}}%
\pgfpathcurveto{\pgfqpoint{3.805882in}{1.125206in}}{\pgfqpoint{4.005020in}{1.261680in}}{\pgfqpoint{4.165467in}{1.437762in}}%
\pgfpathlineto{\pgfqpoint{3.027163in}{2.475000in}}%
\pgfpathlineto{\pgfqpoint{1.717949in}{1.664101in}}%
\pgfpathlineto{\pgfqpoint{1.717949in}{1.664101in}}%
\pgfpathclose%
\pgfusepath{stroke,fill}%
\end{pgfscope}%
\begin{pgfscope}%
\pgfsetbuttcap%
\pgfsetmiterjoin%
\definecolor{currentfill}{rgb}{0.984314,0.501961,0.447059}%
\pgfsetfillcolor{currentfill}%
\pgfsetlinewidth{1.003750pt}%
\definecolor{currentstroke}{rgb}{1.000000,1.000000,1.000000}%
\pgfsetstrokecolor{currentstroke}%
\pgfsetdash{}{0pt}%
\pgfpathmoveto{\pgfqpoint{4.165467in}{1.437762in}}%
\pgfpathcurveto{\pgfqpoint{4.165467in}{1.437762in}}{\pgfqpoint{4.165467in}{1.437762in}}{\pgfqpoint{4.165467in}{1.437762in}}%
\pgfpathlineto{\pgfqpoint{3.027163in}{2.475000in}}%
\pgfpathlineto{\pgfqpoint{4.165467in}{1.437762in}}%
\pgfpathlineto{\pgfqpoint{4.165467in}{1.437762in}}%
\pgfpathclose%
\pgfusepath{stroke,fill}%
\end{pgfscope}%
\begin{pgfscope}%
\pgfsetbuttcap%
\pgfsetmiterjoin%
\definecolor{currentfill}{rgb}{0.501961,0.694118,0.827451}%
\pgfsetfillcolor{currentfill}%
\pgfsetlinewidth{1.003750pt}%
\definecolor{currentstroke}{rgb}{1.000000,1.000000,1.000000}%
\pgfsetstrokecolor{currentstroke}%
\pgfsetdash{}{0pt}%
\pgfpathmoveto{\pgfqpoint{4.165467in}{1.437762in}}%
\pgfpathcurveto{\pgfqpoint{4.293579in}{1.578356in}}{\pgfqpoint{4.394541in}{1.741476in}}{\pgfqpoint{4.463232in}{1.918848in}}%
\pgfpathcurveto{\pgfqpoint{4.531924in}{2.096219in}}{\pgfqpoint{4.567163in}{2.284792in}}{\pgfqpoint{4.567163in}{2.475001in}}%
\pgfpathlineto{\pgfqpoint{3.027163in}{2.475000in}}%
\pgfpathlineto{\pgfqpoint{4.165467in}{1.437762in}}%
\pgfpathlineto{\pgfqpoint{4.165467in}{1.437762in}}%
\pgfpathclose%
\pgfusepath{stroke,fill}%
\end{pgfscope}%
\begin{pgfscope}%
\pgfsetbuttcap%
\pgfsetmiterjoin%
\definecolor{currentfill}{rgb}{0.992157,0.705882,0.384314}%
\pgfsetfillcolor{currentfill}%
\pgfsetlinewidth{1.003750pt}%
\definecolor{currentstroke}{rgb}{1.000000,1.000000,1.000000}%
\pgfsetstrokecolor{currentstroke}%
\pgfsetdash{}{0pt}%
\pgfpathmoveto{\pgfqpoint{4.567163in}{2.475001in}}%
\pgfpathcurveto{\pgfqpoint{4.567163in}{2.475001in}}{\pgfqpoint{4.567163in}{2.475001in}}{\pgfqpoint{4.567163in}{2.475001in}}%
\pgfpathlineto{\pgfqpoint{3.027163in}{2.475000in}}%
\pgfpathlineto{\pgfqpoint{4.567163in}{2.475001in}}%
\pgfpathlineto{\pgfqpoint{4.567163in}{2.475001in}}%
\pgfpathclose%
\pgfusepath{stroke,fill}%
\end{pgfscope}%
\begin{pgfscope}%
\definecolor{textcolor}{rgb}{0.150000,0.150000,0.150000}%
\pgfsetstrokecolor{textcolor}%
\pgfsetfillcolor{textcolor}%
\pgftext[x=3.583970in,y=3.212389in,,]{\color{textcolor}\sffamily\fontsize{12.000000}{14.400000}\selectfont 29.4\%}%
\end{pgfscope}%
\begin{pgfscope}%
\definecolor{textcolor}{rgb}{0.150000,0.150000,0.150000}%
\pgfsetstrokecolor{textcolor}%
\pgfsetfillcolor{textcolor}%
\pgftext[x=2.165522in,y=2.808692in,,]{\color{textcolor}\sffamily\fontsize{12.000000}{14.400000}\selectfont 29.4\%}%
\end{pgfscope}%
\begin{pgfscope}%
\definecolor{textcolor}{rgb}{0.150000,0.150000,0.150000}%
\pgfsetstrokecolor{textcolor}%
\pgfsetfillcolor{textcolor}%
\pgftext[x=2.942078in,y=1.554926in,,]{\color{textcolor}\sffamily\fontsize{12.000000}{14.400000}\selectfont 29.4\%}%
\end{pgfscope}%
\begin{pgfscope}%
\definecolor{textcolor}{rgb}{0.150000,0.150000,0.150000}%
\pgfsetstrokecolor{textcolor}%
\pgfsetfillcolor{textcolor}%
\pgftext[x=3.710146in,y=1.852657in,,]{\color{textcolor}\sffamily\fontsize{12.000000}{14.400000}\selectfont 0.0\%}%
\end{pgfscope}%
\begin{pgfscope}%
\definecolor{textcolor}{rgb}{0.150000,0.150000,0.150000}%
\pgfsetstrokecolor{textcolor}%
\pgfsetfillcolor{textcolor}%
\pgftext[x=3.888805in,y=2.141309in,,]{\color{textcolor}\sffamily\fontsize{12.000000}{14.400000}\selectfont 11.8\%}%
\end{pgfscope}%
\begin{pgfscope}%
\definecolor{textcolor}{rgb}{0.150000,0.150000,0.150000}%
\pgfsetstrokecolor{textcolor}%
\pgfsetfillcolor{textcolor}%
\pgftext[x=3.951163in,y=2.475000in,,]{\color{textcolor}\sffamily\fontsize{12.000000}{14.400000}\selectfont 0.0\%}%
\end{pgfscope}%
\begin{pgfscope}%
\pgfsetbuttcap%
\pgfsetmiterjoin%
\definecolor{currentfill}{rgb}{1.000000,1.000000,1.000000}%
\pgfsetfillcolor{currentfill}%
\pgfsetfillopacity{0.800000}%
\pgfsetlinewidth{1.003750pt}%
\definecolor{currentstroke}{rgb}{0.800000,0.800000,0.800000}%
\pgfsetstrokecolor{currentstroke}%
\pgfsetstrokeopacity{0.800000}%
\pgfsetdash{}{0pt}%
\pgfpathmoveto{\pgfqpoint{3.271524in}{0.458500in}}%
\pgfpathlineto{\pgfqpoint{5.827163in}{0.458500in}}%
\pgfpathquadraticcurveto{\pgfqpoint{5.852163in}{0.458500in}}{\pgfqpoint{5.852163in}{0.483500in}}%
\pgfpathlineto{\pgfqpoint{5.852163in}{1.571829in}}%
\pgfpathquadraticcurveto{\pgfqpoint{5.852163in}{1.596829in}}{\pgfqpoint{5.827163in}{1.596829in}}%
\pgfpathlineto{\pgfqpoint{3.271524in}{1.596829in}}%
\pgfpathquadraticcurveto{\pgfqpoint{3.246524in}{1.596829in}}{\pgfqpoint{3.246524in}{1.571829in}}%
\pgfpathlineto{\pgfqpoint{3.246524in}{0.483500in}}%
\pgfpathquadraticcurveto{\pgfqpoint{3.246524in}{0.458500in}}{\pgfqpoint{3.271524in}{0.458500in}}%
\pgfpathlineto{\pgfqpoint{3.271524in}{0.458500in}}%
\pgfpathclose%
\pgfusepath{stroke,fill}%
\end{pgfscope}%
\begin{pgfscope}%
\pgfsetbuttcap%
\pgfsetmiterjoin%
\definecolor{currentfill}{rgb}{0.552941,0.827451,0.780392}%
\pgfsetfillcolor{currentfill}%
\pgfsetlinewidth{1.003750pt}%
\definecolor{currentstroke}{rgb}{1.000000,1.000000,1.000000}%
\pgfsetstrokecolor{currentstroke}%
\pgfsetdash{}{0pt}%
\pgfpathmoveto{\pgfqpoint{3.296524in}{1.451858in}}%
\pgfpathlineto{\pgfqpoint{3.546524in}{1.451858in}}%
\pgfpathlineto{\pgfqpoint{3.546524in}{1.539358in}}%
\pgfpathlineto{\pgfqpoint{3.296524in}{1.539358in}}%
\pgfpathlineto{\pgfqpoint{3.296524in}{1.451858in}}%
\pgfpathclose%
\pgfusepath{stroke,fill}%
\end{pgfscope}%
\begin{pgfscope}%
\definecolor{textcolor}{rgb}{0.150000,0.150000,0.150000}%
\pgfsetstrokecolor{textcolor}%
\pgfsetfillcolor{textcolor}%
\pgftext[x=3.646524in,y=1.451858in,left,base]{\color{textcolor}\sffamily\fontsize{9.000000}{10.800000}\selectfont No CS Degree, 29.4 \%}%
\end{pgfscope}%
\begin{pgfscope}%
\pgfsetbuttcap%
\pgfsetmiterjoin%
\definecolor{currentfill}{rgb}{1.000000,1.000000,0.701961}%
\pgfsetfillcolor{currentfill}%
\pgfsetlinewidth{1.003750pt}%
\definecolor{currentstroke}{rgb}{1.000000,1.000000,1.000000}%
\pgfsetstrokecolor{currentstroke}%
\pgfsetdash{}{0pt}%
\pgfpathmoveto{\pgfqpoint{3.296524in}{1.268387in}}%
\pgfpathlineto{\pgfqpoint{3.546524in}{1.268387in}}%
\pgfpathlineto{\pgfqpoint{3.546524in}{1.355887in}}%
\pgfpathlineto{\pgfqpoint{3.296524in}{1.355887in}}%
\pgfpathlineto{\pgfqpoint{3.296524in}{1.268387in}}%
\pgfpathclose%
\pgfusepath{stroke,fill}%
\end{pgfscope}%
\begin{pgfscope}%
\definecolor{textcolor}{rgb}{0.150000,0.150000,0.150000}%
\pgfsetstrokecolor{textcolor}%
\pgfsetfillcolor{textcolor}%
\pgftext[x=3.646524in,y=1.268387in,left,base]{\color{textcolor}\sffamily\fontsize{9.000000}{10.800000}\selectfont Bachelors's Degree, 29.4 \%}%
\end{pgfscope}%
\begin{pgfscope}%
\pgfsetbuttcap%
\pgfsetmiterjoin%
\definecolor{currentfill}{rgb}{0.745098,0.729412,0.854902}%
\pgfsetfillcolor{currentfill}%
\pgfsetlinewidth{1.003750pt}%
\definecolor{currentstroke}{rgb}{1.000000,1.000000,1.000000}%
\pgfsetstrokecolor{currentstroke}%
\pgfsetdash{}{0pt}%
\pgfpathmoveto{\pgfqpoint{3.296524in}{1.084915in}}%
\pgfpathlineto{\pgfqpoint{3.546524in}{1.084915in}}%
\pgfpathlineto{\pgfqpoint{3.546524in}{1.172415in}}%
\pgfpathlineto{\pgfqpoint{3.296524in}{1.172415in}}%
\pgfpathlineto{\pgfqpoint{3.296524in}{1.084915in}}%
\pgfpathclose%
\pgfusepath{stroke,fill}%
\end{pgfscope}%
\begin{pgfscope}%
\definecolor{textcolor}{rgb}{0.150000,0.150000,0.150000}%
\pgfsetstrokecolor{textcolor}%
\pgfsetfillcolor{textcolor}%
\pgftext[x=3.646524in,y=1.084915in,left,base]{\color{textcolor}\sffamily\fontsize{9.000000}{10.800000}\selectfont Master's Degree, 29.4 \%}%
\end{pgfscope}%
\begin{pgfscope}%
\pgfsetbuttcap%
\pgfsetmiterjoin%
\definecolor{currentfill}{rgb}{0.984314,0.501961,0.447059}%
\pgfsetfillcolor{currentfill}%
\pgfsetlinewidth{1.003750pt}%
\definecolor{currentstroke}{rgb}{1.000000,1.000000,1.000000}%
\pgfsetstrokecolor{currentstroke}%
\pgfsetdash{}{0pt}%
\pgfpathmoveto{\pgfqpoint{3.296524in}{0.901444in}}%
\pgfpathlineto{\pgfqpoint{3.546524in}{0.901444in}}%
\pgfpathlineto{\pgfqpoint{3.546524in}{0.988944in}}%
\pgfpathlineto{\pgfqpoint{3.296524in}{0.988944in}}%
\pgfpathlineto{\pgfqpoint{3.296524in}{0.901444in}}%
\pgfpathclose%
\pgfusepath{stroke,fill}%
\end{pgfscope}%
\begin{pgfscope}%
\definecolor{textcolor}{rgb}{0.150000,0.150000,0.150000}%
\pgfsetstrokecolor{textcolor}%
\pgfsetfillcolor{textcolor}%
\pgftext[x=3.646524in,y=0.901444in,left,base]{\color{textcolor}\sffamily\fontsize{9.000000}{10.800000}\selectfont Doctorate Degree, 0.0 \%}%
\end{pgfscope}%
\begin{pgfscope}%
\pgfsetbuttcap%
\pgfsetmiterjoin%
\definecolor{currentfill}{rgb}{0.501961,0.694118,0.827451}%
\pgfsetfillcolor{currentfill}%
\pgfsetlinewidth{1.003750pt}%
\definecolor{currentstroke}{rgb}{1.000000,1.000000,1.000000}%
\pgfsetstrokecolor{currentstroke}%
\pgfsetdash{}{0pt}%
\pgfpathmoveto{\pgfqpoint{3.296524in}{0.717972in}}%
\pgfpathlineto{\pgfqpoint{3.546524in}{0.717972in}}%
\pgfpathlineto{\pgfqpoint{3.546524in}{0.805472in}}%
\pgfpathlineto{\pgfqpoint{3.296524in}{0.805472in}}%
\pgfpathlineto{\pgfqpoint{3.296524in}{0.717972in}}%
\pgfpathclose%
\pgfusepath{stroke,fill}%
\end{pgfscope}%
\begin{pgfscope}%
\definecolor{textcolor}{rgb}{0.150000,0.150000,0.150000}%
\pgfsetstrokecolor{textcolor}%
\pgfsetfillcolor{textcolor}%
\pgftext[x=3.646524in,y=0.717972in,left,base]{\color{textcolor}\sffamily\fontsize{9.000000}{10.800000}\selectfont Other Engineering Degree, 11.8 \%}%
\end{pgfscope}%
\begin{pgfscope}%
\pgfsetbuttcap%
\pgfsetmiterjoin%
\definecolor{currentfill}{rgb}{0.992157,0.705882,0.384314}%
\pgfsetfillcolor{currentfill}%
\pgfsetlinewidth{1.003750pt}%
\definecolor{currentstroke}{rgb}{1.000000,1.000000,1.000000}%
\pgfsetstrokecolor{currentstroke}%
\pgfsetdash{}{0pt}%
\pgfpathmoveto{\pgfqpoint{3.296524in}{0.534501in}}%
\pgfpathlineto{\pgfqpoint{3.546524in}{0.534501in}}%
\pgfpathlineto{\pgfqpoint{3.546524in}{0.622001in}}%
\pgfpathlineto{\pgfqpoint{3.296524in}{0.622001in}}%
\pgfpathlineto{\pgfqpoint{3.296524in}{0.534501in}}%
\pgfpathclose%
\pgfusepath{stroke,fill}%
\end{pgfscope}%
\begin{pgfscope}%
\definecolor{textcolor}{rgb}{0.150000,0.150000,0.150000}%
\pgfsetstrokecolor{textcolor}%
\pgfsetfillcolor{textcolor}%
\pgftext[x=3.646524in,y=0.534501in,left,base]{\color{textcolor}\sffamily\fontsize{9.000000}{10.800000}\selectfont Yes but did not finish, 0.0 \%}%
\end{pgfscope}%
\end{pgfpicture}%
\makeatother%
\endgroup%
}
		\caption{first figure}
	\end{minipage}\hspace{-1em}
	\begin{minipage}{0.45\textwidth}
		\centering
		\scalebox{0.7}{%% Creator: Matplotlib, PGF backend
%%
%% To include the figure in your LaTeX document, write
%%   \input{<filename>.pgf}
%%
%% Make sure the required packages are loaded in your preamble
%%   \usepackage{pgf}
%%
%% Also ensure that all the required font packages are loaded; for instance,
%% the lmodern package is sometimes necessary when using math font.
%%   \usepackage{lmodern}
%%
%% Figures using additional raster images can only be included by \input if
%% they are in the same directory as the main LaTeX file. For loading figures
%% from other directories you can use the `import` package
%%   \usepackage{import}
%%
%% and then include the figures with
%%   \import{<path to file>}{<filename>.pgf}
%%
%% Matplotlib used the following preamble
%%   \usepackage{fontspec}
%%   \setmainfont{DejaVuSerif.ttf}[Path=\detokenize{/home/spam/miniconda3/envs/mpl/lib/python3.10/site-packages/matplotlib/mpl-data/fonts/ttf/}]
%%   \setsansfont{DejaVuSans.ttf}[Path=\detokenize{/home/spam/miniconda3/envs/mpl/lib/python3.10/site-packages/matplotlib/mpl-data/fonts/ttf/}]
%%   \setmonofont{DejaVuSansMono.ttf}[Path=\detokenize{/home/spam/miniconda3/envs/mpl/lib/python3.10/site-packages/matplotlib/mpl-data/fonts/ttf/}]
%%
\begingroup%
\makeatletter%
\begin{pgfpicture}%
\pgfpathrectangle{\pgfpointorigin}{\pgfqpoint{5.906660in}{5.000000in}}%
\pgfusepath{use as bounding box, clip}%
\begin{pgfscope}%
\pgfsetbuttcap%
\pgfsetmiterjoin%
\definecolor{currentfill}{rgb}{1.000000,1.000000,1.000000}%
\pgfsetfillcolor{currentfill}%
\pgfsetlinewidth{0.000000pt}%
\definecolor{currentstroke}{rgb}{1.000000,1.000000,1.000000}%
\pgfsetstrokecolor{currentstroke}%
\pgfsetdash{}{0pt}%
\pgfpathmoveto{\pgfqpoint{0.000000in}{0.000000in}}%
\pgfpathlineto{\pgfqpoint{5.906660in}{0.000000in}}%
\pgfpathlineto{\pgfqpoint{5.906660in}{5.000000in}}%
\pgfpathlineto{\pgfqpoint{0.000000in}{5.000000in}}%
\pgfpathlineto{\pgfqpoint{0.000000in}{0.000000in}}%
\pgfpathclose%
\pgfusepath{fill}%
\end{pgfscope}%
\begin{pgfscope}%
\definecolor{textcolor}{rgb}{0.150000,0.150000,0.150000}%
\pgfsetstrokecolor{textcolor}%
\pgfsetfillcolor{textcolor}%
\pgftext[x=3.027163in,y=0.411111in,,top]{\color{textcolor}\sffamily\fontsize{12.000000}{14.400000}\selectfont Computer Science UniversityUniverity Experience}%
\end{pgfscope}%
\begin{pgfscope}%
\pgfsetbuttcap%
\pgfsetmiterjoin%
\definecolor{currentfill}{rgb}{0.552941,0.827451,0.780392}%
\pgfsetfillcolor{currentfill}%
\pgfsetlinewidth{1.003750pt}%
\definecolor{currentstroke}{rgb}{1.000000,1.000000,1.000000}%
\pgfsetstrokecolor{currentstroke}%
\pgfsetdash{}{0pt}%
\pgfpathmoveto{\pgfqpoint{4.567163in}{2.475000in}}%
\pgfpathcurveto{\pgfqpoint{4.567163in}{2.713219in}}{\pgfqpoint{4.511889in}{2.948220in}}{\pgfqpoint{4.405702in}{3.161463in}}%
\pgfpathcurveto{\pgfqpoint{4.299515in}{3.374705in}}{\pgfqpoint{4.145283in}{3.560429in}}{\pgfqpoint{3.955175in}{3.703981in}}%
\pgfpathcurveto{\pgfqpoint{3.765067in}{3.847533in}}{\pgfqpoint{3.544218in}{3.945035in}}{\pgfqpoint{3.310053in}{3.988794in}}%
\pgfpathcurveto{\pgfqpoint{3.075889in}{4.032554in}}{\pgfqpoint{2.834733in}{4.021389in}}{\pgfqpoint{2.605613in}{3.956180in}}%
\pgfpathlineto{\pgfqpoint{3.027163in}{2.475000in}}%
\pgfpathlineto{\pgfqpoint{4.567163in}{2.475000in}}%
\pgfpathlineto{\pgfqpoint{4.567163in}{2.475000in}}%
\pgfpathclose%
\pgfusepath{stroke,fill}%
\end{pgfscope}%
\begin{pgfscope}%
\pgfsetbuttcap%
\pgfsetmiterjoin%
\definecolor{currentfill}{rgb}{1.000000,1.000000,0.701961}%
\pgfsetfillcolor{currentfill}%
\pgfsetlinewidth{1.003750pt}%
\definecolor{currentstroke}{rgb}{1.000000,1.000000,1.000000}%
\pgfsetstrokecolor{currentstroke}%
\pgfsetdash{}{0pt}%
\pgfpathmoveto{\pgfqpoint{2.605613in}{3.956180in}}%
\pgfpathcurveto{\pgfqpoint{2.376493in}{3.890972in}}{\pgfqpoint{2.165597in}{3.773481in}}{\pgfqpoint{1.989567in}{3.612978in}}%
\pgfpathcurveto{\pgfqpoint{1.813536in}{3.452475in}}{\pgfqpoint{1.677124in}{3.253294in}}{\pgfqpoint{1.591094in}{3.031153in}}%
\pgfpathcurveto{\pgfqpoint{1.505064in}{2.809011in}}{\pgfqpoint{1.471740in}{2.569908in}}{\pgfqpoint{1.493751in}{2.332708in}}%
\pgfpathcurveto{\pgfqpoint{1.515762in}{2.095508in}}{\pgfqpoint{1.592513in}{1.866620in}}{\pgfqpoint{1.717949in}{1.664101in}}%
\pgfpathlineto{\pgfqpoint{3.027163in}{2.475000in}}%
\pgfpathlineto{\pgfqpoint{2.605613in}{3.956180in}}%
\pgfpathlineto{\pgfqpoint{2.605613in}{3.956180in}}%
\pgfpathclose%
\pgfusepath{stroke,fill}%
\end{pgfscope}%
\begin{pgfscope}%
\pgfsetbuttcap%
\pgfsetmiterjoin%
\definecolor{currentfill}{rgb}{0.745098,0.729412,0.854902}%
\pgfsetfillcolor{currentfill}%
\pgfsetlinewidth{1.003750pt}%
\definecolor{currentstroke}{rgb}{1.000000,1.000000,1.000000}%
\pgfsetstrokecolor{currentstroke}%
\pgfsetdash{}{0pt}%
\pgfpathmoveto{\pgfqpoint{1.717949in}{1.664101in}}%
\pgfpathcurveto{\pgfqpoint{1.843385in}{1.461582in}}{\pgfqpoint{2.014117in}{1.290903in}}{\pgfqpoint{2.216676in}{1.165531in}}%
\pgfpathcurveto{\pgfqpoint{2.419234in}{1.040158in}}{\pgfqpoint{2.648147in}{0.963479in}}{\pgfqpoint{2.885354in}{0.941543in}}%
\pgfpathcurveto{\pgfqpoint{3.122560in}{0.919607in}}{\pgfqpoint{3.361653in}{0.953006in}}{\pgfqpoint{3.583768in}{1.039106in}}%
\pgfpathcurveto{\pgfqpoint{3.805882in}{1.125206in}}{\pgfqpoint{4.005020in}{1.261680in}}{\pgfqpoint{4.165467in}{1.437762in}}%
\pgfpathlineto{\pgfqpoint{3.027163in}{2.475000in}}%
\pgfpathlineto{\pgfqpoint{1.717949in}{1.664101in}}%
\pgfpathlineto{\pgfqpoint{1.717949in}{1.664101in}}%
\pgfpathclose%
\pgfusepath{stroke,fill}%
\end{pgfscope}%
\begin{pgfscope}%
\pgfsetbuttcap%
\pgfsetmiterjoin%
\definecolor{currentfill}{rgb}{0.984314,0.501961,0.447059}%
\pgfsetfillcolor{currentfill}%
\pgfsetlinewidth{1.003750pt}%
\definecolor{currentstroke}{rgb}{1.000000,1.000000,1.000000}%
\pgfsetstrokecolor{currentstroke}%
\pgfsetdash{}{0pt}%
\pgfpathmoveto{\pgfqpoint{4.165467in}{1.437762in}}%
\pgfpathcurveto{\pgfqpoint{4.165467in}{1.437762in}}{\pgfqpoint{4.165467in}{1.437762in}}{\pgfqpoint{4.165467in}{1.437762in}}%
\pgfpathlineto{\pgfqpoint{3.027163in}{2.475000in}}%
\pgfpathlineto{\pgfqpoint{4.165467in}{1.437762in}}%
\pgfpathlineto{\pgfqpoint{4.165467in}{1.437762in}}%
\pgfpathclose%
\pgfusepath{stroke,fill}%
\end{pgfscope}%
\begin{pgfscope}%
\pgfsetbuttcap%
\pgfsetmiterjoin%
\definecolor{currentfill}{rgb}{0.501961,0.694118,0.827451}%
\pgfsetfillcolor{currentfill}%
\pgfsetlinewidth{1.003750pt}%
\definecolor{currentstroke}{rgb}{1.000000,1.000000,1.000000}%
\pgfsetstrokecolor{currentstroke}%
\pgfsetdash{}{0pt}%
\pgfpathmoveto{\pgfqpoint{4.165467in}{1.437762in}}%
\pgfpathcurveto{\pgfqpoint{4.293579in}{1.578356in}}{\pgfqpoint{4.394541in}{1.741476in}}{\pgfqpoint{4.463232in}{1.918848in}}%
\pgfpathcurveto{\pgfqpoint{4.531924in}{2.096219in}}{\pgfqpoint{4.567163in}{2.284792in}}{\pgfqpoint{4.567163in}{2.475001in}}%
\pgfpathlineto{\pgfqpoint{3.027163in}{2.475000in}}%
\pgfpathlineto{\pgfqpoint{4.165467in}{1.437762in}}%
\pgfpathlineto{\pgfqpoint{4.165467in}{1.437762in}}%
\pgfpathclose%
\pgfusepath{stroke,fill}%
\end{pgfscope}%
\begin{pgfscope}%
\pgfsetbuttcap%
\pgfsetmiterjoin%
\definecolor{currentfill}{rgb}{0.992157,0.705882,0.384314}%
\pgfsetfillcolor{currentfill}%
\pgfsetlinewidth{1.003750pt}%
\definecolor{currentstroke}{rgb}{1.000000,1.000000,1.000000}%
\pgfsetstrokecolor{currentstroke}%
\pgfsetdash{}{0pt}%
\pgfpathmoveto{\pgfqpoint{4.567163in}{2.475001in}}%
\pgfpathcurveto{\pgfqpoint{4.567163in}{2.475001in}}{\pgfqpoint{4.567163in}{2.475001in}}{\pgfqpoint{4.567163in}{2.475001in}}%
\pgfpathlineto{\pgfqpoint{3.027163in}{2.475000in}}%
\pgfpathlineto{\pgfqpoint{4.567163in}{2.475001in}}%
\pgfpathlineto{\pgfqpoint{4.567163in}{2.475001in}}%
\pgfpathclose%
\pgfusepath{stroke,fill}%
\end{pgfscope}%
\begin{pgfscope}%
\definecolor{textcolor}{rgb}{0.150000,0.150000,0.150000}%
\pgfsetstrokecolor{textcolor}%
\pgfsetfillcolor{textcolor}%
\pgftext[x=3.583970in,y=3.212389in,,]{\color{textcolor}\sffamily\fontsize{12.000000}{14.400000}\selectfont 29.4\%}%
\end{pgfscope}%
\begin{pgfscope}%
\definecolor{textcolor}{rgb}{0.150000,0.150000,0.150000}%
\pgfsetstrokecolor{textcolor}%
\pgfsetfillcolor{textcolor}%
\pgftext[x=2.165522in,y=2.808692in,,]{\color{textcolor}\sffamily\fontsize{12.000000}{14.400000}\selectfont 29.4\%}%
\end{pgfscope}%
\begin{pgfscope}%
\definecolor{textcolor}{rgb}{0.150000,0.150000,0.150000}%
\pgfsetstrokecolor{textcolor}%
\pgfsetfillcolor{textcolor}%
\pgftext[x=2.942078in,y=1.554926in,,]{\color{textcolor}\sffamily\fontsize{12.000000}{14.400000}\selectfont 29.4\%}%
\end{pgfscope}%
\begin{pgfscope}%
\definecolor{textcolor}{rgb}{0.150000,0.150000,0.150000}%
\pgfsetstrokecolor{textcolor}%
\pgfsetfillcolor{textcolor}%
\pgftext[x=3.710146in,y=1.852657in,,]{\color{textcolor}\sffamily\fontsize{12.000000}{14.400000}\selectfont 0.0\%}%
\end{pgfscope}%
\begin{pgfscope}%
\definecolor{textcolor}{rgb}{0.150000,0.150000,0.150000}%
\pgfsetstrokecolor{textcolor}%
\pgfsetfillcolor{textcolor}%
\pgftext[x=3.888805in,y=2.141309in,,]{\color{textcolor}\sffamily\fontsize{12.000000}{14.400000}\selectfont 11.8\%}%
\end{pgfscope}%
\begin{pgfscope}%
\definecolor{textcolor}{rgb}{0.150000,0.150000,0.150000}%
\pgfsetstrokecolor{textcolor}%
\pgfsetfillcolor{textcolor}%
\pgftext[x=3.951163in,y=2.475000in,,]{\color{textcolor}\sffamily\fontsize{12.000000}{14.400000}\selectfont 0.0\%}%
\end{pgfscope}%
\begin{pgfscope}%
\pgfsetbuttcap%
\pgfsetmiterjoin%
\definecolor{currentfill}{rgb}{1.000000,1.000000,1.000000}%
\pgfsetfillcolor{currentfill}%
\pgfsetfillopacity{0.800000}%
\pgfsetlinewidth{1.003750pt}%
\definecolor{currentstroke}{rgb}{0.800000,0.800000,0.800000}%
\pgfsetstrokecolor{currentstroke}%
\pgfsetstrokeopacity{0.800000}%
\pgfsetdash{}{0pt}%
\pgfpathmoveto{\pgfqpoint{3.271524in}{0.458500in}}%
\pgfpathlineto{\pgfqpoint{5.827163in}{0.458500in}}%
\pgfpathquadraticcurveto{\pgfqpoint{5.852163in}{0.458500in}}{\pgfqpoint{5.852163in}{0.483500in}}%
\pgfpathlineto{\pgfqpoint{5.852163in}{1.571829in}}%
\pgfpathquadraticcurveto{\pgfqpoint{5.852163in}{1.596829in}}{\pgfqpoint{5.827163in}{1.596829in}}%
\pgfpathlineto{\pgfqpoint{3.271524in}{1.596829in}}%
\pgfpathquadraticcurveto{\pgfqpoint{3.246524in}{1.596829in}}{\pgfqpoint{3.246524in}{1.571829in}}%
\pgfpathlineto{\pgfqpoint{3.246524in}{0.483500in}}%
\pgfpathquadraticcurveto{\pgfqpoint{3.246524in}{0.458500in}}{\pgfqpoint{3.271524in}{0.458500in}}%
\pgfpathlineto{\pgfqpoint{3.271524in}{0.458500in}}%
\pgfpathclose%
\pgfusepath{stroke,fill}%
\end{pgfscope}%
\begin{pgfscope}%
\pgfsetbuttcap%
\pgfsetmiterjoin%
\definecolor{currentfill}{rgb}{0.552941,0.827451,0.780392}%
\pgfsetfillcolor{currentfill}%
\pgfsetlinewidth{1.003750pt}%
\definecolor{currentstroke}{rgb}{1.000000,1.000000,1.000000}%
\pgfsetstrokecolor{currentstroke}%
\pgfsetdash{}{0pt}%
\pgfpathmoveto{\pgfqpoint{3.296524in}{1.451858in}}%
\pgfpathlineto{\pgfqpoint{3.546524in}{1.451858in}}%
\pgfpathlineto{\pgfqpoint{3.546524in}{1.539358in}}%
\pgfpathlineto{\pgfqpoint{3.296524in}{1.539358in}}%
\pgfpathlineto{\pgfqpoint{3.296524in}{1.451858in}}%
\pgfpathclose%
\pgfusepath{stroke,fill}%
\end{pgfscope}%
\begin{pgfscope}%
\definecolor{textcolor}{rgb}{0.150000,0.150000,0.150000}%
\pgfsetstrokecolor{textcolor}%
\pgfsetfillcolor{textcolor}%
\pgftext[x=3.646524in,y=1.451858in,left,base]{\color{textcolor}\sffamily\fontsize{9.000000}{10.800000}\selectfont No CS Degree, 29.4 \%}%
\end{pgfscope}%
\begin{pgfscope}%
\pgfsetbuttcap%
\pgfsetmiterjoin%
\definecolor{currentfill}{rgb}{1.000000,1.000000,0.701961}%
\pgfsetfillcolor{currentfill}%
\pgfsetlinewidth{1.003750pt}%
\definecolor{currentstroke}{rgb}{1.000000,1.000000,1.000000}%
\pgfsetstrokecolor{currentstroke}%
\pgfsetdash{}{0pt}%
\pgfpathmoveto{\pgfqpoint{3.296524in}{1.268387in}}%
\pgfpathlineto{\pgfqpoint{3.546524in}{1.268387in}}%
\pgfpathlineto{\pgfqpoint{3.546524in}{1.355887in}}%
\pgfpathlineto{\pgfqpoint{3.296524in}{1.355887in}}%
\pgfpathlineto{\pgfqpoint{3.296524in}{1.268387in}}%
\pgfpathclose%
\pgfusepath{stroke,fill}%
\end{pgfscope}%
\begin{pgfscope}%
\definecolor{textcolor}{rgb}{0.150000,0.150000,0.150000}%
\pgfsetstrokecolor{textcolor}%
\pgfsetfillcolor{textcolor}%
\pgftext[x=3.646524in,y=1.268387in,left,base]{\color{textcolor}\sffamily\fontsize{9.000000}{10.800000}\selectfont Bachelors's Degree, 29.4 \%}%
\end{pgfscope}%
\begin{pgfscope}%
\pgfsetbuttcap%
\pgfsetmiterjoin%
\definecolor{currentfill}{rgb}{0.745098,0.729412,0.854902}%
\pgfsetfillcolor{currentfill}%
\pgfsetlinewidth{1.003750pt}%
\definecolor{currentstroke}{rgb}{1.000000,1.000000,1.000000}%
\pgfsetstrokecolor{currentstroke}%
\pgfsetdash{}{0pt}%
\pgfpathmoveto{\pgfqpoint{3.296524in}{1.084915in}}%
\pgfpathlineto{\pgfqpoint{3.546524in}{1.084915in}}%
\pgfpathlineto{\pgfqpoint{3.546524in}{1.172415in}}%
\pgfpathlineto{\pgfqpoint{3.296524in}{1.172415in}}%
\pgfpathlineto{\pgfqpoint{3.296524in}{1.084915in}}%
\pgfpathclose%
\pgfusepath{stroke,fill}%
\end{pgfscope}%
\begin{pgfscope}%
\definecolor{textcolor}{rgb}{0.150000,0.150000,0.150000}%
\pgfsetstrokecolor{textcolor}%
\pgfsetfillcolor{textcolor}%
\pgftext[x=3.646524in,y=1.084915in,left,base]{\color{textcolor}\sffamily\fontsize{9.000000}{10.800000}\selectfont Master's Degree, 29.4 \%}%
\end{pgfscope}%
\begin{pgfscope}%
\pgfsetbuttcap%
\pgfsetmiterjoin%
\definecolor{currentfill}{rgb}{0.984314,0.501961,0.447059}%
\pgfsetfillcolor{currentfill}%
\pgfsetlinewidth{1.003750pt}%
\definecolor{currentstroke}{rgb}{1.000000,1.000000,1.000000}%
\pgfsetstrokecolor{currentstroke}%
\pgfsetdash{}{0pt}%
\pgfpathmoveto{\pgfqpoint{3.296524in}{0.901444in}}%
\pgfpathlineto{\pgfqpoint{3.546524in}{0.901444in}}%
\pgfpathlineto{\pgfqpoint{3.546524in}{0.988944in}}%
\pgfpathlineto{\pgfqpoint{3.296524in}{0.988944in}}%
\pgfpathlineto{\pgfqpoint{3.296524in}{0.901444in}}%
\pgfpathclose%
\pgfusepath{stroke,fill}%
\end{pgfscope}%
\begin{pgfscope}%
\definecolor{textcolor}{rgb}{0.150000,0.150000,0.150000}%
\pgfsetstrokecolor{textcolor}%
\pgfsetfillcolor{textcolor}%
\pgftext[x=3.646524in,y=0.901444in,left,base]{\color{textcolor}\sffamily\fontsize{9.000000}{10.800000}\selectfont Doctorate Degree, 0.0 \%}%
\end{pgfscope}%
\begin{pgfscope}%
\pgfsetbuttcap%
\pgfsetmiterjoin%
\definecolor{currentfill}{rgb}{0.501961,0.694118,0.827451}%
\pgfsetfillcolor{currentfill}%
\pgfsetlinewidth{1.003750pt}%
\definecolor{currentstroke}{rgb}{1.000000,1.000000,1.000000}%
\pgfsetstrokecolor{currentstroke}%
\pgfsetdash{}{0pt}%
\pgfpathmoveto{\pgfqpoint{3.296524in}{0.717972in}}%
\pgfpathlineto{\pgfqpoint{3.546524in}{0.717972in}}%
\pgfpathlineto{\pgfqpoint{3.546524in}{0.805472in}}%
\pgfpathlineto{\pgfqpoint{3.296524in}{0.805472in}}%
\pgfpathlineto{\pgfqpoint{3.296524in}{0.717972in}}%
\pgfpathclose%
\pgfusepath{stroke,fill}%
\end{pgfscope}%
\begin{pgfscope}%
\definecolor{textcolor}{rgb}{0.150000,0.150000,0.150000}%
\pgfsetstrokecolor{textcolor}%
\pgfsetfillcolor{textcolor}%
\pgftext[x=3.646524in,y=0.717972in,left,base]{\color{textcolor}\sffamily\fontsize{9.000000}{10.800000}\selectfont Other Engineering Degree, 11.8 \%}%
\end{pgfscope}%
\begin{pgfscope}%
\pgfsetbuttcap%
\pgfsetmiterjoin%
\definecolor{currentfill}{rgb}{0.992157,0.705882,0.384314}%
\pgfsetfillcolor{currentfill}%
\pgfsetlinewidth{1.003750pt}%
\definecolor{currentstroke}{rgb}{1.000000,1.000000,1.000000}%
\pgfsetstrokecolor{currentstroke}%
\pgfsetdash{}{0pt}%
\pgfpathmoveto{\pgfqpoint{3.296524in}{0.534501in}}%
\pgfpathlineto{\pgfqpoint{3.546524in}{0.534501in}}%
\pgfpathlineto{\pgfqpoint{3.546524in}{0.622001in}}%
\pgfpathlineto{\pgfqpoint{3.296524in}{0.622001in}}%
\pgfpathlineto{\pgfqpoint{3.296524in}{0.534501in}}%
\pgfpathclose%
\pgfusepath{stroke,fill}%
\end{pgfscope}%
\begin{pgfscope}%
\definecolor{textcolor}{rgb}{0.150000,0.150000,0.150000}%
\pgfsetstrokecolor{textcolor}%
\pgfsetfillcolor{textcolor}%
\pgftext[x=3.646524in,y=0.534501in,left,base]{\color{textcolor}\sffamily\fontsize{9.000000}{10.800000}\selectfont Yes but did not finish, 0.0 \%}%
\end{pgfscope}%
\end{pgfpicture}%
\makeatother%
\endgroup%
}
		\caption{second figure}
	\end{minipage}
\end{figure}



\begin{figure}[H]
	\scalebox{0.72}{%% Creator: Matplotlib, PGF backend
%%
%% To include the figure in your LaTeX document, write
%%   \input{<filename>.pgf}
%%
%% Make sure the required packages are loaded in your preamble
%%   \usepackage{pgf}
%%
%% Also ensure that all the required font packages are loaded; for instance,
%% the lmodern package is sometimes necessary when using math font.
%%   \usepackage{lmodern}
%%
%% Figures using additional raster images can only be included by \input if
%% they are in the same directory as the main LaTeX file. For loading figures
%% from other directories you can use the `import` package
%%   \usepackage{import}
%%
%% and then include the figures with
%%   \import{<path to file>}{<filename>.pgf}
%%
%% Matplotlib used the following preamble
%%   \usepackage{fontspec}
%%   \setmainfont{DejaVuSerif.ttf}[Path=\detokenize{/home/spam/miniconda3/envs/mpl/lib/python3.10/site-packages/matplotlib/mpl-data/fonts/ttf/}]
%%   \setsansfont{DejaVuSans.ttf}[Path=\detokenize{/home/spam/miniconda3/envs/mpl/lib/python3.10/site-packages/matplotlib/mpl-data/fonts/ttf/}]
%%   \setmonofont{DejaVuSansMono.ttf}[Path=\detokenize{/home/spam/miniconda3/envs/mpl/lib/python3.10/site-packages/matplotlib/mpl-data/fonts/ttf/}]
%%
\begingroup%
\makeatletter%
\begin{pgfpicture}%
\pgfpathrectangle{\pgfpointorigin}{\pgfqpoint{5.906660in}{5.000000in}}%
\pgfusepath{use as bounding box, clip}%
\begin{pgfscope}%
\pgfsetbuttcap%
\pgfsetmiterjoin%
\definecolor{currentfill}{rgb}{1.000000,1.000000,1.000000}%
\pgfsetfillcolor{currentfill}%
\pgfsetlinewidth{0.000000pt}%
\definecolor{currentstroke}{rgb}{1.000000,1.000000,1.000000}%
\pgfsetstrokecolor{currentstroke}%
\pgfsetdash{}{0pt}%
\pgfpathmoveto{\pgfqpoint{0.000000in}{0.000000in}}%
\pgfpathlineto{\pgfqpoint{5.906660in}{0.000000in}}%
\pgfpathlineto{\pgfqpoint{5.906660in}{5.000000in}}%
\pgfpathlineto{\pgfqpoint{0.000000in}{5.000000in}}%
\pgfpathlineto{\pgfqpoint{0.000000in}{0.000000in}}%
\pgfpathclose%
\pgfusepath{fill}%
\end{pgfscope}%
\begin{pgfscope}%
\definecolor{textcolor}{rgb}{0.150000,0.150000,0.150000}%
\pgfsetstrokecolor{textcolor}%
\pgfsetfillcolor{textcolor}%
\pgftext[x=3.027163in,y=0.411111in,,top]{\color{textcolor}\sffamily\fontsize{12.000000}{14.400000}\selectfont Computer Science UniversityUniverity Experience}%
\end{pgfscope}%
\begin{pgfscope}%
\pgfsetbuttcap%
\pgfsetmiterjoin%
\definecolor{currentfill}{rgb}{0.552941,0.827451,0.780392}%
\pgfsetfillcolor{currentfill}%
\pgfsetlinewidth{1.003750pt}%
\definecolor{currentstroke}{rgb}{1.000000,1.000000,1.000000}%
\pgfsetstrokecolor{currentstroke}%
\pgfsetdash{}{0pt}%
\pgfpathmoveto{\pgfqpoint{4.567163in}{2.475000in}}%
\pgfpathcurveto{\pgfqpoint{4.567163in}{2.713219in}}{\pgfqpoint{4.511889in}{2.948220in}}{\pgfqpoint{4.405702in}{3.161463in}}%
\pgfpathcurveto{\pgfqpoint{4.299515in}{3.374705in}}{\pgfqpoint{4.145283in}{3.560429in}}{\pgfqpoint{3.955175in}{3.703981in}}%
\pgfpathcurveto{\pgfqpoint{3.765067in}{3.847533in}}{\pgfqpoint{3.544218in}{3.945035in}}{\pgfqpoint{3.310053in}{3.988794in}}%
\pgfpathcurveto{\pgfqpoint{3.075889in}{4.032554in}}{\pgfqpoint{2.834733in}{4.021389in}}{\pgfqpoint{2.605613in}{3.956180in}}%
\pgfpathlineto{\pgfqpoint{3.027163in}{2.475000in}}%
\pgfpathlineto{\pgfqpoint{4.567163in}{2.475000in}}%
\pgfpathlineto{\pgfqpoint{4.567163in}{2.475000in}}%
\pgfpathclose%
\pgfusepath{stroke,fill}%
\end{pgfscope}%
\begin{pgfscope}%
\pgfsetbuttcap%
\pgfsetmiterjoin%
\definecolor{currentfill}{rgb}{1.000000,1.000000,0.701961}%
\pgfsetfillcolor{currentfill}%
\pgfsetlinewidth{1.003750pt}%
\definecolor{currentstroke}{rgb}{1.000000,1.000000,1.000000}%
\pgfsetstrokecolor{currentstroke}%
\pgfsetdash{}{0pt}%
\pgfpathmoveto{\pgfqpoint{2.605613in}{3.956180in}}%
\pgfpathcurveto{\pgfqpoint{2.376493in}{3.890972in}}{\pgfqpoint{2.165597in}{3.773481in}}{\pgfqpoint{1.989567in}{3.612978in}}%
\pgfpathcurveto{\pgfqpoint{1.813536in}{3.452475in}}{\pgfqpoint{1.677124in}{3.253294in}}{\pgfqpoint{1.591094in}{3.031153in}}%
\pgfpathcurveto{\pgfqpoint{1.505064in}{2.809011in}}{\pgfqpoint{1.471740in}{2.569908in}}{\pgfqpoint{1.493751in}{2.332708in}}%
\pgfpathcurveto{\pgfqpoint{1.515762in}{2.095508in}}{\pgfqpoint{1.592513in}{1.866620in}}{\pgfqpoint{1.717949in}{1.664101in}}%
\pgfpathlineto{\pgfqpoint{3.027163in}{2.475000in}}%
\pgfpathlineto{\pgfqpoint{2.605613in}{3.956180in}}%
\pgfpathlineto{\pgfqpoint{2.605613in}{3.956180in}}%
\pgfpathclose%
\pgfusepath{stroke,fill}%
\end{pgfscope}%
\begin{pgfscope}%
\pgfsetbuttcap%
\pgfsetmiterjoin%
\definecolor{currentfill}{rgb}{0.745098,0.729412,0.854902}%
\pgfsetfillcolor{currentfill}%
\pgfsetlinewidth{1.003750pt}%
\definecolor{currentstroke}{rgb}{1.000000,1.000000,1.000000}%
\pgfsetstrokecolor{currentstroke}%
\pgfsetdash{}{0pt}%
\pgfpathmoveto{\pgfqpoint{1.717949in}{1.664101in}}%
\pgfpathcurveto{\pgfqpoint{1.843385in}{1.461582in}}{\pgfqpoint{2.014117in}{1.290903in}}{\pgfqpoint{2.216676in}{1.165531in}}%
\pgfpathcurveto{\pgfqpoint{2.419234in}{1.040158in}}{\pgfqpoint{2.648147in}{0.963479in}}{\pgfqpoint{2.885354in}{0.941543in}}%
\pgfpathcurveto{\pgfqpoint{3.122560in}{0.919607in}}{\pgfqpoint{3.361653in}{0.953006in}}{\pgfqpoint{3.583768in}{1.039106in}}%
\pgfpathcurveto{\pgfqpoint{3.805882in}{1.125206in}}{\pgfqpoint{4.005020in}{1.261680in}}{\pgfqpoint{4.165467in}{1.437762in}}%
\pgfpathlineto{\pgfqpoint{3.027163in}{2.475000in}}%
\pgfpathlineto{\pgfqpoint{1.717949in}{1.664101in}}%
\pgfpathlineto{\pgfqpoint{1.717949in}{1.664101in}}%
\pgfpathclose%
\pgfusepath{stroke,fill}%
\end{pgfscope}%
\begin{pgfscope}%
\pgfsetbuttcap%
\pgfsetmiterjoin%
\definecolor{currentfill}{rgb}{0.984314,0.501961,0.447059}%
\pgfsetfillcolor{currentfill}%
\pgfsetlinewidth{1.003750pt}%
\definecolor{currentstroke}{rgb}{1.000000,1.000000,1.000000}%
\pgfsetstrokecolor{currentstroke}%
\pgfsetdash{}{0pt}%
\pgfpathmoveto{\pgfqpoint{4.165467in}{1.437762in}}%
\pgfpathcurveto{\pgfqpoint{4.165467in}{1.437762in}}{\pgfqpoint{4.165467in}{1.437762in}}{\pgfqpoint{4.165467in}{1.437762in}}%
\pgfpathlineto{\pgfqpoint{3.027163in}{2.475000in}}%
\pgfpathlineto{\pgfqpoint{4.165467in}{1.437762in}}%
\pgfpathlineto{\pgfqpoint{4.165467in}{1.437762in}}%
\pgfpathclose%
\pgfusepath{stroke,fill}%
\end{pgfscope}%
\begin{pgfscope}%
\pgfsetbuttcap%
\pgfsetmiterjoin%
\definecolor{currentfill}{rgb}{0.501961,0.694118,0.827451}%
\pgfsetfillcolor{currentfill}%
\pgfsetlinewidth{1.003750pt}%
\definecolor{currentstroke}{rgb}{1.000000,1.000000,1.000000}%
\pgfsetstrokecolor{currentstroke}%
\pgfsetdash{}{0pt}%
\pgfpathmoveto{\pgfqpoint{4.165467in}{1.437762in}}%
\pgfpathcurveto{\pgfqpoint{4.293579in}{1.578356in}}{\pgfqpoint{4.394541in}{1.741476in}}{\pgfqpoint{4.463232in}{1.918848in}}%
\pgfpathcurveto{\pgfqpoint{4.531924in}{2.096219in}}{\pgfqpoint{4.567163in}{2.284792in}}{\pgfqpoint{4.567163in}{2.475001in}}%
\pgfpathlineto{\pgfqpoint{3.027163in}{2.475000in}}%
\pgfpathlineto{\pgfqpoint{4.165467in}{1.437762in}}%
\pgfpathlineto{\pgfqpoint{4.165467in}{1.437762in}}%
\pgfpathclose%
\pgfusepath{stroke,fill}%
\end{pgfscope}%
\begin{pgfscope}%
\pgfsetbuttcap%
\pgfsetmiterjoin%
\definecolor{currentfill}{rgb}{0.992157,0.705882,0.384314}%
\pgfsetfillcolor{currentfill}%
\pgfsetlinewidth{1.003750pt}%
\definecolor{currentstroke}{rgb}{1.000000,1.000000,1.000000}%
\pgfsetstrokecolor{currentstroke}%
\pgfsetdash{}{0pt}%
\pgfpathmoveto{\pgfqpoint{4.567163in}{2.475001in}}%
\pgfpathcurveto{\pgfqpoint{4.567163in}{2.475001in}}{\pgfqpoint{4.567163in}{2.475001in}}{\pgfqpoint{4.567163in}{2.475001in}}%
\pgfpathlineto{\pgfqpoint{3.027163in}{2.475000in}}%
\pgfpathlineto{\pgfqpoint{4.567163in}{2.475001in}}%
\pgfpathlineto{\pgfqpoint{4.567163in}{2.475001in}}%
\pgfpathclose%
\pgfusepath{stroke,fill}%
\end{pgfscope}%
\begin{pgfscope}%
\definecolor{textcolor}{rgb}{0.150000,0.150000,0.150000}%
\pgfsetstrokecolor{textcolor}%
\pgfsetfillcolor{textcolor}%
\pgftext[x=3.583970in,y=3.212389in,,]{\color{textcolor}\sffamily\fontsize{12.000000}{14.400000}\selectfont 29.4\%}%
\end{pgfscope}%
\begin{pgfscope}%
\definecolor{textcolor}{rgb}{0.150000,0.150000,0.150000}%
\pgfsetstrokecolor{textcolor}%
\pgfsetfillcolor{textcolor}%
\pgftext[x=2.165522in,y=2.808692in,,]{\color{textcolor}\sffamily\fontsize{12.000000}{14.400000}\selectfont 29.4\%}%
\end{pgfscope}%
\begin{pgfscope}%
\definecolor{textcolor}{rgb}{0.150000,0.150000,0.150000}%
\pgfsetstrokecolor{textcolor}%
\pgfsetfillcolor{textcolor}%
\pgftext[x=2.942078in,y=1.554926in,,]{\color{textcolor}\sffamily\fontsize{12.000000}{14.400000}\selectfont 29.4\%}%
\end{pgfscope}%
\begin{pgfscope}%
\definecolor{textcolor}{rgb}{0.150000,0.150000,0.150000}%
\pgfsetstrokecolor{textcolor}%
\pgfsetfillcolor{textcolor}%
\pgftext[x=3.710146in,y=1.852657in,,]{\color{textcolor}\sffamily\fontsize{12.000000}{14.400000}\selectfont 0.0\%}%
\end{pgfscope}%
\begin{pgfscope}%
\definecolor{textcolor}{rgb}{0.150000,0.150000,0.150000}%
\pgfsetstrokecolor{textcolor}%
\pgfsetfillcolor{textcolor}%
\pgftext[x=3.888805in,y=2.141309in,,]{\color{textcolor}\sffamily\fontsize{12.000000}{14.400000}\selectfont 11.8\%}%
\end{pgfscope}%
\begin{pgfscope}%
\definecolor{textcolor}{rgb}{0.150000,0.150000,0.150000}%
\pgfsetstrokecolor{textcolor}%
\pgfsetfillcolor{textcolor}%
\pgftext[x=3.951163in,y=2.475000in,,]{\color{textcolor}\sffamily\fontsize{12.000000}{14.400000}\selectfont 0.0\%}%
\end{pgfscope}%
\begin{pgfscope}%
\pgfsetbuttcap%
\pgfsetmiterjoin%
\definecolor{currentfill}{rgb}{1.000000,1.000000,1.000000}%
\pgfsetfillcolor{currentfill}%
\pgfsetfillopacity{0.800000}%
\pgfsetlinewidth{1.003750pt}%
\definecolor{currentstroke}{rgb}{0.800000,0.800000,0.800000}%
\pgfsetstrokecolor{currentstroke}%
\pgfsetstrokeopacity{0.800000}%
\pgfsetdash{}{0pt}%
\pgfpathmoveto{\pgfqpoint{3.271524in}{0.458500in}}%
\pgfpathlineto{\pgfqpoint{5.827163in}{0.458500in}}%
\pgfpathquadraticcurveto{\pgfqpoint{5.852163in}{0.458500in}}{\pgfqpoint{5.852163in}{0.483500in}}%
\pgfpathlineto{\pgfqpoint{5.852163in}{1.571829in}}%
\pgfpathquadraticcurveto{\pgfqpoint{5.852163in}{1.596829in}}{\pgfqpoint{5.827163in}{1.596829in}}%
\pgfpathlineto{\pgfqpoint{3.271524in}{1.596829in}}%
\pgfpathquadraticcurveto{\pgfqpoint{3.246524in}{1.596829in}}{\pgfqpoint{3.246524in}{1.571829in}}%
\pgfpathlineto{\pgfqpoint{3.246524in}{0.483500in}}%
\pgfpathquadraticcurveto{\pgfqpoint{3.246524in}{0.458500in}}{\pgfqpoint{3.271524in}{0.458500in}}%
\pgfpathlineto{\pgfqpoint{3.271524in}{0.458500in}}%
\pgfpathclose%
\pgfusepath{stroke,fill}%
\end{pgfscope}%
\begin{pgfscope}%
\pgfsetbuttcap%
\pgfsetmiterjoin%
\definecolor{currentfill}{rgb}{0.552941,0.827451,0.780392}%
\pgfsetfillcolor{currentfill}%
\pgfsetlinewidth{1.003750pt}%
\definecolor{currentstroke}{rgb}{1.000000,1.000000,1.000000}%
\pgfsetstrokecolor{currentstroke}%
\pgfsetdash{}{0pt}%
\pgfpathmoveto{\pgfqpoint{3.296524in}{1.451858in}}%
\pgfpathlineto{\pgfqpoint{3.546524in}{1.451858in}}%
\pgfpathlineto{\pgfqpoint{3.546524in}{1.539358in}}%
\pgfpathlineto{\pgfqpoint{3.296524in}{1.539358in}}%
\pgfpathlineto{\pgfqpoint{3.296524in}{1.451858in}}%
\pgfpathclose%
\pgfusepath{stroke,fill}%
\end{pgfscope}%
\begin{pgfscope}%
\definecolor{textcolor}{rgb}{0.150000,0.150000,0.150000}%
\pgfsetstrokecolor{textcolor}%
\pgfsetfillcolor{textcolor}%
\pgftext[x=3.646524in,y=1.451858in,left,base]{\color{textcolor}\sffamily\fontsize{9.000000}{10.800000}\selectfont No CS Degree, 29.4 \%}%
\end{pgfscope}%
\begin{pgfscope}%
\pgfsetbuttcap%
\pgfsetmiterjoin%
\definecolor{currentfill}{rgb}{1.000000,1.000000,0.701961}%
\pgfsetfillcolor{currentfill}%
\pgfsetlinewidth{1.003750pt}%
\definecolor{currentstroke}{rgb}{1.000000,1.000000,1.000000}%
\pgfsetstrokecolor{currentstroke}%
\pgfsetdash{}{0pt}%
\pgfpathmoveto{\pgfqpoint{3.296524in}{1.268387in}}%
\pgfpathlineto{\pgfqpoint{3.546524in}{1.268387in}}%
\pgfpathlineto{\pgfqpoint{3.546524in}{1.355887in}}%
\pgfpathlineto{\pgfqpoint{3.296524in}{1.355887in}}%
\pgfpathlineto{\pgfqpoint{3.296524in}{1.268387in}}%
\pgfpathclose%
\pgfusepath{stroke,fill}%
\end{pgfscope}%
\begin{pgfscope}%
\definecolor{textcolor}{rgb}{0.150000,0.150000,0.150000}%
\pgfsetstrokecolor{textcolor}%
\pgfsetfillcolor{textcolor}%
\pgftext[x=3.646524in,y=1.268387in,left,base]{\color{textcolor}\sffamily\fontsize{9.000000}{10.800000}\selectfont Bachelors's Degree, 29.4 \%}%
\end{pgfscope}%
\begin{pgfscope}%
\pgfsetbuttcap%
\pgfsetmiterjoin%
\definecolor{currentfill}{rgb}{0.745098,0.729412,0.854902}%
\pgfsetfillcolor{currentfill}%
\pgfsetlinewidth{1.003750pt}%
\definecolor{currentstroke}{rgb}{1.000000,1.000000,1.000000}%
\pgfsetstrokecolor{currentstroke}%
\pgfsetdash{}{0pt}%
\pgfpathmoveto{\pgfqpoint{3.296524in}{1.084915in}}%
\pgfpathlineto{\pgfqpoint{3.546524in}{1.084915in}}%
\pgfpathlineto{\pgfqpoint{3.546524in}{1.172415in}}%
\pgfpathlineto{\pgfqpoint{3.296524in}{1.172415in}}%
\pgfpathlineto{\pgfqpoint{3.296524in}{1.084915in}}%
\pgfpathclose%
\pgfusepath{stroke,fill}%
\end{pgfscope}%
\begin{pgfscope}%
\definecolor{textcolor}{rgb}{0.150000,0.150000,0.150000}%
\pgfsetstrokecolor{textcolor}%
\pgfsetfillcolor{textcolor}%
\pgftext[x=3.646524in,y=1.084915in,left,base]{\color{textcolor}\sffamily\fontsize{9.000000}{10.800000}\selectfont Master's Degree, 29.4 \%}%
\end{pgfscope}%
\begin{pgfscope}%
\pgfsetbuttcap%
\pgfsetmiterjoin%
\definecolor{currentfill}{rgb}{0.984314,0.501961,0.447059}%
\pgfsetfillcolor{currentfill}%
\pgfsetlinewidth{1.003750pt}%
\definecolor{currentstroke}{rgb}{1.000000,1.000000,1.000000}%
\pgfsetstrokecolor{currentstroke}%
\pgfsetdash{}{0pt}%
\pgfpathmoveto{\pgfqpoint{3.296524in}{0.901444in}}%
\pgfpathlineto{\pgfqpoint{3.546524in}{0.901444in}}%
\pgfpathlineto{\pgfqpoint{3.546524in}{0.988944in}}%
\pgfpathlineto{\pgfqpoint{3.296524in}{0.988944in}}%
\pgfpathlineto{\pgfqpoint{3.296524in}{0.901444in}}%
\pgfpathclose%
\pgfusepath{stroke,fill}%
\end{pgfscope}%
\begin{pgfscope}%
\definecolor{textcolor}{rgb}{0.150000,0.150000,0.150000}%
\pgfsetstrokecolor{textcolor}%
\pgfsetfillcolor{textcolor}%
\pgftext[x=3.646524in,y=0.901444in,left,base]{\color{textcolor}\sffamily\fontsize{9.000000}{10.800000}\selectfont Doctorate Degree, 0.0 \%}%
\end{pgfscope}%
\begin{pgfscope}%
\pgfsetbuttcap%
\pgfsetmiterjoin%
\definecolor{currentfill}{rgb}{0.501961,0.694118,0.827451}%
\pgfsetfillcolor{currentfill}%
\pgfsetlinewidth{1.003750pt}%
\definecolor{currentstroke}{rgb}{1.000000,1.000000,1.000000}%
\pgfsetstrokecolor{currentstroke}%
\pgfsetdash{}{0pt}%
\pgfpathmoveto{\pgfqpoint{3.296524in}{0.717972in}}%
\pgfpathlineto{\pgfqpoint{3.546524in}{0.717972in}}%
\pgfpathlineto{\pgfqpoint{3.546524in}{0.805472in}}%
\pgfpathlineto{\pgfqpoint{3.296524in}{0.805472in}}%
\pgfpathlineto{\pgfqpoint{3.296524in}{0.717972in}}%
\pgfpathclose%
\pgfusepath{stroke,fill}%
\end{pgfscope}%
\begin{pgfscope}%
\definecolor{textcolor}{rgb}{0.150000,0.150000,0.150000}%
\pgfsetstrokecolor{textcolor}%
\pgfsetfillcolor{textcolor}%
\pgftext[x=3.646524in,y=0.717972in,left,base]{\color{textcolor}\sffamily\fontsize{9.000000}{10.800000}\selectfont Other Engineering Degree, 11.8 \%}%
\end{pgfscope}%
\begin{pgfscope}%
\pgfsetbuttcap%
\pgfsetmiterjoin%
\definecolor{currentfill}{rgb}{0.992157,0.705882,0.384314}%
\pgfsetfillcolor{currentfill}%
\pgfsetlinewidth{1.003750pt}%
\definecolor{currentstroke}{rgb}{1.000000,1.000000,1.000000}%
\pgfsetstrokecolor{currentstroke}%
\pgfsetdash{}{0pt}%
\pgfpathmoveto{\pgfqpoint{3.296524in}{0.534501in}}%
\pgfpathlineto{\pgfqpoint{3.546524in}{0.534501in}}%
\pgfpathlineto{\pgfqpoint{3.546524in}{0.622001in}}%
\pgfpathlineto{\pgfqpoint{3.296524in}{0.622001in}}%
\pgfpathlineto{\pgfqpoint{3.296524in}{0.534501in}}%
\pgfpathclose%
\pgfusepath{stroke,fill}%
\end{pgfscope}%
\begin{pgfscope}%
\definecolor{textcolor}{rgb}{0.150000,0.150000,0.150000}%
\pgfsetstrokecolor{textcolor}%
\pgfsetfillcolor{textcolor}%
\pgftext[x=3.646524in,y=0.534501in,left,base]{\color{textcolor}\sffamily\fontsize{9.000000}{10.800000}\selectfont Yes but did not finish, 0.0 \%}%
\end{pgfscope}%
\end{pgfpicture}%
\makeatother%
\endgroup%
}
	\caption{spam}
	\label{fig:uniexp}
\end{figure}

\begin{figure}[H]
	\scalebox{0.72}{%% Creator: Matplotlib, PGF backend
%%
%% To include the figure in your LaTeX document, write
%%   \input{<filename>.pgf}
%%
%% Make sure the required packages are loaded in your preamble
%%   \usepackage{pgf}
%%
%% Also ensure that all the required font packages are loaded; for instance,
%% the lmodern package is sometimes necessary when using math font.
%%   \usepackage{lmodern}
%%
%% Figures using additional raster images can only be included by \input if
%% they are in the same directory as the main LaTeX file. For loading figures
%% from other directories you can use the `import` package
%%   \usepackage{import}
%%
%% and then include the figures with
%%   \import{<path to file>}{<filename>.pgf}
%%
%% Matplotlib used the following preamble
%%   \usepackage{fontspec}
%%   \setmainfont{DejaVuSerif.ttf}[Path=\detokenize{/home/spam/miniconda3/envs/mpl/lib/python3.10/site-packages/matplotlib/mpl-data/fonts/ttf/}]
%%   \setsansfont{DejaVuSans.ttf}[Path=\detokenize{/home/spam/miniconda3/envs/mpl/lib/python3.10/site-packages/matplotlib/mpl-data/fonts/ttf/}]
%%   \setmonofont{DejaVuSansMono.ttf}[Path=\detokenize{/home/spam/miniconda3/envs/mpl/lib/python3.10/site-packages/matplotlib/mpl-data/fonts/ttf/}]
%%
\begingroup%
\makeatletter%
\begin{pgfpicture}%
\pgfpathrectangle{\pgfpointorigin}{\pgfqpoint{5.906660in}{5.000000in}}%
\pgfusepath{use as bounding box, clip}%
\begin{pgfscope}%
\pgfsetbuttcap%
\pgfsetmiterjoin%
\definecolor{currentfill}{rgb}{1.000000,1.000000,1.000000}%
\pgfsetfillcolor{currentfill}%
\pgfsetlinewidth{0.000000pt}%
\definecolor{currentstroke}{rgb}{1.000000,1.000000,1.000000}%
\pgfsetstrokecolor{currentstroke}%
\pgfsetdash{}{0pt}%
\pgfpathmoveto{\pgfqpoint{0.000000in}{0.000000in}}%
\pgfpathlineto{\pgfqpoint{5.906660in}{0.000000in}}%
\pgfpathlineto{\pgfqpoint{5.906660in}{5.000000in}}%
\pgfpathlineto{\pgfqpoint{0.000000in}{5.000000in}}%
\pgfpathlineto{\pgfqpoint{0.000000in}{0.000000in}}%
\pgfpathclose%
\pgfusepath{fill}%
\end{pgfscope}%
\begin{pgfscope}%
\definecolor{textcolor}{rgb}{0.150000,0.150000,0.150000}%
\pgfsetstrokecolor{textcolor}%
\pgfsetfillcolor{textcolor}%
\pgftext[x=3.027163in,y=0.411111in,,top]{\color{textcolor}\sffamily\fontsize{12.000000}{14.400000}\selectfont Computer Science UniversityUniverity Experience}%
\end{pgfscope}%
\begin{pgfscope}%
\pgfsetbuttcap%
\pgfsetmiterjoin%
\definecolor{currentfill}{rgb}{0.552941,0.827451,0.780392}%
\pgfsetfillcolor{currentfill}%
\pgfsetlinewidth{1.003750pt}%
\definecolor{currentstroke}{rgb}{1.000000,1.000000,1.000000}%
\pgfsetstrokecolor{currentstroke}%
\pgfsetdash{}{0pt}%
\pgfpathmoveto{\pgfqpoint{4.567163in}{2.475000in}}%
\pgfpathcurveto{\pgfqpoint{4.567163in}{2.713219in}}{\pgfqpoint{4.511889in}{2.948220in}}{\pgfqpoint{4.405702in}{3.161463in}}%
\pgfpathcurveto{\pgfqpoint{4.299515in}{3.374705in}}{\pgfqpoint{4.145283in}{3.560429in}}{\pgfqpoint{3.955175in}{3.703981in}}%
\pgfpathcurveto{\pgfqpoint{3.765067in}{3.847533in}}{\pgfqpoint{3.544218in}{3.945035in}}{\pgfqpoint{3.310053in}{3.988794in}}%
\pgfpathcurveto{\pgfqpoint{3.075889in}{4.032554in}}{\pgfqpoint{2.834733in}{4.021389in}}{\pgfqpoint{2.605613in}{3.956180in}}%
\pgfpathlineto{\pgfqpoint{3.027163in}{2.475000in}}%
\pgfpathlineto{\pgfqpoint{4.567163in}{2.475000in}}%
\pgfpathlineto{\pgfqpoint{4.567163in}{2.475000in}}%
\pgfpathclose%
\pgfusepath{stroke,fill}%
\end{pgfscope}%
\begin{pgfscope}%
\pgfsetbuttcap%
\pgfsetmiterjoin%
\definecolor{currentfill}{rgb}{1.000000,1.000000,0.701961}%
\pgfsetfillcolor{currentfill}%
\pgfsetlinewidth{1.003750pt}%
\definecolor{currentstroke}{rgb}{1.000000,1.000000,1.000000}%
\pgfsetstrokecolor{currentstroke}%
\pgfsetdash{}{0pt}%
\pgfpathmoveto{\pgfqpoint{2.605613in}{3.956180in}}%
\pgfpathcurveto{\pgfqpoint{2.376493in}{3.890972in}}{\pgfqpoint{2.165597in}{3.773481in}}{\pgfqpoint{1.989567in}{3.612978in}}%
\pgfpathcurveto{\pgfqpoint{1.813536in}{3.452475in}}{\pgfqpoint{1.677124in}{3.253294in}}{\pgfqpoint{1.591094in}{3.031153in}}%
\pgfpathcurveto{\pgfqpoint{1.505064in}{2.809011in}}{\pgfqpoint{1.471740in}{2.569908in}}{\pgfqpoint{1.493751in}{2.332708in}}%
\pgfpathcurveto{\pgfqpoint{1.515762in}{2.095508in}}{\pgfqpoint{1.592513in}{1.866620in}}{\pgfqpoint{1.717949in}{1.664101in}}%
\pgfpathlineto{\pgfqpoint{3.027163in}{2.475000in}}%
\pgfpathlineto{\pgfqpoint{2.605613in}{3.956180in}}%
\pgfpathlineto{\pgfqpoint{2.605613in}{3.956180in}}%
\pgfpathclose%
\pgfusepath{stroke,fill}%
\end{pgfscope}%
\begin{pgfscope}%
\pgfsetbuttcap%
\pgfsetmiterjoin%
\definecolor{currentfill}{rgb}{0.745098,0.729412,0.854902}%
\pgfsetfillcolor{currentfill}%
\pgfsetlinewidth{1.003750pt}%
\definecolor{currentstroke}{rgb}{1.000000,1.000000,1.000000}%
\pgfsetstrokecolor{currentstroke}%
\pgfsetdash{}{0pt}%
\pgfpathmoveto{\pgfqpoint{1.717949in}{1.664101in}}%
\pgfpathcurveto{\pgfqpoint{1.843385in}{1.461582in}}{\pgfqpoint{2.014117in}{1.290903in}}{\pgfqpoint{2.216676in}{1.165531in}}%
\pgfpathcurveto{\pgfqpoint{2.419234in}{1.040158in}}{\pgfqpoint{2.648147in}{0.963479in}}{\pgfqpoint{2.885354in}{0.941543in}}%
\pgfpathcurveto{\pgfqpoint{3.122560in}{0.919607in}}{\pgfqpoint{3.361653in}{0.953006in}}{\pgfqpoint{3.583768in}{1.039106in}}%
\pgfpathcurveto{\pgfqpoint{3.805882in}{1.125206in}}{\pgfqpoint{4.005020in}{1.261680in}}{\pgfqpoint{4.165467in}{1.437762in}}%
\pgfpathlineto{\pgfqpoint{3.027163in}{2.475000in}}%
\pgfpathlineto{\pgfqpoint{1.717949in}{1.664101in}}%
\pgfpathlineto{\pgfqpoint{1.717949in}{1.664101in}}%
\pgfpathclose%
\pgfusepath{stroke,fill}%
\end{pgfscope}%
\begin{pgfscope}%
\pgfsetbuttcap%
\pgfsetmiterjoin%
\definecolor{currentfill}{rgb}{0.984314,0.501961,0.447059}%
\pgfsetfillcolor{currentfill}%
\pgfsetlinewidth{1.003750pt}%
\definecolor{currentstroke}{rgb}{1.000000,1.000000,1.000000}%
\pgfsetstrokecolor{currentstroke}%
\pgfsetdash{}{0pt}%
\pgfpathmoveto{\pgfqpoint{4.165467in}{1.437762in}}%
\pgfpathcurveto{\pgfqpoint{4.165467in}{1.437762in}}{\pgfqpoint{4.165467in}{1.437762in}}{\pgfqpoint{4.165467in}{1.437762in}}%
\pgfpathlineto{\pgfqpoint{3.027163in}{2.475000in}}%
\pgfpathlineto{\pgfqpoint{4.165467in}{1.437762in}}%
\pgfpathlineto{\pgfqpoint{4.165467in}{1.437762in}}%
\pgfpathclose%
\pgfusepath{stroke,fill}%
\end{pgfscope}%
\begin{pgfscope}%
\pgfsetbuttcap%
\pgfsetmiterjoin%
\definecolor{currentfill}{rgb}{0.501961,0.694118,0.827451}%
\pgfsetfillcolor{currentfill}%
\pgfsetlinewidth{1.003750pt}%
\definecolor{currentstroke}{rgb}{1.000000,1.000000,1.000000}%
\pgfsetstrokecolor{currentstroke}%
\pgfsetdash{}{0pt}%
\pgfpathmoveto{\pgfqpoint{4.165467in}{1.437762in}}%
\pgfpathcurveto{\pgfqpoint{4.293579in}{1.578356in}}{\pgfqpoint{4.394541in}{1.741476in}}{\pgfqpoint{4.463232in}{1.918848in}}%
\pgfpathcurveto{\pgfqpoint{4.531924in}{2.096219in}}{\pgfqpoint{4.567163in}{2.284792in}}{\pgfqpoint{4.567163in}{2.475001in}}%
\pgfpathlineto{\pgfqpoint{3.027163in}{2.475000in}}%
\pgfpathlineto{\pgfqpoint{4.165467in}{1.437762in}}%
\pgfpathlineto{\pgfqpoint{4.165467in}{1.437762in}}%
\pgfpathclose%
\pgfusepath{stroke,fill}%
\end{pgfscope}%
\begin{pgfscope}%
\pgfsetbuttcap%
\pgfsetmiterjoin%
\definecolor{currentfill}{rgb}{0.992157,0.705882,0.384314}%
\pgfsetfillcolor{currentfill}%
\pgfsetlinewidth{1.003750pt}%
\definecolor{currentstroke}{rgb}{1.000000,1.000000,1.000000}%
\pgfsetstrokecolor{currentstroke}%
\pgfsetdash{}{0pt}%
\pgfpathmoveto{\pgfqpoint{4.567163in}{2.475001in}}%
\pgfpathcurveto{\pgfqpoint{4.567163in}{2.475001in}}{\pgfqpoint{4.567163in}{2.475001in}}{\pgfqpoint{4.567163in}{2.475001in}}%
\pgfpathlineto{\pgfqpoint{3.027163in}{2.475000in}}%
\pgfpathlineto{\pgfqpoint{4.567163in}{2.475001in}}%
\pgfpathlineto{\pgfqpoint{4.567163in}{2.475001in}}%
\pgfpathclose%
\pgfusepath{stroke,fill}%
\end{pgfscope}%
\begin{pgfscope}%
\definecolor{textcolor}{rgb}{0.150000,0.150000,0.150000}%
\pgfsetstrokecolor{textcolor}%
\pgfsetfillcolor{textcolor}%
\pgftext[x=3.583970in,y=3.212389in,,]{\color{textcolor}\sffamily\fontsize{12.000000}{14.400000}\selectfont 29.4\%}%
\end{pgfscope}%
\begin{pgfscope}%
\definecolor{textcolor}{rgb}{0.150000,0.150000,0.150000}%
\pgfsetstrokecolor{textcolor}%
\pgfsetfillcolor{textcolor}%
\pgftext[x=2.165522in,y=2.808692in,,]{\color{textcolor}\sffamily\fontsize{12.000000}{14.400000}\selectfont 29.4\%}%
\end{pgfscope}%
\begin{pgfscope}%
\definecolor{textcolor}{rgb}{0.150000,0.150000,0.150000}%
\pgfsetstrokecolor{textcolor}%
\pgfsetfillcolor{textcolor}%
\pgftext[x=2.942078in,y=1.554926in,,]{\color{textcolor}\sffamily\fontsize{12.000000}{14.400000}\selectfont 29.4\%}%
\end{pgfscope}%
\begin{pgfscope}%
\definecolor{textcolor}{rgb}{0.150000,0.150000,0.150000}%
\pgfsetstrokecolor{textcolor}%
\pgfsetfillcolor{textcolor}%
\pgftext[x=3.710146in,y=1.852657in,,]{\color{textcolor}\sffamily\fontsize{12.000000}{14.400000}\selectfont 0.0\%}%
\end{pgfscope}%
\begin{pgfscope}%
\definecolor{textcolor}{rgb}{0.150000,0.150000,0.150000}%
\pgfsetstrokecolor{textcolor}%
\pgfsetfillcolor{textcolor}%
\pgftext[x=3.888805in,y=2.141309in,,]{\color{textcolor}\sffamily\fontsize{12.000000}{14.400000}\selectfont 11.8\%}%
\end{pgfscope}%
\begin{pgfscope}%
\definecolor{textcolor}{rgb}{0.150000,0.150000,0.150000}%
\pgfsetstrokecolor{textcolor}%
\pgfsetfillcolor{textcolor}%
\pgftext[x=3.951163in,y=2.475000in,,]{\color{textcolor}\sffamily\fontsize{12.000000}{14.400000}\selectfont 0.0\%}%
\end{pgfscope}%
\begin{pgfscope}%
\pgfsetbuttcap%
\pgfsetmiterjoin%
\definecolor{currentfill}{rgb}{1.000000,1.000000,1.000000}%
\pgfsetfillcolor{currentfill}%
\pgfsetfillopacity{0.800000}%
\pgfsetlinewidth{1.003750pt}%
\definecolor{currentstroke}{rgb}{0.800000,0.800000,0.800000}%
\pgfsetstrokecolor{currentstroke}%
\pgfsetstrokeopacity{0.800000}%
\pgfsetdash{}{0pt}%
\pgfpathmoveto{\pgfqpoint{3.271524in}{0.458500in}}%
\pgfpathlineto{\pgfqpoint{5.827163in}{0.458500in}}%
\pgfpathquadraticcurveto{\pgfqpoint{5.852163in}{0.458500in}}{\pgfqpoint{5.852163in}{0.483500in}}%
\pgfpathlineto{\pgfqpoint{5.852163in}{1.571829in}}%
\pgfpathquadraticcurveto{\pgfqpoint{5.852163in}{1.596829in}}{\pgfqpoint{5.827163in}{1.596829in}}%
\pgfpathlineto{\pgfqpoint{3.271524in}{1.596829in}}%
\pgfpathquadraticcurveto{\pgfqpoint{3.246524in}{1.596829in}}{\pgfqpoint{3.246524in}{1.571829in}}%
\pgfpathlineto{\pgfqpoint{3.246524in}{0.483500in}}%
\pgfpathquadraticcurveto{\pgfqpoint{3.246524in}{0.458500in}}{\pgfqpoint{3.271524in}{0.458500in}}%
\pgfpathlineto{\pgfqpoint{3.271524in}{0.458500in}}%
\pgfpathclose%
\pgfusepath{stroke,fill}%
\end{pgfscope}%
\begin{pgfscope}%
\pgfsetbuttcap%
\pgfsetmiterjoin%
\definecolor{currentfill}{rgb}{0.552941,0.827451,0.780392}%
\pgfsetfillcolor{currentfill}%
\pgfsetlinewidth{1.003750pt}%
\definecolor{currentstroke}{rgb}{1.000000,1.000000,1.000000}%
\pgfsetstrokecolor{currentstroke}%
\pgfsetdash{}{0pt}%
\pgfpathmoveto{\pgfqpoint{3.296524in}{1.451858in}}%
\pgfpathlineto{\pgfqpoint{3.546524in}{1.451858in}}%
\pgfpathlineto{\pgfqpoint{3.546524in}{1.539358in}}%
\pgfpathlineto{\pgfqpoint{3.296524in}{1.539358in}}%
\pgfpathlineto{\pgfqpoint{3.296524in}{1.451858in}}%
\pgfpathclose%
\pgfusepath{stroke,fill}%
\end{pgfscope}%
\begin{pgfscope}%
\definecolor{textcolor}{rgb}{0.150000,0.150000,0.150000}%
\pgfsetstrokecolor{textcolor}%
\pgfsetfillcolor{textcolor}%
\pgftext[x=3.646524in,y=1.451858in,left,base]{\color{textcolor}\sffamily\fontsize{9.000000}{10.800000}\selectfont No CS Degree, 29.4 \%}%
\end{pgfscope}%
\begin{pgfscope}%
\pgfsetbuttcap%
\pgfsetmiterjoin%
\definecolor{currentfill}{rgb}{1.000000,1.000000,0.701961}%
\pgfsetfillcolor{currentfill}%
\pgfsetlinewidth{1.003750pt}%
\definecolor{currentstroke}{rgb}{1.000000,1.000000,1.000000}%
\pgfsetstrokecolor{currentstroke}%
\pgfsetdash{}{0pt}%
\pgfpathmoveto{\pgfqpoint{3.296524in}{1.268387in}}%
\pgfpathlineto{\pgfqpoint{3.546524in}{1.268387in}}%
\pgfpathlineto{\pgfqpoint{3.546524in}{1.355887in}}%
\pgfpathlineto{\pgfqpoint{3.296524in}{1.355887in}}%
\pgfpathlineto{\pgfqpoint{3.296524in}{1.268387in}}%
\pgfpathclose%
\pgfusepath{stroke,fill}%
\end{pgfscope}%
\begin{pgfscope}%
\definecolor{textcolor}{rgb}{0.150000,0.150000,0.150000}%
\pgfsetstrokecolor{textcolor}%
\pgfsetfillcolor{textcolor}%
\pgftext[x=3.646524in,y=1.268387in,left,base]{\color{textcolor}\sffamily\fontsize{9.000000}{10.800000}\selectfont Bachelors's Degree, 29.4 \%}%
\end{pgfscope}%
\begin{pgfscope}%
\pgfsetbuttcap%
\pgfsetmiterjoin%
\definecolor{currentfill}{rgb}{0.745098,0.729412,0.854902}%
\pgfsetfillcolor{currentfill}%
\pgfsetlinewidth{1.003750pt}%
\definecolor{currentstroke}{rgb}{1.000000,1.000000,1.000000}%
\pgfsetstrokecolor{currentstroke}%
\pgfsetdash{}{0pt}%
\pgfpathmoveto{\pgfqpoint{3.296524in}{1.084915in}}%
\pgfpathlineto{\pgfqpoint{3.546524in}{1.084915in}}%
\pgfpathlineto{\pgfqpoint{3.546524in}{1.172415in}}%
\pgfpathlineto{\pgfqpoint{3.296524in}{1.172415in}}%
\pgfpathlineto{\pgfqpoint{3.296524in}{1.084915in}}%
\pgfpathclose%
\pgfusepath{stroke,fill}%
\end{pgfscope}%
\begin{pgfscope}%
\definecolor{textcolor}{rgb}{0.150000,0.150000,0.150000}%
\pgfsetstrokecolor{textcolor}%
\pgfsetfillcolor{textcolor}%
\pgftext[x=3.646524in,y=1.084915in,left,base]{\color{textcolor}\sffamily\fontsize{9.000000}{10.800000}\selectfont Master's Degree, 29.4 \%}%
\end{pgfscope}%
\begin{pgfscope}%
\pgfsetbuttcap%
\pgfsetmiterjoin%
\definecolor{currentfill}{rgb}{0.984314,0.501961,0.447059}%
\pgfsetfillcolor{currentfill}%
\pgfsetlinewidth{1.003750pt}%
\definecolor{currentstroke}{rgb}{1.000000,1.000000,1.000000}%
\pgfsetstrokecolor{currentstroke}%
\pgfsetdash{}{0pt}%
\pgfpathmoveto{\pgfqpoint{3.296524in}{0.901444in}}%
\pgfpathlineto{\pgfqpoint{3.546524in}{0.901444in}}%
\pgfpathlineto{\pgfqpoint{3.546524in}{0.988944in}}%
\pgfpathlineto{\pgfqpoint{3.296524in}{0.988944in}}%
\pgfpathlineto{\pgfqpoint{3.296524in}{0.901444in}}%
\pgfpathclose%
\pgfusepath{stroke,fill}%
\end{pgfscope}%
\begin{pgfscope}%
\definecolor{textcolor}{rgb}{0.150000,0.150000,0.150000}%
\pgfsetstrokecolor{textcolor}%
\pgfsetfillcolor{textcolor}%
\pgftext[x=3.646524in,y=0.901444in,left,base]{\color{textcolor}\sffamily\fontsize{9.000000}{10.800000}\selectfont Doctorate Degree, 0.0 \%}%
\end{pgfscope}%
\begin{pgfscope}%
\pgfsetbuttcap%
\pgfsetmiterjoin%
\definecolor{currentfill}{rgb}{0.501961,0.694118,0.827451}%
\pgfsetfillcolor{currentfill}%
\pgfsetlinewidth{1.003750pt}%
\definecolor{currentstroke}{rgb}{1.000000,1.000000,1.000000}%
\pgfsetstrokecolor{currentstroke}%
\pgfsetdash{}{0pt}%
\pgfpathmoveto{\pgfqpoint{3.296524in}{0.717972in}}%
\pgfpathlineto{\pgfqpoint{3.546524in}{0.717972in}}%
\pgfpathlineto{\pgfqpoint{3.546524in}{0.805472in}}%
\pgfpathlineto{\pgfqpoint{3.296524in}{0.805472in}}%
\pgfpathlineto{\pgfqpoint{3.296524in}{0.717972in}}%
\pgfpathclose%
\pgfusepath{stroke,fill}%
\end{pgfscope}%
\begin{pgfscope}%
\definecolor{textcolor}{rgb}{0.150000,0.150000,0.150000}%
\pgfsetstrokecolor{textcolor}%
\pgfsetfillcolor{textcolor}%
\pgftext[x=3.646524in,y=0.717972in,left,base]{\color{textcolor}\sffamily\fontsize{9.000000}{10.800000}\selectfont Other Engineering Degree, 11.8 \%}%
\end{pgfscope}%
\begin{pgfscope}%
\pgfsetbuttcap%
\pgfsetmiterjoin%
\definecolor{currentfill}{rgb}{0.992157,0.705882,0.384314}%
\pgfsetfillcolor{currentfill}%
\pgfsetlinewidth{1.003750pt}%
\definecolor{currentstroke}{rgb}{1.000000,1.000000,1.000000}%
\pgfsetstrokecolor{currentstroke}%
\pgfsetdash{}{0pt}%
\pgfpathmoveto{\pgfqpoint{3.296524in}{0.534501in}}%
\pgfpathlineto{\pgfqpoint{3.546524in}{0.534501in}}%
\pgfpathlineto{\pgfqpoint{3.546524in}{0.622001in}}%
\pgfpathlineto{\pgfqpoint{3.296524in}{0.622001in}}%
\pgfpathlineto{\pgfqpoint{3.296524in}{0.534501in}}%
\pgfpathclose%
\pgfusepath{stroke,fill}%
\end{pgfscope}%
\begin{pgfscope}%
\definecolor{textcolor}{rgb}{0.150000,0.150000,0.150000}%
\pgfsetstrokecolor{textcolor}%
\pgfsetfillcolor{textcolor}%
\pgftext[x=3.646524in,y=0.534501in,left,base]{\color{textcolor}\sffamily\fontsize{9.000000}{10.800000}\selectfont Yes but did not finish, 0.0 \%}%
\end{pgfscope}%
\end{pgfpicture}%
\makeatother%
\endgroup%
}
	\caption{more and more}
	\label{fig:moreandmore}
\end{figure}

\begin{figure}[H]
	\centering
	\scalebox{0.72}{%% Creator: Matplotlib, PGF backend
%%
%% To include the figure in your LaTeX document, write
%%   \input{<filename>.pgf}
%%
%% Make sure the required packages are loaded in your preamble
%%   \usepackage{pgf}
%%
%% Also ensure that all the required font packages are loaded; for instance,
%% the lmodern package is sometimes necessary when using math font.
%%   \usepackage{lmodern}
%%
%% Figures using additional raster images can only be included by \input if
%% they are in the same directory as the main LaTeX file. For loading figures
%% from other directories you can use the `import` package
%%   \usepackage{import}
%%
%% and then include the figures with
%%   \import{<path to file>}{<filename>.pgf}
%%
%% Matplotlib used the following preamble
%%   \usepackage{fontspec}
%%   \setmainfont{DejaVuSerif.ttf}[Path=\detokenize{/home/spam/miniconda3/envs/mpl/lib/python3.10/site-packages/matplotlib/mpl-data/fonts/ttf/}]
%%   \setsansfont{DejaVuSans.ttf}[Path=\detokenize{/home/spam/miniconda3/envs/mpl/lib/python3.10/site-packages/matplotlib/mpl-data/fonts/ttf/}]
%%   \setmonofont{DejaVuSansMono.ttf}[Path=\detokenize{/home/spam/miniconda3/envs/mpl/lib/python3.10/site-packages/matplotlib/mpl-data/fonts/ttf/}]
%%
\begingroup%
\makeatletter%
\begin{pgfpicture}%
\pgfpathrectangle{\pgfpointorigin}{\pgfqpoint{5.906660in}{5.000000in}}%
\pgfusepath{use as bounding box, clip}%
\begin{pgfscope}%
\pgfsetbuttcap%
\pgfsetmiterjoin%
\definecolor{currentfill}{rgb}{1.000000,1.000000,1.000000}%
\pgfsetfillcolor{currentfill}%
\pgfsetlinewidth{0.000000pt}%
\definecolor{currentstroke}{rgb}{1.000000,1.000000,1.000000}%
\pgfsetstrokecolor{currentstroke}%
\pgfsetdash{}{0pt}%
\pgfpathmoveto{\pgfqpoint{0.000000in}{0.000000in}}%
\pgfpathlineto{\pgfqpoint{5.906660in}{0.000000in}}%
\pgfpathlineto{\pgfqpoint{5.906660in}{5.000000in}}%
\pgfpathlineto{\pgfqpoint{0.000000in}{5.000000in}}%
\pgfpathlineto{\pgfqpoint{0.000000in}{0.000000in}}%
\pgfpathclose%
\pgfusepath{fill}%
\end{pgfscope}%
\begin{pgfscope}%
\definecolor{textcolor}{rgb}{0.150000,0.150000,0.150000}%
\pgfsetstrokecolor{textcolor}%
\pgfsetfillcolor{textcolor}%
\pgftext[x=3.027163in,y=0.411111in,,top]{\color{textcolor}\sffamily\fontsize{12.000000}{14.400000}\selectfont Computer Science UniversityUniverity Experience}%
\end{pgfscope}%
\begin{pgfscope}%
\pgfsetbuttcap%
\pgfsetmiterjoin%
\definecolor{currentfill}{rgb}{0.552941,0.827451,0.780392}%
\pgfsetfillcolor{currentfill}%
\pgfsetlinewidth{1.003750pt}%
\definecolor{currentstroke}{rgb}{1.000000,1.000000,1.000000}%
\pgfsetstrokecolor{currentstroke}%
\pgfsetdash{}{0pt}%
\pgfpathmoveto{\pgfqpoint{4.567163in}{2.475000in}}%
\pgfpathcurveto{\pgfqpoint{4.567163in}{2.713219in}}{\pgfqpoint{4.511889in}{2.948220in}}{\pgfqpoint{4.405702in}{3.161463in}}%
\pgfpathcurveto{\pgfqpoint{4.299515in}{3.374705in}}{\pgfqpoint{4.145283in}{3.560429in}}{\pgfqpoint{3.955175in}{3.703981in}}%
\pgfpathcurveto{\pgfqpoint{3.765067in}{3.847533in}}{\pgfqpoint{3.544218in}{3.945035in}}{\pgfqpoint{3.310053in}{3.988794in}}%
\pgfpathcurveto{\pgfqpoint{3.075889in}{4.032554in}}{\pgfqpoint{2.834733in}{4.021389in}}{\pgfqpoint{2.605613in}{3.956180in}}%
\pgfpathlineto{\pgfqpoint{3.027163in}{2.475000in}}%
\pgfpathlineto{\pgfqpoint{4.567163in}{2.475000in}}%
\pgfpathlineto{\pgfqpoint{4.567163in}{2.475000in}}%
\pgfpathclose%
\pgfusepath{stroke,fill}%
\end{pgfscope}%
\begin{pgfscope}%
\pgfsetbuttcap%
\pgfsetmiterjoin%
\definecolor{currentfill}{rgb}{1.000000,1.000000,0.701961}%
\pgfsetfillcolor{currentfill}%
\pgfsetlinewidth{1.003750pt}%
\definecolor{currentstroke}{rgb}{1.000000,1.000000,1.000000}%
\pgfsetstrokecolor{currentstroke}%
\pgfsetdash{}{0pt}%
\pgfpathmoveto{\pgfqpoint{2.605613in}{3.956180in}}%
\pgfpathcurveto{\pgfqpoint{2.376493in}{3.890972in}}{\pgfqpoint{2.165597in}{3.773481in}}{\pgfqpoint{1.989567in}{3.612978in}}%
\pgfpathcurveto{\pgfqpoint{1.813536in}{3.452475in}}{\pgfqpoint{1.677124in}{3.253294in}}{\pgfqpoint{1.591094in}{3.031153in}}%
\pgfpathcurveto{\pgfqpoint{1.505064in}{2.809011in}}{\pgfqpoint{1.471740in}{2.569908in}}{\pgfqpoint{1.493751in}{2.332708in}}%
\pgfpathcurveto{\pgfqpoint{1.515762in}{2.095508in}}{\pgfqpoint{1.592513in}{1.866620in}}{\pgfqpoint{1.717949in}{1.664101in}}%
\pgfpathlineto{\pgfqpoint{3.027163in}{2.475000in}}%
\pgfpathlineto{\pgfqpoint{2.605613in}{3.956180in}}%
\pgfpathlineto{\pgfqpoint{2.605613in}{3.956180in}}%
\pgfpathclose%
\pgfusepath{stroke,fill}%
\end{pgfscope}%
\begin{pgfscope}%
\pgfsetbuttcap%
\pgfsetmiterjoin%
\definecolor{currentfill}{rgb}{0.745098,0.729412,0.854902}%
\pgfsetfillcolor{currentfill}%
\pgfsetlinewidth{1.003750pt}%
\definecolor{currentstroke}{rgb}{1.000000,1.000000,1.000000}%
\pgfsetstrokecolor{currentstroke}%
\pgfsetdash{}{0pt}%
\pgfpathmoveto{\pgfqpoint{1.717949in}{1.664101in}}%
\pgfpathcurveto{\pgfqpoint{1.843385in}{1.461582in}}{\pgfqpoint{2.014117in}{1.290903in}}{\pgfqpoint{2.216676in}{1.165531in}}%
\pgfpathcurveto{\pgfqpoint{2.419234in}{1.040158in}}{\pgfqpoint{2.648147in}{0.963479in}}{\pgfqpoint{2.885354in}{0.941543in}}%
\pgfpathcurveto{\pgfqpoint{3.122560in}{0.919607in}}{\pgfqpoint{3.361653in}{0.953006in}}{\pgfqpoint{3.583768in}{1.039106in}}%
\pgfpathcurveto{\pgfqpoint{3.805882in}{1.125206in}}{\pgfqpoint{4.005020in}{1.261680in}}{\pgfqpoint{4.165467in}{1.437762in}}%
\pgfpathlineto{\pgfqpoint{3.027163in}{2.475000in}}%
\pgfpathlineto{\pgfqpoint{1.717949in}{1.664101in}}%
\pgfpathlineto{\pgfqpoint{1.717949in}{1.664101in}}%
\pgfpathclose%
\pgfusepath{stroke,fill}%
\end{pgfscope}%
\begin{pgfscope}%
\pgfsetbuttcap%
\pgfsetmiterjoin%
\definecolor{currentfill}{rgb}{0.984314,0.501961,0.447059}%
\pgfsetfillcolor{currentfill}%
\pgfsetlinewidth{1.003750pt}%
\definecolor{currentstroke}{rgb}{1.000000,1.000000,1.000000}%
\pgfsetstrokecolor{currentstroke}%
\pgfsetdash{}{0pt}%
\pgfpathmoveto{\pgfqpoint{4.165467in}{1.437762in}}%
\pgfpathcurveto{\pgfqpoint{4.165467in}{1.437762in}}{\pgfqpoint{4.165467in}{1.437762in}}{\pgfqpoint{4.165467in}{1.437762in}}%
\pgfpathlineto{\pgfqpoint{3.027163in}{2.475000in}}%
\pgfpathlineto{\pgfqpoint{4.165467in}{1.437762in}}%
\pgfpathlineto{\pgfqpoint{4.165467in}{1.437762in}}%
\pgfpathclose%
\pgfusepath{stroke,fill}%
\end{pgfscope}%
\begin{pgfscope}%
\pgfsetbuttcap%
\pgfsetmiterjoin%
\definecolor{currentfill}{rgb}{0.501961,0.694118,0.827451}%
\pgfsetfillcolor{currentfill}%
\pgfsetlinewidth{1.003750pt}%
\definecolor{currentstroke}{rgb}{1.000000,1.000000,1.000000}%
\pgfsetstrokecolor{currentstroke}%
\pgfsetdash{}{0pt}%
\pgfpathmoveto{\pgfqpoint{4.165467in}{1.437762in}}%
\pgfpathcurveto{\pgfqpoint{4.293579in}{1.578356in}}{\pgfqpoint{4.394541in}{1.741476in}}{\pgfqpoint{4.463232in}{1.918848in}}%
\pgfpathcurveto{\pgfqpoint{4.531924in}{2.096219in}}{\pgfqpoint{4.567163in}{2.284792in}}{\pgfqpoint{4.567163in}{2.475001in}}%
\pgfpathlineto{\pgfqpoint{3.027163in}{2.475000in}}%
\pgfpathlineto{\pgfqpoint{4.165467in}{1.437762in}}%
\pgfpathlineto{\pgfqpoint{4.165467in}{1.437762in}}%
\pgfpathclose%
\pgfusepath{stroke,fill}%
\end{pgfscope}%
\begin{pgfscope}%
\pgfsetbuttcap%
\pgfsetmiterjoin%
\definecolor{currentfill}{rgb}{0.992157,0.705882,0.384314}%
\pgfsetfillcolor{currentfill}%
\pgfsetlinewidth{1.003750pt}%
\definecolor{currentstroke}{rgb}{1.000000,1.000000,1.000000}%
\pgfsetstrokecolor{currentstroke}%
\pgfsetdash{}{0pt}%
\pgfpathmoveto{\pgfqpoint{4.567163in}{2.475001in}}%
\pgfpathcurveto{\pgfqpoint{4.567163in}{2.475001in}}{\pgfqpoint{4.567163in}{2.475001in}}{\pgfqpoint{4.567163in}{2.475001in}}%
\pgfpathlineto{\pgfqpoint{3.027163in}{2.475000in}}%
\pgfpathlineto{\pgfqpoint{4.567163in}{2.475001in}}%
\pgfpathlineto{\pgfqpoint{4.567163in}{2.475001in}}%
\pgfpathclose%
\pgfusepath{stroke,fill}%
\end{pgfscope}%
\begin{pgfscope}%
\definecolor{textcolor}{rgb}{0.150000,0.150000,0.150000}%
\pgfsetstrokecolor{textcolor}%
\pgfsetfillcolor{textcolor}%
\pgftext[x=3.583970in,y=3.212389in,,]{\color{textcolor}\sffamily\fontsize{12.000000}{14.400000}\selectfont 29.4\%}%
\end{pgfscope}%
\begin{pgfscope}%
\definecolor{textcolor}{rgb}{0.150000,0.150000,0.150000}%
\pgfsetstrokecolor{textcolor}%
\pgfsetfillcolor{textcolor}%
\pgftext[x=2.165522in,y=2.808692in,,]{\color{textcolor}\sffamily\fontsize{12.000000}{14.400000}\selectfont 29.4\%}%
\end{pgfscope}%
\begin{pgfscope}%
\definecolor{textcolor}{rgb}{0.150000,0.150000,0.150000}%
\pgfsetstrokecolor{textcolor}%
\pgfsetfillcolor{textcolor}%
\pgftext[x=2.942078in,y=1.554926in,,]{\color{textcolor}\sffamily\fontsize{12.000000}{14.400000}\selectfont 29.4\%}%
\end{pgfscope}%
\begin{pgfscope}%
\definecolor{textcolor}{rgb}{0.150000,0.150000,0.150000}%
\pgfsetstrokecolor{textcolor}%
\pgfsetfillcolor{textcolor}%
\pgftext[x=3.710146in,y=1.852657in,,]{\color{textcolor}\sffamily\fontsize{12.000000}{14.400000}\selectfont 0.0\%}%
\end{pgfscope}%
\begin{pgfscope}%
\definecolor{textcolor}{rgb}{0.150000,0.150000,0.150000}%
\pgfsetstrokecolor{textcolor}%
\pgfsetfillcolor{textcolor}%
\pgftext[x=3.888805in,y=2.141309in,,]{\color{textcolor}\sffamily\fontsize{12.000000}{14.400000}\selectfont 11.8\%}%
\end{pgfscope}%
\begin{pgfscope}%
\definecolor{textcolor}{rgb}{0.150000,0.150000,0.150000}%
\pgfsetstrokecolor{textcolor}%
\pgfsetfillcolor{textcolor}%
\pgftext[x=3.951163in,y=2.475000in,,]{\color{textcolor}\sffamily\fontsize{12.000000}{14.400000}\selectfont 0.0\%}%
\end{pgfscope}%
\begin{pgfscope}%
\pgfsetbuttcap%
\pgfsetmiterjoin%
\definecolor{currentfill}{rgb}{1.000000,1.000000,1.000000}%
\pgfsetfillcolor{currentfill}%
\pgfsetfillopacity{0.800000}%
\pgfsetlinewidth{1.003750pt}%
\definecolor{currentstroke}{rgb}{0.800000,0.800000,0.800000}%
\pgfsetstrokecolor{currentstroke}%
\pgfsetstrokeopacity{0.800000}%
\pgfsetdash{}{0pt}%
\pgfpathmoveto{\pgfqpoint{3.271524in}{0.458500in}}%
\pgfpathlineto{\pgfqpoint{5.827163in}{0.458500in}}%
\pgfpathquadraticcurveto{\pgfqpoint{5.852163in}{0.458500in}}{\pgfqpoint{5.852163in}{0.483500in}}%
\pgfpathlineto{\pgfqpoint{5.852163in}{1.571829in}}%
\pgfpathquadraticcurveto{\pgfqpoint{5.852163in}{1.596829in}}{\pgfqpoint{5.827163in}{1.596829in}}%
\pgfpathlineto{\pgfqpoint{3.271524in}{1.596829in}}%
\pgfpathquadraticcurveto{\pgfqpoint{3.246524in}{1.596829in}}{\pgfqpoint{3.246524in}{1.571829in}}%
\pgfpathlineto{\pgfqpoint{3.246524in}{0.483500in}}%
\pgfpathquadraticcurveto{\pgfqpoint{3.246524in}{0.458500in}}{\pgfqpoint{3.271524in}{0.458500in}}%
\pgfpathlineto{\pgfqpoint{3.271524in}{0.458500in}}%
\pgfpathclose%
\pgfusepath{stroke,fill}%
\end{pgfscope}%
\begin{pgfscope}%
\pgfsetbuttcap%
\pgfsetmiterjoin%
\definecolor{currentfill}{rgb}{0.552941,0.827451,0.780392}%
\pgfsetfillcolor{currentfill}%
\pgfsetlinewidth{1.003750pt}%
\definecolor{currentstroke}{rgb}{1.000000,1.000000,1.000000}%
\pgfsetstrokecolor{currentstroke}%
\pgfsetdash{}{0pt}%
\pgfpathmoveto{\pgfqpoint{3.296524in}{1.451858in}}%
\pgfpathlineto{\pgfqpoint{3.546524in}{1.451858in}}%
\pgfpathlineto{\pgfqpoint{3.546524in}{1.539358in}}%
\pgfpathlineto{\pgfqpoint{3.296524in}{1.539358in}}%
\pgfpathlineto{\pgfqpoint{3.296524in}{1.451858in}}%
\pgfpathclose%
\pgfusepath{stroke,fill}%
\end{pgfscope}%
\begin{pgfscope}%
\definecolor{textcolor}{rgb}{0.150000,0.150000,0.150000}%
\pgfsetstrokecolor{textcolor}%
\pgfsetfillcolor{textcolor}%
\pgftext[x=3.646524in,y=1.451858in,left,base]{\color{textcolor}\sffamily\fontsize{9.000000}{10.800000}\selectfont No CS Degree, 29.4 \%}%
\end{pgfscope}%
\begin{pgfscope}%
\pgfsetbuttcap%
\pgfsetmiterjoin%
\definecolor{currentfill}{rgb}{1.000000,1.000000,0.701961}%
\pgfsetfillcolor{currentfill}%
\pgfsetlinewidth{1.003750pt}%
\definecolor{currentstroke}{rgb}{1.000000,1.000000,1.000000}%
\pgfsetstrokecolor{currentstroke}%
\pgfsetdash{}{0pt}%
\pgfpathmoveto{\pgfqpoint{3.296524in}{1.268387in}}%
\pgfpathlineto{\pgfqpoint{3.546524in}{1.268387in}}%
\pgfpathlineto{\pgfqpoint{3.546524in}{1.355887in}}%
\pgfpathlineto{\pgfqpoint{3.296524in}{1.355887in}}%
\pgfpathlineto{\pgfqpoint{3.296524in}{1.268387in}}%
\pgfpathclose%
\pgfusepath{stroke,fill}%
\end{pgfscope}%
\begin{pgfscope}%
\definecolor{textcolor}{rgb}{0.150000,0.150000,0.150000}%
\pgfsetstrokecolor{textcolor}%
\pgfsetfillcolor{textcolor}%
\pgftext[x=3.646524in,y=1.268387in,left,base]{\color{textcolor}\sffamily\fontsize{9.000000}{10.800000}\selectfont Bachelors's Degree, 29.4 \%}%
\end{pgfscope}%
\begin{pgfscope}%
\pgfsetbuttcap%
\pgfsetmiterjoin%
\definecolor{currentfill}{rgb}{0.745098,0.729412,0.854902}%
\pgfsetfillcolor{currentfill}%
\pgfsetlinewidth{1.003750pt}%
\definecolor{currentstroke}{rgb}{1.000000,1.000000,1.000000}%
\pgfsetstrokecolor{currentstroke}%
\pgfsetdash{}{0pt}%
\pgfpathmoveto{\pgfqpoint{3.296524in}{1.084915in}}%
\pgfpathlineto{\pgfqpoint{3.546524in}{1.084915in}}%
\pgfpathlineto{\pgfqpoint{3.546524in}{1.172415in}}%
\pgfpathlineto{\pgfqpoint{3.296524in}{1.172415in}}%
\pgfpathlineto{\pgfqpoint{3.296524in}{1.084915in}}%
\pgfpathclose%
\pgfusepath{stroke,fill}%
\end{pgfscope}%
\begin{pgfscope}%
\definecolor{textcolor}{rgb}{0.150000,0.150000,0.150000}%
\pgfsetstrokecolor{textcolor}%
\pgfsetfillcolor{textcolor}%
\pgftext[x=3.646524in,y=1.084915in,left,base]{\color{textcolor}\sffamily\fontsize{9.000000}{10.800000}\selectfont Master's Degree, 29.4 \%}%
\end{pgfscope}%
\begin{pgfscope}%
\pgfsetbuttcap%
\pgfsetmiterjoin%
\definecolor{currentfill}{rgb}{0.984314,0.501961,0.447059}%
\pgfsetfillcolor{currentfill}%
\pgfsetlinewidth{1.003750pt}%
\definecolor{currentstroke}{rgb}{1.000000,1.000000,1.000000}%
\pgfsetstrokecolor{currentstroke}%
\pgfsetdash{}{0pt}%
\pgfpathmoveto{\pgfqpoint{3.296524in}{0.901444in}}%
\pgfpathlineto{\pgfqpoint{3.546524in}{0.901444in}}%
\pgfpathlineto{\pgfqpoint{3.546524in}{0.988944in}}%
\pgfpathlineto{\pgfqpoint{3.296524in}{0.988944in}}%
\pgfpathlineto{\pgfqpoint{3.296524in}{0.901444in}}%
\pgfpathclose%
\pgfusepath{stroke,fill}%
\end{pgfscope}%
\begin{pgfscope}%
\definecolor{textcolor}{rgb}{0.150000,0.150000,0.150000}%
\pgfsetstrokecolor{textcolor}%
\pgfsetfillcolor{textcolor}%
\pgftext[x=3.646524in,y=0.901444in,left,base]{\color{textcolor}\sffamily\fontsize{9.000000}{10.800000}\selectfont Doctorate Degree, 0.0 \%}%
\end{pgfscope}%
\begin{pgfscope}%
\pgfsetbuttcap%
\pgfsetmiterjoin%
\definecolor{currentfill}{rgb}{0.501961,0.694118,0.827451}%
\pgfsetfillcolor{currentfill}%
\pgfsetlinewidth{1.003750pt}%
\definecolor{currentstroke}{rgb}{1.000000,1.000000,1.000000}%
\pgfsetstrokecolor{currentstroke}%
\pgfsetdash{}{0pt}%
\pgfpathmoveto{\pgfqpoint{3.296524in}{0.717972in}}%
\pgfpathlineto{\pgfqpoint{3.546524in}{0.717972in}}%
\pgfpathlineto{\pgfqpoint{3.546524in}{0.805472in}}%
\pgfpathlineto{\pgfqpoint{3.296524in}{0.805472in}}%
\pgfpathlineto{\pgfqpoint{3.296524in}{0.717972in}}%
\pgfpathclose%
\pgfusepath{stroke,fill}%
\end{pgfscope}%
\begin{pgfscope}%
\definecolor{textcolor}{rgb}{0.150000,0.150000,0.150000}%
\pgfsetstrokecolor{textcolor}%
\pgfsetfillcolor{textcolor}%
\pgftext[x=3.646524in,y=0.717972in,left,base]{\color{textcolor}\sffamily\fontsize{9.000000}{10.800000}\selectfont Other Engineering Degree, 11.8 \%}%
\end{pgfscope}%
\begin{pgfscope}%
\pgfsetbuttcap%
\pgfsetmiterjoin%
\definecolor{currentfill}{rgb}{0.992157,0.705882,0.384314}%
\pgfsetfillcolor{currentfill}%
\pgfsetlinewidth{1.003750pt}%
\definecolor{currentstroke}{rgb}{1.000000,1.000000,1.000000}%
\pgfsetstrokecolor{currentstroke}%
\pgfsetdash{}{0pt}%
\pgfpathmoveto{\pgfqpoint{3.296524in}{0.534501in}}%
\pgfpathlineto{\pgfqpoint{3.546524in}{0.534501in}}%
\pgfpathlineto{\pgfqpoint{3.546524in}{0.622001in}}%
\pgfpathlineto{\pgfqpoint{3.296524in}{0.622001in}}%
\pgfpathlineto{\pgfqpoint{3.296524in}{0.534501in}}%
\pgfpathclose%
\pgfusepath{stroke,fill}%
\end{pgfscope}%
\begin{pgfscope}%
\definecolor{textcolor}{rgb}{0.150000,0.150000,0.150000}%
\pgfsetstrokecolor{textcolor}%
\pgfsetfillcolor{textcolor}%
\pgftext[x=3.646524in,y=0.534501in,left,base]{\color{textcolor}\sffamily\fontsize{9.000000}{10.800000}\selectfont Yes but did not finish, 0.0 \%}%
\end{pgfscope}%
\end{pgfpicture}%
\makeatother%
\endgroup%
}
	\caption{previnteractive}
	\label{fig:previnteractive}
\end{figure}

\begin{figure}[H]
	\scalebox{0.72}{%% Creator: Matplotlib, PGF backend
%%
%% To include the figure in your LaTeX document, write
%%   \input{<filename>.pgf}
%%
%% Make sure the required packages are loaded in your preamble
%%   \usepackage{pgf}
%%
%% Also ensure that all the required font packages are loaded; for instance,
%% the lmodern package is sometimes necessary when using math font.
%%   \usepackage{lmodern}
%%
%% Figures using additional raster images can only be included by \input if
%% they are in the same directory as the main LaTeX file. For loading figures
%% from other directories you can use the `import` package
%%   \usepackage{import}
%%
%% and then include the figures with
%%   \import{<path to file>}{<filename>.pgf}
%%
%% Matplotlib used the following preamble
%%   \usepackage{fontspec}
%%   \setmainfont{DejaVuSerif.ttf}[Path=\detokenize{/home/spam/miniconda3/envs/mpl/lib/python3.10/site-packages/matplotlib/mpl-data/fonts/ttf/}]
%%   \setsansfont{DejaVuSans.ttf}[Path=\detokenize{/home/spam/miniconda3/envs/mpl/lib/python3.10/site-packages/matplotlib/mpl-data/fonts/ttf/}]
%%   \setmonofont{DejaVuSansMono.ttf}[Path=\detokenize{/home/spam/miniconda3/envs/mpl/lib/python3.10/site-packages/matplotlib/mpl-data/fonts/ttf/}]
%%
\begingroup%
\makeatletter%
\begin{pgfpicture}%
\pgfpathrectangle{\pgfpointorigin}{\pgfqpoint{5.906660in}{5.000000in}}%
\pgfusepath{use as bounding box, clip}%
\begin{pgfscope}%
\pgfsetbuttcap%
\pgfsetmiterjoin%
\definecolor{currentfill}{rgb}{1.000000,1.000000,1.000000}%
\pgfsetfillcolor{currentfill}%
\pgfsetlinewidth{0.000000pt}%
\definecolor{currentstroke}{rgb}{1.000000,1.000000,1.000000}%
\pgfsetstrokecolor{currentstroke}%
\pgfsetdash{}{0pt}%
\pgfpathmoveto{\pgfqpoint{0.000000in}{0.000000in}}%
\pgfpathlineto{\pgfqpoint{5.906660in}{0.000000in}}%
\pgfpathlineto{\pgfqpoint{5.906660in}{5.000000in}}%
\pgfpathlineto{\pgfqpoint{0.000000in}{5.000000in}}%
\pgfpathlineto{\pgfqpoint{0.000000in}{0.000000in}}%
\pgfpathclose%
\pgfusepath{fill}%
\end{pgfscope}%
\begin{pgfscope}%
\definecolor{textcolor}{rgb}{0.150000,0.150000,0.150000}%
\pgfsetstrokecolor{textcolor}%
\pgfsetfillcolor{textcolor}%
\pgftext[x=3.027163in,y=0.411111in,,top]{\color{textcolor}\sffamily\fontsize{12.000000}{14.400000}\selectfont Computer Science UniversityUniverity Experience}%
\end{pgfscope}%
\begin{pgfscope}%
\pgfsetbuttcap%
\pgfsetmiterjoin%
\definecolor{currentfill}{rgb}{0.552941,0.827451,0.780392}%
\pgfsetfillcolor{currentfill}%
\pgfsetlinewidth{1.003750pt}%
\definecolor{currentstroke}{rgb}{1.000000,1.000000,1.000000}%
\pgfsetstrokecolor{currentstroke}%
\pgfsetdash{}{0pt}%
\pgfpathmoveto{\pgfqpoint{4.567163in}{2.475000in}}%
\pgfpathcurveto{\pgfqpoint{4.567163in}{2.713219in}}{\pgfqpoint{4.511889in}{2.948220in}}{\pgfqpoint{4.405702in}{3.161463in}}%
\pgfpathcurveto{\pgfqpoint{4.299515in}{3.374705in}}{\pgfqpoint{4.145283in}{3.560429in}}{\pgfqpoint{3.955175in}{3.703981in}}%
\pgfpathcurveto{\pgfqpoint{3.765067in}{3.847533in}}{\pgfqpoint{3.544218in}{3.945035in}}{\pgfqpoint{3.310053in}{3.988794in}}%
\pgfpathcurveto{\pgfqpoint{3.075889in}{4.032554in}}{\pgfqpoint{2.834733in}{4.021389in}}{\pgfqpoint{2.605613in}{3.956180in}}%
\pgfpathlineto{\pgfqpoint{3.027163in}{2.475000in}}%
\pgfpathlineto{\pgfqpoint{4.567163in}{2.475000in}}%
\pgfpathlineto{\pgfqpoint{4.567163in}{2.475000in}}%
\pgfpathclose%
\pgfusepath{stroke,fill}%
\end{pgfscope}%
\begin{pgfscope}%
\pgfsetbuttcap%
\pgfsetmiterjoin%
\definecolor{currentfill}{rgb}{1.000000,1.000000,0.701961}%
\pgfsetfillcolor{currentfill}%
\pgfsetlinewidth{1.003750pt}%
\definecolor{currentstroke}{rgb}{1.000000,1.000000,1.000000}%
\pgfsetstrokecolor{currentstroke}%
\pgfsetdash{}{0pt}%
\pgfpathmoveto{\pgfqpoint{2.605613in}{3.956180in}}%
\pgfpathcurveto{\pgfqpoint{2.376493in}{3.890972in}}{\pgfqpoint{2.165597in}{3.773481in}}{\pgfqpoint{1.989567in}{3.612978in}}%
\pgfpathcurveto{\pgfqpoint{1.813536in}{3.452475in}}{\pgfqpoint{1.677124in}{3.253294in}}{\pgfqpoint{1.591094in}{3.031153in}}%
\pgfpathcurveto{\pgfqpoint{1.505064in}{2.809011in}}{\pgfqpoint{1.471740in}{2.569908in}}{\pgfqpoint{1.493751in}{2.332708in}}%
\pgfpathcurveto{\pgfqpoint{1.515762in}{2.095508in}}{\pgfqpoint{1.592513in}{1.866620in}}{\pgfqpoint{1.717949in}{1.664101in}}%
\pgfpathlineto{\pgfqpoint{3.027163in}{2.475000in}}%
\pgfpathlineto{\pgfqpoint{2.605613in}{3.956180in}}%
\pgfpathlineto{\pgfqpoint{2.605613in}{3.956180in}}%
\pgfpathclose%
\pgfusepath{stroke,fill}%
\end{pgfscope}%
\begin{pgfscope}%
\pgfsetbuttcap%
\pgfsetmiterjoin%
\definecolor{currentfill}{rgb}{0.745098,0.729412,0.854902}%
\pgfsetfillcolor{currentfill}%
\pgfsetlinewidth{1.003750pt}%
\definecolor{currentstroke}{rgb}{1.000000,1.000000,1.000000}%
\pgfsetstrokecolor{currentstroke}%
\pgfsetdash{}{0pt}%
\pgfpathmoveto{\pgfqpoint{1.717949in}{1.664101in}}%
\pgfpathcurveto{\pgfqpoint{1.843385in}{1.461582in}}{\pgfqpoint{2.014117in}{1.290903in}}{\pgfqpoint{2.216676in}{1.165531in}}%
\pgfpathcurveto{\pgfqpoint{2.419234in}{1.040158in}}{\pgfqpoint{2.648147in}{0.963479in}}{\pgfqpoint{2.885354in}{0.941543in}}%
\pgfpathcurveto{\pgfqpoint{3.122560in}{0.919607in}}{\pgfqpoint{3.361653in}{0.953006in}}{\pgfqpoint{3.583768in}{1.039106in}}%
\pgfpathcurveto{\pgfqpoint{3.805882in}{1.125206in}}{\pgfqpoint{4.005020in}{1.261680in}}{\pgfqpoint{4.165467in}{1.437762in}}%
\pgfpathlineto{\pgfqpoint{3.027163in}{2.475000in}}%
\pgfpathlineto{\pgfqpoint{1.717949in}{1.664101in}}%
\pgfpathlineto{\pgfqpoint{1.717949in}{1.664101in}}%
\pgfpathclose%
\pgfusepath{stroke,fill}%
\end{pgfscope}%
\begin{pgfscope}%
\pgfsetbuttcap%
\pgfsetmiterjoin%
\definecolor{currentfill}{rgb}{0.984314,0.501961,0.447059}%
\pgfsetfillcolor{currentfill}%
\pgfsetlinewidth{1.003750pt}%
\definecolor{currentstroke}{rgb}{1.000000,1.000000,1.000000}%
\pgfsetstrokecolor{currentstroke}%
\pgfsetdash{}{0pt}%
\pgfpathmoveto{\pgfqpoint{4.165467in}{1.437762in}}%
\pgfpathcurveto{\pgfqpoint{4.165467in}{1.437762in}}{\pgfqpoint{4.165467in}{1.437762in}}{\pgfqpoint{4.165467in}{1.437762in}}%
\pgfpathlineto{\pgfqpoint{3.027163in}{2.475000in}}%
\pgfpathlineto{\pgfqpoint{4.165467in}{1.437762in}}%
\pgfpathlineto{\pgfqpoint{4.165467in}{1.437762in}}%
\pgfpathclose%
\pgfusepath{stroke,fill}%
\end{pgfscope}%
\begin{pgfscope}%
\pgfsetbuttcap%
\pgfsetmiterjoin%
\definecolor{currentfill}{rgb}{0.501961,0.694118,0.827451}%
\pgfsetfillcolor{currentfill}%
\pgfsetlinewidth{1.003750pt}%
\definecolor{currentstroke}{rgb}{1.000000,1.000000,1.000000}%
\pgfsetstrokecolor{currentstroke}%
\pgfsetdash{}{0pt}%
\pgfpathmoveto{\pgfqpoint{4.165467in}{1.437762in}}%
\pgfpathcurveto{\pgfqpoint{4.293579in}{1.578356in}}{\pgfqpoint{4.394541in}{1.741476in}}{\pgfqpoint{4.463232in}{1.918848in}}%
\pgfpathcurveto{\pgfqpoint{4.531924in}{2.096219in}}{\pgfqpoint{4.567163in}{2.284792in}}{\pgfqpoint{4.567163in}{2.475001in}}%
\pgfpathlineto{\pgfqpoint{3.027163in}{2.475000in}}%
\pgfpathlineto{\pgfqpoint{4.165467in}{1.437762in}}%
\pgfpathlineto{\pgfqpoint{4.165467in}{1.437762in}}%
\pgfpathclose%
\pgfusepath{stroke,fill}%
\end{pgfscope}%
\begin{pgfscope}%
\pgfsetbuttcap%
\pgfsetmiterjoin%
\definecolor{currentfill}{rgb}{0.992157,0.705882,0.384314}%
\pgfsetfillcolor{currentfill}%
\pgfsetlinewidth{1.003750pt}%
\definecolor{currentstroke}{rgb}{1.000000,1.000000,1.000000}%
\pgfsetstrokecolor{currentstroke}%
\pgfsetdash{}{0pt}%
\pgfpathmoveto{\pgfqpoint{4.567163in}{2.475001in}}%
\pgfpathcurveto{\pgfqpoint{4.567163in}{2.475001in}}{\pgfqpoint{4.567163in}{2.475001in}}{\pgfqpoint{4.567163in}{2.475001in}}%
\pgfpathlineto{\pgfqpoint{3.027163in}{2.475000in}}%
\pgfpathlineto{\pgfqpoint{4.567163in}{2.475001in}}%
\pgfpathlineto{\pgfqpoint{4.567163in}{2.475001in}}%
\pgfpathclose%
\pgfusepath{stroke,fill}%
\end{pgfscope}%
\begin{pgfscope}%
\definecolor{textcolor}{rgb}{0.150000,0.150000,0.150000}%
\pgfsetstrokecolor{textcolor}%
\pgfsetfillcolor{textcolor}%
\pgftext[x=3.583970in,y=3.212389in,,]{\color{textcolor}\sffamily\fontsize{12.000000}{14.400000}\selectfont 29.4\%}%
\end{pgfscope}%
\begin{pgfscope}%
\definecolor{textcolor}{rgb}{0.150000,0.150000,0.150000}%
\pgfsetstrokecolor{textcolor}%
\pgfsetfillcolor{textcolor}%
\pgftext[x=2.165522in,y=2.808692in,,]{\color{textcolor}\sffamily\fontsize{12.000000}{14.400000}\selectfont 29.4\%}%
\end{pgfscope}%
\begin{pgfscope}%
\definecolor{textcolor}{rgb}{0.150000,0.150000,0.150000}%
\pgfsetstrokecolor{textcolor}%
\pgfsetfillcolor{textcolor}%
\pgftext[x=2.942078in,y=1.554926in,,]{\color{textcolor}\sffamily\fontsize{12.000000}{14.400000}\selectfont 29.4\%}%
\end{pgfscope}%
\begin{pgfscope}%
\definecolor{textcolor}{rgb}{0.150000,0.150000,0.150000}%
\pgfsetstrokecolor{textcolor}%
\pgfsetfillcolor{textcolor}%
\pgftext[x=3.710146in,y=1.852657in,,]{\color{textcolor}\sffamily\fontsize{12.000000}{14.400000}\selectfont 0.0\%}%
\end{pgfscope}%
\begin{pgfscope}%
\definecolor{textcolor}{rgb}{0.150000,0.150000,0.150000}%
\pgfsetstrokecolor{textcolor}%
\pgfsetfillcolor{textcolor}%
\pgftext[x=3.888805in,y=2.141309in,,]{\color{textcolor}\sffamily\fontsize{12.000000}{14.400000}\selectfont 11.8\%}%
\end{pgfscope}%
\begin{pgfscope}%
\definecolor{textcolor}{rgb}{0.150000,0.150000,0.150000}%
\pgfsetstrokecolor{textcolor}%
\pgfsetfillcolor{textcolor}%
\pgftext[x=3.951163in,y=2.475000in,,]{\color{textcolor}\sffamily\fontsize{12.000000}{14.400000}\selectfont 0.0\%}%
\end{pgfscope}%
\begin{pgfscope}%
\pgfsetbuttcap%
\pgfsetmiterjoin%
\definecolor{currentfill}{rgb}{1.000000,1.000000,1.000000}%
\pgfsetfillcolor{currentfill}%
\pgfsetfillopacity{0.800000}%
\pgfsetlinewidth{1.003750pt}%
\definecolor{currentstroke}{rgb}{0.800000,0.800000,0.800000}%
\pgfsetstrokecolor{currentstroke}%
\pgfsetstrokeopacity{0.800000}%
\pgfsetdash{}{0pt}%
\pgfpathmoveto{\pgfqpoint{3.271524in}{0.458500in}}%
\pgfpathlineto{\pgfqpoint{5.827163in}{0.458500in}}%
\pgfpathquadraticcurveto{\pgfqpoint{5.852163in}{0.458500in}}{\pgfqpoint{5.852163in}{0.483500in}}%
\pgfpathlineto{\pgfqpoint{5.852163in}{1.571829in}}%
\pgfpathquadraticcurveto{\pgfqpoint{5.852163in}{1.596829in}}{\pgfqpoint{5.827163in}{1.596829in}}%
\pgfpathlineto{\pgfqpoint{3.271524in}{1.596829in}}%
\pgfpathquadraticcurveto{\pgfqpoint{3.246524in}{1.596829in}}{\pgfqpoint{3.246524in}{1.571829in}}%
\pgfpathlineto{\pgfqpoint{3.246524in}{0.483500in}}%
\pgfpathquadraticcurveto{\pgfqpoint{3.246524in}{0.458500in}}{\pgfqpoint{3.271524in}{0.458500in}}%
\pgfpathlineto{\pgfqpoint{3.271524in}{0.458500in}}%
\pgfpathclose%
\pgfusepath{stroke,fill}%
\end{pgfscope}%
\begin{pgfscope}%
\pgfsetbuttcap%
\pgfsetmiterjoin%
\definecolor{currentfill}{rgb}{0.552941,0.827451,0.780392}%
\pgfsetfillcolor{currentfill}%
\pgfsetlinewidth{1.003750pt}%
\definecolor{currentstroke}{rgb}{1.000000,1.000000,1.000000}%
\pgfsetstrokecolor{currentstroke}%
\pgfsetdash{}{0pt}%
\pgfpathmoveto{\pgfqpoint{3.296524in}{1.451858in}}%
\pgfpathlineto{\pgfqpoint{3.546524in}{1.451858in}}%
\pgfpathlineto{\pgfqpoint{3.546524in}{1.539358in}}%
\pgfpathlineto{\pgfqpoint{3.296524in}{1.539358in}}%
\pgfpathlineto{\pgfqpoint{3.296524in}{1.451858in}}%
\pgfpathclose%
\pgfusepath{stroke,fill}%
\end{pgfscope}%
\begin{pgfscope}%
\definecolor{textcolor}{rgb}{0.150000,0.150000,0.150000}%
\pgfsetstrokecolor{textcolor}%
\pgfsetfillcolor{textcolor}%
\pgftext[x=3.646524in,y=1.451858in,left,base]{\color{textcolor}\sffamily\fontsize{9.000000}{10.800000}\selectfont No CS Degree, 29.4 \%}%
\end{pgfscope}%
\begin{pgfscope}%
\pgfsetbuttcap%
\pgfsetmiterjoin%
\definecolor{currentfill}{rgb}{1.000000,1.000000,0.701961}%
\pgfsetfillcolor{currentfill}%
\pgfsetlinewidth{1.003750pt}%
\definecolor{currentstroke}{rgb}{1.000000,1.000000,1.000000}%
\pgfsetstrokecolor{currentstroke}%
\pgfsetdash{}{0pt}%
\pgfpathmoveto{\pgfqpoint{3.296524in}{1.268387in}}%
\pgfpathlineto{\pgfqpoint{3.546524in}{1.268387in}}%
\pgfpathlineto{\pgfqpoint{3.546524in}{1.355887in}}%
\pgfpathlineto{\pgfqpoint{3.296524in}{1.355887in}}%
\pgfpathlineto{\pgfqpoint{3.296524in}{1.268387in}}%
\pgfpathclose%
\pgfusepath{stroke,fill}%
\end{pgfscope}%
\begin{pgfscope}%
\definecolor{textcolor}{rgb}{0.150000,0.150000,0.150000}%
\pgfsetstrokecolor{textcolor}%
\pgfsetfillcolor{textcolor}%
\pgftext[x=3.646524in,y=1.268387in,left,base]{\color{textcolor}\sffamily\fontsize{9.000000}{10.800000}\selectfont Bachelors's Degree, 29.4 \%}%
\end{pgfscope}%
\begin{pgfscope}%
\pgfsetbuttcap%
\pgfsetmiterjoin%
\definecolor{currentfill}{rgb}{0.745098,0.729412,0.854902}%
\pgfsetfillcolor{currentfill}%
\pgfsetlinewidth{1.003750pt}%
\definecolor{currentstroke}{rgb}{1.000000,1.000000,1.000000}%
\pgfsetstrokecolor{currentstroke}%
\pgfsetdash{}{0pt}%
\pgfpathmoveto{\pgfqpoint{3.296524in}{1.084915in}}%
\pgfpathlineto{\pgfqpoint{3.546524in}{1.084915in}}%
\pgfpathlineto{\pgfqpoint{3.546524in}{1.172415in}}%
\pgfpathlineto{\pgfqpoint{3.296524in}{1.172415in}}%
\pgfpathlineto{\pgfqpoint{3.296524in}{1.084915in}}%
\pgfpathclose%
\pgfusepath{stroke,fill}%
\end{pgfscope}%
\begin{pgfscope}%
\definecolor{textcolor}{rgb}{0.150000,0.150000,0.150000}%
\pgfsetstrokecolor{textcolor}%
\pgfsetfillcolor{textcolor}%
\pgftext[x=3.646524in,y=1.084915in,left,base]{\color{textcolor}\sffamily\fontsize{9.000000}{10.800000}\selectfont Master's Degree, 29.4 \%}%
\end{pgfscope}%
\begin{pgfscope}%
\pgfsetbuttcap%
\pgfsetmiterjoin%
\definecolor{currentfill}{rgb}{0.984314,0.501961,0.447059}%
\pgfsetfillcolor{currentfill}%
\pgfsetlinewidth{1.003750pt}%
\definecolor{currentstroke}{rgb}{1.000000,1.000000,1.000000}%
\pgfsetstrokecolor{currentstroke}%
\pgfsetdash{}{0pt}%
\pgfpathmoveto{\pgfqpoint{3.296524in}{0.901444in}}%
\pgfpathlineto{\pgfqpoint{3.546524in}{0.901444in}}%
\pgfpathlineto{\pgfqpoint{3.546524in}{0.988944in}}%
\pgfpathlineto{\pgfqpoint{3.296524in}{0.988944in}}%
\pgfpathlineto{\pgfqpoint{3.296524in}{0.901444in}}%
\pgfpathclose%
\pgfusepath{stroke,fill}%
\end{pgfscope}%
\begin{pgfscope}%
\definecolor{textcolor}{rgb}{0.150000,0.150000,0.150000}%
\pgfsetstrokecolor{textcolor}%
\pgfsetfillcolor{textcolor}%
\pgftext[x=3.646524in,y=0.901444in,left,base]{\color{textcolor}\sffamily\fontsize{9.000000}{10.800000}\selectfont Doctorate Degree, 0.0 \%}%
\end{pgfscope}%
\begin{pgfscope}%
\pgfsetbuttcap%
\pgfsetmiterjoin%
\definecolor{currentfill}{rgb}{0.501961,0.694118,0.827451}%
\pgfsetfillcolor{currentfill}%
\pgfsetlinewidth{1.003750pt}%
\definecolor{currentstroke}{rgb}{1.000000,1.000000,1.000000}%
\pgfsetstrokecolor{currentstroke}%
\pgfsetdash{}{0pt}%
\pgfpathmoveto{\pgfqpoint{3.296524in}{0.717972in}}%
\pgfpathlineto{\pgfqpoint{3.546524in}{0.717972in}}%
\pgfpathlineto{\pgfqpoint{3.546524in}{0.805472in}}%
\pgfpathlineto{\pgfqpoint{3.296524in}{0.805472in}}%
\pgfpathlineto{\pgfqpoint{3.296524in}{0.717972in}}%
\pgfpathclose%
\pgfusepath{stroke,fill}%
\end{pgfscope}%
\begin{pgfscope}%
\definecolor{textcolor}{rgb}{0.150000,0.150000,0.150000}%
\pgfsetstrokecolor{textcolor}%
\pgfsetfillcolor{textcolor}%
\pgftext[x=3.646524in,y=0.717972in,left,base]{\color{textcolor}\sffamily\fontsize{9.000000}{10.800000}\selectfont Other Engineering Degree, 11.8 \%}%
\end{pgfscope}%
\begin{pgfscope}%
\pgfsetbuttcap%
\pgfsetmiterjoin%
\definecolor{currentfill}{rgb}{0.992157,0.705882,0.384314}%
\pgfsetfillcolor{currentfill}%
\pgfsetlinewidth{1.003750pt}%
\definecolor{currentstroke}{rgb}{1.000000,1.000000,1.000000}%
\pgfsetstrokecolor{currentstroke}%
\pgfsetdash{}{0pt}%
\pgfpathmoveto{\pgfqpoint{3.296524in}{0.534501in}}%
\pgfpathlineto{\pgfqpoint{3.546524in}{0.534501in}}%
\pgfpathlineto{\pgfqpoint{3.546524in}{0.622001in}}%
\pgfpathlineto{\pgfqpoint{3.296524in}{0.622001in}}%
\pgfpathlineto{\pgfqpoint{3.296524in}{0.534501in}}%
\pgfpathclose%
\pgfusepath{stroke,fill}%
\end{pgfscope}%
\begin{pgfscope}%
\definecolor{textcolor}{rgb}{0.150000,0.150000,0.150000}%
\pgfsetstrokecolor{textcolor}%
\pgfsetfillcolor{textcolor}%
\pgftext[x=3.646524in,y=0.534501in,left,base]{\color{textcolor}\sffamily\fontsize{9.000000}{10.800000}\selectfont Yes but did not finish, 0.0 \%}%
\end{pgfscope}%
\end{pgfpicture}%
\makeatother%
\endgroup%
}
	\caption{question test}
	\label{fig:question}
\end{figure}

\begin{figure}[H]
	\centering
	\scalebox{0.67}{%% Creator: Matplotlib, PGF backend
%%
%% To include the figure in your LaTeX document, write
%%   \input{<filename>.pgf}
%%
%% Make sure the required packages are loaded in your preamble
%%   \usepackage{pgf}
%%
%% Also ensure that all the required font packages are loaded; for instance,
%% the lmodern package is sometimes necessary when using math font.
%%   \usepackage{lmodern}
%%
%% Figures using additional raster images can only be included by \input if
%% they are in the same directory as the main LaTeX file. For loading figures
%% from other directories you can use the `import` package
%%   \usepackage{import}
%%
%% and then include the figures with
%%   \import{<path to file>}{<filename>.pgf}
%%
%% Matplotlib used the following preamble
%%   \usepackage{fontspec}
%%   \setmainfont{DejaVuSerif.ttf}[Path=\detokenize{/home/spam/miniconda3/envs/mpl/lib/python3.10/site-packages/matplotlib/mpl-data/fonts/ttf/}]
%%   \setsansfont{DejaVuSans.ttf}[Path=\detokenize{/home/spam/miniconda3/envs/mpl/lib/python3.10/site-packages/matplotlib/mpl-data/fonts/ttf/}]
%%   \setmonofont{DejaVuSansMono.ttf}[Path=\detokenize{/home/spam/miniconda3/envs/mpl/lib/python3.10/site-packages/matplotlib/mpl-data/fonts/ttf/}]
%%
\begingroup%
\makeatletter%
\begin{pgfpicture}%
\pgfpathrectangle{\pgfpointorigin}{\pgfqpoint{5.906660in}{5.000000in}}%
\pgfusepath{use as bounding box, clip}%
\begin{pgfscope}%
\pgfsetbuttcap%
\pgfsetmiterjoin%
\definecolor{currentfill}{rgb}{1.000000,1.000000,1.000000}%
\pgfsetfillcolor{currentfill}%
\pgfsetlinewidth{0.000000pt}%
\definecolor{currentstroke}{rgb}{1.000000,1.000000,1.000000}%
\pgfsetstrokecolor{currentstroke}%
\pgfsetdash{}{0pt}%
\pgfpathmoveto{\pgfqpoint{0.000000in}{0.000000in}}%
\pgfpathlineto{\pgfqpoint{5.906660in}{0.000000in}}%
\pgfpathlineto{\pgfqpoint{5.906660in}{5.000000in}}%
\pgfpathlineto{\pgfqpoint{0.000000in}{5.000000in}}%
\pgfpathlineto{\pgfqpoint{0.000000in}{0.000000in}}%
\pgfpathclose%
\pgfusepath{fill}%
\end{pgfscope}%
\begin{pgfscope}%
\definecolor{textcolor}{rgb}{0.150000,0.150000,0.150000}%
\pgfsetstrokecolor{textcolor}%
\pgfsetfillcolor{textcolor}%
\pgftext[x=3.027163in,y=0.411111in,,top]{\color{textcolor}\sffamily\fontsize{12.000000}{14.400000}\selectfont Computer Science UniversityUniverity Experience}%
\end{pgfscope}%
\begin{pgfscope}%
\pgfsetbuttcap%
\pgfsetmiterjoin%
\definecolor{currentfill}{rgb}{0.552941,0.827451,0.780392}%
\pgfsetfillcolor{currentfill}%
\pgfsetlinewidth{1.003750pt}%
\definecolor{currentstroke}{rgb}{1.000000,1.000000,1.000000}%
\pgfsetstrokecolor{currentstroke}%
\pgfsetdash{}{0pt}%
\pgfpathmoveto{\pgfqpoint{4.567163in}{2.475000in}}%
\pgfpathcurveto{\pgfqpoint{4.567163in}{2.713219in}}{\pgfqpoint{4.511889in}{2.948220in}}{\pgfqpoint{4.405702in}{3.161463in}}%
\pgfpathcurveto{\pgfqpoint{4.299515in}{3.374705in}}{\pgfqpoint{4.145283in}{3.560429in}}{\pgfqpoint{3.955175in}{3.703981in}}%
\pgfpathcurveto{\pgfqpoint{3.765067in}{3.847533in}}{\pgfqpoint{3.544218in}{3.945035in}}{\pgfqpoint{3.310053in}{3.988794in}}%
\pgfpathcurveto{\pgfqpoint{3.075889in}{4.032554in}}{\pgfqpoint{2.834733in}{4.021389in}}{\pgfqpoint{2.605613in}{3.956180in}}%
\pgfpathlineto{\pgfqpoint{3.027163in}{2.475000in}}%
\pgfpathlineto{\pgfqpoint{4.567163in}{2.475000in}}%
\pgfpathlineto{\pgfqpoint{4.567163in}{2.475000in}}%
\pgfpathclose%
\pgfusepath{stroke,fill}%
\end{pgfscope}%
\begin{pgfscope}%
\pgfsetbuttcap%
\pgfsetmiterjoin%
\definecolor{currentfill}{rgb}{1.000000,1.000000,0.701961}%
\pgfsetfillcolor{currentfill}%
\pgfsetlinewidth{1.003750pt}%
\definecolor{currentstroke}{rgb}{1.000000,1.000000,1.000000}%
\pgfsetstrokecolor{currentstroke}%
\pgfsetdash{}{0pt}%
\pgfpathmoveto{\pgfqpoint{2.605613in}{3.956180in}}%
\pgfpathcurveto{\pgfqpoint{2.376493in}{3.890972in}}{\pgfqpoint{2.165597in}{3.773481in}}{\pgfqpoint{1.989567in}{3.612978in}}%
\pgfpathcurveto{\pgfqpoint{1.813536in}{3.452475in}}{\pgfqpoint{1.677124in}{3.253294in}}{\pgfqpoint{1.591094in}{3.031153in}}%
\pgfpathcurveto{\pgfqpoint{1.505064in}{2.809011in}}{\pgfqpoint{1.471740in}{2.569908in}}{\pgfqpoint{1.493751in}{2.332708in}}%
\pgfpathcurveto{\pgfqpoint{1.515762in}{2.095508in}}{\pgfqpoint{1.592513in}{1.866620in}}{\pgfqpoint{1.717949in}{1.664101in}}%
\pgfpathlineto{\pgfqpoint{3.027163in}{2.475000in}}%
\pgfpathlineto{\pgfqpoint{2.605613in}{3.956180in}}%
\pgfpathlineto{\pgfqpoint{2.605613in}{3.956180in}}%
\pgfpathclose%
\pgfusepath{stroke,fill}%
\end{pgfscope}%
\begin{pgfscope}%
\pgfsetbuttcap%
\pgfsetmiterjoin%
\definecolor{currentfill}{rgb}{0.745098,0.729412,0.854902}%
\pgfsetfillcolor{currentfill}%
\pgfsetlinewidth{1.003750pt}%
\definecolor{currentstroke}{rgb}{1.000000,1.000000,1.000000}%
\pgfsetstrokecolor{currentstroke}%
\pgfsetdash{}{0pt}%
\pgfpathmoveto{\pgfqpoint{1.717949in}{1.664101in}}%
\pgfpathcurveto{\pgfqpoint{1.843385in}{1.461582in}}{\pgfqpoint{2.014117in}{1.290903in}}{\pgfqpoint{2.216676in}{1.165531in}}%
\pgfpathcurveto{\pgfqpoint{2.419234in}{1.040158in}}{\pgfqpoint{2.648147in}{0.963479in}}{\pgfqpoint{2.885354in}{0.941543in}}%
\pgfpathcurveto{\pgfqpoint{3.122560in}{0.919607in}}{\pgfqpoint{3.361653in}{0.953006in}}{\pgfqpoint{3.583768in}{1.039106in}}%
\pgfpathcurveto{\pgfqpoint{3.805882in}{1.125206in}}{\pgfqpoint{4.005020in}{1.261680in}}{\pgfqpoint{4.165467in}{1.437762in}}%
\pgfpathlineto{\pgfqpoint{3.027163in}{2.475000in}}%
\pgfpathlineto{\pgfqpoint{1.717949in}{1.664101in}}%
\pgfpathlineto{\pgfqpoint{1.717949in}{1.664101in}}%
\pgfpathclose%
\pgfusepath{stroke,fill}%
\end{pgfscope}%
\begin{pgfscope}%
\pgfsetbuttcap%
\pgfsetmiterjoin%
\definecolor{currentfill}{rgb}{0.984314,0.501961,0.447059}%
\pgfsetfillcolor{currentfill}%
\pgfsetlinewidth{1.003750pt}%
\definecolor{currentstroke}{rgb}{1.000000,1.000000,1.000000}%
\pgfsetstrokecolor{currentstroke}%
\pgfsetdash{}{0pt}%
\pgfpathmoveto{\pgfqpoint{4.165467in}{1.437762in}}%
\pgfpathcurveto{\pgfqpoint{4.165467in}{1.437762in}}{\pgfqpoint{4.165467in}{1.437762in}}{\pgfqpoint{4.165467in}{1.437762in}}%
\pgfpathlineto{\pgfqpoint{3.027163in}{2.475000in}}%
\pgfpathlineto{\pgfqpoint{4.165467in}{1.437762in}}%
\pgfpathlineto{\pgfqpoint{4.165467in}{1.437762in}}%
\pgfpathclose%
\pgfusepath{stroke,fill}%
\end{pgfscope}%
\begin{pgfscope}%
\pgfsetbuttcap%
\pgfsetmiterjoin%
\definecolor{currentfill}{rgb}{0.501961,0.694118,0.827451}%
\pgfsetfillcolor{currentfill}%
\pgfsetlinewidth{1.003750pt}%
\definecolor{currentstroke}{rgb}{1.000000,1.000000,1.000000}%
\pgfsetstrokecolor{currentstroke}%
\pgfsetdash{}{0pt}%
\pgfpathmoveto{\pgfqpoint{4.165467in}{1.437762in}}%
\pgfpathcurveto{\pgfqpoint{4.293579in}{1.578356in}}{\pgfqpoint{4.394541in}{1.741476in}}{\pgfqpoint{4.463232in}{1.918848in}}%
\pgfpathcurveto{\pgfqpoint{4.531924in}{2.096219in}}{\pgfqpoint{4.567163in}{2.284792in}}{\pgfqpoint{4.567163in}{2.475001in}}%
\pgfpathlineto{\pgfqpoint{3.027163in}{2.475000in}}%
\pgfpathlineto{\pgfqpoint{4.165467in}{1.437762in}}%
\pgfpathlineto{\pgfqpoint{4.165467in}{1.437762in}}%
\pgfpathclose%
\pgfusepath{stroke,fill}%
\end{pgfscope}%
\begin{pgfscope}%
\pgfsetbuttcap%
\pgfsetmiterjoin%
\definecolor{currentfill}{rgb}{0.992157,0.705882,0.384314}%
\pgfsetfillcolor{currentfill}%
\pgfsetlinewidth{1.003750pt}%
\definecolor{currentstroke}{rgb}{1.000000,1.000000,1.000000}%
\pgfsetstrokecolor{currentstroke}%
\pgfsetdash{}{0pt}%
\pgfpathmoveto{\pgfqpoint{4.567163in}{2.475001in}}%
\pgfpathcurveto{\pgfqpoint{4.567163in}{2.475001in}}{\pgfqpoint{4.567163in}{2.475001in}}{\pgfqpoint{4.567163in}{2.475001in}}%
\pgfpathlineto{\pgfqpoint{3.027163in}{2.475000in}}%
\pgfpathlineto{\pgfqpoint{4.567163in}{2.475001in}}%
\pgfpathlineto{\pgfqpoint{4.567163in}{2.475001in}}%
\pgfpathclose%
\pgfusepath{stroke,fill}%
\end{pgfscope}%
\begin{pgfscope}%
\definecolor{textcolor}{rgb}{0.150000,0.150000,0.150000}%
\pgfsetstrokecolor{textcolor}%
\pgfsetfillcolor{textcolor}%
\pgftext[x=3.583970in,y=3.212389in,,]{\color{textcolor}\sffamily\fontsize{12.000000}{14.400000}\selectfont 29.4\%}%
\end{pgfscope}%
\begin{pgfscope}%
\definecolor{textcolor}{rgb}{0.150000,0.150000,0.150000}%
\pgfsetstrokecolor{textcolor}%
\pgfsetfillcolor{textcolor}%
\pgftext[x=2.165522in,y=2.808692in,,]{\color{textcolor}\sffamily\fontsize{12.000000}{14.400000}\selectfont 29.4\%}%
\end{pgfscope}%
\begin{pgfscope}%
\definecolor{textcolor}{rgb}{0.150000,0.150000,0.150000}%
\pgfsetstrokecolor{textcolor}%
\pgfsetfillcolor{textcolor}%
\pgftext[x=2.942078in,y=1.554926in,,]{\color{textcolor}\sffamily\fontsize{12.000000}{14.400000}\selectfont 29.4\%}%
\end{pgfscope}%
\begin{pgfscope}%
\definecolor{textcolor}{rgb}{0.150000,0.150000,0.150000}%
\pgfsetstrokecolor{textcolor}%
\pgfsetfillcolor{textcolor}%
\pgftext[x=3.710146in,y=1.852657in,,]{\color{textcolor}\sffamily\fontsize{12.000000}{14.400000}\selectfont 0.0\%}%
\end{pgfscope}%
\begin{pgfscope}%
\definecolor{textcolor}{rgb}{0.150000,0.150000,0.150000}%
\pgfsetstrokecolor{textcolor}%
\pgfsetfillcolor{textcolor}%
\pgftext[x=3.888805in,y=2.141309in,,]{\color{textcolor}\sffamily\fontsize{12.000000}{14.400000}\selectfont 11.8\%}%
\end{pgfscope}%
\begin{pgfscope}%
\definecolor{textcolor}{rgb}{0.150000,0.150000,0.150000}%
\pgfsetstrokecolor{textcolor}%
\pgfsetfillcolor{textcolor}%
\pgftext[x=3.951163in,y=2.475000in,,]{\color{textcolor}\sffamily\fontsize{12.000000}{14.400000}\selectfont 0.0\%}%
\end{pgfscope}%
\begin{pgfscope}%
\pgfsetbuttcap%
\pgfsetmiterjoin%
\definecolor{currentfill}{rgb}{1.000000,1.000000,1.000000}%
\pgfsetfillcolor{currentfill}%
\pgfsetfillopacity{0.800000}%
\pgfsetlinewidth{1.003750pt}%
\definecolor{currentstroke}{rgb}{0.800000,0.800000,0.800000}%
\pgfsetstrokecolor{currentstroke}%
\pgfsetstrokeopacity{0.800000}%
\pgfsetdash{}{0pt}%
\pgfpathmoveto{\pgfqpoint{3.271524in}{0.458500in}}%
\pgfpathlineto{\pgfqpoint{5.827163in}{0.458500in}}%
\pgfpathquadraticcurveto{\pgfqpoint{5.852163in}{0.458500in}}{\pgfqpoint{5.852163in}{0.483500in}}%
\pgfpathlineto{\pgfqpoint{5.852163in}{1.571829in}}%
\pgfpathquadraticcurveto{\pgfqpoint{5.852163in}{1.596829in}}{\pgfqpoint{5.827163in}{1.596829in}}%
\pgfpathlineto{\pgfqpoint{3.271524in}{1.596829in}}%
\pgfpathquadraticcurveto{\pgfqpoint{3.246524in}{1.596829in}}{\pgfqpoint{3.246524in}{1.571829in}}%
\pgfpathlineto{\pgfqpoint{3.246524in}{0.483500in}}%
\pgfpathquadraticcurveto{\pgfqpoint{3.246524in}{0.458500in}}{\pgfqpoint{3.271524in}{0.458500in}}%
\pgfpathlineto{\pgfqpoint{3.271524in}{0.458500in}}%
\pgfpathclose%
\pgfusepath{stroke,fill}%
\end{pgfscope}%
\begin{pgfscope}%
\pgfsetbuttcap%
\pgfsetmiterjoin%
\definecolor{currentfill}{rgb}{0.552941,0.827451,0.780392}%
\pgfsetfillcolor{currentfill}%
\pgfsetlinewidth{1.003750pt}%
\definecolor{currentstroke}{rgb}{1.000000,1.000000,1.000000}%
\pgfsetstrokecolor{currentstroke}%
\pgfsetdash{}{0pt}%
\pgfpathmoveto{\pgfqpoint{3.296524in}{1.451858in}}%
\pgfpathlineto{\pgfqpoint{3.546524in}{1.451858in}}%
\pgfpathlineto{\pgfqpoint{3.546524in}{1.539358in}}%
\pgfpathlineto{\pgfqpoint{3.296524in}{1.539358in}}%
\pgfpathlineto{\pgfqpoint{3.296524in}{1.451858in}}%
\pgfpathclose%
\pgfusepath{stroke,fill}%
\end{pgfscope}%
\begin{pgfscope}%
\definecolor{textcolor}{rgb}{0.150000,0.150000,0.150000}%
\pgfsetstrokecolor{textcolor}%
\pgfsetfillcolor{textcolor}%
\pgftext[x=3.646524in,y=1.451858in,left,base]{\color{textcolor}\sffamily\fontsize{9.000000}{10.800000}\selectfont No CS Degree, 29.4 \%}%
\end{pgfscope}%
\begin{pgfscope}%
\pgfsetbuttcap%
\pgfsetmiterjoin%
\definecolor{currentfill}{rgb}{1.000000,1.000000,0.701961}%
\pgfsetfillcolor{currentfill}%
\pgfsetlinewidth{1.003750pt}%
\definecolor{currentstroke}{rgb}{1.000000,1.000000,1.000000}%
\pgfsetstrokecolor{currentstroke}%
\pgfsetdash{}{0pt}%
\pgfpathmoveto{\pgfqpoint{3.296524in}{1.268387in}}%
\pgfpathlineto{\pgfqpoint{3.546524in}{1.268387in}}%
\pgfpathlineto{\pgfqpoint{3.546524in}{1.355887in}}%
\pgfpathlineto{\pgfqpoint{3.296524in}{1.355887in}}%
\pgfpathlineto{\pgfqpoint{3.296524in}{1.268387in}}%
\pgfpathclose%
\pgfusepath{stroke,fill}%
\end{pgfscope}%
\begin{pgfscope}%
\definecolor{textcolor}{rgb}{0.150000,0.150000,0.150000}%
\pgfsetstrokecolor{textcolor}%
\pgfsetfillcolor{textcolor}%
\pgftext[x=3.646524in,y=1.268387in,left,base]{\color{textcolor}\sffamily\fontsize{9.000000}{10.800000}\selectfont Bachelors's Degree, 29.4 \%}%
\end{pgfscope}%
\begin{pgfscope}%
\pgfsetbuttcap%
\pgfsetmiterjoin%
\definecolor{currentfill}{rgb}{0.745098,0.729412,0.854902}%
\pgfsetfillcolor{currentfill}%
\pgfsetlinewidth{1.003750pt}%
\definecolor{currentstroke}{rgb}{1.000000,1.000000,1.000000}%
\pgfsetstrokecolor{currentstroke}%
\pgfsetdash{}{0pt}%
\pgfpathmoveto{\pgfqpoint{3.296524in}{1.084915in}}%
\pgfpathlineto{\pgfqpoint{3.546524in}{1.084915in}}%
\pgfpathlineto{\pgfqpoint{3.546524in}{1.172415in}}%
\pgfpathlineto{\pgfqpoint{3.296524in}{1.172415in}}%
\pgfpathlineto{\pgfqpoint{3.296524in}{1.084915in}}%
\pgfpathclose%
\pgfusepath{stroke,fill}%
\end{pgfscope}%
\begin{pgfscope}%
\definecolor{textcolor}{rgb}{0.150000,0.150000,0.150000}%
\pgfsetstrokecolor{textcolor}%
\pgfsetfillcolor{textcolor}%
\pgftext[x=3.646524in,y=1.084915in,left,base]{\color{textcolor}\sffamily\fontsize{9.000000}{10.800000}\selectfont Master's Degree, 29.4 \%}%
\end{pgfscope}%
\begin{pgfscope}%
\pgfsetbuttcap%
\pgfsetmiterjoin%
\definecolor{currentfill}{rgb}{0.984314,0.501961,0.447059}%
\pgfsetfillcolor{currentfill}%
\pgfsetlinewidth{1.003750pt}%
\definecolor{currentstroke}{rgb}{1.000000,1.000000,1.000000}%
\pgfsetstrokecolor{currentstroke}%
\pgfsetdash{}{0pt}%
\pgfpathmoveto{\pgfqpoint{3.296524in}{0.901444in}}%
\pgfpathlineto{\pgfqpoint{3.546524in}{0.901444in}}%
\pgfpathlineto{\pgfqpoint{3.546524in}{0.988944in}}%
\pgfpathlineto{\pgfqpoint{3.296524in}{0.988944in}}%
\pgfpathlineto{\pgfqpoint{3.296524in}{0.901444in}}%
\pgfpathclose%
\pgfusepath{stroke,fill}%
\end{pgfscope}%
\begin{pgfscope}%
\definecolor{textcolor}{rgb}{0.150000,0.150000,0.150000}%
\pgfsetstrokecolor{textcolor}%
\pgfsetfillcolor{textcolor}%
\pgftext[x=3.646524in,y=0.901444in,left,base]{\color{textcolor}\sffamily\fontsize{9.000000}{10.800000}\selectfont Doctorate Degree, 0.0 \%}%
\end{pgfscope}%
\begin{pgfscope}%
\pgfsetbuttcap%
\pgfsetmiterjoin%
\definecolor{currentfill}{rgb}{0.501961,0.694118,0.827451}%
\pgfsetfillcolor{currentfill}%
\pgfsetlinewidth{1.003750pt}%
\definecolor{currentstroke}{rgb}{1.000000,1.000000,1.000000}%
\pgfsetstrokecolor{currentstroke}%
\pgfsetdash{}{0pt}%
\pgfpathmoveto{\pgfqpoint{3.296524in}{0.717972in}}%
\pgfpathlineto{\pgfqpoint{3.546524in}{0.717972in}}%
\pgfpathlineto{\pgfqpoint{3.546524in}{0.805472in}}%
\pgfpathlineto{\pgfqpoint{3.296524in}{0.805472in}}%
\pgfpathlineto{\pgfqpoint{3.296524in}{0.717972in}}%
\pgfpathclose%
\pgfusepath{stroke,fill}%
\end{pgfscope}%
\begin{pgfscope}%
\definecolor{textcolor}{rgb}{0.150000,0.150000,0.150000}%
\pgfsetstrokecolor{textcolor}%
\pgfsetfillcolor{textcolor}%
\pgftext[x=3.646524in,y=0.717972in,left,base]{\color{textcolor}\sffamily\fontsize{9.000000}{10.800000}\selectfont Other Engineering Degree, 11.8 \%}%
\end{pgfscope}%
\begin{pgfscope}%
\pgfsetbuttcap%
\pgfsetmiterjoin%
\definecolor{currentfill}{rgb}{0.992157,0.705882,0.384314}%
\pgfsetfillcolor{currentfill}%
\pgfsetlinewidth{1.003750pt}%
\definecolor{currentstroke}{rgb}{1.000000,1.000000,1.000000}%
\pgfsetstrokecolor{currentstroke}%
\pgfsetdash{}{0pt}%
\pgfpathmoveto{\pgfqpoint{3.296524in}{0.534501in}}%
\pgfpathlineto{\pgfqpoint{3.546524in}{0.534501in}}%
\pgfpathlineto{\pgfqpoint{3.546524in}{0.622001in}}%
\pgfpathlineto{\pgfqpoint{3.296524in}{0.622001in}}%
\pgfpathlineto{\pgfqpoint{3.296524in}{0.534501in}}%
\pgfpathclose%
\pgfusepath{stroke,fill}%
\end{pgfscope}%
\begin{pgfscope}%
\definecolor{textcolor}{rgb}{0.150000,0.150000,0.150000}%
\pgfsetstrokecolor{textcolor}%
\pgfsetfillcolor{textcolor}%
\pgftext[x=3.646524in,y=0.534501in,left,base]{\color{textcolor}\sffamily\fontsize{9.000000}{10.800000}\selectfont Yes but did not finish, 0.0 \%}%
\end{pgfscope}%
\end{pgfpicture}%
\makeatother%
\endgroup%
}
	\vspace{-4em}
	\caption{confidence}
	\label{fig:question}
\end{figure}

\begin{figure}[H]
	\centering
	\scalebox{0.67}{%% Creator: Matplotlib, PGF backend
%%
%% To include the figure in your LaTeX document, write
%%   \input{<filename>.pgf}
%%
%% Make sure the required packages are loaded in your preamble
%%   \usepackage{pgf}
%%
%% Also ensure that all the required font packages are loaded; for instance,
%% the lmodern package is sometimes necessary when using math font.
%%   \usepackage{lmodern}
%%
%% Figures using additional raster images can only be included by \input if
%% they are in the same directory as the main LaTeX file. For loading figures
%% from other directories you can use the `import` package
%%   \usepackage{import}
%%
%% and then include the figures with
%%   \import{<path to file>}{<filename>.pgf}
%%
%% Matplotlib used the following preamble
%%   \usepackage{fontspec}
%%   \setmainfont{DejaVuSerif.ttf}[Path=\detokenize{/home/spam/miniconda3/envs/mpl/lib/python3.10/site-packages/matplotlib/mpl-data/fonts/ttf/}]
%%   \setsansfont{DejaVuSans.ttf}[Path=\detokenize{/home/spam/miniconda3/envs/mpl/lib/python3.10/site-packages/matplotlib/mpl-data/fonts/ttf/}]
%%   \setmonofont{DejaVuSansMono.ttf}[Path=\detokenize{/home/spam/miniconda3/envs/mpl/lib/python3.10/site-packages/matplotlib/mpl-data/fonts/ttf/}]
%%
\begingroup%
\makeatletter%
\begin{pgfpicture}%
\pgfpathrectangle{\pgfpointorigin}{\pgfqpoint{5.906660in}{5.000000in}}%
\pgfusepath{use as bounding box, clip}%
\begin{pgfscope}%
\pgfsetbuttcap%
\pgfsetmiterjoin%
\definecolor{currentfill}{rgb}{1.000000,1.000000,1.000000}%
\pgfsetfillcolor{currentfill}%
\pgfsetlinewidth{0.000000pt}%
\definecolor{currentstroke}{rgb}{1.000000,1.000000,1.000000}%
\pgfsetstrokecolor{currentstroke}%
\pgfsetdash{}{0pt}%
\pgfpathmoveto{\pgfqpoint{0.000000in}{0.000000in}}%
\pgfpathlineto{\pgfqpoint{5.906660in}{0.000000in}}%
\pgfpathlineto{\pgfqpoint{5.906660in}{5.000000in}}%
\pgfpathlineto{\pgfqpoint{0.000000in}{5.000000in}}%
\pgfpathlineto{\pgfqpoint{0.000000in}{0.000000in}}%
\pgfpathclose%
\pgfusepath{fill}%
\end{pgfscope}%
\begin{pgfscope}%
\definecolor{textcolor}{rgb}{0.150000,0.150000,0.150000}%
\pgfsetstrokecolor{textcolor}%
\pgfsetfillcolor{textcolor}%
\pgftext[x=3.027163in,y=0.411111in,,top]{\color{textcolor}\sffamily\fontsize{12.000000}{14.400000}\selectfont Computer Science UniversityUniverity Experience}%
\end{pgfscope}%
\begin{pgfscope}%
\pgfsetbuttcap%
\pgfsetmiterjoin%
\definecolor{currentfill}{rgb}{0.552941,0.827451,0.780392}%
\pgfsetfillcolor{currentfill}%
\pgfsetlinewidth{1.003750pt}%
\definecolor{currentstroke}{rgb}{1.000000,1.000000,1.000000}%
\pgfsetstrokecolor{currentstroke}%
\pgfsetdash{}{0pt}%
\pgfpathmoveto{\pgfqpoint{4.567163in}{2.475000in}}%
\pgfpathcurveto{\pgfqpoint{4.567163in}{2.713219in}}{\pgfqpoint{4.511889in}{2.948220in}}{\pgfqpoint{4.405702in}{3.161463in}}%
\pgfpathcurveto{\pgfqpoint{4.299515in}{3.374705in}}{\pgfqpoint{4.145283in}{3.560429in}}{\pgfqpoint{3.955175in}{3.703981in}}%
\pgfpathcurveto{\pgfqpoint{3.765067in}{3.847533in}}{\pgfqpoint{3.544218in}{3.945035in}}{\pgfqpoint{3.310053in}{3.988794in}}%
\pgfpathcurveto{\pgfqpoint{3.075889in}{4.032554in}}{\pgfqpoint{2.834733in}{4.021389in}}{\pgfqpoint{2.605613in}{3.956180in}}%
\pgfpathlineto{\pgfqpoint{3.027163in}{2.475000in}}%
\pgfpathlineto{\pgfqpoint{4.567163in}{2.475000in}}%
\pgfpathlineto{\pgfqpoint{4.567163in}{2.475000in}}%
\pgfpathclose%
\pgfusepath{stroke,fill}%
\end{pgfscope}%
\begin{pgfscope}%
\pgfsetbuttcap%
\pgfsetmiterjoin%
\definecolor{currentfill}{rgb}{1.000000,1.000000,0.701961}%
\pgfsetfillcolor{currentfill}%
\pgfsetlinewidth{1.003750pt}%
\definecolor{currentstroke}{rgb}{1.000000,1.000000,1.000000}%
\pgfsetstrokecolor{currentstroke}%
\pgfsetdash{}{0pt}%
\pgfpathmoveto{\pgfqpoint{2.605613in}{3.956180in}}%
\pgfpathcurveto{\pgfqpoint{2.376493in}{3.890972in}}{\pgfqpoint{2.165597in}{3.773481in}}{\pgfqpoint{1.989567in}{3.612978in}}%
\pgfpathcurveto{\pgfqpoint{1.813536in}{3.452475in}}{\pgfqpoint{1.677124in}{3.253294in}}{\pgfqpoint{1.591094in}{3.031153in}}%
\pgfpathcurveto{\pgfqpoint{1.505064in}{2.809011in}}{\pgfqpoint{1.471740in}{2.569908in}}{\pgfqpoint{1.493751in}{2.332708in}}%
\pgfpathcurveto{\pgfqpoint{1.515762in}{2.095508in}}{\pgfqpoint{1.592513in}{1.866620in}}{\pgfqpoint{1.717949in}{1.664101in}}%
\pgfpathlineto{\pgfqpoint{3.027163in}{2.475000in}}%
\pgfpathlineto{\pgfqpoint{2.605613in}{3.956180in}}%
\pgfpathlineto{\pgfqpoint{2.605613in}{3.956180in}}%
\pgfpathclose%
\pgfusepath{stroke,fill}%
\end{pgfscope}%
\begin{pgfscope}%
\pgfsetbuttcap%
\pgfsetmiterjoin%
\definecolor{currentfill}{rgb}{0.745098,0.729412,0.854902}%
\pgfsetfillcolor{currentfill}%
\pgfsetlinewidth{1.003750pt}%
\definecolor{currentstroke}{rgb}{1.000000,1.000000,1.000000}%
\pgfsetstrokecolor{currentstroke}%
\pgfsetdash{}{0pt}%
\pgfpathmoveto{\pgfqpoint{1.717949in}{1.664101in}}%
\pgfpathcurveto{\pgfqpoint{1.843385in}{1.461582in}}{\pgfqpoint{2.014117in}{1.290903in}}{\pgfqpoint{2.216676in}{1.165531in}}%
\pgfpathcurveto{\pgfqpoint{2.419234in}{1.040158in}}{\pgfqpoint{2.648147in}{0.963479in}}{\pgfqpoint{2.885354in}{0.941543in}}%
\pgfpathcurveto{\pgfqpoint{3.122560in}{0.919607in}}{\pgfqpoint{3.361653in}{0.953006in}}{\pgfqpoint{3.583768in}{1.039106in}}%
\pgfpathcurveto{\pgfqpoint{3.805882in}{1.125206in}}{\pgfqpoint{4.005020in}{1.261680in}}{\pgfqpoint{4.165467in}{1.437762in}}%
\pgfpathlineto{\pgfqpoint{3.027163in}{2.475000in}}%
\pgfpathlineto{\pgfqpoint{1.717949in}{1.664101in}}%
\pgfpathlineto{\pgfqpoint{1.717949in}{1.664101in}}%
\pgfpathclose%
\pgfusepath{stroke,fill}%
\end{pgfscope}%
\begin{pgfscope}%
\pgfsetbuttcap%
\pgfsetmiterjoin%
\definecolor{currentfill}{rgb}{0.984314,0.501961,0.447059}%
\pgfsetfillcolor{currentfill}%
\pgfsetlinewidth{1.003750pt}%
\definecolor{currentstroke}{rgb}{1.000000,1.000000,1.000000}%
\pgfsetstrokecolor{currentstroke}%
\pgfsetdash{}{0pt}%
\pgfpathmoveto{\pgfqpoint{4.165467in}{1.437762in}}%
\pgfpathcurveto{\pgfqpoint{4.165467in}{1.437762in}}{\pgfqpoint{4.165467in}{1.437762in}}{\pgfqpoint{4.165467in}{1.437762in}}%
\pgfpathlineto{\pgfqpoint{3.027163in}{2.475000in}}%
\pgfpathlineto{\pgfqpoint{4.165467in}{1.437762in}}%
\pgfpathlineto{\pgfqpoint{4.165467in}{1.437762in}}%
\pgfpathclose%
\pgfusepath{stroke,fill}%
\end{pgfscope}%
\begin{pgfscope}%
\pgfsetbuttcap%
\pgfsetmiterjoin%
\definecolor{currentfill}{rgb}{0.501961,0.694118,0.827451}%
\pgfsetfillcolor{currentfill}%
\pgfsetlinewidth{1.003750pt}%
\definecolor{currentstroke}{rgb}{1.000000,1.000000,1.000000}%
\pgfsetstrokecolor{currentstroke}%
\pgfsetdash{}{0pt}%
\pgfpathmoveto{\pgfqpoint{4.165467in}{1.437762in}}%
\pgfpathcurveto{\pgfqpoint{4.293579in}{1.578356in}}{\pgfqpoint{4.394541in}{1.741476in}}{\pgfqpoint{4.463232in}{1.918848in}}%
\pgfpathcurveto{\pgfqpoint{4.531924in}{2.096219in}}{\pgfqpoint{4.567163in}{2.284792in}}{\pgfqpoint{4.567163in}{2.475001in}}%
\pgfpathlineto{\pgfqpoint{3.027163in}{2.475000in}}%
\pgfpathlineto{\pgfqpoint{4.165467in}{1.437762in}}%
\pgfpathlineto{\pgfqpoint{4.165467in}{1.437762in}}%
\pgfpathclose%
\pgfusepath{stroke,fill}%
\end{pgfscope}%
\begin{pgfscope}%
\pgfsetbuttcap%
\pgfsetmiterjoin%
\definecolor{currentfill}{rgb}{0.992157,0.705882,0.384314}%
\pgfsetfillcolor{currentfill}%
\pgfsetlinewidth{1.003750pt}%
\definecolor{currentstroke}{rgb}{1.000000,1.000000,1.000000}%
\pgfsetstrokecolor{currentstroke}%
\pgfsetdash{}{0pt}%
\pgfpathmoveto{\pgfqpoint{4.567163in}{2.475001in}}%
\pgfpathcurveto{\pgfqpoint{4.567163in}{2.475001in}}{\pgfqpoint{4.567163in}{2.475001in}}{\pgfqpoint{4.567163in}{2.475001in}}%
\pgfpathlineto{\pgfqpoint{3.027163in}{2.475000in}}%
\pgfpathlineto{\pgfqpoint{4.567163in}{2.475001in}}%
\pgfpathlineto{\pgfqpoint{4.567163in}{2.475001in}}%
\pgfpathclose%
\pgfusepath{stroke,fill}%
\end{pgfscope}%
\begin{pgfscope}%
\definecolor{textcolor}{rgb}{0.150000,0.150000,0.150000}%
\pgfsetstrokecolor{textcolor}%
\pgfsetfillcolor{textcolor}%
\pgftext[x=3.583970in,y=3.212389in,,]{\color{textcolor}\sffamily\fontsize{12.000000}{14.400000}\selectfont 29.4\%}%
\end{pgfscope}%
\begin{pgfscope}%
\definecolor{textcolor}{rgb}{0.150000,0.150000,0.150000}%
\pgfsetstrokecolor{textcolor}%
\pgfsetfillcolor{textcolor}%
\pgftext[x=2.165522in,y=2.808692in,,]{\color{textcolor}\sffamily\fontsize{12.000000}{14.400000}\selectfont 29.4\%}%
\end{pgfscope}%
\begin{pgfscope}%
\definecolor{textcolor}{rgb}{0.150000,0.150000,0.150000}%
\pgfsetstrokecolor{textcolor}%
\pgfsetfillcolor{textcolor}%
\pgftext[x=2.942078in,y=1.554926in,,]{\color{textcolor}\sffamily\fontsize{12.000000}{14.400000}\selectfont 29.4\%}%
\end{pgfscope}%
\begin{pgfscope}%
\definecolor{textcolor}{rgb}{0.150000,0.150000,0.150000}%
\pgfsetstrokecolor{textcolor}%
\pgfsetfillcolor{textcolor}%
\pgftext[x=3.710146in,y=1.852657in,,]{\color{textcolor}\sffamily\fontsize{12.000000}{14.400000}\selectfont 0.0\%}%
\end{pgfscope}%
\begin{pgfscope}%
\definecolor{textcolor}{rgb}{0.150000,0.150000,0.150000}%
\pgfsetstrokecolor{textcolor}%
\pgfsetfillcolor{textcolor}%
\pgftext[x=3.888805in,y=2.141309in,,]{\color{textcolor}\sffamily\fontsize{12.000000}{14.400000}\selectfont 11.8\%}%
\end{pgfscope}%
\begin{pgfscope}%
\definecolor{textcolor}{rgb}{0.150000,0.150000,0.150000}%
\pgfsetstrokecolor{textcolor}%
\pgfsetfillcolor{textcolor}%
\pgftext[x=3.951163in,y=2.475000in,,]{\color{textcolor}\sffamily\fontsize{12.000000}{14.400000}\selectfont 0.0\%}%
\end{pgfscope}%
\begin{pgfscope}%
\pgfsetbuttcap%
\pgfsetmiterjoin%
\definecolor{currentfill}{rgb}{1.000000,1.000000,1.000000}%
\pgfsetfillcolor{currentfill}%
\pgfsetfillopacity{0.800000}%
\pgfsetlinewidth{1.003750pt}%
\definecolor{currentstroke}{rgb}{0.800000,0.800000,0.800000}%
\pgfsetstrokecolor{currentstroke}%
\pgfsetstrokeopacity{0.800000}%
\pgfsetdash{}{0pt}%
\pgfpathmoveto{\pgfqpoint{3.271524in}{0.458500in}}%
\pgfpathlineto{\pgfqpoint{5.827163in}{0.458500in}}%
\pgfpathquadraticcurveto{\pgfqpoint{5.852163in}{0.458500in}}{\pgfqpoint{5.852163in}{0.483500in}}%
\pgfpathlineto{\pgfqpoint{5.852163in}{1.571829in}}%
\pgfpathquadraticcurveto{\pgfqpoint{5.852163in}{1.596829in}}{\pgfqpoint{5.827163in}{1.596829in}}%
\pgfpathlineto{\pgfqpoint{3.271524in}{1.596829in}}%
\pgfpathquadraticcurveto{\pgfqpoint{3.246524in}{1.596829in}}{\pgfqpoint{3.246524in}{1.571829in}}%
\pgfpathlineto{\pgfqpoint{3.246524in}{0.483500in}}%
\pgfpathquadraticcurveto{\pgfqpoint{3.246524in}{0.458500in}}{\pgfqpoint{3.271524in}{0.458500in}}%
\pgfpathlineto{\pgfqpoint{3.271524in}{0.458500in}}%
\pgfpathclose%
\pgfusepath{stroke,fill}%
\end{pgfscope}%
\begin{pgfscope}%
\pgfsetbuttcap%
\pgfsetmiterjoin%
\definecolor{currentfill}{rgb}{0.552941,0.827451,0.780392}%
\pgfsetfillcolor{currentfill}%
\pgfsetlinewidth{1.003750pt}%
\definecolor{currentstroke}{rgb}{1.000000,1.000000,1.000000}%
\pgfsetstrokecolor{currentstroke}%
\pgfsetdash{}{0pt}%
\pgfpathmoveto{\pgfqpoint{3.296524in}{1.451858in}}%
\pgfpathlineto{\pgfqpoint{3.546524in}{1.451858in}}%
\pgfpathlineto{\pgfqpoint{3.546524in}{1.539358in}}%
\pgfpathlineto{\pgfqpoint{3.296524in}{1.539358in}}%
\pgfpathlineto{\pgfqpoint{3.296524in}{1.451858in}}%
\pgfpathclose%
\pgfusepath{stroke,fill}%
\end{pgfscope}%
\begin{pgfscope}%
\definecolor{textcolor}{rgb}{0.150000,0.150000,0.150000}%
\pgfsetstrokecolor{textcolor}%
\pgfsetfillcolor{textcolor}%
\pgftext[x=3.646524in,y=1.451858in,left,base]{\color{textcolor}\sffamily\fontsize{9.000000}{10.800000}\selectfont No CS Degree, 29.4 \%}%
\end{pgfscope}%
\begin{pgfscope}%
\pgfsetbuttcap%
\pgfsetmiterjoin%
\definecolor{currentfill}{rgb}{1.000000,1.000000,0.701961}%
\pgfsetfillcolor{currentfill}%
\pgfsetlinewidth{1.003750pt}%
\definecolor{currentstroke}{rgb}{1.000000,1.000000,1.000000}%
\pgfsetstrokecolor{currentstroke}%
\pgfsetdash{}{0pt}%
\pgfpathmoveto{\pgfqpoint{3.296524in}{1.268387in}}%
\pgfpathlineto{\pgfqpoint{3.546524in}{1.268387in}}%
\pgfpathlineto{\pgfqpoint{3.546524in}{1.355887in}}%
\pgfpathlineto{\pgfqpoint{3.296524in}{1.355887in}}%
\pgfpathlineto{\pgfqpoint{3.296524in}{1.268387in}}%
\pgfpathclose%
\pgfusepath{stroke,fill}%
\end{pgfscope}%
\begin{pgfscope}%
\definecolor{textcolor}{rgb}{0.150000,0.150000,0.150000}%
\pgfsetstrokecolor{textcolor}%
\pgfsetfillcolor{textcolor}%
\pgftext[x=3.646524in,y=1.268387in,left,base]{\color{textcolor}\sffamily\fontsize{9.000000}{10.800000}\selectfont Bachelors's Degree, 29.4 \%}%
\end{pgfscope}%
\begin{pgfscope}%
\pgfsetbuttcap%
\pgfsetmiterjoin%
\definecolor{currentfill}{rgb}{0.745098,0.729412,0.854902}%
\pgfsetfillcolor{currentfill}%
\pgfsetlinewidth{1.003750pt}%
\definecolor{currentstroke}{rgb}{1.000000,1.000000,1.000000}%
\pgfsetstrokecolor{currentstroke}%
\pgfsetdash{}{0pt}%
\pgfpathmoveto{\pgfqpoint{3.296524in}{1.084915in}}%
\pgfpathlineto{\pgfqpoint{3.546524in}{1.084915in}}%
\pgfpathlineto{\pgfqpoint{3.546524in}{1.172415in}}%
\pgfpathlineto{\pgfqpoint{3.296524in}{1.172415in}}%
\pgfpathlineto{\pgfqpoint{3.296524in}{1.084915in}}%
\pgfpathclose%
\pgfusepath{stroke,fill}%
\end{pgfscope}%
\begin{pgfscope}%
\definecolor{textcolor}{rgb}{0.150000,0.150000,0.150000}%
\pgfsetstrokecolor{textcolor}%
\pgfsetfillcolor{textcolor}%
\pgftext[x=3.646524in,y=1.084915in,left,base]{\color{textcolor}\sffamily\fontsize{9.000000}{10.800000}\selectfont Master's Degree, 29.4 \%}%
\end{pgfscope}%
\begin{pgfscope}%
\pgfsetbuttcap%
\pgfsetmiterjoin%
\definecolor{currentfill}{rgb}{0.984314,0.501961,0.447059}%
\pgfsetfillcolor{currentfill}%
\pgfsetlinewidth{1.003750pt}%
\definecolor{currentstroke}{rgb}{1.000000,1.000000,1.000000}%
\pgfsetstrokecolor{currentstroke}%
\pgfsetdash{}{0pt}%
\pgfpathmoveto{\pgfqpoint{3.296524in}{0.901444in}}%
\pgfpathlineto{\pgfqpoint{3.546524in}{0.901444in}}%
\pgfpathlineto{\pgfqpoint{3.546524in}{0.988944in}}%
\pgfpathlineto{\pgfqpoint{3.296524in}{0.988944in}}%
\pgfpathlineto{\pgfqpoint{3.296524in}{0.901444in}}%
\pgfpathclose%
\pgfusepath{stroke,fill}%
\end{pgfscope}%
\begin{pgfscope}%
\definecolor{textcolor}{rgb}{0.150000,0.150000,0.150000}%
\pgfsetstrokecolor{textcolor}%
\pgfsetfillcolor{textcolor}%
\pgftext[x=3.646524in,y=0.901444in,left,base]{\color{textcolor}\sffamily\fontsize{9.000000}{10.800000}\selectfont Doctorate Degree, 0.0 \%}%
\end{pgfscope}%
\begin{pgfscope}%
\pgfsetbuttcap%
\pgfsetmiterjoin%
\definecolor{currentfill}{rgb}{0.501961,0.694118,0.827451}%
\pgfsetfillcolor{currentfill}%
\pgfsetlinewidth{1.003750pt}%
\definecolor{currentstroke}{rgb}{1.000000,1.000000,1.000000}%
\pgfsetstrokecolor{currentstroke}%
\pgfsetdash{}{0pt}%
\pgfpathmoveto{\pgfqpoint{3.296524in}{0.717972in}}%
\pgfpathlineto{\pgfqpoint{3.546524in}{0.717972in}}%
\pgfpathlineto{\pgfqpoint{3.546524in}{0.805472in}}%
\pgfpathlineto{\pgfqpoint{3.296524in}{0.805472in}}%
\pgfpathlineto{\pgfqpoint{3.296524in}{0.717972in}}%
\pgfpathclose%
\pgfusepath{stroke,fill}%
\end{pgfscope}%
\begin{pgfscope}%
\definecolor{textcolor}{rgb}{0.150000,0.150000,0.150000}%
\pgfsetstrokecolor{textcolor}%
\pgfsetfillcolor{textcolor}%
\pgftext[x=3.646524in,y=0.717972in,left,base]{\color{textcolor}\sffamily\fontsize{9.000000}{10.800000}\selectfont Other Engineering Degree, 11.8 \%}%
\end{pgfscope}%
\begin{pgfscope}%
\pgfsetbuttcap%
\pgfsetmiterjoin%
\definecolor{currentfill}{rgb}{0.992157,0.705882,0.384314}%
\pgfsetfillcolor{currentfill}%
\pgfsetlinewidth{1.003750pt}%
\definecolor{currentstroke}{rgb}{1.000000,1.000000,1.000000}%
\pgfsetstrokecolor{currentstroke}%
\pgfsetdash{}{0pt}%
\pgfpathmoveto{\pgfqpoint{3.296524in}{0.534501in}}%
\pgfpathlineto{\pgfqpoint{3.546524in}{0.534501in}}%
\pgfpathlineto{\pgfqpoint{3.546524in}{0.622001in}}%
\pgfpathlineto{\pgfqpoint{3.296524in}{0.622001in}}%
\pgfpathlineto{\pgfqpoint{3.296524in}{0.534501in}}%
\pgfpathclose%
\pgfusepath{stroke,fill}%
\end{pgfscope}%
\begin{pgfscope}%
\definecolor{textcolor}{rgb}{0.150000,0.150000,0.150000}%
\pgfsetstrokecolor{textcolor}%
\pgfsetfillcolor{textcolor}%
\pgftext[x=3.646524in,y=0.534501in,left,base]{\color{textcolor}\sffamily\fontsize{9.000000}{10.800000}\selectfont Yes but did not finish, 0.0 \%}%
\end{pgfscope}%
\end{pgfpicture}%
\makeatother%
\endgroup%
}
	\caption{On average how often do you use command line applications or terminal based tools?}
	\label{fig:question}
\end{figure}

\begin{figure}[H]
	\centering
	\scalebox{0.67}{%% Creator: Matplotlib, PGF backend
%%
%% To include the figure in your LaTeX document, write
%%   \input{<filename>.pgf}
%%
%% Make sure the required packages are loaded in your preamble
%%   \usepackage{pgf}
%%
%% Also ensure that all the required font packages are loaded; for instance,
%% the lmodern package is sometimes necessary when using math font.
%%   \usepackage{lmodern}
%%
%% Figures using additional raster images can only be included by \input if
%% they are in the same directory as the main LaTeX file. For loading figures
%% from other directories you can use the `import` package
%%   \usepackage{import}
%%
%% and then include the figures with
%%   \import{<path to file>}{<filename>.pgf}
%%
%% Matplotlib used the following preamble
%%   \usepackage{fontspec}
%%   \setmainfont{DejaVuSerif.ttf}[Path=\detokenize{/home/spam/miniconda3/envs/mpl/lib/python3.10/site-packages/matplotlib/mpl-data/fonts/ttf/}]
%%   \setsansfont{DejaVuSans.ttf}[Path=\detokenize{/home/spam/miniconda3/envs/mpl/lib/python3.10/site-packages/matplotlib/mpl-data/fonts/ttf/}]
%%   \setmonofont{DejaVuSansMono.ttf}[Path=\detokenize{/home/spam/miniconda3/envs/mpl/lib/python3.10/site-packages/matplotlib/mpl-data/fonts/ttf/}]
%%
\begingroup%
\makeatletter%
\begin{pgfpicture}%
\pgfpathrectangle{\pgfpointorigin}{\pgfqpoint{5.906660in}{5.000000in}}%
\pgfusepath{use as bounding box, clip}%
\begin{pgfscope}%
\pgfsetbuttcap%
\pgfsetmiterjoin%
\definecolor{currentfill}{rgb}{1.000000,1.000000,1.000000}%
\pgfsetfillcolor{currentfill}%
\pgfsetlinewidth{0.000000pt}%
\definecolor{currentstroke}{rgb}{1.000000,1.000000,1.000000}%
\pgfsetstrokecolor{currentstroke}%
\pgfsetdash{}{0pt}%
\pgfpathmoveto{\pgfqpoint{0.000000in}{0.000000in}}%
\pgfpathlineto{\pgfqpoint{5.906660in}{0.000000in}}%
\pgfpathlineto{\pgfqpoint{5.906660in}{5.000000in}}%
\pgfpathlineto{\pgfqpoint{0.000000in}{5.000000in}}%
\pgfpathlineto{\pgfqpoint{0.000000in}{0.000000in}}%
\pgfpathclose%
\pgfusepath{fill}%
\end{pgfscope}%
\begin{pgfscope}%
\definecolor{textcolor}{rgb}{0.150000,0.150000,0.150000}%
\pgfsetstrokecolor{textcolor}%
\pgfsetfillcolor{textcolor}%
\pgftext[x=3.027163in,y=0.411111in,,top]{\color{textcolor}\sffamily\fontsize{12.000000}{14.400000}\selectfont Computer Science UniversityUniverity Experience}%
\end{pgfscope}%
\begin{pgfscope}%
\pgfsetbuttcap%
\pgfsetmiterjoin%
\definecolor{currentfill}{rgb}{0.552941,0.827451,0.780392}%
\pgfsetfillcolor{currentfill}%
\pgfsetlinewidth{1.003750pt}%
\definecolor{currentstroke}{rgb}{1.000000,1.000000,1.000000}%
\pgfsetstrokecolor{currentstroke}%
\pgfsetdash{}{0pt}%
\pgfpathmoveto{\pgfqpoint{4.567163in}{2.475000in}}%
\pgfpathcurveto{\pgfqpoint{4.567163in}{2.713219in}}{\pgfqpoint{4.511889in}{2.948220in}}{\pgfqpoint{4.405702in}{3.161463in}}%
\pgfpathcurveto{\pgfqpoint{4.299515in}{3.374705in}}{\pgfqpoint{4.145283in}{3.560429in}}{\pgfqpoint{3.955175in}{3.703981in}}%
\pgfpathcurveto{\pgfqpoint{3.765067in}{3.847533in}}{\pgfqpoint{3.544218in}{3.945035in}}{\pgfqpoint{3.310053in}{3.988794in}}%
\pgfpathcurveto{\pgfqpoint{3.075889in}{4.032554in}}{\pgfqpoint{2.834733in}{4.021389in}}{\pgfqpoint{2.605613in}{3.956180in}}%
\pgfpathlineto{\pgfqpoint{3.027163in}{2.475000in}}%
\pgfpathlineto{\pgfqpoint{4.567163in}{2.475000in}}%
\pgfpathlineto{\pgfqpoint{4.567163in}{2.475000in}}%
\pgfpathclose%
\pgfusepath{stroke,fill}%
\end{pgfscope}%
\begin{pgfscope}%
\pgfsetbuttcap%
\pgfsetmiterjoin%
\definecolor{currentfill}{rgb}{1.000000,1.000000,0.701961}%
\pgfsetfillcolor{currentfill}%
\pgfsetlinewidth{1.003750pt}%
\definecolor{currentstroke}{rgb}{1.000000,1.000000,1.000000}%
\pgfsetstrokecolor{currentstroke}%
\pgfsetdash{}{0pt}%
\pgfpathmoveto{\pgfqpoint{2.605613in}{3.956180in}}%
\pgfpathcurveto{\pgfqpoint{2.376493in}{3.890972in}}{\pgfqpoint{2.165597in}{3.773481in}}{\pgfqpoint{1.989567in}{3.612978in}}%
\pgfpathcurveto{\pgfqpoint{1.813536in}{3.452475in}}{\pgfqpoint{1.677124in}{3.253294in}}{\pgfqpoint{1.591094in}{3.031153in}}%
\pgfpathcurveto{\pgfqpoint{1.505064in}{2.809011in}}{\pgfqpoint{1.471740in}{2.569908in}}{\pgfqpoint{1.493751in}{2.332708in}}%
\pgfpathcurveto{\pgfqpoint{1.515762in}{2.095508in}}{\pgfqpoint{1.592513in}{1.866620in}}{\pgfqpoint{1.717949in}{1.664101in}}%
\pgfpathlineto{\pgfqpoint{3.027163in}{2.475000in}}%
\pgfpathlineto{\pgfqpoint{2.605613in}{3.956180in}}%
\pgfpathlineto{\pgfqpoint{2.605613in}{3.956180in}}%
\pgfpathclose%
\pgfusepath{stroke,fill}%
\end{pgfscope}%
\begin{pgfscope}%
\pgfsetbuttcap%
\pgfsetmiterjoin%
\definecolor{currentfill}{rgb}{0.745098,0.729412,0.854902}%
\pgfsetfillcolor{currentfill}%
\pgfsetlinewidth{1.003750pt}%
\definecolor{currentstroke}{rgb}{1.000000,1.000000,1.000000}%
\pgfsetstrokecolor{currentstroke}%
\pgfsetdash{}{0pt}%
\pgfpathmoveto{\pgfqpoint{1.717949in}{1.664101in}}%
\pgfpathcurveto{\pgfqpoint{1.843385in}{1.461582in}}{\pgfqpoint{2.014117in}{1.290903in}}{\pgfqpoint{2.216676in}{1.165531in}}%
\pgfpathcurveto{\pgfqpoint{2.419234in}{1.040158in}}{\pgfqpoint{2.648147in}{0.963479in}}{\pgfqpoint{2.885354in}{0.941543in}}%
\pgfpathcurveto{\pgfqpoint{3.122560in}{0.919607in}}{\pgfqpoint{3.361653in}{0.953006in}}{\pgfqpoint{3.583768in}{1.039106in}}%
\pgfpathcurveto{\pgfqpoint{3.805882in}{1.125206in}}{\pgfqpoint{4.005020in}{1.261680in}}{\pgfqpoint{4.165467in}{1.437762in}}%
\pgfpathlineto{\pgfqpoint{3.027163in}{2.475000in}}%
\pgfpathlineto{\pgfqpoint{1.717949in}{1.664101in}}%
\pgfpathlineto{\pgfqpoint{1.717949in}{1.664101in}}%
\pgfpathclose%
\pgfusepath{stroke,fill}%
\end{pgfscope}%
\begin{pgfscope}%
\pgfsetbuttcap%
\pgfsetmiterjoin%
\definecolor{currentfill}{rgb}{0.984314,0.501961,0.447059}%
\pgfsetfillcolor{currentfill}%
\pgfsetlinewidth{1.003750pt}%
\definecolor{currentstroke}{rgb}{1.000000,1.000000,1.000000}%
\pgfsetstrokecolor{currentstroke}%
\pgfsetdash{}{0pt}%
\pgfpathmoveto{\pgfqpoint{4.165467in}{1.437762in}}%
\pgfpathcurveto{\pgfqpoint{4.165467in}{1.437762in}}{\pgfqpoint{4.165467in}{1.437762in}}{\pgfqpoint{4.165467in}{1.437762in}}%
\pgfpathlineto{\pgfqpoint{3.027163in}{2.475000in}}%
\pgfpathlineto{\pgfqpoint{4.165467in}{1.437762in}}%
\pgfpathlineto{\pgfqpoint{4.165467in}{1.437762in}}%
\pgfpathclose%
\pgfusepath{stroke,fill}%
\end{pgfscope}%
\begin{pgfscope}%
\pgfsetbuttcap%
\pgfsetmiterjoin%
\definecolor{currentfill}{rgb}{0.501961,0.694118,0.827451}%
\pgfsetfillcolor{currentfill}%
\pgfsetlinewidth{1.003750pt}%
\definecolor{currentstroke}{rgb}{1.000000,1.000000,1.000000}%
\pgfsetstrokecolor{currentstroke}%
\pgfsetdash{}{0pt}%
\pgfpathmoveto{\pgfqpoint{4.165467in}{1.437762in}}%
\pgfpathcurveto{\pgfqpoint{4.293579in}{1.578356in}}{\pgfqpoint{4.394541in}{1.741476in}}{\pgfqpoint{4.463232in}{1.918848in}}%
\pgfpathcurveto{\pgfqpoint{4.531924in}{2.096219in}}{\pgfqpoint{4.567163in}{2.284792in}}{\pgfqpoint{4.567163in}{2.475001in}}%
\pgfpathlineto{\pgfqpoint{3.027163in}{2.475000in}}%
\pgfpathlineto{\pgfqpoint{4.165467in}{1.437762in}}%
\pgfpathlineto{\pgfqpoint{4.165467in}{1.437762in}}%
\pgfpathclose%
\pgfusepath{stroke,fill}%
\end{pgfscope}%
\begin{pgfscope}%
\pgfsetbuttcap%
\pgfsetmiterjoin%
\definecolor{currentfill}{rgb}{0.992157,0.705882,0.384314}%
\pgfsetfillcolor{currentfill}%
\pgfsetlinewidth{1.003750pt}%
\definecolor{currentstroke}{rgb}{1.000000,1.000000,1.000000}%
\pgfsetstrokecolor{currentstroke}%
\pgfsetdash{}{0pt}%
\pgfpathmoveto{\pgfqpoint{4.567163in}{2.475001in}}%
\pgfpathcurveto{\pgfqpoint{4.567163in}{2.475001in}}{\pgfqpoint{4.567163in}{2.475001in}}{\pgfqpoint{4.567163in}{2.475001in}}%
\pgfpathlineto{\pgfqpoint{3.027163in}{2.475000in}}%
\pgfpathlineto{\pgfqpoint{4.567163in}{2.475001in}}%
\pgfpathlineto{\pgfqpoint{4.567163in}{2.475001in}}%
\pgfpathclose%
\pgfusepath{stroke,fill}%
\end{pgfscope}%
\begin{pgfscope}%
\definecolor{textcolor}{rgb}{0.150000,0.150000,0.150000}%
\pgfsetstrokecolor{textcolor}%
\pgfsetfillcolor{textcolor}%
\pgftext[x=3.583970in,y=3.212389in,,]{\color{textcolor}\sffamily\fontsize{12.000000}{14.400000}\selectfont 29.4\%}%
\end{pgfscope}%
\begin{pgfscope}%
\definecolor{textcolor}{rgb}{0.150000,0.150000,0.150000}%
\pgfsetstrokecolor{textcolor}%
\pgfsetfillcolor{textcolor}%
\pgftext[x=2.165522in,y=2.808692in,,]{\color{textcolor}\sffamily\fontsize{12.000000}{14.400000}\selectfont 29.4\%}%
\end{pgfscope}%
\begin{pgfscope}%
\definecolor{textcolor}{rgb}{0.150000,0.150000,0.150000}%
\pgfsetstrokecolor{textcolor}%
\pgfsetfillcolor{textcolor}%
\pgftext[x=2.942078in,y=1.554926in,,]{\color{textcolor}\sffamily\fontsize{12.000000}{14.400000}\selectfont 29.4\%}%
\end{pgfscope}%
\begin{pgfscope}%
\definecolor{textcolor}{rgb}{0.150000,0.150000,0.150000}%
\pgfsetstrokecolor{textcolor}%
\pgfsetfillcolor{textcolor}%
\pgftext[x=3.710146in,y=1.852657in,,]{\color{textcolor}\sffamily\fontsize{12.000000}{14.400000}\selectfont 0.0\%}%
\end{pgfscope}%
\begin{pgfscope}%
\definecolor{textcolor}{rgb}{0.150000,0.150000,0.150000}%
\pgfsetstrokecolor{textcolor}%
\pgfsetfillcolor{textcolor}%
\pgftext[x=3.888805in,y=2.141309in,,]{\color{textcolor}\sffamily\fontsize{12.000000}{14.400000}\selectfont 11.8\%}%
\end{pgfscope}%
\begin{pgfscope}%
\definecolor{textcolor}{rgb}{0.150000,0.150000,0.150000}%
\pgfsetstrokecolor{textcolor}%
\pgfsetfillcolor{textcolor}%
\pgftext[x=3.951163in,y=2.475000in,,]{\color{textcolor}\sffamily\fontsize{12.000000}{14.400000}\selectfont 0.0\%}%
\end{pgfscope}%
\begin{pgfscope}%
\pgfsetbuttcap%
\pgfsetmiterjoin%
\definecolor{currentfill}{rgb}{1.000000,1.000000,1.000000}%
\pgfsetfillcolor{currentfill}%
\pgfsetfillopacity{0.800000}%
\pgfsetlinewidth{1.003750pt}%
\definecolor{currentstroke}{rgb}{0.800000,0.800000,0.800000}%
\pgfsetstrokecolor{currentstroke}%
\pgfsetstrokeopacity{0.800000}%
\pgfsetdash{}{0pt}%
\pgfpathmoveto{\pgfqpoint{3.271524in}{0.458500in}}%
\pgfpathlineto{\pgfqpoint{5.827163in}{0.458500in}}%
\pgfpathquadraticcurveto{\pgfqpoint{5.852163in}{0.458500in}}{\pgfqpoint{5.852163in}{0.483500in}}%
\pgfpathlineto{\pgfqpoint{5.852163in}{1.571829in}}%
\pgfpathquadraticcurveto{\pgfqpoint{5.852163in}{1.596829in}}{\pgfqpoint{5.827163in}{1.596829in}}%
\pgfpathlineto{\pgfqpoint{3.271524in}{1.596829in}}%
\pgfpathquadraticcurveto{\pgfqpoint{3.246524in}{1.596829in}}{\pgfqpoint{3.246524in}{1.571829in}}%
\pgfpathlineto{\pgfqpoint{3.246524in}{0.483500in}}%
\pgfpathquadraticcurveto{\pgfqpoint{3.246524in}{0.458500in}}{\pgfqpoint{3.271524in}{0.458500in}}%
\pgfpathlineto{\pgfqpoint{3.271524in}{0.458500in}}%
\pgfpathclose%
\pgfusepath{stroke,fill}%
\end{pgfscope}%
\begin{pgfscope}%
\pgfsetbuttcap%
\pgfsetmiterjoin%
\definecolor{currentfill}{rgb}{0.552941,0.827451,0.780392}%
\pgfsetfillcolor{currentfill}%
\pgfsetlinewidth{1.003750pt}%
\definecolor{currentstroke}{rgb}{1.000000,1.000000,1.000000}%
\pgfsetstrokecolor{currentstroke}%
\pgfsetdash{}{0pt}%
\pgfpathmoveto{\pgfqpoint{3.296524in}{1.451858in}}%
\pgfpathlineto{\pgfqpoint{3.546524in}{1.451858in}}%
\pgfpathlineto{\pgfqpoint{3.546524in}{1.539358in}}%
\pgfpathlineto{\pgfqpoint{3.296524in}{1.539358in}}%
\pgfpathlineto{\pgfqpoint{3.296524in}{1.451858in}}%
\pgfpathclose%
\pgfusepath{stroke,fill}%
\end{pgfscope}%
\begin{pgfscope}%
\definecolor{textcolor}{rgb}{0.150000,0.150000,0.150000}%
\pgfsetstrokecolor{textcolor}%
\pgfsetfillcolor{textcolor}%
\pgftext[x=3.646524in,y=1.451858in,left,base]{\color{textcolor}\sffamily\fontsize{9.000000}{10.800000}\selectfont No CS Degree, 29.4 \%}%
\end{pgfscope}%
\begin{pgfscope}%
\pgfsetbuttcap%
\pgfsetmiterjoin%
\definecolor{currentfill}{rgb}{1.000000,1.000000,0.701961}%
\pgfsetfillcolor{currentfill}%
\pgfsetlinewidth{1.003750pt}%
\definecolor{currentstroke}{rgb}{1.000000,1.000000,1.000000}%
\pgfsetstrokecolor{currentstroke}%
\pgfsetdash{}{0pt}%
\pgfpathmoveto{\pgfqpoint{3.296524in}{1.268387in}}%
\pgfpathlineto{\pgfqpoint{3.546524in}{1.268387in}}%
\pgfpathlineto{\pgfqpoint{3.546524in}{1.355887in}}%
\pgfpathlineto{\pgfqpoint{3.296524in}{1.355887in}}%
\pgfpathlineto{\pgfqpoint{3.296524in}{1.268387in}}%
\pgfpathclose%
\pgfusepath{stroke,fill}%
\end{pgfscope}%
\begin{pgfscope}%
\definecolor{textcolor}{rgb}{0.150000,0.150000,0.150000}%
\pgfsetstrokecolor{textcolor}%
\pgfsetfillcolor{textcolor}%
\pgftext[x=3.646524in,y=1.268387in,left,base]{\color{textcolor}\sffamily\fontsize{9.000000}{10.800000}\selectfont Bachelors's Degree, 29.4 \%}%
\end{pgfscope}%
\begin{pgfscope}%
\pgfsetbuttcap%
\pgfsetmiterjoin%
\definecolor{currentfill}{rgb}{0.745098,0.729412,0.854902}%
\pgfsetfillcolor{currentfill}%
\pgfsetlinewidth{1.003750pt}%
\definecolor{currentstroke}{rgb}{1.000000,1.000000,1.000000}%
\pgfsetstrokecolor{currentstroke}%
\pgfsetdash{}{0pt}%
\pgfpathmoveto{\pgfqpoint{3.296524in}{1.084915in}}%
\pgfpathlineto{\pgfqpoint{3.546524in}{1.084915in}}%
\pgfpathlineto{\pgfqpoint{3.546524in}{1.172415in}}%
\pgfpathlineto{\pgfqpoint{3.296524in}{1.172415in}}%
\pgfpathlineto{\pgfqpoint{3.296524in}{1.084915in}}%
\pgfpathclose%
\pgfusepath{stroke,fill}%
\end{pgfscope}%
\begin{pgfscope}%
\definecolor{textcolor}{rgb}{0.150000,0.150000,0.150000}%
\pgfsetstrokecolor{textcolor}%
\pgfsetfillcolor{textcolor}%
\pgftext[x=3.646524in,y=1.084915in,left,base]{\color{textcolor}\sffamily\fontsize{9.000000}{10.800000}\selectfont Master's Degree, 29.4 \%}%
\end{pgfscope}%
\begin{pgfscope}%
\pgfsetbuttcap%
\pgfsetmiterjoin%
\definecolor{currentfill}{rgb}{0.984314,0.501961,0.447059}%
\pgfsetfillcolor{currentfill}%
\pgfsetlinewidth{1.003750pt}%
\definecolor{currentstroke}{rgb}{1.000000,1.000000,1.000000}%
\pgfsetstrokecolor{currentstroke}%
\pgfsetdash{}{0pt}%
\pgfpathmoveto{\pgfqpoint{3.296524in}{0.901444in}}%
\pgfpathlineto{\pgfqpoint{3.546524in}{0.901444in}}%
\pgfpathlineto{\pgfqpoint{3.546524in}{0.988944in}}%
\pgfpathlineto{\pgfqpoint{3.296524in}{0.988944in}}%
\pgfpathlineto{\pgfqpoint{3.296524in}{0.901444in}}%
\pgfpathclose%
\pgfusepath{stroke,fill}%
\end{pgfscope}%
\begin{pgfscope}%
\definecolor{textcolor}{rgb}{0.150000,0.150000,0.150000}%
\pgfsetstrokecolor{textcolor}%
\pgfsetfillcolor{textcolor}%
\pgftext[x=3.646524in,y=0.901444in,left,base]{\color{textcolor}\sffamily\fontsize{9.000000}{10.800000}\selectfont Doctorate Degree, 0.0 \%}%
\end{pgfscope}%
\begin{pgfscope}%
\pgfsetbuttcap%
\pgfsetmiterjoin%
\definecolor{currentfill}{rgb}{0.501961,0.694118,0.827451}%
\pgfsetfillcolor{currentfill}%
\pgfsetlinewidth{1.003750pt}%
\definecolor{currentstroke}{rgb}{1.000000,1.000000,1.000000}%
\pgfsetstrokecolor{currentstroke}%
\pgfsetdash{}{0pt}%
\pgfpathmoveto{\pgfqpoint{3.296524in}{0.717972in}}%
\pgfpathlineto{\pgfqpoint{3.546524in}{0.717972in}}%
\pgfpathlineto{\pgfqpoint{3.546524in}{0.805472in}}%
\pgfpathlineto{\pgfqpoint{3.296524in}{0.805472in}}%
\pgfpathlineto{\pgfqpoint{3.296524in}{0.717972in}}%
\pgfpathclose%
\pgfusepath{stroke,fill}%
\end{pgfscope}%
\begin{pgfscope}%
\definecolor{textcolor}{rgb}{0.150000,0.150000,0.150000}%
\pgfsetstrokecolor{textcolor}%
\pgfsetfillcolor{textcolor}%
\pgftext[x=3.646524in,y=0.717972in,left,base]{\color{textcolor}\sffamily\fontsize{9.000000}{10.800000}\selectfont Other Engineering Degree, 11.8 \%}%
\end{pgfscope}%
\begin{pgfscope}%
\pgfsetbuttcap%
\pgfsetmiterjoin%
\definecolor{currentfill}{rgb}{0.992157,0.705882,0.384314}%
\pgfsetfillcolor{currentfill}%
\pgfsetlinewidth{1.003750pt}%
\definecolor{currentstroke}{rgb}{1.000000,1.000000,1.000000}%
\pgfsetstrokecolor{currentstroke}%
\pgfsetdash{}{0pt}%
\pgfpathmoveto{\pgfqpoint{3.296524in}{0.534501in}}%
\pgfpathlineto{\pgfqpoint{3.546524in}{0.534501in}}%
\pgfpathlineto{\pgfqpoint{3.546524in}{0.622001in}}%
\pgfpathlineto{\pgfqpoint{3.296524in}{0.622001in}}%
\pgfpathlineto{\pgfqpoint{3.296524in}{0.534501in}}%
\pgfpathclose%
\pgfusepath{stroke,fill}%
\end{pgfscope}%
\begin{pgfscope}%
\definecolor{textcolor}{rgb}{0.150000,0.150000,0.150000}%
\pgfsetstrokecolor{textcolor}%
\pgfsetfillcolor{textcolor}%
\pgftext[x=3.646524in,y=0.534501in,left,base]{\color{textcolor}\sffamily\fontsize{9.000000}{10.800000}\selectfont Yes but did not finish, 0.0 \%}%
\end{pgfscope}%
\end{pgfpicture}%
\makeatother%
\endgroup%
}
	\caption{On average how often do you use command line applications or terminal based tools?}
	\label{fig:question}
\end{figure}

\begin{table}[htbp]\centering
\begin{tabular}{|c|c|c|}
\hline
\textbf{Question}                                                                                                                                                      & \begin{tabular}[c]{@{}c@{}}Interactive\\ Correct \%\end{tabular} & \begin{tabular}[c]{@{}c@{}}Non Interactive \\ Correct \%\end{tabular} \\ \hline
\textbf{What does CLI stand for?}                                                                                                                                      & 100.00\%                                                         & 100.00\%                                                              \\ \hline
\textbf{\begin{tabular}[c]{@{}c@{}}Which one of the following best describes \\ the role of flags\\  when issuing a command?\end{tabular}}                             & 63.16\%                                                          & 46.67\%                                                               \\ \hline
\textbf{\begin{tabular}[c]{@{}c@{}}In your own words can you describe \\ what the shell is?\end{tabular}}                                                              & 94.28\%                                                          & 73.34\%                                                               \\ \hline
\textbf{\begin{tabular}[c]{@{}c@{}}How would you count the number of lines in\\  a given file with the word count utility?\end{tabular}}                               & 100.00\%                                                         & 93.34\%                                                               \\ \hline
\textbf{\begin{tabular}[c]{@{}c@{}}Which one of the following flow diagrams\\  best describes textual interaction \\ with the operating system\end{tabular}}           & 63.16\%                                                          & 60.00\%                                                               \\ \hline
\textbf{\begin{tabular}[c]{@{}c@{}}Which of the following is an\\  incorrect usage of flags?\end{tabular}}                                                             & 52.64\%                                                          & 20.00\%                                                               \\ \hline
\textbf{What is the role of the prompt?}                                                                                                                               & 52.63\%                                                          & 60.00\%                                                               \\ \hline
\textbf{\begin{tabular}[c]{@{}c@{}}What is the command to see where you are \\ on your file system and\\  what does the abbreviation stand for?\end{tabular}}          & 94.28\%                                                          & 66.67\%                                                               \\ \hline
\textbf{\begin{tabular}[c]{@{}c@{}}Which of the following structures\\  describes the file system best?\end{tabular}}                                                  & 94.28\%                                                          & 80.00\%                                                               \\ \hline
\textbf{\begin{tabular}[c]{@{}c@{}}What command do you use to \\ list the contents of your current directory?\end{tabular}}                                            & 94.28\%                                                          & 100.00\%                                                              \\ \hline
\textbf{\begin{tabular}[c]{@{}c@{}}Can you explain in what situations\\  the rmdir command\\  will not delete a directory?\end{tabular}}                               & 78.95\%                                                          & 86.67\%                                                               \\ \hline
\textbf{\begin{tabular}[c]{@{}c@{}}What shell keyboard shortcut \\ cancels a command or input?\end{tabular}}                                                           & 89.47\%                                                          & 93.34\%                                                               \\ \hline
\textbf{\begin{tabular}[c]{@{}c@{}}What is the name of the command that brings up \\ documentation about a command? \\ What does this command stand for?\end{tabular}} & 89.47\%                                                          & 80.47\%                                                               \\ \hline
\textbf{Overall Score}                                                                                                                                                 & 82.19\%                                                          & 72.85\%                                                               \\ \hline
\end{tabular}
\caption{Summary of questions answered correctly by method during the evaluation phase.}
\label{tab:evaluation}
\end{table}

