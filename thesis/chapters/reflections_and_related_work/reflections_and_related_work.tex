\chapter{Reflections and Related Work}
% - Related Work and Reflections
%   - related work
%   - future work
%       - improvements to CLI-Tutor
%       - areas to explore in the space
\label{chap:reflection}
\section{Related Work}

Interactive learning tools are not a new area of research in the software
development field. Tutorial software for programming languages has existed  and
been researched for decades \cite{anderson1985lisp,
	anderson1986automatedtutoring, gerdes2012interactive,
	permpool2019interactive, lee2005intelligent, jeuring2011programming,
	holland2009j,  schez2020intelligent}.

Interactive tutorial tools in the software development space are primarily
built for the purpose of introducing specific programming languages
\cite{donehaskell, holland2009j, ajayi2010development, lee2005intelligent}.
Pillay et al. laid out a framework for how to develop programming language
tutors in 2003 \cite{pillay2003developing}. However, Due to the recent growth
of remote learning and remote working, asynchronous learning and computer based
learning are seeing a fresh surge in research. Not only are intelligent
tutoring systems being developed and tested against traditional learning media
but also against out asynchronous computer based systems such as video lectures
\cite{becker201950, ossovski2022comparing}.

There are also a handful of interactive tutorial tools that are built primarily
for the web. \cite{donehaskell, harris_team, harris_team, herweijer,
lee2005intelligent}. This category of tools often go beyond the realm of
programming languages and often serve as interactive elements in the official
documentation of certain web technologies, such as with
\cite{harris_team,team_meta}. Another interesting aspect of these web based
tools is that they try to implement a sandboxed environment to make the tools
more accessible and lower the barrier to use. This sandboxing has long been
identified as a pragmatic deign choice with research dating as far back as the
1985 work of Anderson et al. \cite{anderson1985lisp}, \textit{LISP TUTOR}, one
of the earliest interactive programming tutors. Another more recent work
looking into this sandboxing is \cite{permpool2019interactive}.

\section{Improvements}

The user study and the associated feedback provided useful insights into how subsequent versions of \textit{CLI-Tutor} could be improved. Some users suggested that adopting a more hybrid approach between the interactive and non-interactive tools could improve \textit{CLI-Tutor}.

\begin{quotes}
	It's a great tool, feels great to use. Personally i prefer guide books / video guides simply
	because going back and forth to look for desired info is easier than a more linear interactive
	approach but i can greatly appreciate such "learning games", and would love to see more of it
\end{quotes}

\begin{quotes}
	Still i prefer to be able to quickly look up something in a textbook / scroll back in a video,
	because it allows to connect information more intricately and more adjusted to my way of
	storing / recalling information.
\end{quotes}

These suggestions could improve the utility of \textit{CLI-Tutor} as a
reference tool and allow for continued use of the tool beyond taking the first
lesson. This hybrid documentation and interactive sandbox approach is a design
style that is seeing an increase in popularity recently especially in the web
development documentation world. A very popular  \textit{JavaScript} framework,
\textit{React}\footnote{\url{https://reactjs.org/}} has recently released a
beta version of their documentation that employs exactly this hybrid approach
\cite{team_meta}. Another popular \textit{JavaScript} framework,
\textit{Svelte}\footnote{\url{https://svelte.dev/}}, has also employed a similar
hybrid approach for their introductory tutorials \cite{harris_team}.


Other areas for potential improvement to \textit{CLI-Tutor} include:

\begin{itemize}

	\item  \textit{Dynamic feedback}: Integrating some form of feedback that
	      dynamically adjusts according to the user's input or offers some
	      intelligent suggestions. Such an approach would have the potential to
	      offer more personalised and specific feedback than the approach in the
	      current version of \textit{CLI-Tutor}. \cite{keuning2014strategy,
		      gerdes2017ask, rivers2017data} are all works targeting this domain of
	      intelligent feedback generation.

	\item \textit{Mobile version}: A mobile friendly version of
	      \textit{CLI-Tutor}, particularly the web application, would have the
	      potential of widening the potential user group and simultaneously
	      reducing the barrier to entry for \textit{CLI-Tutor}. In order to
	      implement this, some attention to the sizing of elements in the web
	      application would be necessary. Furthermore, a mobile friendly
	      version would necessitate the integration of a modified on-screen
	      keyboard to ease interaction with the tool, as it is primarily
	      keyboard driven.

	\item \textit{Progress tracking}: The integration of some sort of
	      progress tracking would improve the user experience of
	      \textit{CLI-Tutor}. This could open the doors to elements such as
	      gameificaton or performance analytics and even more refined feedback
	      to be integrated. Implementing statefulness in \textit{CLI-Tutor}
	      could potentially be achieved using lightweight client-side databases
	      such as \textit{sqlite}\footnote{\url{https://sqlite.org/}}.

\end{itemize}


\section{Future Work}
Building upon this work can provide valuable insights into how to make the
command line more approachable and on how to build better interactive learning
software in general. The addition of some suggested improvements in the
preceding chapter can open the doors to a wide array of future research in
this domain. Furthermore, the user study in this work could benefit from a
reproduction with a larger number of participants. Another potential
modification for future research is could be to perform a user study with
absolute beginners, or individuals with no programming or command line
experience.

Another avenue of research could be to look into more advanced topics and see if
the benefits of an interactive learning environment still reflect when the
subject matter is more complicated or targeted at more advanced users.
