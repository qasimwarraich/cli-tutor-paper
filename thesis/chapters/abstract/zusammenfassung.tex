% TODO: PROPERLY TRANSLATE THIS 
\begin{zusammenfassung}
\sloppy

Trotz des wohl veralteten Aussehens, schwierige Lernkurve und praktisch
Nichtexistenz im allgemeinen Personal-Computing-Bereich,
Befehlszeilenschnittstellen (CLIs) haben sich in der Welt der
Softwareentwicklung mehr als bewährt. Es gibt eine Vielzahl äußerst beliebter
Tools und Anwendungen, die konzentrieren sich in erster Linie auf die CLI als
Interaktionsmedium. Einige Beispiele enthalten Versionskontrollsoftware wie
`git', Compiler und Interpreter für Programmiersprachen, Paketmanager und
verschiedene Kerndienstprogramme beliebt in Bereichen wie Softwareentwicklung,
Scripting und System Verwaltung.

Wie bereits erwähnt, hat die Verwendung der Befehlszeile als
Interaktionsparadigma praktisch aus der Perspektive der Mainstream-PC-Nutzung
verschwunden. Dies trägt stark zum Einschüchterungsfaktor und zur
Lernschwierigkeit bei die daran interessiert sind, in die Softwareentwicklung
oder Systemadministration einzusteigen. Diese Ungewohntheit, gepaart mit der
Zwangsläufigkeit der Nutzung von CLIs im Der Entwicklungsbereich unterstreicht
die Notwendigkeit, die Befehlszeile zugänglicher zu machen neue Benutzer, für
die die textbasierte Interaktion mit ihrem Computer ein Fremdwort ist Konzept.
In den letzten Jahren interaktives Lernen unter Verwendung von Tools wie
Sandboxed Umgebungen erfreuen sich zunehmender Beliebtheit und haben das
Potenzial, a geeignetes Medium zum Erlernen der CLI durch praktische Anwendung,
Beispiele und üben.

\end{zusammenfassung}
