\begin{abstract}

Despite the arguably dated appearance, difficult learning curve and practical
non-existence in the modern personal computing space, Command Line Interfaces
(CLIs) have more than stood the test of time in the software development world.
There are a multitude of extremely popular tools and applications that
primarily focus on the command line as an interaction medium. Some examples
include version control software like \textit{git}\footnote{git source code version
control tool: \href{https://git-scm.com/}{https://git-scm.com/} }, compilers
and interpreters for programming languages, package managers and various core
utilities that are popular in areas such as software development, scripting and
system administration. Command line interfaces are also utilised in areas
outside of software development. For example, the infamous Bloomberg Terminal in
the financial sector and in general computing applications such as email e.g. \textit{mutt,
neomutt} and text editing e.g. \textit{Vim, Neovim, Wordstar}. 
%should I add foot notes for all these applications?

As mentioned before, the use of the command line as an interaction paradigm has
effectively disappeared from a mainstream personal computer usage perspective.
This reality contributes greatly to the intimidation factor and learning difficulty for
those interested in getting into software engineering or system administration.
This unfamiliarity, paired with the inevitability of usage of CLIs in the
development space, highlights a need to make the command line more accessible to
new users for whom text-based interaction with their computer is an alien
concept. In recent years, interactive learning tools utilising features such as sandboxed
environments have been gaining in popularity and have the potential to be a
suitable medium for learning command line basics through actual usage, examples
and practice.

In this work, we have created just such an interactive tutoring tool tailored
for the command line. \textit{CLI-Tutor} is a forgiving CLI application that aims to
teach topics such as shell basics and Unix-like core utility usage through the
use of guided lessons with interactive examples and feedback.

\end{abstract}
