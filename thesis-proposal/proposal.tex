\documentclass{task_description}
\usepackage[table]{xcolor}
\usepackage[bottom]{footmisc}

\newcommand{\todox}[1]{{\color{red}#1}}

\begin{document}

\thispagestyle{firstpage}
\vspace*{23mm}%
\hfill\parbox[t]{65mm}{
Herr\\
Qasim Warraich\\
Längassstrasse 54\\
3012 Bern \\[3mm]
Matrikel-Nr. 18-787-796 \\
qasim.warraich@uzh.ch\\[15mm]
\today \\
}
\vspace*{5mm}

\subsection*{Master's Thesis Specification}

\section*{``CLI Tutor''}

\subsection*{Introduction}

Despite the arguably dated appearance, difficult learning curve and practical
non existence in the general personal computing space, Command Line Interfaces
(CLIs) have more than stood the test of time in the software development world.
There are a multitude of extremely popular tools and applications that
primarily focus on the command line as an interaction medium. Some examples
include version control software like `git`, compilers and interpreters for
programming languages, package managers and various core utilities that are
popular in areas such as scripting and system administration.

As mentioned before,the use of command line interfaces has effectively
disappeared from a mainstream personal computer usage perspective. This
contributes greatly to the intimidation factor and learning difficulty for
those interested in getting into software engineering or system administration.
This paired with the inevitability of usage of CLIs in the development space
highlights a need to make the command line more accessible to new users for
whom text based interaction with their computer is an alien concept. In recent
years interactive learning utilising tools such as sandboxed environments have
been gaining in popularity and have the potential to be a suitable medium for
learning command line basics through actual usage, examples and practice. 

\subsection*{The goals of this Master's thesis}

This thesis aims to create a forgiving shell like interface with the goal of
teaching beginners basic CLI usage. The goal is to cover topics as shell
scripting basics and Unix-like core utility usage through the use of
interactive examples. The inspiration for this tool is form the `vimtutor`
\cite{pierce_ware_smith_moolenaar_2019} utility shipping along side the
extremely popular terminal based text editor Vim. A secondary goal would be to
wrap the shell learning tool in a web application in order to make it more
accessible and portable. To validate the tool, a user study will be conducted,
most likely with bachelor students at the University of Zurich.

\subsection*{Tasks}
\paragraph{Paragraph.}

\subsection*{Milestones}

\begin{tabular}{lp{10cm}}
Deadline & What \\
\hline\noalign{\smallskip}
March 1st & Official start of thesis. \\
March 8th  & Literature review complete. \\
March 15th & First draft of curriculum completed. \\
March 21st & Curriculum defined. \\
March 29th & Development of tool begins. \\
May 31st  & Review of tool \& Decision made regarding Web based version. \\
June 7th & Validation and study defined and begun. \\
June 29st  & Study completed. \\
July 5th  & Analysis of findings completed. \\
July 31st & Thesis submitted and presentation completed. \\
\end{tabular}

\subsection*{General thesis guidelines}

The typical rules of academic work must be followed. In
\cite{Bernstein2005-daguide}, Bernstein describes a number of guidelines which
must be followed. At the end of the thesis, a final report has to be
written. The report should clearly be organized, follow the usual academic
s.e.a.l. \LaTeX-template.

Since implementing software is also part of this thesis, state-of-the-art
design, coding, and documentation standards for the software have to be obeyed.

Effective feedback can only be provided to the student if the thesis draft is handed in well before the final deadline!

The diploma thesis has to be concluded with a final presentation for the members
of the Software Evolution and Architecture Lab (s.e.a.l.).

\subsection*{Special remarks}
\paragraph{Copyright.}
In accordance with current regulations, the student retains the copyright to his work, while providing a non-exclusive, non-revocable, time-unlimited license for it to the university. For this particular thesis, the student intends to keep all source code public so as to provide maximum accessibility to the proposed learning aid. 

\vspace{2em}
\noindent\textbf{Responsible assistant}: Dr. Carol Alexandru-Funakoshi

\vspace{2em}
\noindent\textbf{Signatures:}

\vspace{3\baselineskip}
\noindent Student Name\hfill Qasim Warraich
\vspace{2cm}
\bibliographystyle{abbrv}
\bibliography{proposal}

\end{document}

